%!TEX root = Tesis.tex
\chapter{Influence of the inlet geometry} 
\label{cap:geom}

As discussed in the literature review (chapter \ref{cap:liter}) and along the present chapter, different geometric variations of the inlet line near the impeller have been proposed and tested in order to mitigate the negative effect of the flow unstabilities.

While mainly aimed at obtaining a more stable flow to delay deep surge onset to enhance the usable zone of the map and keep compressor thermodynamic efficiency even at low air mass flow operating conditions, the impact of these solutions in the generated and transmitted noise must be also considered.

In this section different experimental measurement campaigns are described, where some of the geometries most commonly proposed in the literature were tested both on the continuous flow test rig and in real engine conditions. 

\section{Simple geometries}

The first of these experimental campaigns was carried by manufacturing a sample of each one of the simplest proposed geometries: a 90º sharp elbow, a tapered duct, and a reservoir. In addition, a short straight pipe was also used as a reference. All of these four geometries can be seen in Fig. \ref{fig:image_simple_geom}.

These simple geometries were installed on the anechoic chamber flow test rig described in chapter \ref{cap:metod}. Both the orifice noise measurement free-field microphone and the radiated noise setup with four equally distributed free-field microphones were kept in order to evaluate the acoustic emission in this preliminary investigation.

In order to simulate more realistic inlet conditions a short straight pipe was added after each of the tested geometries. This short pipe included a three piezoelectric sensor array in order to obtain  validation pressure data for CFD simulations, but no acoustic intensimetry could be performed as the pipes were too short for proper flow development.

\begin{figure}[tb!]
\centering
\includegraphics[width=0.9\textwidth]{figures/cap_geom/CAM00038_2}
\caption{Image of the selected simple geometries during the manufacturing process. The straight pipe at the top was used standalone as a reference and also attached upstream of each geometry in order to measure the instantaneous pressure.}
\label{fig:image_simple_geom}
\end{figure}

\subsection{Surge margin}

Furthermore, raw pressure data from one of these fast sensors was used to characterize with high precision the air mass flow at which the compressor entered into deep surge conditions. This way not only the acoustic output of each geometry could be measured but also their influence regarding the stability of the flow and thus the enlargement of the usable operating conditions.

As in the previously described experimental campaigns, this fast pressure data was recorded by a Yokogawa digital oscilloscope, along with the mass flow and shaft speed data. The rest of the ancillary information about the turbocharger operation was recorded by the in-house DAQ system.

Measurements with each inlet geometric variant were carried out at different air mass flow settings and a constant shaft speed of 160 krpm, from higher mass flow to lower mass flow. 

When conditions were judged to be close enough to deep surge, a continuous recording of the variables started where the mass flow was gradually decreased by slowly closing the backpressure valve until the violent oscillations of deep surge began and the valve was opened completely to stabilize the turbocharger.

\begin{figure}[htb!]
\centering
\includegraphics[width=0.49\textwidth]{figures/cap_geom/AutoSurge_straightData}
\includegraphics[width=0.49\textwidth]{figures/cap_geom/AutoSurge_reservoirData}\\[3mm]
\includegraphics[width=0.49\textwidth]{figures/cap_geom/AutoSurge_elbowData}
\includegraphics[width=0.49\textwidth]{figures/cap_geom/AutoSurge_taperedData}
\caption{Non-dimensional pressure recording of each geometry when air mass flow was progressively decreased until deep surge condition was reached.}
\label{fig:autosurge_simple_geom}
\end{figure}

This pressure recording was then post-processed following a procedure similar to that described in chapter \ref{cap:metod}. In steps of 0.5 s, FFT was applied to obtain the pressure spectra and then a ratio $R$ was computed between the power at 0-50 Hz and the power at 50-100 Hz:

\begin{equation}
	R=\frac{\sum_{f=0}^{50}|\mathbf{P}(f)|^2}{\sum_{f=50}^{100}|\mathbf{P}(f)|^2}
	%, \quad \mathbf{P}(f) = \mathcal{F}}\{p(t_i,t_i+0.5)\}
\end{equation}

When this power ratio reached values over 100, it was considered that deep surge had started at that time. Data was then averaged between the two previous time steps in order to obtain the corresponding value of lowest allowable air mass flow. Time steps and power ratios can be together with non-dimensional pressure tracks in Fig. \ref{fig:autosurge_simple_geom}. The lowest allowable mass flow obtained through this method can be seen in Fig. \ref{fig:minFlow_simple_geom}.

\begin{figure}[h!]
\centering
\includegraphics[width=0.75\textwidth]{figures/cap_geom/GeomComp_SurgeAMF}
\caption{Lowest allowable air mass flow for the different simple geometries tested, obtained through the described algorithm.}
\label{fig:minFlow_simple_geom}
\end{figure}

As seen in this figure, the straight duct is the first one to reach  deep surge conditions. The tapered duct and the 90º elbow pipe have similar but higher margins until surge, while the reservoir delays this damaging condition even more, allowing approximately 10 g/s of further air mass flow reduction.

\subsection{Noise levels}

In addition to the effect on surge margin, the effect on external noise levels was evaluated. As described in previous chapters, two different measurements systems were available in the anechoic chamber. 

To capture the orifice noise, one free-field omnidirectional microphone was situated at a distance of 10 mm from the inlet opening. In this case the microphone had to be repositioned each time the geometry was changed as they had different lengths and, in the case of the elbow, orientation of the orifice.

The second system for measuring externally radiated noise, consisting in the four equispaced free-field microphones placed at 1 m distance form the center of the inlet assembly was kept in the same place for all the tested geometries. 

In both cases the microphones were calibrated using a Brüel \& Kj\ae r pistonphone prior to the measurements and data was captured using a Pulse DAQ. Fig. \ref{fig:Radiated_Orifice_SPL_simple_geom} shows the result of both measurements at the stable operating point most close to surge as characterized by the aforementioned procedure, since it is the point of maximum noise emission.

\begin{figure}[h!]
\centering
\includegraphics[width=0.75\textwidth]{figures/cap_geom/GeomComp_Radiated_Orifice_SPL}
\caption{Comparison of orifice noise and averaged radiated noise measured in the anechoic chamber for each of the simple inlet geometries at the stable operating point most close to surge.}
\label{fig:Radiated_Orifice_SPL_simple_geom}
\end{figure}

It can be seen that again there are significant differences between the inlet geometries tested. These differences were higher when measuring the radiated noise through the average of the four microphones, with the reservoir featuring a 3 dB reduction compared to the reference straight duct. 

On the other hand, the 90º elbow inlet reached a 5 dB increase over the straight inlet level, but this particular result could be influenced by the fact that the direction of the orifice was not perpendicular to the plane of the four microphones. Finally, the tapered duct seemed to radiate at the same level that the straight pipe.

Regarding orifice noise, the reservoir featured again a reduction in noise level of 2 dB. However, in this case the elbow featured the same level as the reference straight pipe, being the tapered duct the one with a higher level at 130 dB, 3 dB more than the reference geometry.

In Fig. \ref{fig:GeomComp_OrificeSpectra_ann} the different spectra recorded by the orifice microphone during each geometry test are plotted to look in detail at these level differences. It can be seen how levels for the tapered duct are higher along the whole frequency range, while the inlets featuring the elbow and the reservoir remain closer to the reference straight pipe.

\begin{figure}[h!]
\centering
\includegraphics[width=0.75\textwidth]{figures/cap_geom/GeomComp_OrificeSpectra_ann}
\caption{Comparison of orifice noise spectra for each of the simple inlet geometries at the stable operating point most close to surge.}
\label{fig:GeomComp_OrificeSpectra_ann}
\end{figure}

The most noticeable difference however, is the resonator effect of the reservoir, that provides an important attenuation at frequencies in the plane wave region of the spectrum. Finally, a small peak can be noticed at 16 kHz, correspondent to the Blade Passing Frequency  associated to the 160 krpm shaft speed that was used during these measurements.

\section{On-engine tests}

In the scope of the on-engine tests described in chapter \ref{cap:engine}, a follow-up experimental campaign was carried out in order to characterize the effect that different geometric variations of the turbocharger inlet line had in performance, surge margin and noise emission.

\subsection{Measurement setup}

The selected geometric variations of the inlet line were chosen taking into account realistic packaging size restrictions of commercial vehicles, so in this case the reservoir case was not tested as it was not possible to fit it to the available compressor inlet. 

Similarly, elbow inlets were not tested as they would require to re-route the inlet line, being its design already frozen at the time of the testing. Focus was then put into small changes like tapered ducts, rotation devices and nozzle-like additions to the inlet pipe.

\begin{figure}[h!]
\centering
\includegraphics[width=0.95\textwidth]{figures/cap_geom/geom_engine}
\caption{Longitudinal section view of the inlet geometries that were considered for the on-engine experimental campaign.}
\label{fig:geom_engine}
\end{figure}

Figure \ref{fig:geom_engine} shows the selected geometry variations based on these selection criteria. On top of the figure the reference geometry is depicted. The turbocharger inducer was milled and substituted by an adaptor to which a straight duct was coupled. At the end of this duct a pipe of the original diameter was then affixed with a tapered transition.

\begin{wrapfigure}{o}[7mm]{0.4\textwidth}
%\begin{figure}[h!]
\centering
\includegraphics[width=0.4\textwidth]{figures/cap_geom/IGV_fitted}
\caption{Picture of the 3D-printed IGV device fitted to the compressor inlet duct, looking towards the impeller.}
\label{fig:IGV_fitted}
%\end{figure}
\end{wrapfigure}

The simplest inlet geometry tested consisted in a tapered duct situated as close as possible to the compressor impeller leading edge. This can be seen at the center left in the figure. Below, it is shown how this tapered duct was complemented with an inlet guide vane (IGV) device.

This was done in order to increase the swirl of the incoming air flow. The device stators featured 0º leading edge angles and $\sim 70$º trailing edge angles, and it was directly 3D-printed from the CAD design. The device can be seen fitted into the inlet pipe in Fig. \ref{fig:IGV_fitted}.

To analyze how the closeness of the tapered duct to the impeller influenced the behaviour of the compressor, the two following geometric variations (center column of Fig. \ref{fig:geom_engine}) were the same except for the separation between the tapered duct and the impeller, which was increased to 52 mm. Again, one measurement was performed with and IGV and other without it.

Lastly, two nozzles were tested (right column of Fig. \ref{fig:geom_engine}): a convergent-divergent nozzle between the tapered adaptor and the impeller (top) and a convergent nozzle (bottom) just after the tapered duct, with its throat as close as possible to the impeller wheel.

In order to evaluate the influence of these geometric variation on the noise output of the compressor, the acoustic measurement setup used in chapter \ref{cap:engine} was maintained, consisting in the straight inlet extension featuring the pressure decomposition array for in-duct intensimetry and the intensity probe aligned with the inlet orifice.

Measurement operating conditions for the noise characterization campaign were chosen \hl{as to be representative of the most common real-life conditions that the engine would experiment in a commercial vehicle, in order to test if these simple inlet modifications could modify the noise output not only on the most extreme conditions but in more conservative settings.}

\subsection{Results}

\hl{Since the objective of this experimental campaign is the comparison between geometries, in this subsection only the results from the orifice intensity probe will be presented for the sake of brevity.}

In Fig. \ref{fig:geom_mot_comparison_map1} the interpolated intensity level map is presented, indicating the points at which each measurement was made. The interpolation of sound intensity was performed between frequencies of 1 and 3 kHz, which are specially relevant regarding the aforementioned ``whoosh'' issue.

\begin{figure}[h!]
\centering
\includegraphics[width=0.7\textwidth]{figures/cap_geom/map_engine_geom_config1}
\caption{Sound intensity level map between 1--3 kHz for the straight inlet used as a reference in the on-engine inlet geometric variation test.}
\label{fig:geom_mot_comparison_map1}
\end{figure}

Note as well that because of the particular regulation mechanism of the facility based on the injection of compressed air into the inlet line, the depicted operating conditions do not represent constant shaft speed lines, so no judgement about the slope of these lines should be made.

However, it can be seen in this figure that the noise levels in the selected band behave as results of presented in previous chapters led to expect. Noise levels increase as pressure ratio increases and air mass flow diminishes, with the dependency on air mass flow being much more noticeable at higher pressure ratios that at lower ones.

\begin{figure}[h!]
\centering
\includegraphics[width=0.45\textwidth]{figures/cap_geom/map_engine_geom_config2}\hspace{0.05\textwidth}
\includegraphics[width=0.45\textwidth]{figures/cap_geom/map_engine_geom_config3}\\[2mm]
\includegraphics[width=0.45\textwidth]{figures/cap_geom/map_engine_geom_config4}\hspace{0.05\textwidth}
\includegraphics[width=0.45\textwidth]{figures/cap_geom/map_engine_geom_config5}\\[2mm]
\includegraphics[width=0.45\textwidth]{figures/cap_geom/map_engine_geom_config6}\hspace{0.05\textwidth}
\includegraphics[width=0.45\textwidth]{figures/cap_geom/map_engine_geom_config7}
\caption{Sound intensity level maps between 1--3 kHz for the different geometries used in the on-engine test.}
\label{fig:geom_mot_comparison_maps}
\end{figure}

Figure \ref{fig:geom_mot_comparison_maps} shows the results for the same frequency band when the reference straight duct was substituted by each of the variants described in Fig. \ref{fig:geom_engine}. 

It is apparent from the figure that different operating conditions were measured in each one of the tests, as the goals in this campaign were to characterize the sound emission across the selected map region and to test how each geometry impacted the performance in regards of surge margin and maximum ingested air mass flow.

In this regard, the figure shows how the geometry changes have an impact on surge margin (left frontier of the map), as some of the modified inlet lines increase the range of low flow operating conditions that can be safely used before going into deep surge, specially at higher shaft speeds; compare for instance the highest row of points of the straight inlet in Fig. \ref{fig:geom_mot_comparison_map1} and that of the convergent-divergent nozzle (bottom left plot of Fig. \ref{fig:geom_mot_comparison_maps}).

Impact of the geometry variation on air mass flow was also tested in regards of maximum flow at 160 krpm, as it is not only important to increase surge margin but also to maintain the possibility of operating the turbocharger at high mass flow levels. 

\begin{figure}[h!]
\centering
\includegraphics[width=0.95\textwidth]{figures/cap_geom/geom_mot_comparison_AMF}
\caption{Comparison of mean sound intensity level for each of the inlet geometries across the normal operating range of the engine.}
\label{fig:geom_mot_comparison_AMF}
\end{figure}

A comparison of the performance of each geometry in both aspects is presented in Fig. \ref{fig:geom_mot_comparison_AMF} where the percentages of improvement from the straight duct reference inlet are plotted. An inverse correlation between maximum flow and surge margin is apparent, only the tapered duct at 52 mm from the impeller improving both factors at the same time and only the tapered duct closer to the impeller and featuring an IGV worsening both.

Largest amount of surge margin improvement was found with the convergent nozzle, but at the cost of an equally large air mass flow reduction at $PR = 2$ and 160 krpm of shaft speed, so in order to implement this solution in real-life applications it should be necessary to mechanically open the nozzle throat at these high flow conditions.

To similarly quantify the influence of the geometries in noise emission at the selected frequency band, in Fig. \ref{fig:geom_mot_comparison_meanSIL} an average sound intensity level (SIL) is presented, as measured by the Brüel \& Kj\ae r intensity probe located pointing directly to the inlet orifice.

\begin{figure}[h!]
\centering
\includegraphics[width=0.95\textwidth]{figures/cap_geom/geom_mot_comparison_meanSIL}
\caption{Comparison of mean sound intensity level for each of the inlet geometries across the normal operating range of the engine.}
\label{fig:geom_mot_comparison_meanSIL}
\end{figure}

Intensity level of the tapered duct is very similar to that of the reference straight pipe; both when located as close as possible to the compressor wheel and when separated 52 mm from it. The addition of the described IGV to those tapered inlets notably reduces noise levels by 3 and 4 dB respectively. 

Regarding the nozzle configurations, it can be seen that the convergent-divergent nozzle reduces noise output in the selected frequency band by 1 dB, whereas the convergent nozzle features a notable reduction of approximately 3 dB compared to the reference.

\section{Elbow parametric study}

Having attested the influence of the inlet geometries not only in surge margin and maximum flow but also regarding acoustic performance, it was decided to use the turbocharger test rig used for the inlet local flow measurements described in chapter \ref{cap:inlet} to study the effect of these geometric variations not only in the global aspects of sound intensity and air mass flow but also regarding local aspects like temperature distribution, skewness, etc.

\hl{The gathering of local flow experimental data will also be useful if CFD simulations of these kind of inlet modifications are undertaken, in order to provide a validation reference to check for the correct solving of not only global values but also local flow field characteristics.}

Since neither during the previously described on-engine experimental campaign nor during the preliminary testing described at the beginning of the present chapter was it possible to properly measure the sound intensity caused by elbow geometries, it was decided that the study should start with this kind of geometry variation.

Another reason to consider this geometry is that it introduces non-uniformities in the flow field that completely axisymmetric geometries such as tapered ducts and nozzles do not, so it would provide a good contrast to check the influence on circumferential temperature distribution skewness and correlation with noise generation.

In order to increase the available data on the relation between inlet-induced flow distortion, temperature skewness and noise output, it was decided that a parametric campaign should be carried out where three 90º elbows of different radius would be designed, manufactured and tested.

\subsection{Geometry design}

\begin{figure}[h!]
\centering
\includegraphics[width=0.9\textwidth]{figures/cap_geom/codo075D}
\caption{Mechanical drawing of the 0.75D elbow inlet designed to fit into the compressor inlet adaptor and house the linear thermocouple array.}
\label{fig:codo075D}
\end{figure}

Requirements for the design of the elbow inlets were such that they could be easily installed into the existing setup used for local measurements, coupling to the adaptor piece and existent measurement pipe featuring the in-duct intensity sensor array.

In addition, they needed to house the available linear thermocouple arrays in order to measure the high temperature reversed flow (it should be noted that the circumferential thermocouple array was installed into the adaptor piece) approximately keeping the distance that was selected for the straight pipe study.

Finally, to assess the influence of the flow distortion, the three selected variants were as follows: a very tight elbow with a central radius of 0.75 inlet diameters, a medium elbow with 1 inlet diameter of radius and finally an elbow with a central radius of 1.5 inlet diameters that would provide a smoother transition and less distorted flow.

To fulfil these design requirements, a decision was made to manufacture the elbow prototypes using 3D printing instead of traditional workshop means. This way, the guiding slots for the thermocouples and retention screws could be directly embedded in the piece, along with the slots for the sealing O-rings. Fig. \ref{fig:codo075D} shows the mechanical drawing of the most abrupt elbow with 0.75D of radius.

\begin{figure}[h!]
\centering
\includegraphics[width=0.95\textwidth]{figures/cap_geom/Comp_codos}
\caption{CAD model of the compressor assembly featuring the 0.75D elbow inlet attached to the common adaptor, including the pressure and temperature instrumentation and the mounting system.}
\label{fig:Comp_codos}
\end{figure}

This design was implemented in a parametric CAD model to easily modify the central radius. Also, a locking system was put into place so the elbow could be affixed tightly to the compressor adaptor using a flange and a set of threaded rods, similarly to the procedure performed for the local flow characterization. The three prototypes generated in this way are shown in Fig. \ref{fig:Comp_codos}.

Once that the the three CAD models were generated, they were sent to a commercial partner for verification and manufacturing. It should be noted that the elbow prototypes were expected to withstand the high temperature backflow coming from the compressor in unstable operating conditions close to deep surge, without undertaking deformations that would have misaligned the thermocouple array.

Taking into account these concerns, it was decided to manufacture the elbow prototypes using an aluminium-filled polyamide through a process known as Selective Laser Sintering (SLS), a technique in which a laser beam is focused onto the surface of a tray containing the powdered material, binding it together in layers to create the desired shape. 

\begin{table}[htbp]
  \centering
  \caption{Specifications of the aluminium-filled polyamide used for the laser-sintering of the elbow inlet prototypes.}
  	\footnotesize
    \begin{tabular}{llcc}
    \addlinespace
    \toprule
    \bf Property & \bf Normative & \bf Unit & \bf Value \\
    \midrule
    Tensile Modulus & DIN EN ISO 527 & MPa   & 3800 $\pm$ 150 \\
    Tensile Strength & DIN EN ISO 527 & MPa   & 48 $\pm$ 3 \\
    Elongation at Break & DIN EN ISO 527 & \%    & 3.5 $\pm$ 1 \\
    Flexural Modulus & DIN EN ISO 178 & N/mm² & 3600 $\pm$ 150 \\
    Charpy – Impact strength & DIN EN ISO 179 & MPa   & 29 $\pm$ 2 \\
    Charpy – Notched Impact Strength & DIN EN ISO 179 & MPa   & 4.6 $\pm$ 0.3 \\
    Shore D/ A-hardness & DIN 53505 &   -   & D 76 $\pm$ 2 \\
    Heat Deflection Temp & ASTM D648
(1.82MPa) & °C    & 130 \\
    Vicat Softening Temperature B/50 & DIN EN ISO 306 & °C    & 169 \\
    Density &   N/A    & g/cm³ & 1.36 $\pm$ 0.05 \\
    \bottomrule
    \end{tabular}
  \label{tab:alumide_data}
\end{table}

Specifications of the selected material can be found in table \ref{tab:alumide_data}. Of special relevance is the heat deflection temperature, also called sometimes heat distortion temperature. It is defined as the temperature at which a sample, under a defined load of 1.82 MPa, will start to deflect at least 0.25 mm. Since in this case the inlet line is approximately at ambient pressure no problem is expected until more elevated temperatures.

\subsection{Setup}

After SLS manufacturing and delivery, the prototype elbows were instrumented with thermocouples and installed into the anechoic chamber test rig. Fig. \ref{fig:elbow_075D_assembled} shows the assembly of the narrower elbow variant.

\begin{figure}[h!]
\centering
\includegraphics[width=0.95\textwidth]{figures/cap_geom/elbow_075D_assembled}
\caption{CAD model of the compressor assembly featuring the 0.75D elbow inlet attached to the common adaptor, including the pressure and temperature instrumentation and the mounting system.}
\label{fig:elbow_075D_assembled}
\end{figure}

It should be noted that the narrow holes for the thermocouple alignment became filled with residual dust during the sintering process, and had to be finished with a manual drill. This presented no problem as the chosen material is easily mechanized.

As it can be seen in Fig. \ref{fig:elbow_075D_assembled} the previously used adaptor piece housing the circumferential temperature array, the inducer miniature pressure probe and a sealing O-ring was kept in place, and the holes in the plane surface added to the elbow were used to align the elbow parallel to the outlet and tighten it using the threaded rods depicted in blue.

\subsection{Results}

During this experimental campaign, operating conditions were selected to mirror those used in the local flow characterization chapter that featured a straight inlet pipe, in order to allow the reconstruction of interpolated maps that were as close as possible to the original reference map.

\begin{figure}[bh!]
\centering
\includegraphics[width=0.6\textwidth]{figures/cap_geom/map_elbows}
\caption{Operating conditions at which data was recorded during the inlet flow characterization tests with a straight duct (as reference) and during the parametric elbow campaign.}
\label{fig:map_elbows}
\end{figure}

Fig. \ref{fig:map_elbows} shows the operating conditions of each measured data point, for both the straight inlet pipe used as reference, and for the three described elbow prototypes. It can be seen that repeatability was achieved in most conditions up to a few g/s of air mass flow and 100's of corrected rpm. 

The most adverse conditions were those of maximum allowable flow, specially at higher pressure ratios where the flow distortion and pressure loss caused by the elbows made difficult to achieve exactly the same conditions.

\subsubsection{Sound intensity}

Using the in-duct piezoelectric sensor arrays, sound intensity level was computed through the wave decomposition method outlined in chapter \ref{cap:metod} for each operating conditions. In order to focus into the previously described ``whoosh'' noise problem sound intensity level was averaged between 0.7 and 3 kHz. and then the interpolating procedure was used in order to reconstruct noise maps for each geometry.

\begin{figure}[thb!]
\centering

\includegraphics[width=0.47\textwidth]{figures/cap_geom/map_diff_mapIin_030316_Inlet}\hspace{4mm}
\includegraphics[width=0.47\textwidth]{figures/cap_geom/map_diff_mapIout_030316_Outlet}\\[4mm]

\includegraphics[width=0.47\textwidth]{figures/cap_geom/map_diff_mapIin_260216_Inlet}\hspace{4mm}
\includegraphics[width=0.47\textwidth]{figures/cap_geom/map_diff_mapIout_260216_Outlet}\\[4mm]

\includegraphics[width=0.47\textwidth]{figures/cap_geom/map_diff_mapIin_150216_Inlet}\hspace{4mm}
\includegraphics[width=0.47\textwidth]{figures/cap_geom/map_diff_mapIout_150216_Outlet}\\[4mm]

\caption{Difference map of inlet (left) and outlet (right) in-duct sound intensity levels for each of the selected inlet elbows compared against the reference straight duct. From top to bottom: 1.5D, 1D \& 0.75D.}
\label{fig:map_diff_in_out}
\end{figure}

To highlight the differences between the reference straight duct noise map and the result obtained for each elbow prototype, and taking into account the aforementioned slight differences between operating conditions at higher air mass flow settings, difference maps were produced by subtracting the reference levels to each one of the newly obtained elbow maps.

It should be remarked that this subtraction procedure was not performed between the raw levels at each measurement setting but between each interpolated data point, and only where such interpolated data existed in both noise maps (elbow and reference), as to avoid erroneous differences due to the different operating conditions. 

The results of this procedure can be seen in Fig. \ref{fig:map_diff_in_out}, where the difference maps for the three elbow configurations regarding inlet and outlet in-duct intensity are presented. Color scale has been kept constant in order to allow a meaningful comparison between each map.

Regarding the issue of the small differences at higher air mass flow and higher shaft speed settings it can be seen in the maps, examining for instance the highest mass flow point of the second to last higher shaft speed (160 krpm), that some information is missing around the black dot that indicated the original reference measurement point. 

As for the evolution of level difference with the variation of the elbow radius, the figure shows how in the case of the inlet sound intensity differences appear to be higher than in the case of the outlet. This increase is specially evident at lower flow conditions and medium-low shaft speeds, reaching up to 6 dB in level increase. 

There also appears to be a particular point with an small increase in level for both inlet and outlet ducts, at medium air mass flow and medium to hight shaft speeds, its relevance diminishing along with the reduction of the elbow radius.

Finally, some operating conditions in both the inlet and outlet present reduced noise output in the selected frequency band, specially at medium-higher mass flows and lower shaft speeds. As with the case of the increases, these reductions are more significant in the inlet duct, whereas positive and negative variations are of smaller magnitude in the outlet duct.

In order to better quantify the SIL differences that these maps reveal, histograms that represent the distribution of the differences have been computed and presented in Fig. \ref{fig:elbow_intens_histog} for both inlet (top) and outlet (bottom) sound intensity results.

It can be seen in the figure how the different elbow configurations affect the distribution of the intensity level differences. In the case of the inlet, the less narrow elbow (1.5D) and the medium elbow (1D) present a similar, almost symmetric distribution with a mean around -1 dB.

\begin{figure}[htb!]
\centering
\includegraphics[width=0.95\textwidth,trim={3cm 11cm 3cm 11cm},clip]{figures/cap_geom/distribution_diff_mapIin}
\includegraphics[width=0.95\textwidth,trim={3cm 11cm 3cm 11cm},clip]{figures/cap_geom/distribution_diff_mapIout}
\caption{Histograms of inlet (top) and outlet (bottom) sound intensity level differences between each one of the tested elbow variants and the reference straight inlet pipe. Y axis indicates the percentage of the interpolated compressor map at which each difference level occurred. A dashed line indicates the average of each configuration.}
\label{fig:elbow_intens_histog}
\end{figure}

The distribution of the most narrow elbow (0.75D) appears to feature a marked skewness towards positive differences of up to 6 dB that correspond with the low flow, medium-low speed region identified in the aforementioned intensity maps. These points shift the mean up to approximately 0.2 dB of increased level.

Regarding the outlet difference distributions, the situation appears to reverse, with the narrower elbow now featuring the most symmetric distribution and the other two variations presenting a marked skewness towards intensity reduction values, again up to a mean of around 2 dB in the most favorable case (1.5D elbow).

\begin{figure}[h!]
\centering
\includegraphics[width=0.88\textwidth]{figures/cap_geom/elbow_comp_1sens}
\caption{Comparison of plane wave range spectra from both inlet and outlet, highlighting the differences between the straight reference duct and each elbow configuration.}
\label{fig:elbow_comp_1sens}
\end{figure}

Focusing now into the detailed aspects of the sound level variation, it is possible to compare the pressure spectra in different points to assess the frequency distribution of the level variation. For instance in Fig. \ref{fig:elbow_comp_1sens} the PSD recorded by inlet and outlet piezoelectric sensors is shown for the different inlet geometries, in the plane wave region.

In the case of the outlet duct, spectra are very similar up to a cut-off point at approximately 1.3 kHz, below which the straight duct presents higher levels than all of the elbows.

On the other hand, spectra on the plane wave region of the inlet duct shows higher differences, consistent with the data already presented in the maps and histograms. 

It is interesting to note however that this increase is not homogeneous in frequency; it can be seen how the small peak at 1.7 kHz present in the straight duct gradually increases with elbow radius reduction, up to a clearly visible broadband between 1.5 and 2.5 kHz.

These are flanked by another two smaller broadband elevations at 0.75 and 3.6 kHz that are also visible in the case of the outlet spectra, and were also very clearly present in the pressure spectra recorded by the diffuser pressure probe, shown for instance in \hl{[fig]}.

It should be noted as well that, while levels for the detected broadband in the inlet show an increase up to 5 dB, at these frequencies all of the outlet spectra plotted in the lower graph remain at the same levels, indicating that while changes in the inlet configuration seem to boost the upstream transmission of the so-called ``whoosh'' broadband noise, no influence on the outlet duct spectra is detected at these frequencies.

\subsubsection{Temperature}

\hl{[Thermographic camera results]}

Besides the described differences in terms of acoustic intensity produced by the different inlet configurations, it is possible to follow a similar processing procedure in order to assess the differences in terms of temperatures in the inlet section that are related to the hot reversed flow coming from after the compressor impeller.

As in the previous experimental campaign, data from the two thermocouple arrays (longitudinal and circumferential) was processed in order to compile maps of the backflow length and flow temperature skewness, respectively.

These maps were then compared with the original reference map measured with a straight inlet duct, and difference maps were produced for each elbow variant. The resulting difference maps can be seen in Fig. \ref{fig:map_diff_std}. The same color scale has been kept in all of them to provide a suitable comparison.

\begin{figure}[h!]
\centering
\includegraphics[width=0.47\textwidth]{figures/cap_geom/map_diff_mapTcorStdDev_030316_}\hspace{4mm}
\includegraphics[width=0.47\textwidth]{figures/cap_geom/map_diff_mapTcorStdDev_260216_}\\[3mm]
\includegraphics[width=0.47\textwidth]{figures/cap_geom/map_diff_mapTcorStdDev_150216_}\hspace{4mm}
\caption{Difference maps of temperature circumferential standard deviation for each of the selected inlet elbows compared against the reference straight duct.}
\label{fig:map_diff_std}
\end{figure}

It can be seen that for the three cases a similar patters appears. An increase in temperature skewness is consistently detected at the lower air mass flow settings, specially at medium shaft speed operating conditions.

At higher air mass flow settings however, a slight reduction in flow temperature skewness is apparent, being much more pronounced in the less narrow elbow configuration (1.5D, top left map). 

This reduction area spans approximately the same conditions for the 1D elbow, but the magnitude of the temperature skewness difference is lower than in the previous case, becoming nearly zero in the case of the most reduced elbow variant (0.75D).

\begin{figure}[h!]
\centering
\includegraphics[width=0.95\textwidth,trim={3cm 11cm 3cm 11cm},clip]{figures/cap_geom/distribution_diff_Tstd}
\caption{Histograms of temperature standard deviation differences between each one of the tested elbow variants and the reference straight inlet pipe. Y axis indicates the percentage of the interpolated compressor map at which each difference level occurred. A dashed line indicates the average of each configuration.}
\label{fig:distribution_diff_std}
\end{figure}

As in the case of sound intensity level differences, it is possible to combine these maps in the form of histograms in order to better quantify the distribution of these differences. These histograms can be seen plotted in Fig. \ref{fig:distribution_diff_std}.

Distributions shown in this figure appear similar in shape, but shifted in mean value. In all three cases the right tail of the distribution is longer but less populated, indicating that the difference increase happens abruptly in a small region of the compressor map. 

On the contrary, the left tails are more populated but shorter, meaning that more regions of the compressor map feature points with slightly lower differences than the average. These averages reach up to 1ºC in the case of the most narrow elbow configuration (0.75D), while the average is almost zero for the case of the less narrow elbow (1.5D).

As mentioned earlier, by processing the temperature profiles measured by the longitudinal thermocouple array that was added to the design of each elbow prototype, it is also possible to estimate the length of the hot reversed backflow that travels upstream of the impeller and into the inlet in unstable operating conditions.

When designing the prototypes, care was put to ensure that the separation between the thermocouples remained equal to that of the straight inlet duct, taking into account their inserted length of half a radius and the varying radius of each elbow pipe.

In Fig. \ref{fig:map_diff_backL} the estimated length along the exterior edge of the elbow (where the thermocouple array was installed, as shown in Fig. \ref{fig:codo075D}) is mapped for the three elbow configurations.

\begin{figure}[h!]
\centering
\includegraphics[width=0.47\textwidth]{figures/cap_geom/map_diff_mapBackLen_030316_}\hspace{4mm}
\includegraphics[width=0.47\textwidth]{figures/cap_geom/map_diff_mapBackLen_260216_}\\[3mm]
\includegraphics[width=0.47\textwidth]{figures/cap_geom/map_diff_mapBackLen_150216_}\hspace{4mm}
\caption{Difference maps of high temperature backflow extent for each of the selected inlet elbows compared against the reference straight duct.}
\label{fig:map_diff_backL}
\end{figure}

In this case differences range from minus half a diameter up to plus a diameter when compared with the reference straight pipe. The patters of these differences along the compressor map appears to be similar for the three geometric variations: a slight reduction at lower mass flow conditions and a slight increase at higher mass flow conditions for all shaft speeds.

Exceptions to this pattern are some of the lowest air mass flow data points at 100 and 120 krpm for the two narrower elbows (0.75D and 1D), where the backflow length appears to feature a large increase. However it is possible that those are spurious results from the interpolation procedure, since the increase region is very small.

\begin{figure}[h!]
\centering
\includegraphics[width=0.95\textwidth,trim={3cm 11cm 3cm 11cm},clip]{figures/cap_geom/distribution_diff_mapBackLen}
\caption{Histograms of high temperature backflow extent differences between each one of the tested elbow variants and the reference straight inlet pipe. Y axis indicates the percentage of the interpolated compressor map at which each difference level occurred. A dashed line indicates the average of each configuration.}
\label{fig:distribution_diff_backL}
\end{figure}

These low count extreme points are best shown when the data is put into histogram form as seen in Fig. \ref{fig:distribution_diff_backL}. In this figure it can be seen how the three backflow length difference distributions are quite similar in shape and values.

The three distributions feature approximately the same mean (around 0.1 diameters or of backflow length reduction from the straight inlet case) although for the higher number of the points (that is, the mode) the difference is practically zero.

\subsubsection{Correlations}

\hl{[Correlation figures for the three elbows]}

\section{Commercial ducts}

\hl{[Comparisons of in-duct \& external noise between straight, Jag and LR commercial inlets?]}

\section{Conclusions}

Through this chapter, three different experimental campaigns aimed to assess the influence of the inlet geometry in the turbocharger acoustic output have been presented. While different in scope and means, the results shown in this chapter allow some global conclusions to be drawn.

First, a series of tests was carried out where simple geometric variants were added to the compressor inlet, with a straight duct of the same diameter than the compressor inducer being as a reference. This reference was maintained for the three experimental campaign described. Both orifice noise recorded with a free-field microphone and orifice noise averaged between four microphones were recorded.

Results of this first test series showed that, apart from an effect in the turbocharger surge margin that had been previously pointed out by \hl{[cites]}, influence of the geometries on both orifice and radiated noise could be found. The tapered duct appeared to increase the orifice noise, while the reservoir acted as a resonator and not only increased surge margin but lowered both orifice (2 dB) and radiated noise (3 dB). Meanwhile, the elbow increased kept the same orifice noise levels.

Another series of geometry tests were carried out in an engine cell and introduced more complex geometry combinations, with tapered ducts featuring an IGV device and the addition of two nozzles to the inlet line, besides the reference straight pipe. In this case noise output was measured by a commercial intensity probe aligned with the inlet orifice and with the in-duct intensimetry method based on beamforming wave decomposition described in chapter \ref{cap:metod}.

Acoustic data from this on-engine campaign confirmed that the selected tapered duct, while demonstrating an improvement in surge margin and maximum flow when located at a certain distance from the impeller, did not demonstrate a significative effect in orifice noise intensity and in any case these differences were harmful with a slight increase in levels, albeit of less than 0.5 dB.

The addition of an IGV device to boost incoming flow swirl proved interesting as it decreased the acoustic intensity recorded by the orifice probe by 3-4 dB depending on the position of the tapered duct. However this device negatively impacted the maximum flow and, depending in the position of the downstream tapered section, surge margin. 

In the case of the inlet geometries equipped with a nozzle, a decrease of orifice noise was found ranging from 1 dB (converge-divergent nozzle) to 3 dB (convergent nozzle). These nozzles also improved surge margin up to 20\%, but, specially the convergent one, decreased the maximum available air mass flow.

Finally, a more through experimental campaign was carried out where a family of 90º elbow inlets with different radii (that were not tested on the engine due to packaging constraints) were installed on the anechoic test ring and instrumented to measure not only in-duct noise but also temperature distribution through the thermocouple arrays described in chapter \ref{cap:inlet}.

Results from this campaign provided higher in-depth insight into the influence of the elbow radius; as in the aforementioned first testing campaign the very sharp elbow showed equal or slightly higher mean sound intensity in the inlet. However, this increase was not uniformly distributed across the compressor map but concentrated on the lowest allowable air mass setting for each shaft speed. This pattern was shared among the three tested elbow configurations, although as the radius grew sound intensity levels in both inlet and outlet reduced up to 1 dB in average.

Regarding the influence of the inlet elbow variants on the temperature distribution, a similar pattern of circumferential skewness increase towards the lowest allowable flow conditions for each shaft speed can be reconstructed. However, reversed flow length appears similar in all cases, with differences of less than 0.5 inlet diameters.
