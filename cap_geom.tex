%!TEX root = Tesis.tex
\chapter{Influence of the inlet geometry} 
\label{cap:geom}

As discussed in the literature review (chapter \ref{cap:liter}) and along the present chapter, different geometric variations of the inlet line near the impeller have been proposed and tested in order to mitigate the negative effect of the flow unstabilities.

While mainly aimed at obtaining a more stable flow to delay deep surge onset to enhance the usable zone of the map and keep compressor thermodynamic efficiency even at low air mass flow operating conditions, the impact of these solutions in the generated and transmitted noise must be also considered.

In this section different experimental measurement campaigns are described, where some of the geometries most commonly proposed in the literature were tested both on the continuous flow test rig and in real engine conditions. 

\section{Simple geometries}

The first of these experimental campaigns was carried by manufacturing a sample of each one of the simplest proposed geometries: a 90º sharp elbow, a tapered duct, and a reservoir. In addition, a short straight pipe was also used as a reference. All of these four geometries can be seen in Fig. \ref{fig:image_simple_geom}.

These simple geometries were installed on the anechoic chamber flow test rig described in chapter \ref{cap:metod}. Both the orifice noise measurement free-field microphone and the radiated noise setup with four equally distributed free-field microphones were kept in order to evaluate the acoustic emission in this preliminary investigation.

In order to simulate more realistic inlet conditions a short straight pipe was added after each of the tested geometries. This short pipe included a three piezoelectric sensor array in order to obtain  validation pressure data for CFD simulations, but no acoustic intensimetry could be performed as the pipes were too short for proper flow development.

\begin{figure}[tb!]
\centering
\includegraphics[width=0.9\textwidth]{figures/cap_geom/CAM00038_2}
\caption{Image of the selected simple geometries during the manufacturing process. The straight pipe at the top was used standalone as a reference and also attached upstream of each geometry in order to measure the instantaneous pressure.}
\label{fig:image_simple_geom}
\end{figure}

\subsection{Surge margin}

Furthermore, raw pressure data from one of these fast sensors was used to characterize with high precision the air mass flow at which the compressor entered into deep surge conditions. This way not only the acoustic output of each geometry could be measured but also their influence regarding the stability of the flow and thus the enlargement of the usable operating conditions.

As in the previously described experimental campaigns, this fast pressure data was recorded by a Yokogawa digital oscilloscope, along with the mass flow and shaft speed data. The rest of the ancillary information about the turbocharger operation was recorded by the in-house DAQ system.

Measurements with each inlet geometric variant were carried out at different air mass flow settings and a constant shaft speed of 160 krpm, from higher mass flow to lower mass flow. 

When conditions were judged to be close enough to deep surge, a continuous recording of the variables started where the mass flow was gradually decreased by slowly closing the backpressure valve until the violent oscillations of deep surge began and the valve was opened completely to stabilize the turbocharger.

\begin{figure}[htb!]
\centering
\includegraphics[width=0.49\textwidth]{figures/cap_geom/AutoSurge_straightData}
\includegraphics[width=0.49\textwidth]{figures/cap_geom/AutoSurge_reservoirData}\\[3mm]
\includegraphics[width=0.49\textwidth]{figures/cap_geom/AutoSurge_elbowData}
\includegraphics[width=0.49\textwidth]{figures/cap_geom/AutoSurge_taperedData}
\caption{Non-dimensional pressure recording of each geometry when air mass flow was progressively decreased until deep surge condition was reached.}
\label{fig:autosurge_simple_geom}
\end{figure}

This pressure recording was then post-processed following a procedure similar to that described in chapter \ref{cap:metod}. In steps of 0.5 s, FFT was applied to obtain the pressure spectra and then a ratio $R$ was computed between the power at 0-50 Hz and the power at 50-100 Hz:

\begin{equation}
	R=\frac{\sum_{f=0}^{50}|\mathbf{P}(f)|^2}{\sum_{f=50}^{100}|\mathbf{P}(f)|^2}
	%, \quad \mathbf{P}(f) = \mathcal{F}}\{p(t_i,t_i+0.5)\}
\end{equation}

When this power ratio reached values over 100, it was considered that deep surge had started at that time. Data was then averaged between the two previous time steps in order to obtain the corresponding value of lowest allowable air mass flow. Time steps and power ratios can be together with non-dimensional pressure tracks in Fig. \ref{fig:autosurge_simple_geom}. The lowest allowable mass flow obtained through this method can be seen in Fig. \ref{fig:minFlow_simple_geom}.

\begin{figure}[h!]
\centering
\includegraphics[width=0.75\textwidth]{figures/cap_geom/GeomComp_SurgeAMF}
\caption{Lowest allowable air mass flow for the different simple geometries tested, obtained through the described algorithm.}
\label{fig:minFlow_simple_geom}
\end{figure}

As seen in this figure, the straight duct is the first one to reach  deep surge conditions. The tapered duct and the 90º elbow pipe have similar but higher margins until surge, while the reservoir delays this damaging condition even more, allowing approximately 10 g/s of further air mass flow reduction.

\subsection{Noise levels}

In addition to the effect on surge margin, the effect on external noise levels was evaluated. As described in previous chapters, two different measurements systems were available in the anechoic chamber. 

To capture the orifice noise, one free-field omnidirectional microphone was situated at a distance of 10 mm from the inlet opening. In this case the microphone had to be repositioned each time the geometry was changed as they had different lengths and, in the case of the elbow, orientation of the orifice.

The second system for measuring externally radiated noise, consisting in the four equispaced free-field microphones placed at 1 m distance form the center of the inlet assembly was kept in the same place for all the tested geometries. 

In both cases the microphones were calibrated using a Brüel \& Kj\ae r pistonphone prior to the measurements and data was captured using a Pulse DAQ. Fig. \ref{fig:Radiated_Orifice_SPL_simple_geom} shows the result of both measurements at the stable operating point most close to surge as characterized by the aforementioned procedure, since it is the point of maximum noise emission.

\begin{figure}[h!]
\centering
\includegraphics[width=0.75\textwidth]{figures/cap_geom/GeomComp_Radiated_Orifice_SPL}
\caption{Comparison of orifice noise and averaged radiated noise measured in the anechoic chamber for each of the simple inlet geometries at the stable operating point most close to surge.}
\label{fig:Radiated_Orifice_SPL_simple_geom}
\end{figure}

It can be seen that again there are significant differences between the inlet geometries tested. These differences were higher when measuring the radiated noise through the average of the four microphones, with the reservoir featuring a 3 dB reduction compared to the reference straight duct. 

On the other hand, the 90º elbow inlet reached a 5 dB increase over the straight inlet level, but this particular result could be influenced by the fact that the direction of the orifice was not perpendicular to the plane of the four microphones. Finally, the tapered duct seemed to radiate at the same level that the straight pipe.

Regarding orifice noise, the reservoir featured again a reduction in noise level of 2 dB. However, in this case the elbow featured the same level as the reference straight pipe, being the tapered duct the one with a higher level at 130 dB, 3 dB more than the reference geometry.

In Fig. \ref{fig:GeomComp_OrificeSpectra_ann} the different spectra recorded by the orifice microphone during each geometry test are plotted to look in detail at these level differences. It can be seen how levels for the tapered duct are higher along the whole frequency range, while the inlets featuring the elbow and the reservoir remain closer to the reference straight pipe.

\begin{figure}[h!]
\centering
\includegraphics[width=0.75\textwidth]{figures/cap_geom/GeomComp_OrificeSpectra_ann}
\caption{Comparison of orifice noise spectra for each of the simple inlet geometries at the stable operating point most close to surge.}
\label{fig:GeomComp_OrificeSpectra_ann}
\end{figure}

The most noticeable difference however, is the resonator effect of the reservoir, that provides an important attenuation at frequencies in the plane wave region of the spectrum. Finally, a small peak can be noticed at 16 kHz, correspondent to the Blade Passing Frequency  associated to the 160 krpm shaft speed that was used during these measurements.

\section{On-engine tests}

In the scope of the on-engine tests described in chapter \ref{cap:engine}, a follow-up experimental campaign was carried out in order to characterize the effect that different geometric variations of the turbocharger inlet line had in performance, surge margin and noise emission.

\subsection{Measurement setup}

The selected geometric variations of the inlet line were chosen taking into account realistic packaging size restrictions of commercial vehicles, so in this case the reservoir case was not tested as it was not possible to fit it to the available compressor inlet. 

Similarly, elbow inlets were not tested as they would require to re-route the inlet line, being its design already frozen at the time of the testing. Focus was then put into small changes like tapered ducts, rotation devices and nozzle-like additions to the inlet pipe.

\begin{figure}[h!]
\centering
\includegraphics[width=0.95\textwidth]{figures/cap_geom/geom_engine}
\caption{Longitudinal section view of the inlet geometries that were considered for the on-engine experimental campaign.}
\label{fig:geom_engine}
\end{figure}

Figure \ref{fig:geom_engine} shows the selected geometry variations based on these selection criteria. On top of the figure the reference geometry is depicted. The turbocharger inducer was milled and substituted by an adaptor to which a straight duct was coupled. At the end of this duct a pipe of the original diameter was then affixed with a tapered transition.

\begin{wrapfigure}{o}[7mm]{0.4\textwidth}
%\begin{figure}[h!]
\centering
\includegraphics[width=0.4\textwidth]{figures/cap_geom/IGV_fitted}
\caption{Picture of the 3D-printed IGV device fitted to the compressor inlet duct, looking towards the impeller.}
\label{fig:IGV_fitted}
%\end{figure}
\end{wrapfigure}

The simplest inlet geometry tested consisted in a tapered duct situated as close as possible to the compressor impeller leading edge. This can be seen at the center left in the figure. Below, it is shown how this tapered duct was complemented with an inlet guide vane (IGV) device.

This was done in order to increase the swirl of the incoming air flow. The device stators featured 0º leading edge angles and $\sim 70$º trailing edge angles, and it was directly 3D-printed from the CAD design. The device can be seen fitted into the inlet pipe in Fig. \ref{fig:IGV_fitted}.

To analyze how the closeness of the tapered duct to the impeller influenced the behaviour of the compressor, the two following geometric variations (center column of Fig. \ref{fig:geom_engine}) were the same except for the separation between the tapered duct and the impeller, which was increased to 52 mm. Again, one measurement was performed with and IGV and other without it.

Lastly, two nozzles were tested (right column of Fig. \ref{fig:geom_engine}): a convergent-divergent nozzle between the tapered adaptor and the impeller (top) and a convergent nozzle (bottom) just after the tapered duct, with its throat as close as possible to the impeller wheel.

In order to evaluate the influence of these geometric variation on the noise output of the compressor, the acoustic measurement setup used in chapter \ref{cap:engine} was maintained, consisting in the straight inlet extension featuring the pressure decomposition array for in-duct intensimetry and the intensity probe aligned with the inlet orifice.

Measurement operating conditions for the noise characterization campaign were chosen \hl{as to be representative of the most common real-life conditions that the engine would experiment in a commercial vehicle, in order to test if these simple inlet modifications could modify the noise output not only on the most extreme conditions but in more conservative settings.}

\subsection{Results}

\hl{Since the objective of this experimental campaign is the comparison between geometries, in this subsection only the results from the orifice intensity probe will be presented for the sake of brevity.}

In Fig. \ref{fig:geom_mot_comparison_map1} the interpolated intensity level map is presented, indicating the points at which each measurement was made. The interpolation of sound intensity was performed between frequencies of 1 and 3 kHz, which are specially relevant regarding the aforementioned ``whoosh'' issue.

\begin{figure}[h!]
\centering
\includegraphics[width=0.7\textwidth]{figures/cap_geom/map_engine_geom_config1}
\caption{Sound intensity level map between 1--3 kHz for the straight inlet used as a reference in the on-engine inlet geometric variation test.}
\label{fig:geom_mot_comparison_map1}
\end{figure}

Note as well that because of the particular regulation mechanism of the facility based on the injection of compressed air into the inlet line, the depicted operating conditions do not represent constant shaft speed lines, so no judgement about the slope of these lines should be made.

However, it can be seen in this figure that the noise levels in the selected band behave as results of presented in previous chapters led to expect. Noise levels increase as pressure ratio increases and air mass flow diminishes, with the dependency on air mass flow being much more noticeable at higher pressure ratios that at lower ones.

\begin{figure}[h!]
\centering
\includegraphics[width=0.45\textwidth]{figures/cap_geom/map_engine_geom_config2}\hspace{0.05\textwidth}
\includegraphics[width=0.45\textwidth]{figures/cap_geom/map_engine_geom_config3}\\[2mm]
\includegraphics[width=0.45\textwidth]{figures/cap_geom/map_engine_geom_config4}\hspace{0.05\textwidth}
\includegraphics[width=0.45\textwidth]{figures/cap_geom/map_engine_geom_config5}\\[2mm]
\includegraphics[width=0.45\textwidth]{figures/cap_geom/map_engine_geom_config6}\hspace{0.05\textwidth}
\includegraphics[width=0.45\textwidth]{figures/cap_geom/map_engine_geom_config7}
\caption{Sound intensity level maps between 1--3 kHz for the different geometries used in the on-engine test.}
\label{fig:geom_mot_comparison_maps}
\end{figure}

Figure \ref{fig:geom_mot_comparison_maps} shows the results for the same frequency band when the reference straight duct was substituted by each of the variants described in Fig. \ref{fig:geom_engine}. 

It is apparent from the figure that different operating conditions were measured in each one of the tests, as the goals in this campaign were to characterize the sound emission across the selected map region and to test how each geometry impacted the performance in regards of surge margin and maximum ingested air mass flow.

In this regard, the figure shows how the geometry changes have an impact on surge margin (left frontier of the map), as some of the modified inlet lines increase the range of low flow operating conditions that can be safely used before going into deep surge, specially at higher shaft speeds; compare for instance the highest row of points of the straight inlet in Fig. \ref{fig:geom_mot_comparison_map1} and that of the convergent-divergent nozzle (bottom left plot of Fig. \ref{fig:geom_mot_comparison_maps}).

Impact of the geometry variation on air mass flow was also tested in regards of maximum flow at 160 krpm, as it is not only important to increase surge margin but also to maintain the possibility of operating the turbocharger at high mass flow levels. 

\begin{figure}[h!]
\centering
\includegraphics[width=0.95\textwidth]{figures/cap_geom/geom_mot_comparison_AMF}
\caption{Comparison of mean sound intensity level for each of the inlet geometries across the normal operating range of the engine.}
\label{fig:geom_mot_comparison_AMF}
\end{figure}

A comparison of the performance of each geometry in both aspects is presented in Fig. \ref{fig:geom_mot_comparison_AMF} where the percentages of improvement from the straight duct reference inlet are plotted. An inverse correlation between maximum flow and surge margin is apparent, only the tapered duct at 52 mm from the impeller improving both factors at the same time and only the tapered duct closer to the impeller and featuring an IGV worsening both.

Largest amount of surge margin improvement was found with the convergent nozzle, but at the cost of an equally large air mass flow reduction at $PR = 2$ and 160 krpm of shaft speed, so in order to implement this solution in real-life applications it should be necessary to mechanically open the nozzle throat at these high flow conditions.

To similarly quantify the influence of the geometries in noise emission at the selected frequency band, in Fig. \ref{fig:geom_mot_comparison_meanSIL} an average sound intensity level (SIL) is presented, as measured by the Brüel \& Kj\ae r intensity probe located pointing directly to the inlet orifice.

\begin{figure}[h!]
\centering
\includegraphics[width=0.95\textwidth]{figures/cap_geom/geom_mot_comparison_meanSIL}
\caption{Comparison of mean sound intensity level for each of the inlet geometries across the normal operating range of the engine.}
\label{fig:geom_mot_comparison_meanSIL}
\end{figure}

Intensity level of the tapered duct is very similar to that of the reference straight pipe; both when located as close as possible to the compressor wheel and when separated 52 mm from it. The addition of the described IGV to those tapered inlets notably reduces noise levels by 3 and 4 dB respectively. 

Regarding the nozzle configurations, it can be seen that the convergent-divergent nozzle reduces noise output in the selected frequency band by 1 dB, whereas the convergent nozzle features a notable reduction of approximately 3 dB compared to the reference.

\section{Elbow parametric study}

\subsection{Setup}

\begin{figure}[h!]
\centering
\includegraphics[width=0.9\textwidth]{figures/cap_geom/codo075D}
\caption{Mechanical drawing of the 0.75D elbow inlet designed to fit into the compressor inlet adaptor.}
\label{fig:codo075D}
\end{figure}

\begin{figure}[h!]
\centering
\includegraphics[width=0.95\textwidth]{figures/cap_geom/Comp_codos}
\caption{CAD model of the compressor assembly featuring the 0.75D elbow inlet attached to the common adaptor, including the pressure and temperature instrumentation and the mounting system.}
\label{fig:Comp_codos}
\end{figure}

\begin{table}[htbp]
  \centering
  \caption{Specifications of the aluminium-filled polyamide used for the laser-synthesis of the elbow inlet prototypes.}
    \begin{tabular}{lcc}
    \addlinespace
    \toprule
    \bf Property & \bf Units & \bf Value \\
    \midrule
    Tensile Modulus & MPa   & 3800 $\pm$ 150 \\
    Tensile Strength & MPa   & 48 $\pm$ 3 \\
    Elongation at Break & \%    & 3.5 $\pm$ 1 \\
    Flexural Modulus & N/mm² & 3600 $\pm$ 150 \\
    Charpy – Impact strength & MPa   & 29 $\pm$ 2 \\
    Charpy – Notched Impact Strength & MPa   & 4.6 $\pm$ 0.3 \\
    Shore D/ A-hardness &       & D 76 $\pm$ 2 \\
    Heat Deflection Temp & °C    & 130 \\
    Vicat Softening Temperature B/50 & °C    & 169 \\
    Density & g/cm³ & 1.36 $\pm$ 0.05 \\
    \bottomrule
    \end{tabular}
  \label{tab:alumide_data}
\end{table}


\begin{figure}[h!]
\centering
\includegraphics[width=0.95\textwidth]{figures/cap_geom/elbow_075D_assembled}
\caption{CAD model of the compressor assembly featuring the 0.75D elbow inlet attached to the common adaptor, including the pressure and temperature instrumentation and the mounting system.}
\label{fig:elbow_075D_assembled}
\end{figure}

\subsection{Results}

\begin{figure}[h!]
\centering
\includegraphics[width=0.65\textwidth]{figures/cap_geom/map_elbows}
\caption{CAD model of the compressor assembly featuring the 0.75D elbow inlet attached to the common adaptor, including the pressure and temperature instrumentation and the mounting system.}
\label{fig:map_elbows}
\end{figure}

\subsubsection{Sound intensity}

\begin{figure}[h!]
\centering
\includegraphics[width=0.65\textwidth]{figures/cap_geom/map_diff_15d}
\includegraphics[width=0.65\textwidth]{figures/cap_geom/map_diff_1d}
\includegraphics[width=0.65\textwidth]{figures/cap_geom/map_diff_075d}
\caption{Difference map of sound intensity levels for each of the selected inlet elbows compared against the reference straight duct. From top to bottom: 1.5D, 1D \& 0.75D.}
\label{fig:map_diff_075d}
\end{figure}

\begin{figure}[h!]
\centering
\includegraphics[width=0.95\textwidth]{figures/cap_geom/distribution_diff}
\caption{CAD model of the compressor assembly featuring the 0.75D elbow inlet attached to the common adaptor, including the pressure and temperature instrumentation and the mounting system.}
\label{fig:distribution_diff}
\end{figure}

\subsubsection{Temperature}

\begin{figure}[h!]
\centering
\includegraphics[width=0.65\textwidth]{figures/cap_geom/map_diff_Tstd_15d}
\includegraphics[width=0.65\textwidth]{figures/cap_geom/map_diff_Tstd_1d}
\includegraphics[width=0.65\textwidth]{figures/cap_geom/map_diff_Tstd_075d}
\caption{Difference map of sound intensity levels for each of the selected inlet elbows compared against the reference straight duct. From top to bottom: 1.5D, 1D \& 0.75D.}
\label{fig:map_diff_075d}
\end{figure}

\begin{figure}[h!]
\centering
\includegraphics[width=0.95\textwidth,trim={3cm 11cm 3cm 11cm},clip]{figures/cap_geom/distribution_diff_Tstd}
\caption{CAD model of the compressor assembly featuring the 0.75D elbow inlet attached to the common adaptor, including the pressure and temperature instrumentation and the mounting system.}
\label{fig:distribution_diff}
\end{figure}

\section{Conclusions}