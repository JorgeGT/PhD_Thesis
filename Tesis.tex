\documentclass[final,twoside,11pt]{book}

% Packages
\usepackage[T1]{fontenc}
\usepackage[utf8]{inputenc}
\usepackage{amsthm}
\usepackage{amsmath}
\usepackage{color}
\usepackage[hyperindex]{hyperref}
\usepackage{graphicx}
\usepackage{subfigure}
\usepackage{booktabs}
\usepackage{units}
\usepackage{multirow}
\usepackage{pdfpages}
\usepackage{placeins}
\usepackage{titlesec}
\usepackage{microtype}
\usepackage{doi}
\usepackage[defaultlines=3,all]{nowidow}
\usepackage[bottom]{footmisc}
\usepackage{marvosym}
\usepackage{rotating}
\usepackage{changepage}
\usepackage{wrapfig}
\usepackage{soul}
\usepackage{bm}
\usepackage{bookmark}

% Serif font
\usepackage[bitstream-charter]{mathdesign}

% Monospaced font
\usepackage[scaled]{beramono}

% Sans serif font
\usepackage[type1,scaled=1.02,condensed]{cabin}

% Caption format
\usepackage{caption}
\DeclareCaptionFont{cabin}{\cabin\raggedleft}
\captionsetup{margin=12pt,font={small,sl},labelfont={cabin,bf},parskip=5mm}

% Colors
% \definecolor{tema}{RGB}{80,86,91}
% \definecolor{tema}{rgb}{0.45,0.45,0.45}
\definecolor{tema}{RGB}{0,50,150}
\definecolor{BLACK}{RGB}{0,0,0}
\hypersetup{
    colorlinks,
    citecolor=tema,
    filecolor=BlueViolet,
    linkcolor=black,
    urlcolor=tema,
%   pdfpagelayout=TwoPageRight,
    pdfauthor={Jorge Garcia Tíscar}
}

% Numbering
\numberwithin{equation}{section}
\numberwithin{figure}{chapter}
\numberwithin{table}{chapter}
\setcounter{secnumdepth}{3}

% Footnotes without numbers
\newcommand\blfootnote[1]{%
  \begingroup
  \renewcommand\thefootnote{}\footnote{#1}%
  \addtocounter{footnote}{-1}%
  \endgroup
}

% Page geometry
\usepackage[papersize={17cm,24cm}]{geometry}
%\usepackage[cam,center,a4]{crop} % To test on A4 paper
% Horizontal
\setlength{\textwidth}{12cm}
\setlength{\parindent}{3mm}
% Vertical
\setlength{\headsep}{5mm}
\setlength{\textheight}{19cm}
\setlength{\topmargin}{-10mm}
\raggedbottom

% Headers
\usepackage{fancyhdr}
\renewcommand{\footrulewidth}{0pt}
\renewcommand{\headrulewidth}{0pt}
\fancypagestyle{plain}{
  \fancyhf{}
  \fancyfoot[C]{\small\cabincondensed\bfseries\color{black}\thepage}
}
\fancypagestyle{empty}{
  \fancyhf{}
}
\pagestyle{plain}

% Clear double page style
% \makeatletter
%   \def\cleardoublepage{\clearpage\if@twoside \ifodd\c@page\else
%   \hbox{}
%   \thispagestyle{empty}
%   \newpage
%   \if@twocolumn\hbox{}\newpage\fi\fi\fi}
% \makeatother

% Solve a problem with transparent PDF figures
\pdfpageattr{/Group << /S /Transparency /I true /CS /DeviceRGB>>}

% Itemize symbols
\renewcommand{\labelitemi}{\textcolor{tema}{\scriptsize\raisebox{0.5mm}{\Circsteel}}}
\renewcommand{\labelitemii}{\textcolor{tema}{\scriptsize\raisebox{0.4mm}{\Squaresteel}}}

% Select only a chapter for faster compilation
% \includeonly{cap_result}

\begin{document}

%-------------------------------------------------------------------%
% FRONT MATTER
\frontmatter

% Cover page
%\includepdf[pagecommand={}]{../test_portada3}
%\cleardoublepage

% Inside cover
\enlargethispage{5cm}
\thispagestyle{empty}
\begin{center}
\Large\cabin\textbf{UNIVERSITAT POLITÈCNICA DE VALÈNCIA}\\[1mm]
\large DEPARTAMENTO DE MÁQUINAS Y MOTORES TÉRMICOS\\
\vspace{1cm}

\rule[0.3\baselineskip]{3.8cm}{0.4pt}
\noindent\cabin\large{\hspace{2mm}DOCTORAL THESIS\hspace{2mm}}
\rule[0.3\baselineskip]{3.8cm}{0.4pt}
\vspace{2mm}\\
\huge\textbf{EXPERIMENTS ON TURBOCHARGER COMPRESSOR ACOUSTICS}\\
\vspace{-2.5mm}
\rule{\textwidth}{0.4pt}
\vspace{1cm}

\includegraphics[scale=1]{logos/logo_upv_negro}

\vspace{1.3cm}

{\normalsize\bfseries\cabincondensed PRESENTED BY}\\
\rm\Large D. Jorge García Tíscar\\[4mm]
{\normalsize\bfseries\cabincondensed SUPERVISED BY}\\
\rm\Large Dr. D. Alberto Broatch Jacobi\\[8mm]

{\normalsize\bfseries\cabincondensed IN FULFILLMENT OF THE REQUIREMENTS FOR THE DEGREE OF}\\[2mm]
\it\Large Doctor of Philosophy

\vspace{1.2cm}

\footnotesize\rm\cabin \hl{JANUARY 2017}
\end{center}
\cleardoublepage

% Thesis details
\vspace*{0.25\textheight}
{

\centering
{\normalsize\bfseries\cabincondensed DOCTORAL THESIS}\\[4mm]
\textit{\large ``Experiments on turbochager compressor acoustics''}\\[2cm]

\begin{tabular}{rl}
{\cabincondensed Presented by:} & D. Jorge García Tíscar\\
{\cabincondensed Supervised by:} & Dr. D. Alberto Broatch Jacobi\\[1cm]
\end{tabular}

{\normalsize\bfseries\cabincondensed THESIS EXAMINERS}\\[4mm]
Dr. D. John Doe\\
Dr. D. John Doe\\
Dr. D. John Doe\\[1cm]


{\normalsize\bfseries\cabincondensed DEFENSE COMITEE}\\[4mm]
\begin{tabular}{rl}
{\cabincondensed Chairman:} & Dr. D. John Doe\\
{\cabincondensed Secretary:} & Dr. D. John Doe\\
{\cabincondensed Member:} & Dr. D. John Doe\\[1.5cm]
\end{tabular}

{\cabincondensed Valencia, January 2017}

}
\cleardoublepage

% Dedication
\null\vspace{5cm}
\begin{flushright}
{\itshape Dedicatoria}
\end{flushright}
\cleardoublepage

% Abstract ENG
\titleformat{\chapter}[hang]{}{}{0mm}{\large\bfseries\sffamily\Huge}
\chapter*{Abstract}

\hl{As turbocharging requirements raise to face an increasingly stricter environmental regulation of internal combustion engines, concerns regarding their acoustic emission become more pressing. Since downsized engines require higher boost pressures and lower mass flow rates, the turbocharger compressor is forced to work at more unstable regimes, where flow patterns become more complex and noise levels rise.}

This thesis aims to investigate these issues, proposing methodologies to characterize the acoustic emission of turbocharger compressors and implementing them in different experiments with a special focus on the the link between acoustic emission and flow behaviour at unstable conditions.

A literature review is carried out in order to assess the state of the art regarding specially experimental techniques related to this issue but also including the latest developments in terms of understanding the flow characteristics through numerical simulations. Different methodologies are consequently proposed and implemented into a custom flow rig inside an anechoic chamber as to experimentally measure and post-process the acoustic output of the compressor.

From this measurement campaign a characterization of the noise spectral content across the compressor map is obtained and described, identifying different acoustical phenomena such as blade passing tonal noise, low frequency content associated to deep surge, and a sub-synchronous broadband noise known as ``whoosh'' in the literature, of special concern for automotive manufacturers. This particular phenomenon is detected even at more stable conditions at higher flow rate, and rising in level as flow rate is diminished to the point of being masked by lower frequency content.

After a validation of the selected procedure in realistic engine conditions, experimental data is compared against a numerical model of the compressor developed in a parallel work to assess its validity and propose different post-processing techniques to extract additional insights about the behaviour of the flow at different conditions, hinting at the main generation mechanism for ``whoosh'' being located at the compressor diffuser.

Since numerical simulations predict a reduced amount of reversed unstable flow at conditions where ``whoosh'' noise is still measured, an experimental campaign is performed were detailed measures of local temperature near the compressor wheel are used to determine the evolution of reversed flow, with supplemental readings through pressure probes in the inducer and diffuser being used to link this evolution with the fluctuation of spectral content. Temperature measurements are also correlated with measurements of the velocity field through particle imaging. Results show that even inducer content being highly affected by the onset of reversed flow, ``whoosh'' noise is detected at all tested conditions

Different experimental campaigns are then described were modifications of the inlet geometry immediately upstream the compressor are performed in order to assess how the air flow presentation can influence acoustic performance. Geometries featuring reservoirs, nozzles and guide vanes are shown to potentially reduce noise levels. A 90º elbow parametric study is performed, demonstrating how the inner radius of the elbow influences circumferential temperature differences and noise levels, leading to the hypothesis of geometry influence on ``whoosh'' noise being related to air presentation changes that promote lower or higher growth of reversed backflow, which in turns carries upstream the spectral content being generated in the compressor diffuser. 

Lastly, additional experimental and numerical studies are proposed to further explore these issues, in order to provide a better understanding of how inlet designs may delay and mitigate the appearance and transmission of these adverse acoustical phenomena.

\cleardoublepage

% Abstract ESP
\chapter*{Resumen}

\cleardoublepage

% Abstract CAT
\chapter*{Resum}

\cleardoublepage

% List of publications
\chapter*{List of publications}

The following papers have been published following the work performed in the framework of this doctoral thesis and its associated projects:

\begin{itemize}

  \item \nocite{broatch2014methodology}
  A.~Broatch, J.~Galindo, R.~Navarro, J.~Garc{\'\i}a-T{\'\i}scar, \emph{Methodology for experimental validation of a CFD model for predicting noise generation in centrifugal compressors}, International Journal of Heat and Fluid Flow 50  (2014) 134 -- 144. \newblock \href {http://dx.doi.org/10.1016/j.ijheatfluidflow.2014.06.006}{\path{doi:10.1016/j.ijheatfluidflow.2014.06.006}}.

  \item \nocite{broatch2015simulations}
  A.~Broatch, J.~Galindo, R.~Navarro, J.~Garc{\'\i}a-T{\'\i}scar, A.~Daglish, R.~Sharma, \emph{Simulations and measurements of automotive turbocharger compressor whoosh noise}, Engineering Applications of Computational Fluid Mechanics 9~(1) (2015) 12--20. \newblock \href {http://dx.doi.org/10.1080/19942060.2015.1004788}{\path{doi:10.1080/19942060.2015.1004788}}.

  \item \nocite{torregrosa2014aco}
  A.~Torregrosa, A.~Broatch, R.~Navarro, J.~Garc{\'\i}a-T{\'\i}scar, \emph{Acoustic characterization of automotive turbocompressors}, International Journal of Engine Research 16~(1) (2015) 31--37. \newblock \href {http://dx.doi.org/10.1177/1468087414562866}{\path{doi:10.1177/1468087414562866}}.

  \item \nocite{broatch2016numerical}
  A.~Broatch, J.~Galindo, R.~Navarro, J.~Garc{\'\i}a-T{\'\i}scar, \emph{Numerical and experimental analysis of automotive turbocharger compressor aeroacoustics at different operating conditions}, International Journal of Heat and Fluid Flow (In Press). \newblock \href {http://dx.doi.org/10.1016/j.ijheatfluidflow.2016.04.003}{\path{doi:10.1016/j.ijheatfluidflow.2016.04.003}}.

  \item \nocite{torregrosa2016experimental}
  A.~J. Torregrosa, A.~Broatch, X.~Margot, J.~Garc{\'\i}a-T{\'\i}scar, \emph{Experimental methodology for turbocompressor in-duct noise evaluation based on beamforming wave decomposition}, Journal of Sound and Vibration 376 (2016) 60--71. \newblock \href {http://dx.doi.org/10.1016/j.jsv.2016.04.035}{\path{doi:10.1016/j.jsv.2016.04.035}}.

\end{itemize}

\hl{[Additional papers are under preparation]}

\newpage

{\bfseries\sffamily\Large\noindent Division of work between authors}\\[-1mm]

The work leading up to this thesis was done in collaboration with other researchers. The respondent is the co-author of all papers on which
this thesis is based, with author signatures being in order of seniority in the Institute. The respondent implemented the proposed methodology into the different experimental setups, performed the measurements, processed the experimental results and extracted the presented conclusions. Discussions were also performed in collaboration with supervisor Prof. Broatch and the rest of the co-authors. CFD model setup and CFD postprocessing, with the exception of Dynamic Mode Decomposition, were not performed by the respondent, and are thus gratefully acknowledged.

\cleardoublepage

% Acknowledgements
\chapter*{Acknowledgements}
\hl{[Acknowledgements]}
\cleardoublepage

% Quotes
\null\vspace{2cm}

\begin{flushright}
\begin{minipage}[t]{0.75\textwidth}

% \linespread{0.97}\selectfont
% {\itshape\small``The story is set in the year 1922 and has its main character, Mr Engelbert, moving to Prague from the countryside to live out his retirement years and struggling to cope with the aggressive modernity of the city. The horror is not the power of money or the arrogance of the people, but the noise; not the age-old noise of a thunderstorm or a hammer, but the new noise of engines, especially of automobiles and motor-cycles, the explosive internal combustion monsters.''}\\[-3mm]

% {\cabincondensed\small\hfill \textbf{Milan Kundera}, \emph{The Curtain.}}\\[10mm]

\linespread{0.97}\selectfont
{\itshape\small``Hundreds of noises wove themselves into a wiry texture of sound with barbs protruding here and there, smart edges running along it and subsiding again, with clear notes splintering off and dissipating. By this noise alone, whose special quality cannot be captured in words, a man returning after years of absence would have been able to tell with his eyes shut that he was back in the Imperial Capital and Royal City of Vienna.''}\\[-3mm]

{\cabincondensed\small\hfill \textbf{Robert Musil}, \emph{The Man Without Qualities.}}
\end{minipage}
\end{flushright}

\cleardoublepage

% Table of contents
\tableofcontents
\cleardoublepage

% List of figures
\listoffigures
\cleardoublepage

% List of tables
\listoftables
\cleardoublepage

% List of symbols
%!TEX root = Tesis.tex

\chapter*{List of symbols}
\small
\vspace{-1cm}
\begin{tabular}{lll}
\multicolumn{2}{l} {\sffamily\bfseries\Large {Latin and Greek}} \\[3mm]
$A$ & Area & \si{m^2} \\
$a$ & Speed of sound & \si{m.s^{-1}} \\
$c_p$ & Specific heat capacity at constant pressure & \si{J.kg^{-1}.K^{-1}} \\
$D$ & Duct diameter & m \\
$d_s$ & Sensor separation & m \\
$f$ & Frequency & Hz \\
\textit{Re} & Helmholtz number & $-$\\
$I$ & Sound Intensity & \si{W.m^{-2}} \\
%$k$ & Turbulent kinetic energy & \si{m^2.s^{-2}} \\
$L$ & Level (SPL, SIL or PVL) & dB \\
$\dot{m}$ & Mass flow rate & \si{kg.s^{-1}} \\
$M$ & Mach number & $-$ \\
$N$ & Compressor rotational speed & rpm \\
$p$ & Pressure & Pa \\
$R$ & Specific gas constant & \si{J.kg^{-1}.K^{-1}} \\
\textit{Re} & Reynolds number & $-$ \\
\textit{St} & Strouhal number & $-$ \\
$t$ & Time & s \\
$T$ & Temperature & K \\
$U$ & Mean velocity & \si{im.s^{-1}} \\
$W_u$ & Compressor specific work & \si{m^2.s^{-2}} \\
$\Delta f,t$& Frequency or time step & Hz or s\\
$\gamma$& Ratio of specific heats & $-$ \\
$\varepsilon_R$ & Relative error & \% \\
$\eta_s$ & Isentropic efficiency & \% \\
$\theta$ & Direction of arrival & rad\\
$\lambda$ & DMD eigenvalue & $-$ \\
$\varPi$ & Pressure ratio & $-$ \\
$\rho$ & Density & \si{kg.m^{-3}} \\
$\tau$ & Compressor torque & \si{kg.m^{2}.s^{-2}} \\
$\Phi$ & DMD mode & $-$\\
$\varphi$ & Flow coefficient & $-$ \\
$\psi$ & Generic variable & $-$ \\ 
$\Omega$ & Rotational speed & \si{rad.s^{-1}} \\
%$\omega$ & Specific turbulence dissipation rate& \si{s^{-1}}
\end{tabular}

\vspace{5mm}

\begin{tabular}{ll}
\multicolumn{2}{l} {\sffamily\bfseries\Large {Sub- and superscripts}} \\[3mm]
${+}$ & Forward-travelling variable  \\
${-}$ & Backward-travelling variable   \\
${*}$ & Corrected variable  \\
${1,2,3}$ & Related to $1\textsuperscript{st}, 2\textsuperscript{nd}$ or $3\textsuperscript{rd}$ downstream sensor\\
a & (Freq.) related to an asymmetric acoustic mode\\
back & Backward travelling wave\\
CFD & Variable simulated through CFD\\
c & (Freq.) related to a circular acoustic mode\\
d & Related to sensor spacing \\
exp & Variable measured experimentally \\
in & Related to the inlet duct \\
k & Discrete time step \\
forw & Forward-travelling wave\\
n & (Freq.) related to the spatial Nyquist criterion\\
out & Related to the outlet duct \\
s & Isentropic\\
ref & Reference value \\
rms & Root mean squared \\
T & Total (or stagnation) variable \\
TT & Total to total ratio \\
tip & Related to the blade tip \\
W, A, S & Wall, axis or cross-section monitor, respectively\\
\end{tabular}

\vspace{5mm}

\begin{tabular}{ll}
\multicolumn{2}{l} {\sffamily\bfseries\Large {Acronyms}} \\[3mm]
AMF & Air Mass FLow \\
BPF & Blade Passing Frequency\\
CAD & Computer-Aided Design \\
CCD & Charge-coupled device \\
CFD & Computational Fluid Dynamics \\
DAQ & Data Acquisition \\
DES & Detached Eddy Simulation\\
DMD & Dynamic Mode Decomposition \\
DOA & Direction Of Arrival \\
ECU & Engine Control Unit\\
FFT & Fast Fourier Transform \\
HWA & How Wire Anemometry \\
IGV & Inlet Guide Vanes \\
LDA & Laser Doppler Anemometry \\
LCMV & Linearly-Constrained Minimum Variance \\
LE & Leading edge \\ 
LES & Large Eddy Simulation \\
MEMS & Micro Electro-Mechanical System \\
MoC & Method of Characteristics \\
Nd:YAG & Neodymium-doped Yttrium Aluminium Garnet\\
NRBC & Non-Reflecting Boundary Condition \\
NVH & Noise, Vibration \& Harshness \\
PIV & Particle Image Velocimetry \\
POD & Proper Orthogonal Decomposition \\
PS & Pressure Side \\
PSD & Power Spectral Density \\
PR & Pressure Ratio \\
PVL & Particle Velocity Level \\
RANS & Reynolds-Averaged Navier-Stokes \\
RO & Rotating Order \\
ROI & Region Of Interest \\
RPM & Revolutions Per Minute \\
TCN & Tip Clearance Noise \\
TMM & Two Microphone Method \\
SIL & Sound Intensity Level \\
SLS & Selective Laser Sintering \\
SPL & Sound Pressure Level \\
SST & Shear Stress Transport \\
SS & Suction Side \\
SVD & Singular Value Decomposition \\
VGT & Variable Geometry Turbine \\
\end{tabular}
\normalsize

\cleardoublepage

%-------------------------------------------------------------------%
% MAIN MATTER (style)
\mainmatter

% Chapter
\titleformat{\chapter}[hang]{}{}{0mm}{\large\bfseries\cabincondensed\textcolor{tema}{\hspace{0mm}CHAPTER \thechapter}\\[-1mm]\Huge}
% Section
\titleformat{\section}[hang]{}{}{0mm}{\bfseries\sffamily\Large\textcolor{tema}{\arabic{chapter}.\arabic{section}. }}[\vspace{-1mm}]
% Subsection
\titleformat{\subsection}[hang]{}{}{0mm}{\bfseries\large\sffamily\textcolor{tema}{\arabic{chapter}.\arabic{section}.\arabic{subsection}. }}[\vspace{1mm}]
% Subsubsection
\titleformat{\subsubsection}[hang]{}{}{0mm}{\bfseries\sffamily\textcolor{tema}{\arabic{chapter}.\arabic{section}.\arabic{subsection}.\arabic{subsubsection}. }}[\vspace{1mm}]

% Main pagestyle
\fancyhf{}
\fancyhead[RO]{\small\cabincondensed\bfseries \rightmark\quad|\quad\color{black}\thepage}
\fancyhead[LE]{\small\cabincondensed\bfseries \thepage\quad\textcolor{tema}{|}\quad\leftmark}
\pagestyle{fancy}

% Chapter mark
\renewcommand{\chaptermark}[1]{\markboth{\chaptername\ \color{tema}{\thechapter. \color{black}\itshape\ #1}}{}}
% Section mark
\renewcommand{\sectionmark}[1]{\markright{\color{tema}{\thesection.\color{black}\itshape\ #1}}}

\hypersetup{
  linkcolor = tema
}

%-------------------------------------------------------------------%
% BODY MATTER (include content)

%!TEX root = Tesis.tex
\chapter{Introduction} \label{cap:intro}

% Quote
\null\vspace{-10mm}
\begin{flushright}
\begin{minipage}[t]{0.66\textwidth}
{\itshape\small``The story is set in the year 1922 and has its main character, Mr Engelbert, moving to Prague from the countryside to live out his retirement years and struggling to cope with the aggressive modernity of the city. The horror is not the power of money or the arrogance of the people, but the noise; not the age-old noise of a thunderstorm or a hammer, but the new noise of engines, especially of automobiles and motor-cycles, the explosive internal combustion monsters.''}\\[-3mm]

{\cabincondensed\small\hfill \textbf{Milan Kundera}, \emph{The Curtain.}}\\[5mm]
\end{minipage}
\end{flushright}

% Chapter text
\noindent As Milan Kundera shows, the noise produced by internal combustion vehicles has radically changed the way we understand cities; even many decades before we understood the dangers that chemical emissions pose to the environment and our health, noise emissions were without a doubt a commotion for city inhabitants worldwide: an horrific, roaring hallmark of accelerated progress.

\begin{figure}[t!]
\centering
\includegraphics[width=\textwidth]{figures/cap_intro/cmt_noise6}
\caption{Global noise levels of the city of Valencia around Universitat Politècnica (upper right), showing the importance of road traffic as a source of urban noise pollution. Scale shows normalized levels from quieter (1) to noisier (6). Data courtesy of Valencia City Council.}
\label{fig:noise_valencia}
\end{figure}

Nowadays, no one is surprised anymore by the sound of an automobile or a ``motor-cycle''. Even the mere absence of this familiar, background urban sound is unsettling to many. Road traffic is, however, the main source of noise pollution in cities. As an example, Fig \ref{fig:noise_valencia} depicts global noise levels in the city of Valencia.

It can be clearly seen how automotive vehicles define the noise environment in our cities; high-traffic roads appear as higher level sources of noise that propagates towards less travelled areas. The (traffic-restricted) campus of Universitat Politècnica de València, where this thesis has been developed, appears as a blue, quieter island at the upper right part of the figure.

Noise emission by automotive vehicles is, however, a sum of very complex phenomena. One could consider, intuitively, that the dominant factor of this sum must be the the combustion noise itself: several explosions going off every second of operation, their vibration propagating through the metallic engine block and radiating through the air.

However, many of other sources of noise exist in a vehicle: aerodynamic noise produced as air flows around the structures and through the intake and exhaust pipes, friction noise of the tires against the asphalt of the road, mechanical noise of the engine parts working together, auxiliary systems such as fuel injection, pumps and, increasingly important, the noise produced by the turbocharger group.

\section{Turbochargers}

The idea of using a compressor to increase the density of air available in the combustion chamber of an engine is now more than a century old. While Gottlieb Daimler's 1885 patent included a supercharging mechanical pump, and Louis Renault's 1902 patent featured a centrifugal supercharger, the first design for a turbocharger (this is, a supercharging compressor driven by the exhaust gasses) was presented in Alfred Büchi's 1905 patent Nº204630 from the Imperial Patent Office of the German Reich.

While some racing and luxury sport cars had supercharged engines (the popular Mercedes \emph{Kompressor} series started in 1921), most mass produced cars used simpler, naturally aspirating engines.

During the first half of the XX century, the primary area of application of turbocharging where marine and, specially, aviation engines. The addition of turbocharging to aircraft engines allowed them to counter the diminishing air density at high altitude, increasing the available power and operating ceiling. For instance, the first product made by Cliff Garrett's corporation was a supercharged air cooler for the B-17 Flying Fortress.

It was not until the sixties that turbocharging made its appearance into the popular automotive market, with the Chevrolet Corvair Monza and the aptly named Oldsmobile Jetfire. However, technology was still not very reliable, and they were quickly retired from the dealerships.

The scene changed radically during the seventies. Several factors (the oil crisis, enhanced reliability, stringent emission requirements, turbocharged cars winning Formula 1 and Le Mans races) made possible the first successful passenger diesel cars, that grew during the 80s and 90s to dominate the diesel automotive market.

To illustrate this evolution, Fig. \ref{fig:google_ngram} shows how words like ``turbocharger'' and ``traffic noise'' started to appear in English literature. We can see a sudden bump of ``turbocharger'' mentions following the start of WWII (where turbocharging aircraft quickly became norm), but the peak of popularity appears in the 80s. Concern for ``traffic noise'' on the other hand suddenly boosts from 1965 to 1975.

\begin{figure}[b!]
\centering
\includegraphics[width=\textwidth]{figures/cap_intro/google_ngram}
\caption{Normalized relevance of selected English terms in literature, as indexed by Google Books. Data licensed under CC-BY 3.0 license terms.}
\label{fig:google_ngram}
\end{figure}

This figure also shows how from the last years of the XX century, social concern for the nocive effects of pollutant emissions has grown. These include NO${}_\text x$, CO, CO${}_\text 2$, particulate matter (PM) and so on. In a bid to lower emissions, a trend of \emph{downsizing} engines has become popular, aiming to increase efficiency and lower contaminants by reducing the displacement and cylinder count, in both petrol and diesel markets. Upcoming regulations are even more stringent \cite{deBeeck2013upcoming}, so this tendency is only expected to grow.

As power and torque must be maintained in these reduced displacement engines, turbocharging requirements increase, becoming more and more popular in the petrol market as well. Further information on the relationship between stringent emissions policies, downsizing and turbocharging can be found in the work of Schumann et al.\cite{schumann2012potential}.

\section{Noise emissions}
Another important aspect of increasing concern for both regulators and consumers is the so-called NVH (Noise, Vibration and Harshness) problems that opened this chapter.

While downsizing increases efficiency and allows engines to better meet emission requirements, this strategy has some downsides. For instance, it has been shown \cite{stoffels2003nvh} that at certain operating conditions, a downsized engine will radiate higher noise levels than an equivalent engine of larger displacement.

Increased turbocharging requirements also imply that the compressor must work near its limiting conditions, such as the \emph{surge} limit \cite{teng2009investigation}, where compressor airflow is so low that it reverses, allowing hot air to escape the compressor upstream instead of downstream. This potentially destructive condition is initiated with flow instabilities that are also linked to increased noise output, as will be further explored along this work.

Another factor that must be taken into account during the analysis of acoustical emissions is that raw noise \emph{quantity} (overall level) is not the only concern, for noise \emph{quality} (for example, its particular frequency distribution) is also an important problem, regarding both the possibilities of sound mitigation and psychoacoustic response by the customer, as shown in \cite{brizon2012combining} and the work of Nor et al.\cite{nor2008index}.

\section{State of the art}

These concerns have led to a thriving research effort regarding turbocharging efficiency, usefulness to reduce pollution and contribution to noise emissions.

Several approaches to this problem are being tested. Some researchers conduct experimental analysis of the internal fluid flow of the compressor. Others focus their interest in the externally radiated acoustic field. Computational fluid dynamic (CFD) simulations are carried out in order to visualize the different flow phenomena inside and near the compressor.

Aside from experimental or numerical characterization of the acoustic problem, its mitigation is also being actively investigated. Options range from the modification of inlet or casing geometries, to the addition of different silencers and resonators on the compressor pipes.

Through this research, different acoustical phenomena have been identified. While some like the tonal noise produced by the blade pass (known as Blade Pass Frequency or BPF) have a clear, straightforward explanation, others are not yet fully understood.

For example, a certain wideband noise known as ``whoosh'' in the literature has been lately of interest, with explanations for its generation ranging from turbulent vorticity to fluid interaction with tip clearance (the gap between the wheel blades and the casing wall). Further research is currently ongoing on this particular phenomenon.

%\section{Motivation}

\section{Objectives}

The objective of this thesis is threefold, covering not only the experimental acoustic analysis itself but also its methodological and application aspects. To sum up, the different but related objectives can be enumerated as:

\begin{itemize}
	\item Enlarging the corpus of published experimental results of turbocharger compressor acoustic measurements across its different operating conditions: as the literature is still reduced, expanding it will improve not only understanding of known phenomena but will also identify other areas that may be of interest for further research.

	\item Presenting a comprehensive experimental methodology, from instrumentation to measuring procedures and signal processing, in order to consistently and reliably characterize acoustic emissions, ensuring at the same time the robustness of these techniques when faced with real engine conditions.

	\item Exploring how this experimental methodology can be applied not only to the characterization of the internal sound field, but also to the initialization and validation of numerical simulations and the study of the influence of different inlet geometries on noise generated by the compressor.
\end{itemize}

\section{Thesis outline}

After this introduction that constitutes chapter \ref{cap:intro}, a comprehensive literature review on turbocharger compressor noise research will be carried out in chapter \ref{cap:liter}. This review will introduce relevant developments regarding experimental compressor testing, but also some literature concerning numerical simulations that are of vital importance to improve the understanding of the flow field causing the acoustical emission.

Chapter \ref{cap:metod} will present the different methodologies used in the experimental campaign performed during this thesis, along with the facilities and equipment used to this effect. These methodologies include procedures for the characterization of the internal sound field, but external radiation is also considered to test for relevant correlations, along with the measurement of relevant thermodynamic variables near the compressor inlet. Theoretical background, limitations and best practices regarding the use of the presented methodologies will be also discussed.

The results of the experimental campaign carried out using the described methodology will constitute chapter \ref{cap:results}. As with the preceding chapter, internal and external noise field will be presented separately. However, the last section of the chapter will introduce correlations between the two results. Detailed measurement of thermodynamic variables next to the compressor wheel will enable further comprehension of the particularities of the fluid field and its relation with the acoustical sources.

On chapter \ref{cap:engine}, procedures described on chapter \ref{cap:metod} will then be implemented on an engine test cell in order to assess their performance in real engine conditions. This performance will be compared to established methods such as the two-microphone method and the pressure-pressure intensity probe. While external measurements are difficult in a non-anechoic environment such as the engine test cell, an example of the acoustical particle velocity method will be tested to address this limitation.

Chapter \ref{cap:cfd} will explore how the obtained experimental results can be used to initialize and validate numerical simulations. In turn, these simulations can provide insight into particularities of the flow field and suggest further experimental measurements that can be relevant for the understanding of the phenomena involved in the generation of noise.

Particularly, influence of the inlet geometry immediately upstream of the compressor wheel has been proved of significant relevance in both compressor performance and noise emission. Chapter \ref{cap:geom} will present an experimental parametric study of different geometries and their influence on the the compressor behaviour.

Finally, chapter \ref{cap:conclus} will offer the conclusions that have been reached through the experimental campaign and analysis of results, along with suggestions of future works that could be carried out to advance knowledge of this topic.


%!TEX root = Tesis.tex
\chapter{Literature review} \label{cap:liter}

% Chapter text
\noindent In order to provide a better context for the development of this work, a literature review has been carried out. Special emphasis has been put into existing experimental research of automotive turbocharger acoustics, although techniques and results from related fields are also presented when applicable.

For instance, numerical simulations on the topic are becoming more numerous. They offer further insight into the underlying flow structures even if clear, causal correlations have not yet been demonstrated. As such, these studies merit their own specific section. 

Another point related to these flow structures that deserves interest is the analysis of how the particularities of the inlet geometry immediately upstream of the compressor wheel influence the performance and the noise emission of the system. Not only this influence offers clues of which mechanisms are responsible for certain acoustic phenomena, but they also present an interesting possibility of mitigation.

\section{Experimental research}

While the acoustical emission of axial compressors has been a  subject of extensive research since more than fifty years ago \cite{nvemec1967noise,griffiths1964spectrum}, studies performed on centrifugal turbocharger are more comparatively more scarce. 

Of the existing literature dealing with centrifugal turbomachinery, many efforts were directed at large fans or centrifugal pumps, with small turbochargers have begun only recently to be more relevant.

Among the first it is worth to mention the extensive research carried out at Pennsylvania State University, where at least since the nineties \cite{mongeau1993sound} an instrumented centrifugal pump was installed in an anechoic flow bench to characterize its acoustical performance.

This setup allowed several researchers to conduct experiments analysing not only in-duct noise, as in the works of Mongeau et al \cite{mongeau1993sound,mongeau1995method} and radiated noise, as shown by Choi et al \cite{choi1994aerodynamic,choi2003experiments}, but also pressure distribution on the blade surface \cite{mongeau1993sound,choi1994aerodynamic}.

A related experimental campaign, starting at least from the turn of the millennium, was carried out at the University of Siegen. Here, Wolfram and Carolus \cite{wolfram2010experimental} used again a centrifugal pump inside an anechoic chamber. 

In this case, measurements were done relying in anemometers directed at the blades that could be rotated at will in order to analyse azimuthal rotating modes \cite{wolfram2009detection}. Radiated noise could be also measured using a single microphone \cite{carolus2000experimental}.

Both Penn State and Siegen campaigns show significant aspects of interest such as the identification of Blade Pass Frequency (BPF) peaks on the acoustic spectrum and hints on the link between stall and noise \cite{mongeau1993sound}. Relevant dimensionless numbers such as the Strouhal and Helmoltz numbers are also suggested \cite{mongeau1995method}. 

However, it is important to notice that there are important differences between these large centrifugal pumps and small turbocharger compressor for automotive purposes.

These differences are found not only in the geometrical design aspect but also in the order of rotational speed. While the considered pumps do not exceed 3600 rpm, small automotive turbochargers can exceed 200 krpm. 

Also, they are affected by the pulsating flow of the alternating engine, which will be present in the acoustic signature as well. This later aspect will be shown in chapter \ref{cap:engine}.

In order to address these differences and focus the literature review in the specific issues presented by turbocharger compressors, selected works from the literature will be reviewed in the following subsections, placing special interest in both the measurement techniques and the acoustical results obtained.

\subsection{Measurement methods}

Although not numerous, acoustical research works on small, centrifugal turbocharger compressors can be found in the literature. Many of them are published by automotive or compressor manufacturers, who are among the most concerned about compressor noise emissions.

Starting from the work of Evans and Ward \cite{evans2005minimizing,evans2006reduction} up to more recent investigations such as those of Åbom, Kabral and Rammal \cite{kabral2013acoustical,kabral2014investigation} at the Swedish Royal Institute of Technology (KTH), several different methodologies for characterising the acoustical emission of automotive turbochargers can be found. 

A first division can be made between those studies aiming to measure \emph{internal} flow noise, placing sensors inside the compressor housing or inlet and outlet ducts, and those works based on external measurements of the \emph{radiated} noise. 

Evans and Ward \cite{evans2005minimizing,evans2006reduction} opted for a simple setup of four microphones, situated on the same plane and directed to a turbocharger-equipped engine. They were distributed at 90º with a separation of one meter from the engine. From the pressure readings of these four microphones, Sound Pressure Level (SPL) was computed and then averaged among them, in order to get a single signal from which spectrograms and spectra could be derived.

\subsection{Relevant results}
\section{Numerical simulations} 
\section{Influence of inlet geometry} 

%!TEX root = Tesis.tex
\chapter{Noise measurement methodology} \label{cap:metod}

% Quote
% \null\vspace{-10mm}
% \begin{flushright}
% \begin{minipage}[t]{0.66\textwidth}
% {\itshape\small``The story is set in the year 1922 and has its main character, Mr Engelbert, moving to Prague from the countryside to live out his retirement years and struggling to cope with the aggressive modernity of the city. The horror is not the power of money or the arrogance of the people, but the noise; not the age-old noise of a thunderstorm or a hammer, but the new noise of engines, especially of automobiles and motor-cycles, the explosive internal combustion monsters.''}\\[-3mm]

% {\cabincondensed\small\hfill \textbf{Milan Kundera}, \emph{The Curtain.}}\\[5mm]
% \end{minipage}
% \end{flushright}

% Chapter text
\noindent As stated in the prior chapter, one of the main objectives of the present work has been to develop and implement robust measurement methodologies for the acoustical characterization of turbocharger compressors.
\blfootnote{Work on this chapter has been partly published in the following paper:
\begin{itemize}
\item \emph{Experimental methodology for turbocompressor in-duct noise evaluation based on beamforming wave decomposition} \cite{torregrosa2016experimental}
\end{itemize}
}

In this chapter the theoretical background of these methodologies will be presented, along with the details of their practical implementation and a brief review of the facilities at where they have been implemented.

Apart from the purely experimental implementation, the step-by-step signal processing required to produce final, compiled results will be introduced as well, including sample results in order to illustrate each step.

Special care will be put in considering the different theoretical and practical limitations and restrictions of experimental measurements and processing techniques, to ensure that results are significant and robust.

\section{Facilities} % Descripción instalaciones

Most of the experimental work presented in this thesis has been carried out at laboratory 5K of Universitat Politècnica de València. This facility is equipped with two main installations.

The first one is a flow bench powered by a Diesel engine linked to an asynchronous \hl{dyno} and a screw compressor. This flow bench can be configured to function as a turbocharger test stand or to provide air flow to other facilities. Subsection \ref{sub:flow_bench} will be dedicated to an extended description of this installation.

Next to the flow bench, there is a large anechoic chamber that can be configured to either receive external air flow from the flow bench or to host a standalone engine by itself. For this project, air was provided by the flow bench when the turbocharger test stand was installed inside the chamber. This arrangement is illustrated by Fig. \ref{fig:floorplan}, while subsection \ref{sub:anec_chamber} will describe in detail this particular chamber.

Additionally, an engine test cell located on laboratory 6D has been used for the on-engine validation tests described on chapter \ref{cap:engine}, where the inlet intake of a commercial engine was modified to measure the compressor acoustic emission.

Other facilities included an impulse test bench that was used for sensor calibration, and a workshop were the required customized parts for measuring and instrumenting were manufactured.

\begin{figure}[t!]
\centering
\includegraphics[width=\textwidth]{figures/cap_metod/schematic}
\caption{Floor plan of lab 5K, containing a flow rig and its associated turbocharger installation inside an anechoic chamber with their auxiliary systems.}
\label{fig:floorplan}
\end{figure}

\subsection{Flow test rig} \label{sub:flow_bench}

The flow test rig has been used as flow provider to power the turbocharger turbine both directly, with the turbocharger installed in the flow rig cell itself and, as shown in Fig. \ref{fig:floorplan}, with the turbocharger mounted in a supporting frame inside the anechoic chamber.

At the core of the rig there is a heavy duty MIDR 06.20.45 Diesel engine able to provide up to 260 kW of power. This engine is linked to an asynchronous dyno and, through a coupling gear, to an Atlas Copco ZA 110-3.5 volumetric screw compressor. An overview of the whole installation can be seen in Fig. \ref{fig:flowrig_render}.

\begin{figure}[b!]
\centering
\includegraphics[width=\textwidth]{figures/cap_metod/flow_rig2}
\caption{Render of the flow rig showing the engine core, asynchronous dyno with its support cabinet, screw compressor and air distribution system, including control valves and reservoirs.}
\label{fig:flowrig_render}
\end{figure}

Purpose of this compressor is twofold. First, it supercharges the Diesel engine as its original turbocharger group was removed. Second, compressed flow can be routed to either a turbocharger installed in the same cell or to the adjacent chambers.

Another possibility of this test rig is to directly feed the exhaust gases of the engine to an external turbocharger. This is the reason why the original turbocharger was removed during installation. Direct flow will contain the pulsating content of the engine, allowing simulation of real engine conditions.
The number of cylinders firing can be selected from 1 to 6 in order to simulate different pulsating content. Also, if the hot exhaust gas is desired but the pulsations are not, flow can be routed through a settling tank that will smooth them out, resulting in continuous flow.

For the studies carried out in this facility during the development of this thesis, pulsating flow was not selected. Continuous flow was used in both the test rig and the anechoic chamber. However, work presented in chapter \ref{cap:engine} used real engine conditions to validate the proposed techniques and obtained results.

As for the flow control, another reservoir equipped with a discharging valve can be used to distribute the compressed air between the engine inlet manifold and the external circuit, which can be either directed to another chamber or vented to the outside ambient through a silencer.

If cold flow is desired, a water intercooler can be used to lower the temperature produced by the compression process. This temperature, along with the rest of flow properties like mass flows, pressures and other temperatures, are recorded through a dedicated control system.

Engine and brake are controlled though their own independent instrumentation, while the instantaneous turbocharger variables, including those needed for acoustical characterization, are recorded through a dedicated high speed Yokogawa DAQ.

\subsection{Anechoic chamber} \label{sub:anec_chamber}

The anechoic chamber at laboratory 5K is a solid $9.5 \times 9 \times 7.2$ m reinforced concrete structure, isolated from the rest of the building and resting on a waterproof support that dampens outside vibrations.

The internal walls of the chamber are completely covered in fiberglass wedges of 0.8 m of longitude and a square base of $0.2 \times 0.2$ m. These wedges leave an interior space of $7.5 \times 6.5 \times 6$ m. Usually a further 1 m in each direction is left as an additional safety margin, the rest of the volume being equivalent to acoustic free field conditions.

Such free field conditions, where no acoustic reflection of the walls occurs, are certified in this chamber up to a cut-off frequency of 100 Hz. Low frequency waves may start to reflect below this limit.

A lightweight, acoustically-transparent, grated metallic floor is situated over the floor wedges, supporting personnel and light equipment. Four round steel pillars affixed to the concrete floor are available to support heavy loads such as engines.

\begin{figure}[h!]
\centering
\includegraphics[width=\textwidth]{figures/cap_metod/anechoic}
\caption{View of the inside of the anechoic chamber, showing the turbocharger setup with the mounting frame and the pipes required to operate the installation.}
\label{fig:anechoic_chamber}
\end{figure}

During the work described in this thesis, the load-bearing pillars were unused, since the turbocharger frame was lightweight enough to be supported by the grated floor. Figure \ref{fig:anechoic_chamber} shows the tested turbocharger installed in this frame.

In order to allow the operation of the turbocharger, compressed air flow is fed to the turbine from the flow rig described in the prior subsection. The turbine discharges to a reservoir equipped with a conical nozzle that decreases velocity and acoustic emissions.

Meanwhile, the compressor takes ambient air from the chamber, while its exhaust is redirected to the adjacent auxiliary room for discharging, as seen in Fig \ref{fig:floorplan}. A closed-loop lubricating oil system is also present. Water is also provided to the cooling jackets of the sensors that require it, specially those located on the turbine pipes and the compressor outlet.

A control system similar to that of the flow rig monitors the operation of the system and gathers operational data (pipe temperatures, pressures, mass flow, compressor speed, etc.), allowing regulation of oil temperature, and compressor operation settings (speed and mass flow). Scientific data (mainly dynamic pressure) is recorded with a Yokogawa high speed DAQ.

\subsection{Engine test cell} \label{sub:engine_cell}

As mentioned early in the chapter, while most of the experimental work was carried out using the two facilities described already, the measurement methodology was later put to the test on an engine test cell to verify its reliability on real engine conditions.

The test cell is equipped with an asynchronous electromagnetic brake that allows both steady and transient automatic testing. Operative parameters and control for the engine and brake are integrated into a custom system, including gravimetric fuel consumption sensors, hot plate anemometers, emission measurements and temperature and pressure control of the air pipes and coolant flows.

Access and online control of the engine ECU is also available, and a special pneumatic system can be used to inject compressed air into the inlet manifold downstream of the compressor.

This air injection reduces the air that the compressor ingests, thus lowering compressor air flow and allowing the control of this relevant parameter independently of the engine necessities. The system was used to bring the compressor to surge and near-surge conditions during our research.

In order to implement and validate the in-duct acoustical measurement methodology presented in this chapter, additional sensors and equipment were fitted into the test cell, including a Yokogawa high speed DAQ and a Brüel \& Kjær PULSE DAQ system with a pressure-pressure sound intensity probe used as reference.

\begin{figure}[b!]
\centering
\includegraphics[width=\textwidth]{figures/cap_metod/sala_limpia}
\caption{View of the inside of the engine test cell with the engine in its default configuration, before being modified for the acoustic measurements described in chapter \ref{cap:engine}.}
\label{fig:engine_cell_clear}
\end{figure}

\section{Tested turbochargers}

Different turbochargers have been used in the development of the investigation, as dictated by the requirements of the different projects and facilities that this thesis comprises. All of them are however similar, small to medium sized automotive turbochargers. The most relevant model is the \hl{Garrett GT17} model that was used in chapters \ref{cap:metod}, \ref{cap:results}, \ref{cap:cfd} and part of chapter \ref{cap:geom}. 

\begin{figure}[h!]
\centering
\includegraphics[width=0.95\textwidth]{figures/cap_metod/comp_dimensions}
\caption{Dimensional drawing of the compressor used in [most] of the development of the investigation, marking some relevant dimensions.}
\label{fig:comp_dimens}
\end{figure}

This particular model was digitized to obtain its detailed geometry, which is shown in Fig.~\ref{fig:comp_dimens} along with the most relevant dimensions. \hl{[More data]}

\section{In-duct noise measurement} % Articulo methodology

In this section, a methodology is proposed to evaluate the noise emission of the compressor through its inlet and outlet pipes across selected zones of the compressor map of operating conditions. It allows a simple and clear visualization of noise phenomena and their frequency distributions at all possible operation points of the turbocharger system.

Although compressor manufacturers sometimes provide these kind of ``noise maps'' \cite{gaude2008experimental}, they usually refrain from providing an adequately referenced and step-by-step methodology to reproduce them.

Reviewing the existing literature on turbocompressor noise a variety of very different measurement techniques can be found, from the two-sensor, in-duct approach of Tiikoja et al \cite{tiikoja2011inves} to simple single-sensor pressure levels and external commercial noise-meters \cite{figurella2012noise}.

However, these works focus on the research of different phenomena (transmission loss \cite{tiikoja2011inves}, effect of flow incidence angle \cite{figurella2012noise}, sound generation by rotating stall \cite{mongeau1993sound}, source characterization \cite{mongeau1995method}, etc.), not on the acoustical methodology itself nor on the particular setup considerations and restrictions that each measurement technique imposes.

This work wishes to address these shortcomings of the existing literature by proposing an experimental methodology and discussing the details of its theoretical background, its implementation, its range of application, and the  processing required to produce standardized results.

In order to provide some illustrative results, the methodology proposed in this paper has been applied to a series of experimental tests in a flow bench where a small automotive turbocharging group has been installed.

\subsection{Theoretical background}
\label{sec:theory}

\subsubsection{Sound intensity in flow ducts}
\label{sub:sound_intensity_in_flow_ducts}

While it is possible \cite{figurella2012noise} to rely on a single sensor to measure the scalar \emph{sound pressure level} (SPL) of the flow at a certain location of a duct, more sophisticated approaches are needed to estimate the \emph{sound intensity level} (SIL) that is propagating through the duct.

It is important to consider that the scalar magnitude of sound pressure level at a point can be influenced by the precise geometry of that section and by nodes and reflections that may occur in the duct, whereas sound intensity is a vectorial magnitude that remains almost constant along the duct (except for small dissipation losses), making it a more robust indicator of noise emission.

The basis of such intensity measurement methods is to consider that the pressure signal $x(t)$ measured at a given point of a duct is the linear superposition of a pressure wave $x^+$ travelling downstream and another pressure wave $x^-$ travelling upstream (sometimes referred to as forward and backward waves, respectively \cite{pinero2000estimation}) so that:

\begin{equation}
  x(t)=x^+(t)+x^-(t)
\end{equation}

Comparing the pressure information at two or more spatial positions it is thus possible to infer how the waves are propagating along the duct in each direction, a possibility unavailable by means of a single sensor.

When using this wave decomposition approach, the sound intensity can be estimated following the expression proposed by Morfey \cite{morfey1971sound} and successfully derived by Dokumaci \cite{dokumaci2000calculation} from physical principles:

\begin{equation}\label{eq:metod_intensity}
    I=\frac{1}{\rho a}\left(|\mathbf X^+ |^2(1+M)^2-|\mathbf X^- |^2(1-M)^2\right)
\end{equation}

Here $\mathbf X^+$ and $\mathbf X^-$ are, respectively, the complex spectra of the downstream and upstream pressure waves, $\rho$ is the mean density, $a$ the mean sound speed, and $M$ the mean Mach number of the flow.

\subsubsection{Beamforming wave decomposition}
For a more intuitive comprehension of the beamformer, one can consider its transmitting equivalent: a phased array emitter where the same signal $x(t)$, with its phase shifted by a quantity $w_n^*$, is fed to an $n$-element linear array of fixed transmitters, forming a plane wave emitted in the $\theta$ direction which can be steered at will adjusting each $w_n^*$.
The beamforming method provides a way to mathematically tune the sensitivity of the overall system in order to isolate the downstream and upstream information of the acquired signal. A scheme for a narrowband beamformer can be found in Fig. \ref{fig:schematic}.

\begin{figure}[b!]
\centering
\includegraphics[scale=0.85]{figures/cap_metod/NB_Beamformer}
\caption{Diagram of a narrowband beamformer with three elements tuned to a Direction of Arrival (DOA) of $\theta$.}
\label{fig:schematic}
\end{figure}

The approach in this caase is the inverse: by tuning the weights $w_n^*$ that multiply each recorded pressure signal $x_n(t)$ the pressure wave coming from the direction of arrival (DOA) $\theta$ that we are interested in may be resolved. Details can be found in \cite{van1988beamforming}.

\begin{figure}[t!]
\centering
\includegraphics[width=0.8\textwidth]{figures/cap_metod/Decomposition_t}
\caption{Sample result of the decomposed pressure; at the inlet the amplitude of the pressure wave travelling downstream (towards the compressor) is smaller than the one coming from the compressor; the reverse situation occurs at the outlet.}
\label{fig:decomposition}
\end{figure}

\begin{figure}[t!]
%\centering
\includegraphics[width=0.90\textwidth]{figures/cap_metod/Spectra}
\caption{Sample of raw inlet and outlet pressure power spectral densities (at 160 krpm and 60 g/s), including frequency limitations described in equations \ref{eq:f_nyquist} and \ref{eq:f_modes}. The characteristic blade pass frequency can also be identified, specially in the outlet spectra.}
\label{fig:spectra}
\end{figure}

The extension to a wideband beamformer is relatively straightforward \cite{piper2011broadband}. First the Fast Fourier Transform (FFT, denoted by $\mathcal F$) of the recorded pressure signals is computed:

\begin{equation}
  \mathbf X_n(f_k)=\mathcal{F}\{x_n(t_k)\}
\end{equation}

Subscript $k$ indicates that the signal is acquired at discrete time steps. The described narrowband procedure is then followed for each discrete frequency to finally obtain the desired signal through the inverse transform:

\begin{equation}
    x(t)=\mathcal{F}^{-1}\left\{\mathbf w^H(f_k)\; \mathbf X(f_k)\right\}
\end{equation}

Here, $\mathbf X(f_k)$ is the $k \times n$ matrix of transformed signals and $\mathbf w^H(f_k)$ is the matrix of weights for each frequency. In order to compute the optimal weights for the desired DOAs ($\theta=-90^o$ for downstream and $\theta=90^o$ for upstream waves, assuming a typical flow DOA of $\theta=90^o$), several schemes can be used.

For this study a Linearly Constrained Minimum Variance (LCMV) beamformer was  used, a well established procedure, which aims at minimizing the overall output power (variance) of the signal, while maintaining unitary gain in the precise desired direction.

While several beamforming strategies and implementations are available, \ref{ann:beamformer} offers a simple guide on how to compute $\mathbf w^H(f_k)$ and thus $\mathbf X^+$ and $\mathbf X^-$ using this particular approach. Figure \ref{fig:decomposition} shows the result of this decomposition in the time domain.

Further information on its application, including a comparison of various of these schemes against simulated and experimental data is available in the work of Piñero et al. \cite{pinero2000estimation}.

\subsection{Experimental procedure}
\label{sec:experimental_procedure}

\subsubsection{Turbocharger instrumentation}
\label{sub:turbocharger_instrumentation}

The correct design and installation of the fast pressure sensor arrays is a critical part of the experimental setup.

Ideally, the sensors should be mounted flush (with water-cooled adapters if necessary) in straight pipes, with diameters adjusted to preserve the cross sectional area of the adjacent duct in order to avoid complex structures caused by singularities.

\begin{figure}[b!]
\centering
\includegraphics[scale=0.80]{figures/cap_metod/setup_t}
\caption{Scheme of the proposed turbocharger instrumentation setup, highlighting in color the three-sensor beamforming arrays and their clearance requirements.}
\label{fig:setup_schem}
\end{figure}

As a further measure, enough straight clearance should be left before and after the array (as shown on figure \ref{fig:setup_schem}), to ensure essentially plane wave flow at the measurement section \cite{holland2000measurement}. There should be a minimum of 6 diameters upstream of the array and a further 4 diameters downstream, as recommended in \cite{torregrosa2005exp}.

When placing the transducers, it is important that the 1/5 and 1/3 nodes of the established standing wave pattern are avoided \cite{dowling1983sound}. Also, a compromise in the spacing $d_s$ between them must be reached, taking into consideration errors in the  low-frequency \cite{aabom1988error} and high-frequency \cite{seybert1988two} bands.

Specially, it should be noted that spatial aliasing effects will occur above the Nyquist frequency imposed by the spacing $d_s$. The time of arrival of a wave with DOA $\theta$ and speed $a$ to two consecutive sensors spaced by a distance $d_s$ is:
\begin{equation}
  T_d = \frac{d_s \sin \theta}{a}
\end{equation}

This imposes a Nyquist-type frequency criterion, as the measured wave frequency $f_n$ should be maximum half the frequency $f_d=1/T_d$ naturally associated with $T_d$ to ensure that no high frequency spatial aliasing effects are present:
\begin{equation}\label{eq:f_nyquist}
  f_n \leq \frac{f_d}{2} = \frac{1}{2T_d}=\frac{a}{2d_s \sin \theta}=\frac{a}{2d_s}
\end{equation}

A further restricting phenomenon is the acoustic mode propagation. As the beamforming method proposed is based on the assumption of one-dimensional wave propagation, it will become unreliable with the onset of 3D effects.

According to Eriksson \cite{eriksson1980higher}, it may be assumed that waves in a circular duct propagate in a planar fashion (i.e., pressure is constant in every cross-sectional area of the duct) below a cut-off frequency of:
\begin{equation}\label{eq:f_modes}
f_a = 1.84 \frac{a}{\pi D}\sqrt{1-M^2}
\end{equation}

Here, $a$ is the sound speed, $D$ the duct diameter and $M$ the mean Mach number. At higher frequencies, the so-called first asymmetric mode will start to propagate, thus rendering the assumption of one-dimensional propagation invalid.

Special care should be taken when computing limiting frequencies $f_n$ and $f_a$ as they are not constant but dependent on the operating condition, since some of their components vary with temperature and mass flow. Figure \ref{fig:spectra} shows both limits for a certain operating point.

As the figure shows, the Nyquist criterion accurately pinpoints where differences between sensors start to grow, and it is clearly a more restrictive criterion than the acoustic mode onset. This limit should be properly computed for each condition when multi-sensor decomposition techniques are used.

Also, figure \ref{fig:spectra} shows other phenomena, for instance the Blade Passing Frequency (BPF) tonal noise that is identified in many of the works presented in chapter \ref{cap:liter}. As shown in table \ref{tab:lit_gaude_sources}, the main BPF can be estimated by taking into account the shaft speed and the number of main blades:
\begin{equation}
  \text{BPF} \sim 160\,\text{krpm}\,/\,60\,\frac{\text{s}}{\text{min}} \times 6\,\text{main blades} \sim 16\,\text{kHz}
\end{equation}

It is also important to note that the closer the sensors are mounted together for increasing the $f_n$ limit, the poorer the spatial resolution of low frequency waves will be, as the wavelengths $\lambda=a/f$ become too long and the difference between consecutive sensors decreases below the noise threshold. This effect is intrinsically related to each particular sensor model, but it must be considered if very low frequencies are of interest.

Finally, it is essential to ensure that the response of the three sensors of each array is coherent. Before mounting the sensors in the final linear array form, they should be mounted radially in the same section of a duct and individually calibrated, to ensure that the difference between sensors is not due to individual transducer differences, but only to wave propagation.

In this case, the calibration is performed in an impulse test rig (described in detail in \cite{payri2000modified}) so as to obtain a clear reference pressure wave in all sensors. Both the time and frequency responses are observed, and both the amplifier offset and gain are adjusted to find the best match. An example can be seen on figure \ref{fig:6031_calibration}.

\begin{figure}[t!]
\centering
\includegraphics[width=0.85\textwidth]{figures/cap_metod/6031_calibration_t}
\caption{Results of the calibration of the outlet sensors in an impulse test rig, showing a pressure measurement match in both time and frequency domains.}
\label{fig:6031_calibration}
\end{figure}

In addition to the fast pressure sensors data, more information is required to compile the compressor operating conditions map. These maps are usually presented as plots of the operation points on a total-to-total pressure ratio $(\Pi_\text{TT})$ versus corrected air mass flow $(\dot m^*)$ axes.

\subsubsection{Steady state measurements}
\label{sub:steady_state_measurements}

The usual approach for measuring the operating points of the compressor map involves reducing the air flow by progressively closing the back-pressure valve (shown in figure \ref{fig:setup_schem}) until the desired air mass flow is reached, while the turbine power is adjusted to preserve a given corrected compressor speed $(N^*)$ and to achieve stable operation conditions.

Eventually the compressor blades will begin to stall, until a deep surge condition is reached, as shown in the top plot of figure \ref{fig:surge_datapoints}. In order to avoid this destructive condition, the valve will be opened again and another constant corrected regime line will be measured, until the desired region of the compressor map is completely covered.

Several reference conditions can be selected when correcting the speed and the air mass flow; in this study the mean sea level standard day (defined as one of 15ºC of temperature and 1 atm of pressure) of the International Standard Atmosphere will be used:
\begin{equation}
    \dot m^* = \dot m\frac{101325}{p_{T}}\sqrt{\frac{T_{T}}{288.15}}
    \qquad N^*=N\sqrt{\frac{288.15}{T_{T}}}
\end{equation}

Here, $p_{T}$ and $T_{T}$ refer to the compressor inlet total pressure and temperature, respectively. These are calculated with the air mass flow and the cross sectional area of the inlet duct, using the continuity and ideal gas relations and assuming adiabatic flow:

\begin{figure}[bth!]
\centering
\includegraphics[width=0.85\textwidth]{figures/cap_metod/datapoints_surge.pdf}
\caption{Operating conditions for each measured data point (right plot), including the precise determination of the deep surge limit: top plot on the left shows the temporal evolution of the raw relative pressure recorded by one piezoelectric sensor; bottom plot on the left depicts the correlation between the frequency content in the 0-20 Hz band and deep surge onset.}
\label{fig:surge_datapoints}
\end{figure}

\begin{equation}
    \rho =\cfrac{p}{RT} \quad
    V =\cfrac{\dot m}{\rho A} \quad
    T_{T}=T+\cfrac{V^2}{2c_p} \quad
    p_{T}=p\left(\cfrac{T_{T}}{T}\right)^\frac{\gamma}{\gamma-1}
\end{equation}

In this case $R$ denotes the air-specific ideal gas constant and $\rho,P,T$ and $V$ the density, pressure, temperature and flow speed in the duct, respectively. Also, $A$ denotes the cross sectional area, and $c_p$ the air-specific heat capacity.

Figure \ref{fig:surge_datapoints} (bottom) shows the recorded data points of the compressor map, after computing the total-to-total pressure ratio and
applying the standard day correction.

\subsubsection{Surge limit characterization}
\label{sub:surge_limit_characterization}

When measuring a turbocharger operation map, it is particularly important to  characterize accurately the surge limit, especially when an acoustic evaluation is sought out \cite{galindo2006surge}.

In order to measure different mass flows, the back-pressure valve is usually closed slowly until a certain surge criterion is met, and then opened to avoid damaging the turbocharger. Several surge criteria for centrifugal turbomachinery have been proposed in the literature, for tests on flow benches \cite{galindo2006surge} and on engine test rigs \cite{galindo2013engine}.

An average of power spectrum of a single pressure sensor in the 0 to 20 Hz range, shown by Galindo et al \cite{galindo2006surge} to be a good surge indicator, was computed once for discrete time steps $T$ of 0.5 seconds, providing a clear indication of deep surge onset, as marked in figure \ref{fig:surge_datapoints} (top) with a solid line.

\begin{equation}
 \mathbf X_T=\left|\mathcal F\{x(t)\}\right|^2,\; t\in[t_T,t_{T+1}]
\end{equation}

To obtain the required near-surge data points to be displayed on the map, represented by $\blacksquare$ in the bottom plot of figure \ref{fig:surge_datapoints}, a 1 second average of all recorded data was computed 1 second before surge as indicated by the dashed lines of the left plot, in order to calculate the map coordinates of the point.

\begin{figure*}[t!]
\centering
\includegraphics[width=0.85\textwidth]{figures/cap_metod/gr_mapa_ruido_dB_1000-3000.pdf}
\caption{Sample of noise level in the 1 -- 3 kHz band obtained through the processing described in subsection \ref{sub:noise_maps_and_spectrograms}, including the measured data points $(\circ)$ and the paths later expanded as spectrograms shown in figure \ref{fig:spec50-5k}. The left hand side map represents the inlet duct, whereas the right hand side map represents the outlet duct. }
\label{fig:map1k-3k}
\end{figure*}

\subsubsection{Signal postprocessing: maps and spectrograms}
\label{sub:noise_maps_and_spectrograms}

In addition to the intensity spectra determined for each operating point, further results can be obtained by applying an interpolation procedure to the spectra associated with all the points tested.

For instance, it is possible to calculate the overall level of sound intensity for a given frequency range at each operating point $i$ of the compressor operating map. The sound intensity level at a characteristic band (between $f_1$ and $f_2$) can be calculated for each point as:
\begin{equation}\label{eq:point_level}
  L_i=10\log_{10}\left(10^{12}\sum_{f=f_1}^{f_2}I^2(f)\right)
\end{equation}

Once that several $L_i$ levels have been measured for different conditions of a selected map region, a cubic spline interpolation function $\Phi_{f_1-f_2}$ is used to project the data onto a much more fine and regularly spaced grid, allowing the representation of a map of noise intensity across the measured region of the compressor map, as shown in figure \ref{fig:map1k-3k}.

These ``noise maps'' synthesize the information of the acoustic output of a frequency band of interest in a simple and visual result, allowing an easy estimation of the noise emission through desired operating conditions even if these have not been specifically measured.

Expanding on the necessity of analysing the frequency content evolution as the operating conditions of the compressor change, a different approach can be used, where the measured discrete information can be compiled to estimate a continuous spectrogram.

This is accomplished by sequentially performing the previous map processing for a number $N$ of much smaller frequency bands of width $\Delta f$, resulting in a collection of $N$ interpolation functions $\Phi_{f_j-f_{j+1}}$, where $f_{j+1}-f_j = \Delta f$.

Using these functions, the intensity levels $\mathbf L$ on each small frequency band $k$ are interpolated for a set of $M$ pairs of the compressor map coordinates, such that:

\begin{equation}
 \mathbf L_j=\Phi_{f_j-f_{j+1}}(\bm{\dot m^*},\bm{\Pi_\text{TT}})
\end{equation}

An estimation of the spectrogram $\mathbf S$ that would be obtained in a quasi-stationary measurement by following the set of operation conditions $(\bm{\dot m^*},\bm{\Pi_\text{TT}})$ pairs can then be compiled gathering the levels $\mathbf L_j$. Choosing one coordinate of the set of pairs (in this case $\dot m^*$) as reference for the X axis:

\begin{equation}
  \mathbf S(f_j,\dot m^*)=\begin{bmatrix}
    \mathbf L_1(\dot m^*_1) & \hdots & \mathbf L_1(\dot m^*_M)\\
    \vdots & \ddots & \vdots \\
    \mathbf L_N(\dot m^*_1) & \hdots & \mathbf L_N(\dot m^*_M)
  \end{bmatrix}
\end{equation}

Thus the sound intensity level value $\mathbf S$ for each $\dot m^*$ and each frequency band $j$ can be plotted. An example of this processing for frequencies between 50 and 5000 Hz binned with $\Delta f = 50$ Hz is presented on figure \ref{fig:spec50-5k}, where coordinates $(\bm{\dot{m}^*},\bm{\Pi_\text{TT}})$ correspond to the 160 krpm iso-speed line highlighted in figure \ref{fig:map1k-3k}.

\begin{figure*}[t!]
\centering
\includegraphics[width=0.85\textwidth]{figures/cap_metod/gr_spec_ruido_dB_50-5000.pdf}
\caption{Sample of spectrograms calculated using the procedure outlined in subsection \ref{sub:noise_maps_and_spectrograms}, following the 160 krpm paths marked in figure \ref{fig:map1k-3k}. The 1--3 kHz band selected for the maps in figure \ref{fig:map1k-3k} is highlighted, along with some particular flow phenomena discussed in section \ref{sec:results_and_discussion}.}
\label{fig:spec50-5k}
\end{figure*}

\subsection{Sample results}
\label{sec:results_and_discussion}

\hl{[Quitar?]}

In order to illustrate the methodology presented in this section, a measurement of the tested turbocharger was first carried out in the flow bench that is used to provide flow to the anechoic chamber, configured as shown in figure \ref{fig:setup_schem}. Since in this case the only concern was the validation of the in-duct methodology, installation in the anechoic was not required. 

Kistler type 7031 piezoelectric transducers were used for the inlet duct; the outlet duct was fit with type 6031 transducers due to available stock. An array spacing $d_s=32$ mm for the outlet and $d_s=50$ mm for the inlet was selected, allowing for a safe upper limit in terms of the spatial Nyquist criterion $f_n$ of $\sim$6000 and $\sim$3500 Hz respectively (precise values depend on the conditions of each point).

The variables needed in order to characterize the map point (mass flow, speed, temperatures and pressures) were averaged during 30 seconds using the in-house DAQ system, while the six piezoelectric sensors acquired $10^5$ pressure samples during 1 second using a Yokogawa digital recorder.

Pressure signals on recorded points (shown in figure \ref{fig:surge_datapoints}) were then processed according to the proposed methodology. Samples of the final results are presented in figures \ref{fig:map1k-3k} and \ref{fig:spec50-5k}.

Figure \ref{fig:map1k-3k} shows the 1 -- 3 kHz sound intensity level distribution across selected operating conditions, approximately between maximum efficiency and the measured surge limit. As shown in chapter \ref{cap:liter}, noise in this subsynchronous frquency is often known as ``whoosh''.

Another point of concern that has been identified in the results is that the relevance of this particular phenomenon extends far from the surge region and well into the expected engine operation region.

Looking at either the 160 krpm line on figure \ref{fig:map1k-3k} or at the expanded spectrograms of this line in figure \ref{fig:spec50-5k} where this band has been marked, there is a local maximum in this band between 60 and 70 g/s. For the 180 krpm points on the map of figure \ref{fig:map1k-3k}, the ``whoosh'' noise may be heard even for up to 80 g/s mass flow rate.

Apart from ``whoosh'', both spectrograms show a clear rise in low frequency content when the mass flow is reduced and surge conditions start to appear, even if deep surge conditions have not yet been reached. This is consistent with spectrograms measured experimentally by Galindo et al \cite{galindo2009effect}.

% \begin{wrapfigure}[36]{O}[1.6cm]{0.41\textwidth}
% \vspace{-5mm}
% \includegraphics[width=0.42\textwidth,trim={1cm 0 5mm 0},clip]{figures/cap_metod/mapas_comp}
% \caption{\null\\Outlet maps computed at different frequency ranges. From top to bottom: a general view of the plane wave range global intensity; the characteristic band of the ``whoosh noise'', and a narrow low frequency band that indicates surge onset.}
% \label{fig:mapas_comp}
% \end{wrapfigure}

% Furthermore, this kind of low frequency band may be plotted and its distribution compared with other bands, as shown in figure \ref{fig:mapas_comp} where three different frequency limits have been applied to equation \ref{eq:point_level} during signal processing.

% Compared with the global distribution of sound intensity in the plane wave range (left plot), the ``whoosh noise'' band (central plot) appears to be more prevalent at higher pressure ratios, while the low frequency content (right plot) is characteristic of surge onset.

% These observations match the trends observed in the literature \cite{evans2005minimizing,evans2006reduction,sevginer2007investigation}. Other frequency bands could be mapped in a similar manner, and spectrograms estimated for different map paths with no need to repeat the experimental measurements, in order to investigate other phenomena of interest or to test several candidates for actual engine operating lines.

\section{External noise measurements}

While the accurate characterization of the internal sound field is crucial to understand the different phenomena, the transmission of the noise to the ambient is also relevant as external acoustic radiation is ultimately the main concern of the automotive industry.

To that extent, different external measurements have been carried out as part of the experimental campaign. Noise data has been captured in the orifice of the compressor inlet, around the far field of the turbocharger, and along the radiating surfaces near field.

This section will describe how these measurement were made, while results and discussion will be presented in section \ref{sec:external_results} of chapter \ref{cap:results}, including a correlation of orifice noise and in-duct results and the discussion on the contribution to the overall noise produced by each part of the system.

\subsection{Orifice noise}
\label{sub:metod_orifice_noise}

The noise emitted by the open end of the compressor inlet duct is of special concern, as it is the point where the internal sound field meets the exterior ambient. Furthermore, the compressor may not act as the only source of noise in this case, as the noise produced by the flow interaction with the duct orifice during the aspiration process can be relevant as well.

However, when measuring external noise care should be taken to avoid reflections and perturbations that may interfere with the results. This is why it was decided to install the turbocharger inside the anechoic chamber, so that free-field conditions could be recreated and external noise measurements could be taken and compared against in-duct results.

\begin{wrapfigure}[21]{r}[1.5cm]{0.4\textwidth}
\centering
\vspace{-3mm}
\includegraphics[width=0.4\textwidth]{figures/cap_metod/mics_photo}
\caption{Image of the two free-field microphones placed at 10 mm (mic 1) and 20 mm (mic 2) from the orifice of the compressor inlet duct.}
\label{fig:mics_photo}
\end{wrapfigure}

In this chamber, instrumentation consisted of two free-field Brüel \& Kjær \textcolor{red}{[model]} microphones placed next to the compressor inlet orifice, separated 10 cm and 20 cm, respectively. These are shown as ``Mic 1'' and ``Mic 2'' in Fig.~\ref{fig:mics_photo}.

These microphones were calibrated before the measurements with a Brüel \& Kjær \textcolor{red}{[model]} pistonphone. Both the calibration and the data acquisition were performed with a Brüel \& Kjær PULSE system.

As the conditions inside the anechoic chamber can be regarded as being equivalent to those of an acoustic free field, we can estimate the magnitude of the sound intensity at each microphone $i$ as:

\begin{equation}
	|I_i| = \frac{p^2_\text{rms,i}}{\rho c}
\end{equation}

Here, $p^2_\text{rms,i}$ is the root mean squared pressure recorded by microphone $i$, $\rho$ is the ambient density and $c$ the ambient sound speed. These are calculated from pressure and temperature sensors available in the anechoic chamber.

\subsection{Far field}
\label{sub:metod_far_field}

Taking advantage of the free field conditions of the chamber, ambient microphones were installed around the turbocharger group to estimate the far field sound radiation.

To obtain a better estimation of the sound field, four omnidirectional microphones were distributed around the the central point of the installation, at a radial distance of 1 m from it, as shows Fig. \ref{fig:radiated_jag}.

\begin{figure*}[tb!]
\centering
\includegraphics[width=0.95\textwidth]{figures/cap_metod/radiated_jag}
\caption{Radiated noise microphones installed around a commercial set of compressor inlet and outlet pipes that will be sequentially insulated.}
\label{fig:radiated_jag}
\end{figure*}

Data from these four microphones was captured using a Brüel \& Kjær PULSE system, and their calibration was also performed using a Brüel \& Kjær \textcolor{red}{[model]} pistonphone prior to the measurements.

In order to provide a global metric to compare against orifice noise, results for the four microphones was averaged to compute the final sound pressure level.

\subsection{Near field}

\hl{[Informacion sobre velocidad de particula?]}

\section{Conclusions}
\label{sec:conclusions}

In this chapter a detailed experimental methodology for the noise evaluation of turbocharger compressors in flow benches or engine test cells has been proposed. It includes a description of the tested turbocharger\hl{s} and the facilities where the experimental work has been developed.

The characterization method is based on a robust in-duct acoustical beamforming technique that resolves the plane wave range sound intensity, by decomposing the total pressure signal into its upstream and downstream components.

In addition to resolving intensity, another advantage of the wave decomposition technique is the possibility of isolating the pressure data coming from the compressor, in order to enhance the spectral content of interest in situations where actual anechoic setups are not feasible, due to valves, filters, and other elements.

This isolated pressure information is also useful to validate or initialize CFD simulations where the model boundary conditions have to be defined as anechoic because simulating accessories (valves, flexible ducts, etc.) is not worthy or desirable. Such a validation methodology was presented in \cite{broatch2014methodology} and is discussed in chapter \ref{cap:cfd} of this work.

When used to evaluate the acoustic intensity of the noise generated by the compressor, the proposed signal processing allows to easily  understand visual representations in the form of noise maps of selected frequency bands of interest and estimated spectrograms of operating paths. This can provide guidelines for the evaluation of strategies to counter or avoid certain adverse noise phenomena.

The examples presented above illustrate how this technique allows detecting known acoustical phenomena of a centrifugal compressor, such as medium frequency ``whoosh noise'' and low frequency surge onset, as well as mapping its distribution and relevance across different operating conditions of the turbocharger.

In addition to offering the detailed measurement and signal processing steps, this chapter stresses the importance of taking into account restrictive factors such as calibration, sensor setup clearances and locations, frequency aliasing effects and ranges in which theoretical assumptions hold in order to obtain robust and reproducible results.

Additionally, and taking advantage of the anechoic chamber where the turbocharger was installed, the instrumentation that was added to measure external noise (both orifice noise and the noise radiated to the far field by the system surfaces) is described. \hl{[near field?]}

\subsection{Appendix: LCMV beamforming procedure}
\label{ann:beamformer}

\hl{[Donde poner esto?]}

This appendix proposes a step by step guide to compute the desired $\mathbf X^+$ and $\mathbf X^-$ decomposed pressure spectra required in equation \ref{eq:metod_intensity}, using a LCMV (Linearly Restricted Minimum Covariance) strategy as presented by Verdú \cite{verdu2003contribucion}.

Recall that decomposed signals for each frequency $f_k$ are obtained by weighting the transformed measurement matrix $\mathbf X(f_k)=[X_1 X_2 X_3]=\mathcal F\{[x_1 x_2 x_3]\}$ as follows:
\begin{align}
    \mathbf X^+(f_k)&=\mathbf {w^+}^H\mathbf X(f_k)\\
    \mathbf X^-(f_k)&=\mathbf {w^-}^H\mathbf X(f_k)
\end{align}

It can be shown that the corresponding weights for filtering the downstream and upstream signals are obtained for each frequency $f_k$ by:
\begin{align}
    \mathbf w^+ &=\mathbf g^+\left[ \Sigma^{-1}_x\; \mathbf A^H(\Theta)\left[\mathbf A^H(\Theta)\; \Sigma^{-1}_x\; \mathbf A^H(\Theta)\right]^{-1}\right]\\
    \mathbf w^- &=\mathbf g^-\left[ \Sigma^{-1}_x \mathbf A^H(\Theta)\left[\mathbf A^H(\Theta)\; \Sigma^{-1}_x\; \mathbf A^H(\Theta)\right]^{-1}\right]
\end{align}

Where $g^+=[1\;0]^T$ and $g^-=[0\;1]^T$ are the desired response vectors (unitary gain in one direction and zero gain in the opposite). $\Sigma_x$ denotes the covariance matrix. For a certain discrete frequency $f_k$, using the expected value $E$:
\begin{equation}
  \Sigma_x(f_k)=E[\mathbf X(f_k)\;\mathbf X^H(f_k)]
\end{equation}

The constraints matrix $\mathbf A(\Theta)=[\mathbf a^+(\theta)\; \mathbf a^-(\theta)]^T$ contains the beamformer response array (signal lags) in both directions:
\begin{align}
    \mathbf a^+(\theta) &= \mathbf a(-90^\circ)=\left[1,\exp(j\beta^+d_s),\exp(j\beta^+2d_s)\right]^T\\
    \mathbf a^-(\theta) &= \mathbf a(90^\circ)=\left[1,\exp(-j\beta^-d_s),\exp(-j\beta^-2d_s)\right]^T
\end{align}

Here $d_s$ is the distance between sensors and $\beta^\pm$ are complex wave numbers corrected for attenuation and mean flow:
\begin{equation}
    \beta^+=\frac{k+\alpha(1-j)}{1+M}\quad\text{and}\quad\beta^-=\frac{k+\alpha(1-j)}{1-M}
\end{equation}

$M$ represents the Mach number, $k=\omega/c$ the acoustic wave number and $\alpha$ the viscothermal attenuation coefficient, which can be computed as:
\begin{equation}
  \alpha=\frac{1}{r\,a}\left(\frac{\nu\,\omega}{2}\right)^{1/2}\left[1-(\gamma-1)P_r^{-0.5}\right]
\end{equation}

The radius of the duct is denoted by $r$, $a$ is the speed of sound, $\nu$ the cinematic viscosity, $\omega=2\pi f_k$ the angular frequency and $P_r$ is the Prandtl number.

%!TEX root = Tesis.tex

\chapter{Noise measurement results \& analysis} \label{cap:results}

% Chapter text
\noindent In this chapter, the results of applying the described acoustic measurement methodology to the \hl{test turbochargers} will be presented and analyzed. The measurements described in this chapter have been carried out in the turbocharger test rig installed in the anechoic chamber of the Institute.

A first section of the chapter will deal with the in-duct results, both regarding the full spectrum of the turbocharger noise output and the plane wave frequency range, which will be analyzed in detail with the help of the wave decomposition technique to calculate the sound intensity.

Externally radiated noise produced by the turbocharger group will be also discussed, the results comprising both inlet orifice noise and the radiation around the system generated by the turbocharger itself and its inlet and outlet ducts.

Finally, some correlations will be offered between the internal and external noise levels, that could be useful to estimate the external noise levels without the need to test the system in a large anechoic environment.

\section{Internal noise field}\label{sec:res_internal}

The first part of the experimental campaign consisted of gathering sufficient acoustic data across several operating conditions of the compressor, in order to compute not only the acoustic signature in the most typical points but also how this signature varied at different conditions.

Data for this in-duct noise analysis was recorded using the previously described piezoelectric sensor arrays in both inlet and outlet ducts. This data was then processed in the form of pressure and sound intensity and pressure spectra, noise maps, and spectrograms to show different aspects of the turbocharger acoustic behaviour and its evolution across different operating conditions.

These operating conditions at which measurements were performed were selected taking into account the full compressor map provided by the manufacturer and the approximate working line\footnote{Note that at shaft speeds grater than 190 krpm a second turbocharger is activated in this engine, creating a shift the working line noticeable between 2.5 and 3 pressure ratios.} of the engine operating points in real-life conditions and the capabilities of the compressor test bench. 

Regarding air mass flow, conditions were measured from approximately the maximum efficiency point to a limit slightly beyond the conservative deep surge line indicated by the manufacturer. Pressure ratios were measured up to the maximum capability of the available screw compressor to provide continuous flow.

\begin{figure}[tbh!]
\centering
\vspace{3mm}
\includegraphics[width=0.8\textwidth]{figures/cap_result/map_comp}
\caption{Comparison of the compressor map provided by the manufacturer and the critical operating points measured in the experimental campaign, including the deep surge limit measured in the gas stand and the approximate lug line of the target engine.}
\label{fig:result_map_comp}
\end{figure}

The two limits conform a measurement subset of the compressor map depicted in Fig. \ref{fig:result_map_comp} in color. This subset encompasses the working line up to a shaft speed of 170 krpm. In addition, the deep surge limit was measured in a separate gas stand to locate more accurately the true limit.

These surge measurements were performed with a tapered duct in the compressor inlet line to match the gas stand pipe diameter, which has an influence on the surge margin. Although, these tests confirmed that the manufacturer margin was close enough to the deep surge measured according to the methodology of low frequency content increase described in chapter \ref{cap:metod}. 

As it can be seen in Fig. \ref{fig:result_map_comp} for most of the measured shaft speed lines the last data point (the one with less air mass flow) was situated approximately in between the conservative limit established by the manufacturer and the limit experimentally measured in the gas stand. 

This allowed the comparison between optimum compressor operating points situated at near the maximum thermodynamic efficiency conditions and the most critical conditions near deep surge, passing through the unstable mild surge (or marginal surge) region between the two extremes.

\subsection{Spectral content}

Inlet and outlet pressure spectra were measured at each one of the selected points, allowing a comparison in terms of frequency content of the pressure signals and distribution of acoustic intensity across the measured compressor map.

In this subsection results will be divided between plane wave frequency range, where the wave decomposition algorithm based on beamforming can be applied to the array signals to compute the forward and backward waves, and the higher frequencies where acoustic modes are present and thus the decomposition cannot be performed.

To provide a better insight on how the spectral content of the pressure signals changes across the different operating conditions two subsets of the measured map have been highlighted.

The first of these subsets is a constant shaft speed line at 160 krpm where the air mass flow is progressively reduced from approximately highest efficiency. This will allow the analysis of how the spectral content changes with mass flow from relatively stable conditions to unstable ones.

A second subset forms a distribution of data points following the working line of the real engine which is depicted in Fig. \ref{fig:result_map_comp}. Since this line is situated at an  approximately constant distance from the surge limit it allows to track the evolution of spectral content given by the change in shaft speed while keeping similarly mild surge operating conditions. 

\subsubsection{Plane wave range}

Though dependent on pipe diameter, flow temperature and velocity, typical automotive turbocharger compressor setups can maintain plane wave conditions up to 5--7 kHz \cite{gaude2008experimental}. It is specially relevant because it comprises the frequencies of acoustic phenomena such as ``whoosh'' noise, tonal noise induced by BPF harmonics, deep surge, etc.

From the research point of view, this frequency range is interesting because considering the pressure change constant along each pipe section allows for easier analytical and numerical modelling. 

Experimentally, working in this range allows the use of wave decomposition procedures to isolate the contributions coming forward or backward through the compressor inlet and outlet pipes. These decomposition procedures can also be used together to compute acoustic intensity through the duct.

\subsubsection{Higher frequencies}

\subsubsection{Nondimensional analysis}

\subsection{Sound intensity evolution}


\section{External noise}
\label{sec:external_results}
\subsection{Orifice noise}
\subsection{Surface radiation}

\section{Correlations}
\label{sec:correlations}

%!TEX root = Tesis.tex
\chapter{On-engine validation\\ \textcolor{red}{[Article expMethod - rewrite]}}
\label{cap:engine}
% Quote
% \null\vspace{-10mm}
% \begin{flushright}
% \begin{minipage}[t]{0.66\textwidth}
% {\itshape\small``The story is set in the year 1922 and has its main character, Mr Engelbert, moving to Prague from the countryside to live out his retirement years and struggling to cope with the aggressive modernity of the city. The horror is not the power of money or the arrogance of the people, but the noise; not the age-old noise of a thunderstorm or a hammer, but the new noise of engines, especially of automobiles and motor-cycles, the explosive internal combustion monsters.''}\\[-3mm]

% {\cabincondensed\small\hfill \textbf{Milan Kundera}, \emph{The Curtain.}}\\[5mm]
% \end{minipage}
% \end{flushright}

% Chapter text
\noindent \section{Introduction}\label{sec:intro}
Turbomachinery-generated noise is nowadays one of the main concerns of automotive engineering, either aerial or terrestrial, as industry and society at large are today well aware of the many problems originated by acoustic pollution.

Once exclusive to jet aircraft and high performance engines, the use of turbomachinery is now widespread in internal combustion engines as a result of the current trend to downsize engines \cite{knecht2008diesel}, since otherwise it would not be possible to maintain the same power output while reducing the displacement. In order to achieve this, the compressor must provide higher pressure ratios over a wide range of air mass flow, consequently operating in zones close to its surge limit \cite{galindo2008experiments} and farther from its maximum efficiency peak. At the same time it must provide the required mass flow without overspeeding.

As a result of this demanding requirements, even if the control system may assure the overall safety of the system, the noise generated by the turbocharger is becoming a noticeable issue that also needs to be controlled \cite{gonzalez2003sound}, either actively or passively. In fact, engine noise not only has negative effects on the environment in terms of acoustic pollution, but can also cause discomfort to vehicle passengers, as the engine noise quality is affected \cite{brizon2012combining,nor2008index}.

For this reason, the acoustical characterization of the compressor across critical or demanding operation zones or even across its entire operation map becomes increasingly necessary, and several studies on compressor noise characterization, analysis and mitigation can be found in the recent literature \cite{evans2005minimizing,soh2003noise,teng2009investigation}.

In this paper, various procedures that enable the synthesis of this acoustic information in the form of sound maps that cover a selected zone of the compressor map for relevant frequency ranges are assessed. Such representations provide a simple and clear visualization of noise phenomena for the possible operation points of the turbocharger in that zone.

An important consideration is that robust methodologies are needed to perform this acoustic characterization in the environment of production engine test cells, which are usually neither anechoic nor designed for acoustic measurements.

\section{Experimental characterization}
\label{sec:experimental_characterization}

In order to obtain a noise map that characterizes noise generation by the compressor it is necessary to know the sound spectrum at several operating points and then extract the level of the different frequency bands of interest.

Reviewing the available literature on noise characterization of rotating machines, two different approaches are found. One of them is based on measuring radiated noise by means of exterior microphones inside an anechoic environment as done, for instance, by Cudina \cite{cudina2000noise}. In engine conditions, however, it is difficult to separate the radiation of the compressor from the engine noise.

The other approach focuses on in-duct noise \cite{tiikoja2011inves}, using fast pressure sensors mounted flush inside the duct, in order to better identify the different internal flow phenomena involved in the generation of compressor noise. As this last approach can be more reliably applied to a non-anechoic engine test, this investigation is focused on it.

\subsection{In-duct noise measurements}
\label{sub:in_duct_noise_measurements}

Some authors \cite{figurella2012noise} rely on a single sensor to estimate the in-duct signal spectrum, whereas others use more sophisticated approaches introducing more sensors \cite{tiikoja2011inves}. While the one-sensor method allows the calculation of sound pressure level (SPL), a scalar magnitude which only gives information of the sum of all waves at a particular point, sound wave decomposition enables the computation of the sound intensity level (SIL), a vector magnitude which offers information on the sound coming from a particular direction.

The basis of decomposition methods is to consider that the pressure signal $p$ measured at a given point of a duct is the linear superposition of a pressure wave traveling forward in the flow direction, $p^+$, and another pressure wave traveling backwards, $p^-$.
According to Morfey \cite{morfey1971sound}, the total intensity can be regarded as the sum of the upstream and downstream pressure spectra ($\hat p^+$ and $\hat p^-$) corrected with the mean Mach number ($M_0$) and characteristic acoustic impedance ($\rho_0 a_0$):

\begin{equation}\label{eq:intensity}
	I=I^+-I^-=\frac{1}{\rho_0 a_0}\left(\left|\hat p^+\right|^2\left(1+M_0\right)^2-\left|\hat p^-\right|^2\left(1-M_0\right)^2\right)
\end{equation}

In order to use this expression, it is thus necessary to select a measurement system that allows for wave decomposition. One of the standards is the two-microphone method used for example by Holland and Davies \cite{holland2000measurement} and Tiikoja et al. \cite{tiikoja2011inves}, which uses the correlated spectra measured by two consecutive wall-flush mounted microphones to estimate the reflection coefficient and thus resolve both waves.

However, the two-microphone method is not free of drawbacks. The most important are the negative influence of interfering signals, which is not considered with only two microphones \cite{chung1980transfer}, some viscothermal wave attenuation effects \cite{aabom1988error} and incoherence of the forward and backward signals, the last two being common in waves propagating in flows \cite{dokumaci1997note,holland2000measurement}.

Some of these problems can be addressed using another method denominated beamforming,  based in phased sensor arrays, which has been successfully used for the study of acoustic and pressure fields in ducts with flow \cite{benajes2001estimation,desantes2001experiments,torregrosa2005note}.

The beamforming method can be applied to arrays of two or more microphones, weighting the signal of each one to ``steer'' the sensitivity of the array in the direction of interest (in this case, upstream and downstream of the pipe). A Linearly Constrained Minimum Covariance algorithm was applied in this case to find such weights (see \cite{pinero2000estimation} for details).

The experimental installation includes thus an array of piezoelectric pressure sensors flush-mounted in an extension of the compressor inlet pipe. These sensors were used to test the beam-forming method, the two-microphone method, and the sound pressure level (SPL) of a single sensor.
A commercial intensity probe (B\&K Type 4197) consisting of two phase-matched \nicefrac{1}{2}'' microphones was also installed at the inlet mouth as a reference, surrounded by acoustic insulation mats in order to avoid reflections from the walls or other components.  Fig. \ref{fig:setup} shows this experimental setup.

\begin{figure*}[t!]
\centering
\includegraphics[width=1\textwidth]{figures/cap_motor/setup}
\caption{Experimental setup for the acoustic measurements.}
\label{fig:setup}
\end{figure*}

Before applying the selected methods to the whole chart of the compressor, it is necessary to evaluate the restrictions that they impose in the frequency domain. While the SPL is only limited by the frequency response of the particular sensor, the decomposition algorithms of both the two-microphone method and the beamforming method present some common additional restrictions.

The first one is that both methods are formulated considering plane wave propagation, and are thus theoretically restricted to the frequency range below the onset of higher order modes. According to Eriksson \cite{eriksson1980higher}, the cut-off frequencies for the first asymmetric and first radial modes can be approximated by

\begin{equation}\label{eq:modes}
	f_\text{asim}=1.84\frac{a_0}{\pi D}\sqrt{1-M_0^2}\qquad
	f_\text{radial}=3.83\frac{a_0}{\pi D}\sqrt{1-M_0^2}
\end{equation}

being $D$ the pipe diameter. Also, around 10 diameters of straight pipe should be included at both sides of the array, and the nodes of the standing waves corresponding to \nicefrac{1}{3} and \nicefrac{1}{5} of the pipe length must be avoided.

\begin{figure*}[tb!]
\centering
\includegraphics[width=0.95\textwidth]{figures/cap_motor/espectros}
\caption{Inlet in-duct sound spectra estimated by the three considered methods (SPL, beamforming and two-microphone) and orifice sound intensity spectrum as measured by the B\&K intensity probe.}
\label{fig:espectros}
\end{figure*}

Another restriction is related to the spatial resolution of the consecutive sensors \cite{seybert1988two}, which imposes a cut-off frequency similar in concept to a Nyquist limit for spatial (rather than temporal) sampling, based on the separation ($d_s$) of the sensors. Above this frequency limit, erroneous spatial aliasing effects will be introduced. A simple approximation for this criterion is:

\begin{equation}\label{eq:alias}
	f_\text{asim}=\frac{a_0}{2 d_s}
\end{equation}

In the low frequency range, it is necessary to consider that at sufficiently large wavelengths, the signal differences between consecutive sensors decrease below their sensitivity limit, invalidating the results. With the separator used between the microphones of the intensity probe this limit is at 250 Hz.

Fig. \ref{fig:espectros} shows the in-duct sound spectra computed with the three proposed methods: SPL with one sensor, two-microphone approach, and beamforming with three probes. The spectra measured at the orifice by the reference probe is superimposed in black. In this region, the beamforming method appears to follow the reference more closely.

It is also shown how spurious aliasing peaks occur at the frequency predicted by Eq. \ref{eq:alias} when more than one sensor is used and how the onset of the 1st radial mode predicted by Eq. \ref{eq:modes} marks a clear difference between the in-duct sound field and the sound field radiated by the inlet orifice.

\begin{figure*}[b!]
\centering
\includegraphics[width=0.95\textwidth]{figures/cap_motor/mapasSIL_250-2650}
\caption{In duct SIL maps of the compressor noise estimated by the beamforming method (top left), twomicrophone method (top right) and SIL map of orifice noise as measured by the probe (top center), including correlations between the three measurements (bottom).}
\label{fig:mapasSIL_250-2650}
\end{figure*}

The results of applying Eq. \ref{eq:intensity} to compute SIL in the plane wave region of the spectra at different operating conditions of the compressor are shown in Fig. \ref{fig:mapasSIL_250-2650}. Absolute levels at each working point were interpolated in order to get a continuous map. The beamforming and the two-microphone methods were used for in-duct SIL estimation and the intensity probe for measuring the reference orifice SIL. Correlations of these measurements are also provided, showing that in the plane wave range the beamforming method appears to perform up to 30\% better in terms of $R^2$.

Fig. \ref{fig:mapasSPL_250-2650} shows a comparison between the SPL computed by selecting only one sensor of the array and the reference SIL of the probe. The correlation of these two measurements is better in terms of $R^2$ than the two-microphone method but still worse than the beamforming method.

\begin{figure*}[tb!]
\centering
\includegraphics[width=0.95\textwidth]{figures/cap_motor/mapasSPL_250-2650}
\caption{Comparison of compressor sound maps (plane wave region) obtained through interpolation of SPL at the inlet 1st sensor and SIL at the probe, including a correlation between both.}
\label{fig:mapasSPL_250-2650}
\end{figure*}

However, it should be noted that simple sound pressure level measurements can be influenced by the complex geometry of actual production pipes, which are remarkably more complex than the straight pipe used in this study. Also, it can be seen in Fig. \ref{fig:mapasSPL_250-2650} that, while at low pressure the sound maps are quite similar, SPL of one probe predicts a high sound level at high pressure ratio and high mass flow conditions, which is not the case for the intensity probe.

These high sound level points at high pressure ratio can be clearly seen as points lying out-side the 95\% observation confidence bounds of the correlation in Fig. \ref{fig:mapasSPL_250-2650}. A similar problem can be seen in the correlation of the two-microphone method in Fig. \ref{fig:mapasSIL_250-2650}, in which some outliers appear to dis-tort the map at some operation zones. The beamforming method, however, appears to be more robust.

\subsection{Radiated noise measurement}
\label{sub:radiated_noise_measurement}

Besides in-duct measurements, another acoustic characterization approach consists in the measurement of noise radiated by the turbocharger and its pipes, in most cases using free field microphones at a given distance from the setup.

This method has the advantage of allowing measurements with the real geometry, without any modification in the pipes to mount the in-duct measurement sensors. However, for accurate measurements it is desirable that the test cell be anechoic in order to prevent disturbances in measurements caused by reflections in the engine cell walls and resonances created by the cavities of the engine or its ducts that appear as other noise sources.

As testing the whole engine setup requires a large and expensive anechoic chamber, the setup is sometimes reduced to just the turbocharger group, which is powered by an external compressor or electric motor, losing in this way the possibility of performing on-engine tests.

However, a novel procedure based on measuring acoustic particle velocity level (PVL) may solve these issues \cite{jacobsen2005comparison} and enable on-engine testing in non-anechoic environments. This method makes use of two microscopic filaments acting as hot-wire flowmeters. Measuring the small voltage drop between the two, it is possible to derivate the acoustic particle velocity in the plane of the sensor.

Such a method, in this case a commercial solution providing acoustic holography-like maps of radiated noise level \cite{comesana2013scan}, has been tested on the (non-anechoic) engine test cell. The measuring probe of this device houses the two filaments on a MEMS chip and a traditional microphone, allowing the comparison of the performance of acoustic particle velocity measurements against traditional free-field sound pressure level (SPL) measurements.

\begin{figure*}[tb!]
\centering
\includegraphics[width=0.95\textwidth]{figures/cap_motor/PVL}
\caption{Near field noise holograms (800--2000 Hz) obtained through SPL (top) and PVL (bottom), including detail of the spectra of two regions of interest: turbocharger group (blue) and compressor inlet (red).}
\label{fig:PVL}
\end{figure*}

Fig. \ref{fig:PVL} shows this comparison. SPL (top) and PVL (bottom) maps of the measured engine environment are provided by tracking the probe along the near field of the surfaces of interest. The frequencies considered in these maps have been restricted to the 800--2000 Hz range, where it is common that a broadband emission known as ``whoosh noise'' is produced by the turbocharger \cite{evans2005minimizing}. For two particular regions (turbocharger group and inlet duct, marked on the maps) the full pressure (top) and particle velocity (bottom) spectra are also provided.

The particle velocity technique clearly distinguishes the turbocharger as the source of an 800--2000 Hz broadband noise at this particular operating condition, identified as whoosh noise. The pressure method also appears to identify the turbocharger as the source, but the contrast with the surrounding areas is much less pronounced. This can be clearly seen in the SPL spectra, in which the level of this frequency band is similar in both zones (turbocharger and inlet in this case). In the case of the PVL spectra the turbocharger contribution is visibly higher.

However, the pressure method appears to be more sensitive to reflections in the near field and was thus unable to provide a clear identification. It should be noted that if characterization through in-duct techniques of this particular whoosh noise is intended, special attention should be placed on the outlet pipe, as this phenomenon is often found exclusively downstream of the turbo-charger \cite{broatch2014methodology}.

\section{Conclusions}
\label{sec:conclusions}

Characterization of the acoustic performance of a given compressor under engine conditions must start with the definition of test means. If an anechoic chamber capable of housing the entire engine is available, then free field microphones can be used to estimate the acoustic radiation, but even in this case it would be difficult to isolate the contribution of the compressor.

In order to quantify the compressor noise emission, an intensity probe can be located at the inlet orifice. However, intensity probes are quite expensive, and sufficient acoustic insulation is necessary to prevent reflections. Also, characteristics of the internal field that are not externally propagated will be lost.

In-duct noise measurements offer a cost-effective way to characterize the sound field generated by the compressor, if the ducts can be modified so as to include straight measurement sections. Single-sensor SPL may be used to investigate the upper frequency bands, but generally sound intensity is a better and more meaningful metric of the acoustic power output of the source.

Calculation of in-duct sound intensity can be achieved by using an additional consecutive sensor and decomposing the waves through the two-microphone method. However, the addition of a third sensor and the decomposition through the beamforming method appears to be more reliable along the full plane wave range, and especially in the lower frequencies (250--1000 Hz).

However, the restrictions of these intensity-based methods must be carefully considered, specially the influence of higher order modes and the aliasing frequency limit introduced by the spatial sampling imposed by the sensor spacing.

Regarding the external radiated noise, the particle velocity method has been able to provide more contrast than the pressure method and to correctly identify the turbocharger as the source of noise radiation at that particular ``whoosh noise'' band, in accordance with previous experience.

The particle velocity method could therefore be useful to reconstruct noise maps similar to those obtained using in-duct measurements, isolating the signal from the acoustic near field of the area of interest at different working conditions, without the need of modifying the original duct geometry or ensuring anechoic conditions. More research is being carried out on this topic.


%!TEX root = Tesis.tex
\chapter{Comparison with simulations}cking 
\label{cap:cfd}

\noindent

\section{CFD simulation setup} %..... Breve descripción setup

\section{Validation methodology} %.... Artículo metodología (res.)
\section{Selected results}
\section{Fluid field insights} %.... Artículos CFD I y II

%!TEX root = Tesis.tex
\chapter{Influence of the inlet flow field}
\label{cap:inlet}

\section{Local measurement methodology}

As seen on section \ref{sec:liter_influence} of the literature review, conditions in the compressor inlet immediately upstream the wheel seem to be highly relevant to the development of surge and broadband noise emission. 

An effort was made during the experimental campaign to allow the measurement of thermodynamic conditions in this region: temperature, pressure and velocity. \hl{Comparisons with CFD simulations provided not only validation of the numerical prediction but also further insight as to how the precise geometry of the inlet influenced these conditions.}

\begin{figure}[htb!]
\centering
\includegraphics[width=0.65\textwidth]{figures/cap_inlet/locales_sin_brida}
\caption{CAD view of the proposed local temperature and pressure measurement instrumentation, showing the distribution of thermocouple arrays and pressure probes.}
\label{fig:metod_local_meas}
\end{figure}

% \begin{wrapfigure}[18]{O}[1.6cm]{0.45\textwidth}
% \vspace{-4mm}
% \includegraphics[width=0.45\textwidth]{figures/cap_inlet/medidas_locales_corte}
% \caption{Annotated section view of the local pressure and temperature instrumentation.}
% \label{fig:metod_local_meas_dimen}
% \end{wrapfigure}

\begin{figure}[b!]
\centering
\includegraphics[width=0.5\textwidth]{figures/cap_inlet/medidas_locales_corte}
\caption{Annotated section view of the pressure probes (purple) and thermocouples (green) location.}
\label{fig:metod_local_meas_dimen}
\end{figure}

\subsection{Temperature}

It is well established on the literature that reversing backflows are one of the most telling phenomena of compressor instability and stall inception. As these backflows are coming from downstream the wheel they are thus at higher temperature, due to the non-isentropic compression process. Consequently, temperature measurement provides a method for characterizing the inception of these backflows and their upstream extent.

Two thermocouple arrays were thus installed on the compressor inlet. A circular one covers a section of the pipe just 14 mm from the compressor wheel plane, providing temperature measurements each 45º. The second array is linear, extending in the longitudinal direction of the pipe up to approximately 2 inlet diameters.

The first one is intended to tell if angular variations of temperature distribution exist, and the second one aims to characterize the length of the recirculating backflows at each operating condition of the compressor.

\subsection{Pressure}

While pressure measurements are carried out in both inlet and outlet ducts of the compressor, a further effort has been made to characterize the spectral signature of the pressure in two critical locations: on the inducer, upstream the compressor wheel and on the diffuser, downstream of the wheel. Miniaturized piezoelectric pressure probes from Kistler were used in both locations.

\subsection{Velocity}

\begin{figure}[b!]
\centering
\includegraphics[width=0.65\textwidth]{figures/cap_inlet/Assembly_PIV_straight_3D}
\caption{CAD view of the straight transparent duct for PIV measurements with its mounting assembly.}
\label{fig:metod_local_meas_dimen}
\end{figure}

Besides punctual measurements of pressure and temperature, the characterization of the whole flow field is also crucial. Not only for understanding how the backflows interact with the downstream flow, but also to validate CFD predictions of macroscopic flow structures.

In order to allow a characterization of these flow field, a Particle Image Velocimetry (PIV) system was installed on the anechoic chamber. This method uses a laser sheet which illuminates seeding particles introduced in the air stream. Capturing snapshots of the distribution of these particles with a very fast camera allows, through a correlation postprocessing technique, to compute how the particles have moved between snapshots and thus, the 2D velocity field in the plane illuminated by the laser sheet.

Specifically in this study a 135 mJ Nd:YAG laser was used to illuminate the desired planes, and the flow was seeded with $\sim 1\mathrm{\mu m}$ oil droplets from a custom-made generator. Seeding control was proven to be critical, especially in the case of the longitudinal measurements that were made through the walls of the glass pipe.

Even during measurements of the longitudinal plane at stable conditions (high mass flow) where the stalled backflow does not reach upstream the leading edge of the blades, oil accumulated in the inner wall of the glass pipe, obstructing the view and introducing reflections that rendered the measurement impossible through the soiled parts.

The issue was aggravated when the backflow generated by the partially stalled compressor blades reached the glass pipe. As there is a zone where the axial velocity is null, oil particles become trapped and start to accumulate, forming a clearly visible fluid ``ring'' that blocks the view. Also, the high transversal velocity forces the particles to impinge onto the inner wall, forming visible helicoidal tracks along the surface.

As a consequence, a test procedure was followed where the turbo was settled at the desired operating conditions, and then seeding particles were introduced during a very brief time. This allowed approximately 10 usable pairs of snapshots, after which the turbo was stopped and the glass pipe cleaned for the next run.

Measurement of the transversal planes was done pointing the camera directly towards the compressor, thereby avoiding the need to record through the glass. This allowed for longer test runs because the soiling of the wall did not affect the measurement. 

However, measuring the transversal planes was not without issues. In this configuration, the rotor blades are illuminated by the scattered light, preventing the seeding particles from contrasting enough to be clearly correlated in some places, and generating bright spots that are misidentified as particles moving.

In order to mitigate this problem, a system was devised to feed the pulse train generated by the Hall effect RPM sensor into the laser and camera trigger controller, so that each pair of snapshots were taken at almost the same blade position.

This enabled a postprocessing technique where the average intensity of several non-seeded snapshots was subtracted from each seeded image, greatly reducing the brightness of the illuminated rotor and providing a better particle correlation. In the case of the longitudinal measurements, this was not deemed necessary since the back of the glass was painted matte and thus provided a good background.

While this resulted in overall enhanced quality of the reconstructed field, small reflections such as that of a small notch at the axis nut could not be removed. Also, in some occasions small glitches in the operation of the sensor resulted in pairs where the two laser pulses were included in the first frame, thereby rendering the pair useless. Fortunately, those cases were easy to identify and remove.

Since apart from the influence of the background the transverse velocity flow was much more chaotic than the longitudinal flow field, the quality of each individual snapshot pair correlation was worse than those of the longitudinal case. However, as the oil soiling of the wall was not as problematic, a greater number of valid pairs could be captured so that the averaged velocity field could be reliable.

\section{Local measurement results}

For this investigation, a production automotive turbocharger was chosen. However, prior to its installation in the gas stand, modifications were made to accommodate the required local temperature and pressure sensors and to allow for the installation of transparent ducting for velocity field measurement at a later stage.

The compressor housing was modified by removing the original inducer up to the plane of the blade leading edges and substituting it by a removable straight adaptor piece which incorporated guides for a circumferential array of thermocouples and a miniaturized pressure probe. This adaptor is depicted in red in Figs. \ref{fig:metod_local_meas} and \ref{fig:metod_local_meas_dimen}.

A detachable straight pipe section with a linear thermocouple guide was affixed to this adaptor. This configuration offers the possibility of easily substituting the straight section for others with different geometries such as elbows or tapered ducts.

\hl{For the characterization of the velocity field, both the adaptor and the straight pipe segment were replaced by a transparent straight duct that allowed the use of a Particle Image Velocimetry (PIV) measurement system.}

\begin{figure}[htb!]
\centering
\includegraphics[width=0.60\textwidth]{figures/cap_inlet/locales_sin_brida}
\caption{CAD view of the proposed local temperature and pressure measurement instrumentation, showing the distribution of thermocouple arrays and pressure probes.}
\label{fig:metod_local_meas}
\end{figure}

\begin{figure}[b!]
\centering
\includegraphics[width=0.40\textwidth]{figures/cap_inlet/medidas_locales_corte.PDF}
\raisebox{6mm}{\includegraphics[width=0.55\textwidth]{figures/cap_inlet/front_ann.PDF}}
\caption{Annotated section view of the pressure probes (purple) and thermocouples (green) location.}
\label{fig:metod_local_meas_dimen}
\end{figure}

\subsection{Temperature}
\label{sub:temperature}

\subsubsection{Radial profile}
Before the measurement of the whole map, a sweep of temperature measurements along the inlet radius was performed for both the thermocouple located at 14mm and the thermocouple located at 84mm. For these measurements, the turbocharger compressor was brought to conditions close to deep surge at 80 krpm.

The results can be seen in Fig. \ref{fig:barrido_temp}. As expected, overall temperature profile is higher at 14mm from the leading edge plane than at 84mm. Temperature seems to be approximately constant up to a distance of 5mm from the internal wall, then decaying up to a distance of approximately 10mm from the centreline (which is at 21mm from the wall). From there onwards, temperature is again approximately constant.

\begin{figure}[tb!]
\centering
\includegraphics[width=0.70\textwidth]{figures/cap_inlet/barridoTemp_isol}
\caption{Evolution of temperature measured by the first and last thermocouples of the linear array for different distances from the wall.}
\label{fig:barrido_temp}
\end{figure}

Ambient temperature and surface temperature of the adaptor piece (seen if Figs. \ref{fig:metod_local_meas} and \ref{fig:metod_local_meas_dimen} in red) can also be seen in Fig. \ref{fig:barrido_temp}. As expected, centreline temperature at the most upstream point is almost coincident with the ambient, while fluid temperatures near the wall at the most downstream point (at just 14mm from the wheel) are almost equal to those of the piece (the small initial difference is due to the cold start and thermal resistance).

\begin{figure}[htb!]%[34]{e}[1.25cm]{0.55\textwidth}
\centering
\includegraphics[width=0.48\textwidth]{figures/cap_inlet/temperatures_rise_line-80.pdf}
\includegraphics[width=0.48\textwidth]{figures/cap_inlet/temperatures_rise_line-100.pdf}
\includegraphics[width=0.48\textwidth]{figures/cap_inlet/temperatures_rise_line-120.pdf}
\includegraphics[width=0.48\textwidth]{figures/cap_inlet/temperatures_rise_line-140.pdf}
\includegraphics[width=0.48\textwidth]{figures/cap_inlet/temperatures_rise_line-160.pdf}
\caption{Evolution of temperature rise measured by the linear array thermocouples for different mass flow settings and different shaft speeds, taking the most upstream temperature as reference.}
\label{fig:temp_line_linear}
\end{figure}

From this results it is apparent that measuring the fluid temperature near the wall is not practical due to the thermal transfer from the wall to the thermocouple. Standardized procedures for characterization of flow temperature call for measurements at R/2 and R/3. In this case, this would correspond to distances of 7mm and 10.5mm from the wall. It can be seen in Fig. \ref{fig:barrido_temp} that those distances show indeed a depart from wall-influenced temperatures. Ultimately, the distance of R/2 (10.5mm) was chosen for this study.

\subsubsection{Backflow growth}

By analysing the evolution of temperature results over different operating conditions (speed and air mass flow) it is possible to estimate the growth of the recirculating backflow.

Figure \ref{fig:temp_line_linear} shows different temperature profiles measured by the linear thermocouple array at a same shaft speed and different mass flows. It can be seen how at higher mass flows the temperature along the inlet pipe is approximately equal to that of the ambient.

When mass flows is reduced a temperature increase coming from the compressor wheel starts to be noticeable, denoting the presence of air heated by the compression process which is reversing in direction due to flow instabilities on the wheel channels. The operating condition of backflow onset appears to correspond to that of maximum pressure ratio at each shaft speed.

Temperature increase grows both in difference from the ambient and distance from the wheel. However, this growth appears to be highly dependent of the particular shaft speed. At lower shaft speeds, growth reaches greater distances from the wheel at lower air flows, but maximum temperature difference is lower. Conversely, at higher shaft speed temperature difference is greater (due to higher pressure ratio) but backflow extent along the inlet pipe is more constrained.

The reason for this constriction of backflow length at higher shaft speed could be probably attributed to \hl{higher inlet flow momentum} at those operating conditions. Reduced momentum at lower shaft speeds would instead allow the compressed hot backflow to reach farthest into the inlet line when backflow occurs. 

\hl{In order to ensure that the temperature increase along the inlet pipe measured through the thermocouples is indeed caused by compressor backflow and not by thermal diffusion from the compressor housing through the metallic pipe, thermographic images were captured.

These were synchronized with the operating conditions at which flow temperatures were measured inside the duct. Inlet pipe radiated wall temperature profiles were extracted to compare their growth against those of the internal flow. In this case, profiles appear continuous and coherent with heat diffusion. No onset point of temperature difference with air mass flow could be identified, contrary to the case of internal flow measurement.}

\subsubsection{Circumferential profile}

\begin{figure*}[htb!]
\centering
\includegraphics[width=0.25\textwidth]{figures/cap_inlet/temperatures_circ_line-80-100216.pdf}
\includegraphics[width=0.25\textwidth]{figures/cap_inlet/temperatures_circ_line-100-100216.pdf}
\includegraphics[width=0.25\textwidth]{figures/cap_inlet/temperatures_circ_line-120-100216.pdf}
\includegraphics[width=0.25\textwidth]{figures/cap_inlet/temperatures_circ_line-140-100216.pdf}
\includegraphics[width=0.25\textwidth]{figures/cap_inlet/temperatures_circ_line-160-100216.pdf}
\caption{Evolution of temperature measured by the circumferential array thermocouples as air mass flow was reduced keeping shaft speed constant. A polar temperature plot is shown for each speed, colors indicating mass flow from higher (blues) to lower (reds). Legends for colors at each speed are available in Fig. \ref{fig:temp_line_linear}. See Fig. \ref{fig:metod_local_meas_dimen} for reference position of the compressor volute.}
\label{fig:temp_circ_all}
\end{figure*}

Temperature measurements were taken using the circumferential thermocouple array situated at 14mm from the compressor leading edge. The objective in this case was to characterize the angular distribution of temperature at different operating conditions.

An example of these circumferential profiles can be seen in Fig. \ref{fig:temp_circ_all}, where temperature readings at different mass flow conditions are plotted, for the different shaft speeds considered in the study. 

\begin{figure*}[htb!]
\centering
\includegraphics[width=0.98\textwidth]{figures/cap_inlet/temp_maps.pdf}
\caption{\hl{Evolution of temperature rise measured by the linear array thermocouples for different mass flow settings at 140 krpm, taking the most upstream temperature as reference.}}
\label{fig:temp_maps}
\end{figure*}

It can be seen that the skewness of the circumferential profile grows with the reduction of mass flow. At higher flow, skewness is practically zero, the profile being completely regular. However, again at the point of higher pressure ratio, some skewness starts to appear along with an increase in temperature due to hot backflow.

As air mass flow is further decreased towards compressor surge limit, both overall temperature and profile skewness increased. \hl{The pattern of the skewness appeared to be similar at all conditions: temperature was lower at angles corresponding to the start of the volute and higher at angles corresponding to the end of the volute, where its cross-section is larger.}

\subsection{Pressure}
\label{sub:pressure}

\subsubsection{Local pressure}

As shown in Figs. \ref{fig:metod_local_meas} and \ref{fig:metod_local_meas_dimen}, miniature pressure probes were fitted in both the inducer and the diffuser of the compressor.

A simple metric of sound level, used by most authors, is Sound Pressure Level (SPL). It is a scalar magnitude, and can be calculated from the pressure signal as:
\begin{equation}
  \text{SPL}_\text{dB} = 20\log_{10}\left(\frac{\bar P}{p_0}\right)
\end{equation}

Where $\bar P$ is the RMS mean of the pressure wave amplitude, restricted at the desired frequency range, and $p_0$ is a reference pressure value of $2\cdot 10^{-5}$ Pa.

\subsection{Velocity}
\label{sub:velocity}

\subsubsection{Longitudinal plane}

\begin{figure*}[htb!]
\centering
\includegraphics[width=1.05\textwidth]{figures/cap_inlet/longitudinal}
\caption{Results of the axial velocity field measurement (longitudinal plane) for the two operating points considered: low}
\label{fig:longitudinal}
\end{figure*}

As described in the previous section, longitudinal plane measurements were the most difficult since the seeding oil particles blocked the line of sight of the camera. However with the right combination of seeding flow, oil dilution and quick operation of the system, valid snapshot pairs were obtained.

In order to capture the difference between stable and partially stalled conditions (also called mild surge or marginal surge conditions) two back-pressure valve positions were tested, while maintaining the same shaft speed of 140 krpm.

PIV postprocessing was carried out for 10 snapshot pairs for each conditions. While the lower third of the frames were obscured by a reflection, at least 2/3 of the longitudinal plane could be resolved in sufficient quality. The two resulting averaged velocity fields can be seen in Fig. \ref{fig:longitudinal}.

In these test there was no direct measurement of inlet air mass flow, so an average of the measured velocity field is taken to identify the precise operating point of the compressor.

At the higher mass flow, it can be seen that the velocity field is approximately homogeneous along the duct and only near the wall the velocity diminishes. However, when flow is reduced, the spatial distribution of velocity changes. The expected compressor backflow is apparent as the low velocity region near the wall increases in thickness, reducing the useful diameter from $\sim 19\,$mm to $\sim 13\,$mm.

This reduction in diameter extends to approximately $\sim 55\,$mm, this is, approximately half of the transparent glass. It can also be seen in Fig. \ref{fig:longitudinal} how this reduction in useful diameter appears to be paired with an increase in flow velocity. 

To better quantify this increase in speed, velocity data along the turbocharger axis line (marked with dotted line in Fig. \ref{fig:longitudinal}) from each valid snapshot has been plotted in Fig. \ref{fig:line_comp_long} for both operating conditions. 

While at the high flow operating point the flow exhibits an approximately constant velocity along the line, only decreasing adjacent to the axis nut, at the lower flow condition it can be seen how after a steady initial segment the velocity increases to a top average speed of 91.73 ms${}^{-1}$. The average speed at the initial segment between 80 and 90 mm was found to be 43.68 ms${}^{-1}$.

Using this data it is possible to draw a simple correlation between the area constriction and the increase of speed. Using the mass conservation equation between the initial segment (1) and the top speed point (2):

\begin{equation}
  \dot{m}_1 = \dot{m_2} = \rho_1 V_1 \pi r_1^2 = \rho_2 V_2 \pi r_2^2
\end{equation}

If a small enough increase of temperature between point (1) and point (2) is assumed so that $\rho_1\sim\rho_2$ the expression can be rearranged to obtain the relationship between radius and velocity changes:

\begin{equation}
  \frac{V_1}{V_2} = \frac{r_2^2}{r_1^2}
\end{equation}

If now the aforementioned approximate values for radii and velocities are substituted into the equation, we can obtain an estimation of the relative error $\varepsilon$ between the theoretical value and the experimental observation:

\begin{equation}
  \varepsilon\, (\%) = \left|\frac{V_1/V_2-r_2^2/r_1^2}{r_2^2/r_1^2}\right| \cdot 100 = 1.71\%
\end{equation}
% \left|\frac{43.68/91.73-13^2/19^2}{13^2/19^2}\right| \cdot 100 = 

The small relative difference between the prediction and the observation reinforces the assumption that the measurement technique is in fact able to characterize the reduction in useful area associated with the backflow ans its associated increase in velocity, even if the axial speed of backflow itself cannot be quantitatively measured.

\begin{figure}[htb!]
\centering
\includegraphics[width=0.65\textwidth]{figures/cap_inlet/line_comparison_longCent_ann}
\caption{Velocity magnitude over the centreline of the longitudinal plane at high and low flow. Solid line indicates the mean and points correspond to individual snapshot pairs.}
\label{fig:line_comp_long}
\end{figure}

\subsubsection{Transversal plane}

\begin{figure*}[htb!]
\centering
\includegraphics[width=1.05\textwidth]{figures/cap_inlet/slices_2}
\caption{CAD view of the straight transparent duct for PIV measurements with its mounting assembly.}
\label{fig:res_trans_20mm}
\end{figure*}

\begin{figure*}[b!]
\centering
\includegraphics[width=1.05\textwidth]{figures/cap_inlet/slices_3}
\caption{Velocity magnitude averaged over the all the captured frame pairs for three different transversal planes.}
\label{fig:res_trans_20mm}
\end{figure*}

\begin{figure}[b!]
\centering
\includegraphics[width=0.65\textwidth]{figures/cap_inlet/line_comparison_3slices}
\caption{Velocity magnitude over the centreline of three transversal planes. Solid line indicates the mean and points correspond to individual snapshot pairs.}
\label{fig:line_comp_3slices}
\end{figure}

\subsection{Correlations}

\subsubsection{Backflow and sound intensity}

\begin{figure}[htb!]
\centering
\includegraphics[width=0.6\textwidth]{figures/cap_inlet/corr_delta_int.pdf}
\caption{Correlation between standard deviation $\sigma$ at the circumferential thermocouple array and inlet sound intensity level in the 1--3 kHz band. Size indicates shaft speed from low to high and color indicates mass flow from higher (blues) to lower (reds).}
\label{fig:corr1}
\end{figure}

In order to attest for a linkage between inlet flow instabilities at marginal surge conditions and the broadband noise at subsynchronous speeds identified by several authors, it is possible to correlate the aforementioned thermal phenomena with sound level in that range.

Figure \ref{fig:corr1} shows a plot of \hl{inlet sound intensity level against standard deviation of temperature in the circumferential thermocouple array}. It can be seen how for lower \hl{temperature deviation (near 0.5$^{\circ}$C, which is the expected deviation of the thermocouples)} sound intensity appears to be uncorrelated, varying mainly due to higher shaft speeds. After \hl{temperature deviation reaches approximately 1$^{\circ}$C} however, a linear trend emerges. 

A least squares linear fitting was performed, showing that there is a good correlation between increase in sound intensity at the selected frequency band and increase in the skewness of the inlet circumferential temperature distribution, thereby suggsting a link between the two phenomena.

\subsubsection{Velocity and temperature profiles}

\section{Variations of inlet geometry}
\label{sec:inlet_geo_var}

\subsection{Simple geometries}
\subsection{Engine test cell}
\subsection{Elbow parametric study}


%!TEX root = Tesis.tex
\chapter{Influence of the inlet geometry} 
\label{cap:geom}

As discussed in the literature review performed in chapter \ref{cap:liter}, different geometric variations of the inlet line near the impeller have been proposed and tested in order to mitigate the negative effect of the flow unstabilities.

While mainly aimed at obtaining a more stable flow to delay deep surge onset to enhance the usable zone of the map and keep compressor thermodynamic efficiency even at low air mass flow operating conditions, the impact of these solutions in the generated and transmitted noise must be also considered.

In this chapter different experimental measurement campaigns are described, where some of the geometries most commonly proposed in the literature were tested both in the continuous flow test rig and in real engine conditions. 

\section{Simple geometries}
\label{sec:geom_simple}

The first of these experimental campaigns was carried by manufacturing a sample of each one of the simplest proposed geometries: a 90º sharp elbow, a tapered duct, and a reservoir. In addition, a short straight pipe was also used as a reference. All of these four geometries can be seen in Fig. \ref{fig:image_simple_geom}.

These simple geometries were installed on the anechoic chamber flow test rig described in chapter \ref{cap:metod}. Both the orifice noise measurement free-field microphone and the radiated noise setup with four equally distributed free-field microphones were kept in order to evaluate the acoustic emission in this preliminary investigation.

In order to simulate more realistic inlet conditions a short straight pipe was added upstream of each tested geometry. This short pipe included a three piezoelectric sensor array in order to obtain  validation pressure data for CFD simulations, but no acoustic intensimetry could be performed as the pipes were too short for proper flow development.

\begin{figure}[tb!]
\centering
\includegraphics[width=\textwidth]{figures/cap_geom/geom_anechoic}
\caption[Drawing of the selected simple geometries]{Drawing of the selected simple geometries. The straight pipe at the top was used standalone as a reference and also attached upstream of each geometry in order to measure the instantaneous pressure except for the tapered duct that used a wider pipe.}
\label{fig:image_simple_geom}
\end{figure}

\subsection{Surge margin}

Furthermore, raw pressure data from one of these fast sensors was used to characterize with high precision the air mass flow at which the compressor entered into deep surge conditions. In this way not only the acoustic output of each geometry could be measured but also their influence regarding the stability of the flow and thus the enlargement of the usable operating conditions.

As in the previously described experimental campaigns, this fast pressure data was recorded by a Yokogawa digital oscilloscope, along with the mass flow and shaft speed data. The rest of the ancillary information about the turbocharger operation was recorded by the in-house DAQ system.

Measurements with each inlet geometric variant were carried out at different air mass flow settings and a constant shaft speed of 160 krpm, from higher mass flow to lower mass flow. 

When conditions were judged to be close enough to deep surge, a continuous recording of the variables started where the mass flow was gradually decreased by slowly closing the backpressure valve until the violent oscillations of deep surge began and the valve was opened completely to stabilize the turbocharger.

\begin{figure}[htb!]
\centering
\includegraphics[width=0.49\textwidth]{figures/cap_geom/AutoSurge_straightData}
\includegraphics[width=0.49\textwidth]{figures/cap_geom/AutoSurge_reservoirData}\\[3mm]
\includegraphics[width=0.49\textwidth]{figures/cap_geom/AutoSurge_elbowData}
\includegraphics[width=0.49\textwidth]{figures/cap_geom/AutoSurge_taperedData}
\caption[Non-dimensional pressure recording of each geometry]{Non-dimensional pressure recording of each geometry when air mass flow was progressively decreased until deep surge condition was reached.}
\label{fig:autosurge_simple_geom}
\end{figure}

This pressure recording was then post-processed following a procedure similar to that described in chapter \ref{cap:metod}. In steps of 0.5 s, FFT was applied to obtain the pressure spectra and then a frequency ratio $R_f$ was computed between the power at 0--50 Hz and the power at 50--100 Hz:

\begin{equation}
	R_f=\frac{\sum_{f=0}^{50}|\mathbf{P}(f)|^2}{\sum_{f=50}^{100}|\mathbf{P}(f)|^2}
	%, \quad \mathbf{P}(f) = \mathcal{F}}\{p(t_i,t_i+0.5)\}
\end{equation}

When this power ratio reached values over 100, it was considered that deep surge had started at that time. Data was then averaged between the two previous time steps in order to obtain the corresponding value of lowest allowable air mass flow. Time steps and power ratios can be together with non-dimensional pressure tracks in Fig. \ref{fig:autosurge_simple_geom}. The lowest allowable mass flow obtained through this method can be seen in Fig. \ref{fig:minFlow_simple_geom}.

\begin{figure}[htb!]
\centering
\includegraphics[width=0.75\textwidth]{figures/cap_geom/GeomComp_SurgeAMF}
\caption[Lowest allowable $\dot m^*$ for the different geometries]{Lowest allowable air mass flow for the different simple geometries tested, obtained through the described algorithm.}
\label{fig:minFlow_simple_geom}
\end{figure}

As seen in this figure, the straight duct is the first one to reach  deep surge conditions. The tapered duct and the 90º elbow pipe have similar but higher margins until surge, while the reservoir delays this damaging condition even more, allowing approximately 10 g/s of further air mass flow reduction.

\subsection{Noise levels}

In addition to the effect on surge margin, the effect on external noise levels was evaluated. As described in previous chapters, two different measurements systems were available in the anechoic chamber. 

To capture the orifice noise, one free-field omnidirectional microphone was situated at a distance of 10 mm from the inlet opening. In this case the microphone had to be repositioned each time the geometry was changed as they had different lengths and, in the case of the elbow, orientation of the orifice.

The second system for measuring externally radiated noise, consisting in the four equispaced free-field microphones placed at 1 m distance form the center of the inlet assembly was kept in the same place for all the tested geometries. 

In both cases the microphones were calibrated using a Brüel \& Kj\ae r pistonphone prior to the measurements and data was captured using a PULSE\textsuperscript{TM} DAQ system. Fig. \ref{fig:Radiated_Orifice_SPL_simple_geom} shows the result of both measurements at the stable operating point most close to surge as characterized by the aforementioned procedure, since it is the point of maximum noise emission.

\begin{figure}[htb!]
\centering
\includegraphics[width=0.75\textwidth]{figures/cap_geom/GeomComp_Radiated_Orifice_SPL}
\caption[Comparison of orifice and radiated noise (simple geom.)]{Comparison of orifice noise and averaged radiated noise measured in the anechoic chamber for each of the simple inlet geometries at the stable operating point most close to surge.}
\label{fig:Radiated_Orifice_SPL_simple_geom}
\end{figure}

It can be seen that again there are significant differences between the inlet geometries tested. These differences were higher when measuring the radiated noise through the average of the four microphones, with the reservoir featuring a 3 dB reduction compared to the reference straight duct. 

On the other hand, the 90º elbow inlet reached a 5 dB increase over the straight inlet level, but this particular result could be influenced by the fact that the direction of the orifice was not perpendicular to the plane of the four microphones. Finally, the tapered duct seemed to radiate at the same level that the straight pipe.

Regarding orifice noise, the reservoir featured again a reduction in noise level of 2 dB. However, in this case the elbow featured the same level as the reference straight pipe, being the tapered duct the one with a higher level at 130 dB, 3 dB more than the reference geometry.

In Fig. \ref{fig:GeomComp_OrificeSpectra_ann} the different spectra recorded by the orifice microphone during each geometry test are plotted to look in detail at these level differences. It can be seen how levels for the tapered duct are higher along the whole frequency range, while the inlets featuring the elbow and the reservoir remain closer to the reference straight pipe.

\begin{figure}[htb!]
\centering
\includegraphics[width=0.75\textwidth]{figures/cap_geom/GeomComp_OrificeSpectra_ann}
\caption[Comparison of orifice noise spectra (simple geom.)]{Comparison of orifice noise spectra for each of the simple inlet geometries at the stable operating point most close to surge.}
\label{fig:GeomComp_OrificeSpectra_ann}
\end{figure}

The most noticeable difference however, is the resonator effect of the reservoir, that provides an important attenuation at frequencies in the plane wave region of the spectrum. Compare this against the tapered duct spectrum which raises levels over the whole spectrum. Finally, a small peak can be noticed at 16 kHz, correspondent to the Blade Passing Frequency  associated to the 160 krpm shaft speed that was used during these measurements.

\section{On-engine tests}
\label{sec:geom_motor}

In the scope of the on-engine tests described in chapter \ref{cap:engine}, a follow-up experimental campaign was carried out in order to characterize the effect that different geometric variations of the turbocharger inlet line had in performance, surge margin and noise emission.

\subsection{Measurement setup}

The selected geometric variations of the inlet line were chosen taking into account realistic packaging size restrictions of commercial vehicles, so in this case the reservoir case was not tested as it was not possible to fit it to the available compressor inlet. 

Similarly, elbow inlets were not tested as they would require to re-route the inlet line, being its design already frozen at the time of the testing. Focus was then put into small changes like tapered ducts, rotation devices and nozzle-like additions to the inlet pipe.

\begin{figure}[htb!]
\centering
\includegraphics[width=0.95\textwidth]{figures/cap_geom/geom_engine}
\caption[Section view of the inlet geometries (engine test)]{Longitudinal section view of the inlet geometries that were considered for the on-engine experimental campaign.}
\label{fig:geom_engine}
\end{figure}

Figure \ref{fig:geom_engine} shows the selected geometry variations based on these selection criteria. On top of the figure the reference geometry is depicted. The turbocharger inducer was milled and substituted by an adaptor to which a straight duct was coupled. At the end of this duct a pipe of the original diameter was then affixed with a tapered transition.

The simplest inlet geometry tested consisted in a tapered duct situated as close as possible to the compressor impeller leading edge. This can be seen at the center left in the figure. Below, it is shown how this tapered duct was complemented with an inlet guide vanes (IGV) device.

\begin{wrapfigure}[15]{o}[7mm]{0.4\textwidth}
%\begin{figure}[htb!]
\centering
\includegraphics[width=0.4\textwidth]{figures/cap_geom/IGV_fitted}
\caption[Image of the 3D-printed IGV device]{Image of the 3D-printed IGV device fitted to the compressor inlet duct, looking towards the impeller.}
\label{fig:IGV_fitted}
%\end{figure}
\end{wrapfigure}

This was done in order to increase the swirl of the incoming air flow. The device stators featured 0º leading edge angles and $\sim 70$º trailing edge angles, and it was directly 3D-printed from the CAD design. The device can be seen fitted into the inlet pipe in Fig. \ref{fig:IGV_fitted}.

To analyze how the closeness of the tapered duct to the impeller influenced the behaviour of the compressor, the two following geometric variations (center column of Fig. \ref{fig:geom_engine}) were the same except for the separation between the tapered duct and the impeller, which was increased to 52 mm. Again, one of the measurements was performed with an IGV and the other without it.

Lastly, two nozzles were tested (right column of Fig. \ref{fig:geom_engine}): a convergent-divergent nozzle between the tapered adaptor and the impeller (top) and a convergent nozzle (bottom) just after the tapered duct, with its throat as close as possible to the impeller wheel.

In order to evaluate the influence of these geometric variation on the noise output of the compressor, the acoustic measurement setup used in chapter \ref{cap:engine} was maintained, consisting in the straight inlet extension featuring the pressure decomposition array for in-duct intensimetry and the intensity probe aligned with the inlet orifice.

Measurement operating conditions for the noise characterization campaign were chosen as to be representative of the most common real-life conditions. This was done in order to test if these simple inlet variations could modify the noise output not only on the most extreme conditions but also when more conservative settings are applied.

\subsection{Results}

Since the objective of this experimental campaign is the comparison between geometries, in this subsection only the results from the orifice intensity probe will be presented for the sake of brevity, as it was shown in section \ref{sec:motor_method_valid} that they were well correlated with in-duct noise.

In Fig. \ref{fig:geom_mot_comparison_map1} the interpolated intensity level map is presented, indicating the points at which each measurement was made. The interpolation of sound intensity was performed between frequencies of 1 and 3 kHz, which are specially relevant regarding the aforementioned \emph{whoosh} issue.

\begin{figure}[htb!]
\centering
\includegraphics[width=0.65\textwidth]{figures/cap_geom/map_engine_geom_config1}
\caption[SIL map in the 1--3 kHz band (straight)]{Sound intensity level map between 1--3 kHz for the straight inlet used as a reference in the on-engine inlet geometric variation test.}
\label{fig:geom_mot_comparison_map1}
\end{figure}

Note as well that because of the particular regulation mechanism of the facility based on the injection of compressed air into the inlet line, the depicted operating conditions do not represent constant shaft speed lines, so no judgement about the slope of these lines should be made.

However, it can be seen in this figure that the noise levels in the selected band behave as expected, based on results described in previous chapters. Noise levels increase as pressure ratio increases and air mass flow diminishes, with the dependency on air mass flow being much more noticeable at higher pressure ratios that at lower ones.

\begin{figure}[htb!]
\centering
\includegraphics[width=0.45\textwidth]{figures/cap_geom/map_engine_geom_config2}\hspace{0.05\textwidth}
\includegraphics[width=0.45\textwidth]{figures/cap_geom/map_engine_geom_config3}\\[2mm]
\includegraphics[width=0.45\textwidth]{figures/cap_geom/map_engine_geom_config4}\hspace{0.05\textwidth}
\includegraphics[width=0.45\textwidth]{figures/cap_geom/map_engine_geom_config5}\\[2mm]
\includegraphics[width=0.45\textwidth]{figures/cap_geom/map_engine_geom_config6}\hspace{0.05\textwidth}
\includegraphics[width=0.45\textwidth]{figures/cap_geom/map_engine_geom_config7}
\caption[SIL maps in the 1--3 kHz band (comparison)]{Sound intensity level maps between 1--3 kHz for the different geometries used in the on-engine test.}
\label{fig:geom_mot_comparison_maps}
\end{figure}

Figure \ref{fig:geom_mot_comparison_maps} shows the results for the same frequency band when the reference straight duct was substituted by each of the variants described in Fig.~\ref{fig:geom_engine}. 

It is apparent from the figure that different operating conditions were measured in each one of the tests, as the goals in this campaign were to characterize the sound emission across the selected map region and to test how each geometry impacted the performance in regards of surge margin and maximum ingested air mass flow.

In this regard, the figure shows how the geometry changes have an impact on surge margin (left frontier of the map), as some of the modified inlet lines increase the range of low flow operating conditions that can be safely used before going into deep surge, specially at higher shaft speeds; compare for instance the highest row of points of the straight inlet in Fig. \ref{fig:geom_mot_comparison_map1} and that of the convergent-divergent nozzle (bottom left plot of Fig. \ref{fig:geom_mot_comparison_maps}).

Impact of the geometry variation on air mass flow was also tested in regards of maximum flow at 160 krpm, as it is not only important to increase surge margin but also to maintain the possibility of operating the turbocharger at high mass flow levels. 

\begin{figure}[t!]
\centering
\includegraphics[width=0.95\textwidth]{figures/cap_geom/geom_mot_comparison_AMF}
\caption[Comparison of performance variation]{Comparison of performance variation for each of the inlet geometries across the normal operating range of the engine.}
\label{fig:geom_mot_comparison_AMF}
\end{figure}

A comparison of the performance of each geometry in both aspects is presented in Fig. \ref{fig:geom_mot_comparison_AMF} where the percentages of improvement from the straight duct reference inlet are plotted. An inverse correlation between maximum flow and surge margin is apparent, only the tapered duct at 52 mm from the impeller improving both factors at the same time and only the tapered duct closer to the impeller and featuring an IGV worsening both.

Largest amount of surge margin improvement was found with the convergent nozzle, but at the cost of an equally large air mass flow reduction at $\Pi_\text{TT} = 2$ and 160 krpm of shaft speed, so in order to implement this solution in real-life applications it should be necessary to mechanically open the nozzle throat at these high flow conditions.

To similarly quantify the influence of the geometries in noise emission at the selected frequency band, in Fig. \ref{fig:geom_mot_comparison_meanSIL} an average sound intensity level (SIL) is presented, as measured by the Brüel \& Kj\ae r intensity probe located pointing directly to the inlet orifice.

\begin{figure}[htb!]
\centering
\includegraphics[width=0.95\textwidth]{figures/cap_geom/geom_mot_comparison_meanSIL}
\caption[Comparison of mean sound intensity level]{Comparison of mean sound intensity level for each of the inlet geometries across the normal operating range of the engine.}
\label{fig:geom_mot_comparison_meanSIL}
\end{figure}

Intensity level of the tapered duct is very similar to that of the reference straight pipe; both when located as close as possible to the compressor wheel and when separated 52 mm from it. The addition of the described IGV to those tapered inlets notably reduces noise levels by 3 and 4 dB respectively. 

Regarding the nozzle configurations, it can be seen that the convergent-divergent nozzle reduces noise output in the selected frequency band by 1 dB, whereas the convergent nozzle features a notable reduction of approximately 3 dB compared to the reference.

\section{Elbow parametric study}
\label{sec:geom_elbows}

Having attested the influence of the inlet geometries not only in surge margin and maximum flow but also regarding acoustic performance, it was decided to use the turbocharger test rig used for the inlet local flow measurements described in chapter \ref{cap:inlet} to study the effect of these geometric variations not only in the global aspects of sound intensity and air mass flow but also regarding local aspects like temperature distribution, skewness, etc.

The gathering of local flow experimental data will also be useful if CFD simulations of these kind of inlet modifications are undertaken, in order to provide a validation reference to check for the correct solving of not only global values but also local flow field characteristics.

Since neither during the previously described on-engine experimental campaign nor during the preliminary testing described at the beginning of the present chapter was it possible to properly measure the sound intensity caused by elbow geometries, it was decided that the study should start with this kind of geometry variation.

Another reason to consider this geometry is that it introduces non-unifor\-mi\-ties in the flow field that completely axisymmetric geometries such as tapered ducts and nozzles do not, so it would provide a good contrast to check the influence on circumferential temperature distribution skewness and correlation with noise generation.

In order to increase the available data on the relation between inlet-induced flow distortion, temperature skewness and noise output, it was decided that a parametric campaign should be carried out where three 90º elbows of different radius would be designed, manufactured and tested.

\subsection{Geometry design}

Requirements for the design of the elbow inlets were such that they could be easily installed into the existing setup used for local measurements, coupling to the adaptor piece and existent measurement pipe featuring the in-duct intensity sensor array.

In addition, they needed to house the available linear thermocouple arrays in order to measure the high temperature reversed flow (it should be noted that the circumferential thermocouple array was installed into the adaptor piece) approximately keeping the distance that was selected for the straight pipe study.

\begin{figure}[htb!]
\centering
\includegraphics[width=0.9\textwidth]{figures/cap_geom/codo075D}
\caption[Mechanical drawing of the 0.75D elbow inlet]{Mechanical drawing of the 0.75D elbow inlet designed to fit into the compressor inlet adaptor and house the linear thermocouple array.}
\label{fig:codo075D}
\end{figure}

Finally, to assess the influence of the flow distortion, the three selected variants were as follows: a very tight elbow with a central radius of 0.75 compressor inlet diameters (marked as D in Fig.~\ref{fig:codo075D}), a medium elbow with 1D of radius and finally an elbow with a central radius of 1.5D that would provide a smoother transition and less distorted flow.

To fulfil these design requirements, a decision was made to manufacture the elbow prototypes using 3D printing instead of traditional workshop means. This way, the guiding slots for the thermocouples and retention screws could be directly embedded in the piece, along with the slots for the sealing O-rings. Fig. \ref{fig:codo075D} shows the mechanical drawing of the most abrupt elbow with 0.75D of radius.

This design was implemented in a parametric CAD model to easily modify the central radius. Also, a locking system was put into place so the elbow could be affixed tightly to the compressor adaptor using a flange and a set of threaded rods, similarly to the procedure performed for the local flow characterization. The three prototypes generated in this way are shown in Fig. \ref{fig:Comp_codos}.

Once that the three CAD models were generated, they were sent to a commercial partner for verification and manufacturing. It should be noted that the elbow prototypes were expected to withstand the high temperature backflow coming from the compressor in unstable operating conditions close to deep surge, without undertaking deformations that would have misaligned the thermocouple array.

Taking into account these concerns, it was decided to manufacture the elbow prototypes using an aluminium-filled polyamide through a process known as Selective Laser Sintering (SLS), a technique in which a laser beam is focused onto the surface of a tray containing the powdered material, binding it together in layers to create the desired shape. 

\begin{figure}[t!]
\centering
\includegraphics[width=0.95\textwidth]{figures/cap_geom/Comp_codos}
\caption[CAD models of the compressor elbow inlets]{CAD models of the compressor elbow inlets of different radius (decreasing in radius from left to right) that were used to create the 3D-printed prototypes.}
\label{fig:Comp_codos}
\end{figure}

\begin{table}[b!]
  \centering
  \caption[Specifications of the aluminium-filled polyamide]{Specifications of the aluminium-filled polyamide used for the laser-sintering of the elbow inlet prototypes.}
  	\footnotesize
    \begin{tabular}{llcc}
    \addlinespace
    \toprule
    \bf Property & \bf Normative & \bf Unit & \bf Value \\
    \midrule
    Tensile Modulus & DIN EN ISO 527 & MPa   & 3800 $\pm$ 150 \\
    Tensile Strength & DIN EN ISO 527 & MPa   & 48 $\pm$ 3 \\
    Elongation at Break & DIN EN ISO 527 & \%    & 3.5 $\pm$ 1 \\
    Flexural Modulus & DIN EN ISO 178 & N/mm² & 3600 $\pm$ 150 \\
    Charpy – Impact strength & DIN EN ISO 179 & MPa   & 29 $\pm$ 2 \\
    Charpy – Notched Impact Strength & DIN EN ISO 179 & MPa   & 4.6 $\pm$ 0.3 \\
    Shore D/ A-hardness & DIN 53505 &   -   & D 76 $\pm$ 2 \\
    Heat Deflection Temp & ASTM D648
(1.82MPa) & °C    & 130 \\
    Vicat Softening Temperature B/50 & DIN EN ISO 306 & °C    & 169 \\
    Density &   N/A    & g/cm³ & 1.36 $\pm$ 0.05 \\
    \bottomrule
    \end{tabular}
  \label{tab:alumide_data}
\end{table}

Specifications of the selected material can be found in table \ref{tab:alumide_data}. Of special relevance is the heat deflection temperature, also called sometimes heat distortion temperature. It is defined as the temperature at which a sample, under a defined load of 1.82 MPa, will start to deflect at least 0.25 mm. Since in this case the inlet line is approximately at ambient pressure no problem is expected until more elevated temperatures.

\subsection{Setup}

After SLS manufacturing and delivery, the prototype elbows were instrumented with thermocouples and installed into the anechoic chamber test rig. Fig. \ref{fig:elbow_075D_assembled} shows the assembly of the narrower elbow variant.

\begin{figure}[b!]
\centering
\includegraphics[width=0.9\textwidth]{figures/cap_geom/elbow_075D_assembled}
\caption[CAD model of the compressor assembly]{CAD model of the compressor assembly featuring the 0.75D elbow inlet attached to the common adaptor, including the pressure and temperature instrumentation and the mounting system.}
\label{fig:elbow_075D_assembled}
\end{figure}

It should be noted that the narrow holes for the thermocouple alignment became filled with residual dust during the sintering process, and had to be finished with a manual drill. This presented no problem as the chosen material is easily mechanized.

As it can be seen in Fig. \ref{fig:elbow_075D_assembled} the previously used adaptor piece housing the circumferential temperature array, the inducer miniature pressure probe and a sealing O-ring was kept in place, and the holes in the plane surface added to the elbow were used to align the elbow parallel to the outlet and tighten it using the threaded rods depicted in blue.

\subsection{Results}

During this experimental campaign, operating conditions were selected to mirror those used in the local flow characterization chapter that featured a straight inlet pipe, in order to allow the reconstruction of interpolated maps that were as close as possible to the original reference map.

\begin{figure}[bht!]
\centering
\includegraphics[width=0.7\textwidth]{figures/cap_geom/map_elbows}
\caption[Operating conditions measured for each geometry]{Operating conditions at which data was recorded during the inlet flow characterization tests with a straight duct (as reference) and during the parametric elbow campaign. This data points will be used in the present section for every map.}
\label{fig:map_elbows}
\end{figure}

Fig. \ref{fig:map_elbows} shows the operating conditions of each measured data point, for both the straight inlet pipe used as reference, and for the three described elbow prototypes. It can be seen that repeatability was achieved in most conditions up to a few g/s of air mass flow and $\sim$100 corrected rpm. 

The most adverse conditions were those of maximum allowable flow, specially at higher pressure ratios where the flow distortion and pressure loss caused by the elbows made difficult to achieve exactly the same conditions.

\subsubsection{Sound intensity}

Using the in-duct piezoelectric sensor arrays, sound intensity level was computed through the wave decomposition method outlined in chapter \ref{cap:metod} for each operating conditions. In order to focus into the previously described \emph{whoosh} noise problem, sound intensity level was averaged between 0.7 and 3 kHz. and then the interpolating procedure was used in order to reconstruct noise maps for each geometry.

\begin{figure}[thb!]
\centering

\includegraphics[width=0.47\textwidth]{figures/cap_geom/map_diff_mapIin_030316_Inlet}\hspace{4mm}
\includegraphics[width=0.47\textwidth]{figures/cap_geom/map_diff_mapIout_030316_Outlet}\\[4mm]

\includegraphics[width=0.47\textwidth]{figures/cap_geom/map_diff_mapIin_260216_Inlet}\hspace{4mm}
\includegraphics[width=0.47\textwidth]{figures/cap_geom/map_diff_mapIout_260216_Outlet}\\[4mm]

\includegraphics[width=0.47\textwidth]{figures/cap_geom/map_diff_mapIin_150216_Inlet}\hspace{4mm}
\includegraphics[width=0.47\textwidth]{figures/cap_geom/map_diff_mapIout_150216_Outlet}\\[4mm]

\caption[Difference maps for SIL]{Difference map of inlet (left) and outlet (right) in-duct sound intensity levels for each of the selected inlet elbows compared against the reference straight duct. From top to bottom: 1.5D, 1D \& 0.75D.}
\label{fig:map_diff_in_out}
\end{figure}

To highlight the differences between the reference straight duct noise map and the result obtained for each elbow prototype, and taking into account the aforementioned slight differences between operating conditions at higher air mass flow settings, difference maps were produced by subtracting the reference levels to each one of the newly obtained elbow maps.

It should be remarked that this subtraction procedure was not performed between the raw levels at each measurement setting but between each interpolated data point, and only where such interpolated data existed in both noise maps (elbow and reference), as to avoid erroneous differences due to the different operating conditions. 

The results of this procedure can be seen in Fig. \ref{fig:map_diff_in_out}, where the difference maps for the three elbow configurations regarding inlet and outlet in-duct intensity are presented. Color scale has been kept constant in order to allow a meaningful comparison between each map.

It can be seen in the maps, examining for instance some of the highest mass flow rate points, that some information is missing around the black dot that indicates the original measurement point before interpolation. This missing data comes from slight differences in measurements between geometries, as the subtraction process is only performed where data exists for both of them.

As for the evolution of level difference with the variation of the elbow radius, the figure shows how in the case of the inlet sound intensity differences appear to be higher than in the case of the outlet. This increase is specially evident at lower flow conditions and medium-low shaft speeds, reaching up to 6 dB in level increase. 

There also appears to be a particular point with an small increase in level for both inlet and outlet ducts, at medium air mass flow and medium to hight shaft speeds, its relevance diminishing along with the reduction of the elbow radius.

Finally, some operating conditions in both the inlet and outlet present reduced noise output in the selected frequency band, specially at medium-higher mass flows and lower shaft speeds. As with the case of the increases, these reductions are more significant in the inlet duct, whereas positive and negative variations are of smaller magnitude in the outlet duct.

In order to better quantify the SIL differences that these maps reveal, histograms that represent the distribution of the differences have been computed and presented in Fig. \ref{fig:elbow_intens_histog} for both inlet (top) and outlet (bottom) sound intensity results.

It can be seen in the figure how the different elbow configurations affect the distribution of the intensity level differences. In the case of the inlet, the less narrow elbow (1.5D) and the medium elbow (1D) present a similar, almost symmetric distribution with a mean around -1 dB.

\begin{figure}[htb!]
\centering
\includegraphics[width=0.95\textwidth,trim={3cm 11cm 3cm 11cm},clip]{figures/cap_geom/distribution_diff_mapIin}
\includegraphics[width=0.95\textwidth,trim={3cm 11cm 3cm 11cm},clip]{figures/cap_geom/distribution_diff_mapIout}
\caption[Difference histograms for SIL]{Histograms of inlet (top) and outlet (bottom) sound intensity level differences between each one of the tested elbow variants and the reference straight inlet pipe. Y axis indicates the percentage of the interpolated compressor map at which each difference level occurred. A dashed line indicates the average of each configuration.}
\label{fig:elbow_intens_histog}
\end{figure}

The distribution of the most narrow elbow (0.75D) appears to feature a marked skewness towards positive differences of up to 6 dB that correspond with the low flow, medium-low speed region identified in the aforementioned intensity maps. These points shift the mean up to approximately 0.2 dB of increased level.

Regarding the outlet difference distributions, the situation appears to reverse, with the narrower elbow now featuring the most symmetric distribution and the other two variations presenting a marked skewness towards intensity reduction values, again up to a mean of around 2 dB in the most favourable case (1.5D elbow).

\begin{figure}[htb!]
\centering
\includegraphics[width=0.88\textwidth]{figures/cap_geom/elbow_comp_1sens}
\caption[Comparison of plane wave range spectra]{Comparison of plane wave range spectra from both inlet and outlet, highlighting the differences between the straight reference duct and each elbow configuration.}
\label{fig:elbow_comp_1sens}
\end{figure}

Focusing now into the detailed aspects of the sound level variation, it is possible to compare the pressure spectra in different points to assess the frequency distribution of the level variation. For instance in Fig. \ref{fig:elbow_comp_1sens} the PSD in the plane wave region recorded by inlet and outlet piezoelectric sensors is shown for the different inlet geometries at an operating condition where differences where high.

In the case of the outlet duct, spectra are very similar up to a cut-off point at approximately 1.3 kHz, below which the straight duct presents higher levels than all of the elbows.

On the other hand, spectra on the plane wave region of the inlet duct shows higher differences, consistent with the data already presented in the maps and histograms. 

It is interesting to note however that this increase is not homogeneous in frequency; it can be seen how the small peak at 1.7 kHz present in the straight duct gradually increases with elbow radius reduction, up to a clearly visible broadband between 1.5 and 2.5 kHz.

These are flanked by another two smaller broadband elevations at 0.75 and 3.6 kHz that are also visible in the case of the outlet spectra, and were also very clearly present in the pressure spectra recorded by the diffuser pressure probe, shown for instance in Fig.~\ref{fig:probes_5K} of the previous chapter.

It should be noted as well that, while levels for the detected broadband in the inlet show an increase up to 5 dB, at these frequencies all of the outlet spectra plotted in the lower graph remain at the same levels, indicating that while changes in the inlet configuration seem to boost the upstream transmission of the so-called \emph{whoosh} broadband noise, no influence on the outlet duct spectra is detected at these frequencies.

\subsubsection{Temperature}

Besides the described differences in terms of acoustic intensity produced by the different inlet configurations, it is possible to follow a similar processing procedure in order to assess the differences in terms of temperatures in the inlet section that are related to the hot reversed flow coming from after the compressor impeller.

As in the previous experimental campaign, data from the two thermocouple arrays (longitudinal and circumferential) was processed in order to compile maps of the backflow length and flow temperature skewness, respectively.

These maps were then compared with the original reference map measured with a straight inlet duct, and difference maps were produced for each elbow variant. The resulting difference maps can be seen in Fig. \ref{fig:map_diff_std}. The same color scale has been kept in all of them to provide a suitable comparison.

\begin{figure}[tb!]
\centering
\includegraphics[width=0.47\textwidth]{figures/cap_geom/map_diff_mapTcorStdDev_030316_}\hspace{4mm}
\includegraphics[width=0.47\textwidth]{figures/cap_geom/map_diff_mapTcorStdDev_260216_}\\[3mm]
\includegraphics[width=0.47\textwidth]{figures/cap_geom/map_diff_mapTcorStdDev_150216_}\hspace{4mm}
\caption[Difference maps for temperature skewness]{Difference maps of temperature circumferential standard deviation for each of the selected inlet elbows compared against the reference straight duct.}
\label{fig:map_diff_std}
\end{figure}

It can be seen that for the three cases a similar patters appears. An increase in temperature skewness is consistently detected at the lower air mass flow settings, specially at medium shaft speed operating conditions.

At higher air mass flow settings however, a slight reduction in flow temperature skewness is apparent, being much more pronounced in the less narrow elbow configuration (1.5D, top left map). 

This reduction area spans approximately the same conditions for the 1D elbow, but the magnitude of the temperature skewness difference is lower than in the previous case, becoming nearly zero in the case of the most reduced elbow variant (0.75D).

\begin{figure}[htb!]
\centering
\includegraphics[width=0.95\textwidth,trim={3cm 11cm 3cm 11cm},clip]{figures/cap_geom/distribution_diff_Tstd}
\caption[Difference histogram for temperature skewness]{Histograms of temperature standard deviation differences between each one of the tested elbow variants and the reference straight inlet pipe. Y axis indicates the percentage of the interpolated compressor map at which each difference level occurred. A dashed line indicates the average of each configuration.}
\label{fig:distribution_diff_std}
\end{figure}

As in the case of sound intensity level differences, it is possible to combine these maps in the form of histograms in order to better quantify the distribution of these differences. These histograms can be seen plotted in Fig. \ref{fig:distribution_diff_std}.

Distributions shown in this figure appear similar in shape, but shifted in mean value. In all three cases the right tail of the distribution is longer but less populated, indicating that the difference increase happens abruptly in a small region of the compressor map. 

On the contrary, the left tails are more populated but shorter, meaning that more regions of the compressor map feature points with slightly lower differences than the average. These averages reach up to 1ºC in the case of the most narrow elbow configuration (0.75D), while the average is almost zero for the case of the less narrow elbow (1.5D).

As mentioned earlier, by processing the temperature profiles measured by the longitudinal thermocouple array that was added to the design of each elbow prototype, it is also possible to estimate the length of the hot reversed backflow that travels upstream of the impeller and into the inlet in unstable operating conditions.

When designing the prototypes, care was put to ensure that the separation between the thermocouples remained equal to that of the straight inlet duct, taking into account their inserted length of half a radius and the varying radius of each elbow pipe.

In Fig. \ref{fig:map_diff_backL} the estimated length along the exterior edge of the elbow (where the thermocouple array was installed, as shown in Fig. \ref{fig:codo075D}) is mapped for the three elbow configurations.

\begin{figure}[htb!]
\centering
\includegraphics[width=0.47\textwidth]{figures/cap_geom/map_diff_mapBackLen_030316_}\hspace{4mm}
\includegraphics[width=0.47\textwidth]{figures/cap_geom/map_diff_mapBackLen_260216_}\\[3mm]
\includegraphics[width=0.47\textwidth]{figures/cap_geom/map_diff_mapBackLen_150216_}\hspace{4mm}
\caption[Difference maps for backflow extent]{Difference maps of high temperature backflow extent for each of the selected inlet elbows compared against the reference straight duct.}
\label{fig:map_diff_backL}
\end{figure}

In this case, differences range from minus half a diameter up to plus a diameter when compared with the reference straight pipe. The patterns of these differences along the compressor map appears to be similar for the three geometric variations: a slight reduction at lower mass flow conditions and a slight increase at higher mass flow conditions for all shaft speeds.

Exceptions to this pattern are some of the lowest air mass flow data points at 100 and 120 krpm for the two narrower elbows (0.75D and 1D), where the backflow length appears to feature a large increase. However it is possible that those are spurious results from the interpolation procedure, since the increase region is very small.

\begin{figure}[htb!]
\centering
\includegraphics[width=0.95\textwidth,trim={3cm 11cm 3cm 11cm},clip]{figures/cap_geom/distribution_diff_mapBackLen}
\caption[Difference histogram for backflow extent]{Histogram of high temperature backflow extent differences between each one of the tested elbow variants and the reference straight inlet pipe. Y axis indicates the percentage of the interpolated compressor map at which each difference level occurred. A dashed line indicates the average of each configuration.}
\label{fig:distribution_diff_backL}
\end{figure}

These low count extreme points are best shown when the data is put into histogram form as seen in Fig. \ref{fig:distribution_diff_backL}. In this figure it can be seen how the three backflow length difference distributions are quite similar in shape and values.

The three distributions feature approximately the same mean, around 0.1 diameters of backflow length reduction from the straight inlet case. However, the most usual difference (that is, the mode) the difference is practically zero. It must be also considered that estimation of backflow is only performed along the outer edge of the elbow, while the modified flow pattern could mean that a very non-symmetric backflow is propagating upstream.

\subsubsection{Correlations between SIL and temperature}

In Fig.~\ref{fig:comp_corrs_skew_sil} the correlation between sound intensity level and circumferential temperature skewness has been performed again for the three elbow inlet cases as it was made in chapter \ref{cap:inlet} for the case of the straight pipe, which has been included as a reference keeping the same axis limits of the elbow cases.

It can be seen in this figure how for the straight pipe and the more straight elbow (that with an central radius equal to 1.5D), both pictured on the top plots, a clear separation is maintained between higher flow rate conditions (indicated as blue circles) which exhibit higher noise levels even when no skewness is detected, and those of lower flow rate where skewness is well correlated with the increase in noise levels.

\begin{figure}[t!]
\centering
\includegraphics[width=0.45\textwidth]{figures/cap_geom/corr_int_skew_elbow0}\hspace{4mm}
\includegraphics[width=0.45\textwidth]{figures/cap_geom/corr_int_skew_elbow3}\\[4mm]
\includegraphics[width=0.45\textwidth]{figures/cap_geom/corr_int_skew_elbow2}\hspace{4mm}
\includegraphics[width=0.45\textwidth]{figures/cap_geom/corr_int_skew_elbow1}
\caption[SIL vs. $\sigma$ temp. correlation for each geometry]{Correlations between standard deviation at the circumferential thermocouple array and inlet sound intensity level in the 1--3 kHz band for the straight pipe and the three elbow inlets. Size indicates shaft speed from low to high and color indicates mass flow from higher (blues) to lower (reds).}
\label{fig:comp_corrs_skew_sil}
\end{figure}

In the case of the more narrow elbows, pictured in the bottom plots of the figure, the difference between the higher and lower flow rate cases becomes blurred, as even points of the higher flow rate demonstrate measurable temperature skewness as the elbow radius decreases. 

By comparing the four plots of the figure, it can also be seen how the maximum values of temperature skewness increase as the radius of the pipe decreases, with the same trend being followed by the sound intensity level in the selected frequency band, as was clearly seen previously in Fig.~\ref{fig:elbow_intens_histog} where the characteristic ``hump'' of \emph{whoosh} noise becomes more pronounced as the radius of the elbow gets narrower (while a similar influence is not observed in the outlet spectra).

\section{Conclusions}

Through this chapter, three different experimental campaigns aimed at assessing the influence of the inlet geometry in the turbocharger acoustic output have been presented. While different in scope and means, the results shown in this chapter allow some global conclusions to be drawn.

First, a series of tests was carried out where simple geometric variants were added to the compressor inlet, with a straight duct of the same diameter than the compressor inducer being as a reference. This reference was maintained for the three described experimental campaigns. Both orifice noise recorded with a free-field microphone and orifice noise averaged between four microphones were recorded.

Results of this first test series showed that, apart from an effect in the turbocharger surge margin that had been previously pointed out by Galindo et al. \cite{galindo2007potential}, influence of the geometries on both orifice and radiated noise could be found. The tapered duct appeared to increase the orifice noise, while the reservoir acted as a resonator and not only increased surge margin but lowered both orifice (2 dB) and radiated noise (3 dB). Meanwhile, the elbow increased kept the same orifice noise levels.

Another series of geometry tests were carried out in an engine cell and introduced more complex geometry combinations, with tapered ducts featuring an IGV device and the addition of two nozzles to the inlet line, besides the reference straight pipe. In this case noise output was measured by a commercial intensity probe aligned with the inlet orifice and with the in-duct intensimetry method based on beamforming wave decomposition described in chapter \ref{cap:metod}.

Acoustic data from this on-engine campaign confirmed that the selected tapered duct, while revealing an improvement in surge margin and maximum flow when located at a certain distance from the impeller, did not evidence a significant effect in orifice noise intensity and in any case these differences were harmful with a slight increase in levels, albeit of less than 0.5 dB.

The addition of an IGV device to boost incoming flow swirl proved interesting as it decreased the acoustic intensity recorded by the orifice probe by 3-4 dB depending on the position of the tapered duct. However this device negatively impacted the maximum flow and, depending on the position of the downstream tapered section, also the surge margin. 

In the case of the inlet geometries equipped with a nozzle, a decrease of orifice noise was found ranging from 1 dB (converge-divergent nozzle) to 3 dB (convergent nozzle). These nozzles also improved surge margin up to 20\%, but, specially the convergent one, decreased the maximum available air mass flow.

Finally, a more through experimental campaign was carried out where a family of 90º elbow inlets with different radii (that were not tested on the engine due to packaging constraints) were installed on the anechoic test ring and instrumented to measure not only in-duct noise but also temperature distribution through the thermocouple arrays described in chapter \ref{cap:inlet}.

Results from this campaign provided higher in-depth insight into the influence of the elbow radius; as in the aforementioned first testing campaign the very sharp elbow showed equal or slightly higher mean sound intensity in the inlet. However, this increase was not uniformly distributed across the compressor map but concentrated on the lowest allowable air mass setting for each shaft speed. This pattern was shared among the three tested elbow configurations, although as the radius grew sound intensity levels in both inlet and outlet reduced up to 1 dB in average.

Regarding the influence of the inlet elbow variants on the temperature distribution, a similar pattern of circumferential skewness increase towards the lowest allowable flow conditions for each shaft speed could be reconstructed. However, reversed flow length appears similar in all cases, with differences of less than 0.5 inlet diameters.

Correlations of \emph{whoosh} noise content and temperature skewness indicated an increase of both variables as the elbow radius became narrower. Reasons for these correlated increases could be related to the uneven presentation of the air flow caused by the progressively narrow 90º elbows\footnote{See the work of Kalpakli et al. \cite{kalpakli2013vortical,kalpakli2013turbulent} for an in-depth numerical analysis of the particular flow field caused by these kind of 90º elbows, including its POD decomposition.}, which could create a region of lower pressure that facilitated the upstream propagation of reversed flow and thus enhanced the convection of \emph{whoosh} noise content from the diffuser to the inlet duct, while also increasing the differences in temperature across the different zones.

This would explain why elbow radius shows a clear effect on the \emph{whoosh} noise content in the inlet spectra shown in Fig.~\ref{fig:elbow_comp_1sens} but not in the outlet, and also why this effect is restricted to the lower mass flow rate zones of the compressor map (see Fig.~\ref{fig:map_diff_in_out} for the effect on sound intensity and Fig.~\ref{fig:map_diff_std} for the effect on temperature skewness) where reversed flow is more prominent, as instabilities grow while inflow momentum diminishes. 

In any case, further research, specially if performed together with CFD simulations, is needed to characterize the effect of the inlet flow field at unstable conditions and its relation to coherent flow structures acting as noise sources.


%!TEX root = Tesis.tex
\chapter{Conclusions and future works} %.... Conclusiones / trab. futuros
\label{cap:conclus}

%-------------------------------------------------------------------%
% BACK MATTER
\backmatter 

% References
% Title format
\titleformat{\chapter}[hang]{}{}{0mm}{\bfseries\sffamily\Huge}
% Print bibliography
\small
\bibliographystyle{elsarticle-num}
\bibliography{biblioTesis}

% Back cover
% \includepdf{../test_portada3_back}

\end{document}
