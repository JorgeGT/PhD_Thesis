%!TEX root = Tesis.tex
\chapter{On-engine validation\\ \textcolor{red}{[Article expMethod - rewrite]}}
\label{cap:engine}
% Quote
% \null\vspace{-10mm}
% \begin{flushright}
% \begin{minipage}[t]{0.66\textwidth}
% {\itshape\small``The story is set in the year 1922 and has its main character, Mr Engelbert, moving to Prague from the countryside to live out his retirement years and struggling to cope with the aggressive modernity of the city. The horror is not the power of money or the arrogance of the people, but the noise; not the age-old noise of a thunderstorm or a hammer, but the new noise of engines, especially of automobiles and motor-cycles, the explosive internal combustion monsters.''}\\[-3mm]

% {\cabincondensed\small\hfill \textbf{Milan Kundera}, \emph{The Curtain.}}\\[5mm]
% \end{minipage}
% \end{flushright}

% Chapter text
\noindent \section{Introduction}\label{sec:intro}
Turbomachinery-generated noise is nowadays one of the main concerns of automotive engineering, either aerial or terrestrial, as industry and society at large are today well aware of the many problems originated by acoustic pollution.

Once exclusive to jet aircraft and high performance engines, the use of turbomachinery is now widespread in internal combustion engines as a result of the current trend to downsize engines \cite{knecht2008diesel}, since otherwise it would not be possible to maintain the same power output while reducing the displacement. In order to achieve this, the compressor must provide higher pressure ratios over a wide range of air mass flow, consequently operating in zones close to its surge limit \cite{galindo2008experiments} and farther from its maximum efficiency peak. At the same time it must provide the required mass flow without overspeeding.

As a result of this demanding requirements, even if the control system may assure the overall safety of the system, the noise generated by the turbocharger is becoming a noticeable issue that also needs to be controlled \cite{gonzalez2003sound}, either actively or passively. In fact, engine noise not only has negative effects on the environment in terms of acoustic pollution, but can also cause discomfort to vehicle passengers, as the engine noise quality is affected \cite{brizon2012combining,nor2008index}.

For this reason, the acoustical characterization of the compressor across critical or demanding operation zones or even across its entire operation map becomes increasingly necessary, and several studies on compressor noise characterization, analysis and mitigation can be found in the recent literature \cite{evans2005minimizing,soh2003noise,teng2009investigation}.

In this paper, various procedures that enable the synthesis of this acoustic information in the form of sound maps that cover a selected zone of the compressor map for relevant frequency ranges are assessed. Such representations provide a simple and clear visualization of noise phenomena for the possible operation points of the turbocharger in that zone.

An important consideration is that robust methodologies are needed to perform this acoustic characterization in the environment of production engine test cells, which are usually neither anechoic nor designed for acoustic measurements.

\section{Experimental characterization}
\label{sec:experimental_characterization}

In order to obtain a noise map that characterizes noise generation by the compressor it is necessary to know the sound spectrum at several operating points and then extract the level of the different frequency bands of interest.

Reviewing the available literature on noise characterization of rotating machines, two different approaches are found. One of them is based on measuring radiated noise by means of exterior microphones inside an anechoic environment as done, for instance, by Cudina \cite{cudina2000noise}. In engine conditions, however, it is difficult to separate the radiation of the compressor from the engine noise.

The other approach focuses on in-duct noise \cite{tiikoja2011inves}, using fast pressure sensors mounted flush inside the duct, in order to better identify the different internal flow phenomena involved in the generation of compressor noise. As this last approach can be more reliably applied to a non-anechoic engine test, this investigation is focused on it.

\subsection{In-duct noise measurements}
\label{sub:in_duct_noise_measurements}

Some authors \cite{figurella2012noise} rely on a single sensor to estimate the in-duct signal spectrum, whereas others use more sophisticated approaches introducing more sensors \cite{tiikoja2011inves}. While the one-sensor method allows the calculation of sound pressure level (SPL), a scalar magnitude which only gives information of the sum of all waves at a particular point, sound wave decomposition enables the computation of the sound intensity level (SIL), a vector magnitude which offers information on the sound coming from a particular direction.

The basis of decomposition methods is to consider that the pressure signal $p$ measured at a given point of a duct is the linear superposition of a pressure wave traveling forward in the flow direction, $p^+$, and another pressure wave traveling backwards, $p^-$.
According to Morfey \cite{morfey1971sound}, the total intensity can be regarded as the sum of the upstream and downstream pressure spectra ($\hat p^+$ and $\hat p^-$) corrected with the mean Mach number ($M_0$) and characteristic acoustic impedance ($\rho_0 a_0$):

\begin{equation}\label{eq:intensity}
	I=I^+-I^-=\frac{1}{\rho_0 a_0}\left(\left|\hat p^+\right|^2\left(1+M_0\right)^2-\left|\hat p^-\right|^2\left(1-M_0\right)^2\right)
\end{equation}

In order to use this expression, it is thus necessary to select a measurement system that allows for wave decomposition. One of the standards is the two-microphone method used for example by Holland and Davies \cite{holland2000measurement} and Tiikoja et al. \cite{tiikoja2011inves}, which uses the correlated spectra measured by two consecutive wall-flush mounted microphones to estimate the reflection coefficient and thus resolve both waves.

However, the two-microphone method is not free of drawbacks. The most important are the negative influence of interfering signals, which is not considered with only two microphones \cite{chung1980transfer}, some viscothermal wave attenuation effects \cite{aabom1988error} and incoherence of the forward and backward signals, the last two being common in waves propagating in flows \cite{dokumaci1997note,holland2000measurement}.

Some of these problems can be addressed using another method denominated beamforming,  based in phased sensor arrays, which has been successfully used for the study of acoustic and pressure fields in ducts with flow \cite{benajes2001estimation,desantes2001experiments,torregrosa2005note}.

The beamforming method can be applied to arrays of two or more microphones, weighting the signal of each one to ``steer'' the sensitivity of the array in the direction of interest (in this case, upstream and downstream of the pipe). A Linearly Constrained Minimum Covariance algorithm was applied in this case to find such weights (see \cite{pinero2000estimation} for details).

The experimental installation includes thus an array of piezoelectric pressure sensors flush-mounted in an extension of the compressor inlet pipe. These sensors were used to test the beam-forming method, the two-microphone method, and the sound pressure level (SPL) of a single sensor.
A commercial intensity probe (B\&K Type 4197) consisting of two phase-matched \nicefrac{1}{2}'' microphones was also installed at the inlet mouth as a reference, surrounded by acoustic insulation mats in order to avoid reflections from the walls or other components.  Fig. \ref{fig:setup} shows this experimental setup.

\begin{figure*}[t!]
\centering
\includegraphics[width=1\textwidth]{figures/cap_motor/setup}
\caption{Experimental setup for the acoustic measurements.}
\label{fig:setup}
\end{figure*}

Before applying the selected methods to the whole chart of the compressor, it is necessary to evaluate the restrictions that they impose in the frequency domain. While the SPL is only limited by the frequency response of the particular sensor, the decomposition algorithms of both the two-microphone method and the beamforming method present some common additional restrictions.

The first one is that both methods are formulated considering plane wave propagation, and are thus theoretically restricted to the frequency range below the onset of higher order modes. According to Eriksson \cite{eriksson1980higher}, the cut-off frequencies for the first asymmetric and first radial modes can be approximated by

\begin{equation}\label{eq:modes}
	f_\text{asim}=1.84\frac{a_0}{\pi D}\sqrt{1-M_0^2}\qquad
	f_\text{radial}=3.83\frac{a_0}{\pi D}\sqrt{1-M_0^2}
\end{equation}

being $D$ the pipe diameter. Also, around 10 diameters of straight pipe should be included at both sides of the array, and the nodes of the standing waves corresponding to \nicefrac{1}{3} and \nicefrac{1}{5} of the pipe length must be avoided.

\begin{figure*}[tb!]
\centering
\includegraphics[width=0.95\textwidth]{figures/cap_motor/espectros}
\caption{Inlet in-duct sound spectra estimated by the three considered methods (SPL, beamforming and two-microphone) and orifice sound intensity spectrum as measured by the B\&K intensity probe.}
\label{fig:espectros}
\end{figure*}

Another restriction is related to the spatial resolution of the consecutive sensors \cite{seybert1988two}, which imposes a cut-off frequency similar in concept to a Nyquist limit for spatial (rather than temporal) sampling, based on the separation ($d_s$) of the sensors. Above this frequency limit, erroneous spatial aliasing effects will be introduced. A simple approximation for this criterion is:

\begin{equation}\label{eq:alias}
	f_\text{asim}=\frac{a_0}{2 d_s}
\end{equation}

In the low frequency range, it is necessary to consider that at sufficiently large wavelengths, the signal differences between consecutive sensors decrease below their sensitivity limit, invalidating the results. With the separator used between the microphones of the intensity probe this limit is at 250 Hz.

Fig. \ref{fig:espectros} shows the in-duct sound spectra computed with the three proposed methods: SPL with one sensor, two-microphone approach, and beamforming with three probes. The spectra measured at the orifice by the reference probe is superimposed in black. In this region, the beamforming method appears to follow the reference more closely.

It is also shown how spurious aliasing peaks occur at the frequency predicted by Eq. \ref{eq:alias} when more than one sensor is used and how the onset of the 1st radial mode predicted by Eq. \ref{eq:modes} marks a clear difference between the in-duct sound field and the sound field radiated by the inlet orifice.

\begin{figure*}[b!]
\centering
\includegraphics[width=0.95\textwidth]{figures/cap_motor/mapasSIL_250-2650}
\caption{In duct SIL maps of the compressor noise estimated by the beamforming method (top left), twomicrophone method (top right) and SIL map of orifice noise as measured by the probe (top center), including correlations between the three measurements (bottom).}
\label{fig:mapasSIL_250-2650}
\end{figure*}

The results of applying Eq. \ref{eq:intensity} to compute SIL in the plane wave region of the spectra at different operating conditions of the compressor are shown in Fig. \ref{fig:mapasSIL_250-2650}. Absolute levels at each working point were interpolated in order to get a continuous map. The beamforming and the two-microphone methods were used for in-duct SIL estimation and the intensity probe for measuring the reference orifice SIL. Correlations of these measurements are also provided, showing that in the plane wave range the beamforming method appears to perform up to 30\% better in terms of $R^2$.

Fig. \ref{fig:mapasSPL_250-2650} shows a comparison between the SPL computed by selecting only one sensor of the array and the reference SIL of the probe. The correlation of these two measurements is better in terms of $R^2$ than the two-microphone method but still worse than the beamforming method.

\begin{figure*}[tb!]
\centering
\includegraphics[width=0.95\textwidth]{figures/cap_motor/mapasSPL_250-2650}
\caption{Comparison of compressor sound maps (plane wave region) obtained through interpolation of SPL at the inlet 1st sensor and SIL at the probe, including a correlation between both.}
\label{fig:mapasSPL_250-2650}
\end{figure*}

However, it should be noted that simple sound pressure level measurements can be influenced by the complex geometry of actual production pipes, which are remarkably more complex than the straight pipe used in this study. Also, it can be seen in Fig. \ref{fig:mapasSPL_250-2650} that, while at low pressure the sound maps are quite similar, SPL of one probe predicts a high sound level at high pressure ratio and high mass flow conditions, which is not the case for the intensity probe.

These high sound level points at high pressure ratio can be clearly seen as points lying out-side the 95\% observation confidence bounds of the correlation in Fig. \ref{fig:mapasSPL_250-2650}. A similar problem can be seen in the correlation of the two-microphone method in Fig. \ref{fig:mapasSIL_250-2650}, in which some outliers appear to dis-tort the map at some operation zones. The beamforming method, however, appears to be more robust.

\subsection{Radiated noise measurement}
\label{sub:radiated_noise_measurement}

Besides in-duct measurements, another acoustic characterization approach consists in the measurement of noise radiated by the turbocharger and its pipes, in most cases using free field microphones at a given distance from the setup.

This method has the advantage of allowing measurements with the real geometry, without any modification in the pipes to mount the in-duct measurement sensors. However, for accurate measurements it is desirable that the test cell be anechoic in order to prevent disturbances in measurements caused by reflections in the engine cell walls and resonances created by the cavities of the engine or its ducts that appear as other noise sources.

As testing the whole engine setup requires a large and expensive anechoic chamber, the setup is sometimes reduced to just the turbocharger group, which is powered by an external compressor or electric motor, losing in this way the possibility of performing on-engine tests.

However, a novel procedure based on measuring acoustic particle velocity level (PVL) may solve these issues \cite{jacobsen2005comparison} and enable on-engine testing in non-anechoic environments. This method makes use of two microscopic filaments acting as hot-wire flowmeters. Measuring the small voltage drop between the two, it is possible to derivate the acoustic particle velocity in the plane of the sensor.

Such a method, in this case a commercial solution providing acoustic holography-like maps of radiated noise level \cite{comesana2013scan}, has been tested on the (non-anechoic) engine test cell. The measuring probe of this device houses the two filaments on a MEMS chip and a traditional microphone, allowing the comparison of the performance of acoustic particle velocity measurements against traditional free-field sound pressure level (SPL) measurements.

\begin{figure*}[tb!]
\centering
\includegraphics[width=0.95\textwidth]{figures/cap_motor/PVL}
\caption{Near field noise holograms (800--2000 Hz) obtained through SPL (top) and PVL (bottom), including detail of the spectra of two regions of interest: turbocharger group (blue) and compressor inlet (red).}
\label{fig:PVL}
\end{figure*}

Fig. \ref{fig:PVL} shows this comparison. SPL (top) and PVL (bottom) maps of the measured engine environment are provided by tracking the probe along the near field of the surfaces of interest. The frequencies considered in these maps have been restricted to the 800--2000 Hz range, where it is common that a broadband emission known as ``whoosh noise'' is produced by the turbocharger \cite{evans2005minimizing}. For two particular regions (turbocharger group and inlet duct, marked on the maps) the full pressure (top) and particle velocity (bottom) spectra are also provided.

The particle velocity technique clearly distinguishes the turbocharger as the source of an 800--2000 Hz broadband noise at this particular operating condition, identified as whoosh noise. The pressure method also appears to identify the turbocharger as the source, but the contrast with the surrounding areas is much less pronounced. This can be clearly seen in the SPL spectra, in which the level of this frequency band is similar in both zones (turbocharger and inlet in this case). In the case of the PVL spectra the turbocharger contribution is visibly higher.

However, the pressure method appears to be more sensitive to reflections in the near field and was thus unable to provide a clear identification. It should be noted that if characterization through in-duct techniques of this particular whoosh noise is intended, special attention should be placed on the outlet pipe, as this phenomenon is often found exclusively downstream of the turbo-charger \cite{broatch2014methodology}.

\section{Conclusions}
\label{sec:conclusions}

Characterization of the acoustic performance of a given compressor under engine conditions must start with the definition of test means. If an anechoic chamber capable of housing the entire engine is available, then free field microphones can be used to estimate the acoustic radiation, but even in this case it would be difficult to isolate the contribution of the compressor.

In order to quantify the compressor noise emission, an intensity probe can be located at the inlet orifice. However, intensity probes are quite expensive, and sufficient acoustic insulation is necessary to prevent reflections. Also, characteristics of the internal field that are not externally propagated will be lost.

In-duct noise measurements offer a cost-effective way to characterize the sound field generated by the compressor, if the ducts can be modified so as to include straight measurement sections. Single-sensor SPL may be used to investigate the upper frequency bands, but generally sound intensity is a better and more meaningful metric of the acoustic power output of the source.

Calculation of in-duct sound intensity can be achieved by using an additional consecutive sensor and decomposing the waves through the two-microphone method. However, the addition of a third sensor and the decomposition through the beamforming method appears to be more reliable along the full plane wave range, and especially in the lower frequencies (250--1000 Hz).

However, the restrictions of these intensity-based methods must be carefully considered, specially the influence of higher order modes and the aliasing frequency limit introduced by the spatial sampling imposed by the sensor spacing.

Regarding the external radiated noise, the particle velocity method has been able to provide more contrast than the pressure method and to correctly identify the turbocharger as the source of noise radiation at that particular ``whoosh noise'' band, in accordance with previous experience.

The particle velocity method could therefore be useful to reconstruct noise maps similar to those obtained using in-duct measurements, isolating the signal from the acoustic near field of the area of interest at different working conditions, without the need of modifying the original duct geometry or ensuring anechoic conditions. More research is being carried out on this topic.
