%!TEX root = Tesis.tex
\chapter{Noise measurement methodology} \label{cap:metod}

% Quote
% \null\vspace{-10mm}
% \begin{flushright}
% \begin{minipage}[t]{0.66\textwidth}
% {\itshape\small``The story is set in the year 1922 and has its main character, Mr Engelbert, moving to Prague from the countryside to live out his retirement years and struggling to cope with the aggressive modernity of the city. The horror is not the power of money or the arrogance of the people, but the noise; not the age-old noise of a thunderstorm or a hammer, but the new noise of engines, especially of automobiles and motor-cycles, the explosive internal combustion monsters.''}\\[-3mm]

% {\cabincondensed\small\hfill \textbf{Milan Kundera}, \emph{The Curtain.}}\\[5mm]
% \end{minipage}
% \end{flushright}

% Chapter text
\noindent As stated in the prior chapter, one of the main objectives of the present work has been to develop and implement robust measurement methodologies for the acoustical characterization of turbocharger compressors.
\blfootnote{Work on this chapter has been partly published in the following paper:
\begin{itemize}
\item \emph{Experimental methodology for turbocompressor in-duct noise evaluation based on beamforming wave decomposition} \cite{torregrosa2016experimental}
\end{itemize}
}

In this chapter the theoretical background of these methodologies will be presented, along with the details of their practical implementation and a brief review of the facilities at where they have been implemented.

Apart from the purely experimental implementation, the step-by-step signal processing required to produce final, compiled results will be introduced as well, including sample results in order to illustrate each step.

Special care will be put in considering the different theoretical and practical limitations and restrictions of experimental measurements and processing techniques, to ensure that results are significant and robust.

\section{Facilities}

Most of the experimental work presented in this thesis has been carried out at laboratory 5K of Universitat Politècnica de València. This facility is equipped with two main installations.

The first one is a flow bench powered by a Diesel engine linked to an asynchronous dynamometer and a screw compressor. This flow bench can be configured to function as a turbocharger test stand or to provide air flow to other facilities. Subsection \ref{sub:flow_bench} will be dedicated to an extended description of this installation.

Next to the flow bench, there is a large anechoic chamber that can be configured to either receive external air flow from the flow bench or to host a standalone engine by itself. For this project, air was provided by the flow bench when the turbocharger test stand was installed inside the chamber. This arrangement is illustrated by Fig. \ref{fig:floorplan}, while subsection \ref{sub:anec_chamber} will describe in detail this particular chamber.

Additionally, an engine test cell located on laboratory 6D has been used for the on-engine validation tests described on chapter \ref{cap:engine}, where the inlet intake of a commercial engine was modified to measure the compressor acoustic emission.

Other facilities included an impulse test bench that was used for sensor calibration, and a workshop were the required customized parts for measuring and instrumenting were manufactured.

\begin{figure}[t!]
\centering
\includegraphics[width=\textwidth]{figures/cap_metod/schematic}
\caption[Floor plan of laboratory 5K]{Floor plan of lab 5K, containing a flow rig and its associated turbocharger installation inside an anechoic chamber with their auxiliary systems.}
\label{fig:floorplan}
\end{figure}

\subsection{Flow test rig} \label{sub:flow_bench}

The flow test rig has been used as flow provider to power the turbocharger turbine both directly, with the turbocharger installed in the flow rig cell itself and, as shown in Fig. \ref{fig:floorplan}, with the turbocharger mounted in a supporting frame inside the anechoic chamber.

At the core of the rig there is a heavy duty MIDR 06.20.45 Diesel engine able to provide up to 260 kW of power. This engine is linked to an asynchronous dynamometer and, through a coupling gear, to an Atlas Copco ZA 110-3.5 volumetric screw compressor. An overview of the whole installation can be seen in Fig. \ref{fig:flowrig_render}.

\begin{figure}[b!]
\centering
\includegraphics[width=\textwidth]{figures/cap_metod/flow_rig2}
\caption[Render of the flow rig used to power the turbocharger]{Render of the flow rig showing the engine core, asynchronous dynomometer with its support cabinet, screw compressor and air distribution system, including control valves and reservoirs.}
\label{fig:flowrig_render}
\end{figure}

Purpose of this compressor is twofold. First, it supercharges the Diesel engine as its original turbocharger group was removed. Second, compressed flow can be routed to either a turbocharger installed in the same cell or to the adjacent chambers.

Another possibility of this test rig is to directly feed the exhaust gases of the engine to an external turbocharger. This is the reason why the original turbocharger was removed during installation. Direct flow will contain the pulsating content of the engine, allowing simulation of real engine conditions.
The number of cylinders firing can be selected from 1 to 6 in order to simulate different pulsating content. Also, if the hot exhaust gas is desired but the pulsations are not, flow can be routed through a settling tank that will smooth them out, resulting in continuous flow.

For the studies carried out in this facility during the development of this thesis, pulsating flow was not selected. Continuous flow was used in both the test rig and the anechoic chamber. However, work presented in chapter \ref{cap:engine} used real engine conditions to validate the proposed techniques and obtained results.

As for the flow control, another reservoir equipped with a discharging valve can be used to distribute the compressed air between the engine inlet manifold and the external circuit, which can be either directed to another chamber or vented to the outside ambient through a silencer.

If cold flow is desired, a water intercooler can be used to lower the temperature produced by the compression process. This temperature, along with the rest of flow properties like mass flows, pressures and other temperatures, are recorded through a dedicated control system.

Engine and dynamometer are controlled though their own independent instrumentation, while the instantaneous turbocharger variables, including those needed for acoustical characterization, are recorded through the dedicated high speed Yokogawa DL716 DAQ system.

\subsection{Anechoic chamber} \label{sub:anec_chamber}

The anechoic chamber at laboratory 5K is a solid $9.5 \times 9 \times 7.2$ m reinforced concrete structure, isolated from the rest of the building and resting on a waterproof support that dampens outside vibrations.

The internal walls of the chamber are completely covered in fiberglass wedges of 0.8 m of longitude and a square base of $0.2 \times 0.2$ m. These wedges leave an interior space of $7.5 \times 6.5 \times 6$ m. Usually a further 1 m in each direction is left as an additional safety margin, the rest of the volume being equivalent to acoustic free field conditions.

Such free field conditions, where no acoustic reflection of the walls occurs, are certified in this chamber up to a cut-off frequency of 100 Hz. Low frequency waves may start to reflect below this limit.

A lightweight, acoustically-transparent, grated metallic floor is situated over the floor wedges, supporting personnel and light equipment. Four round steel pillars affixed to the concrete floor are available to support heavy loads such as engines.

\begin{figure}[t!]
\centering
\includegraphics[width=\textwidth]{figures/cap_metod/anechoic}
\caption[View of the anechoic chamber with the turbocharger setup]{View of the inside of the anechoic chamber, showing the turbocharger setup with the mounting frame and the pipes required to operate the installation.}
\label{fig:anechoic_chamber}
\end{figure}

During the work described in this thesis, the load-bearing pillars were unused, since the turbocharger frame was lightweight enough to be supported by the grated floor. Figure \ref{fig:anechoic_chamber} shows the tested turbocharger installed in this frame.

In order to allow the operation of the turbocharger, compressed air flow is fed to the turbine from the flow rig described in the prior subsection. The turbine discharges to a reservoir equipped with a conical nozzle that decreases velocity and acoustic emissions.

Meanwhile, the compressor takes ambient air from the chamber, while its exhaust is redirected to the adjacent auxiliary room for discharging, as seen in Fig \ref{fig:floorplan}. A closed-loop lubricating oil system is also present. Water is also provided to the cooling jackets of the sensors that require it, specially those located on the turbine pipes and the compressor outlet.

A control system similar to that of the flow rig monitors the operation of the system and gathers operational data (pipe temperatures, pressures, mass flow, compressor speed, etc.), allowing regulation of oil temperature, and compressor operation settings (speed and mass flow). Scientific data (mainly dynamic pressure) is recorded with a Yokogawa DL716 DAQ system.

\subsection{Engine test cell} \label{sub:engine_cell}

As mentioned early in the chapter, while most of the experimental work was carried out using the two facilities described already, the measurement methodology was later put to the test on an engine test cell to verify its reliability on real engine conditions.

The test cell is equipped with an asynchronous electromagnetic dynamometer that allows both steady and transient automatic testing. Operative parameters and control for the engine and dyno are integrated into a custom system, including gravimetric fuel consumption sensors, hot plate anemometers, emission measurements and temperature and pressure control of the air pipes and coolant flows.

Access and online control of the engine ECU is also available, and a special pneumatic system can be used to inject compressed air into the inlet manifold downstream of the compressor.

This air injection reduces the air that the compressor ingests, thus lowering compressor air flow and allowing the control of this relevant parameter independently of the engine necessities. The system was used to bring the compressor to surge and near-surge conditions during our research.

In order to implement and validate the in-duct acoustical measurement methodology presented in this chapter, additional sensors and equipment were fitted into the test cell, including a Yokogawa DL716 high speed DAQ system and a Brüel \& Kjær PULSE\textsuperscript{TM} DAQ system with a type 4197 pressure-pressure sound intensity probe used as reference.

\begin{figure}[b!]
\centering
\includegraphics[width=\textwidth]{figures/cap_metod/sala_limpia}
\caption[View of the inside of the engine test cell]{View of the inside of the engine test cell with the engine in its default configuration, before being modified for the acoustic measurements described in chapter \ref{cap:engine}.}
\label{fig:engine_cell_clear}
\end{figure}

\section{Tested turbochargers}

Different turbochargers have been used in the development of the investigation, as dictated by the requirements of the different projects and facilities that this thesis comprises. All of them are however similar, small to medium sized automotive turbochargers. The most relevant model is the Garrett GT17 model that was used in chapters \ref{cap:metod}, \ref{cap:results}, \ref{cap:cfd} and section \ref{sec:geom_simple}. 

\begin{figure}[h!]
\centering
\includegraphics[width=0.92\textwidth]{figures/cap_metod/comp_dimensions}
\caption[Dimensional drawing of the compressor]{Dimensional drawing of the compressor used in most of the development of the investigation, including some relevant dimensions.}
\label{fig:comp_dimens}
\end{figure}

This particular model, featuring six main and six splitter blades, with a vaneless diffuser, was digitized to obtain its detailed geometry, which is shown in Fig.~\ref{fig:comp_dimens} along with its most relevant dimensions. Similar turbochargers were used for engine tests on chapter \ref{cap:engine} and section \ref{sec:geom_motor} and another for chapter \ref{cap:inlet} and section \ref{sec:geom_elbows}, both of them modified in order to fit different inlet geometries.

\section{In-duct noise measurement} % Articulo methodology
\label{sec:method_induct}

In this section, a methodology is proposed to evaluate the noise emission of the compressor through its inlet and outlet pipes across selected zones of the compressor map of operating conditions. It allows a simple and clear visualization of noise phenomena and their frequency distributions at all possible operation points of the turbocharger system.

Although compressor manufacturers sometimes provide these kind of ``noise maps'' \cite{gaude2008experimental}, they usually refrain from providing an adequately referenced and step-by-step methodology to reproduce them.

Reviewing the existing literature on turbocompressor noise a variety of very different measurement techniques can be found, from the two-sensor, in-duct approach of Tiikoja et al \cite{tiikoja2011inves} to simple single-sensor pressure levels and external commercial noise-meters \cite{figurella2012noise}.

However, these works focus on the research of different phenomena (transmission loss \cite{tiikoja2011inves}, effect of flow incidence angle \cite{figurella2012noise}, sound generation by rotating stall \cite{mongeau1993sound}, source characterization \cite{mongeau1995method}, etc.), not on the acoustical methodology itself nor on the particular setup considerations and restrictions that each measurement technique imposes.

In this chapter these shortcomings in existing literature are addressed by proposing a suitable experimental methodology and discussing the details of its theoretical background, its implementation, restrictions and processing required to produce useful results.

\subsection{Theoretical background}
\label{sec:theory}

The first step is thus to review the theoretical background of the determination of acoustic output in the specific situation of ducts with mean flow, and decide techniques to experimentally obtain the required data for their implementation.

\subsubsection{Sound intensity in flow ducts}
\label{sub:sound_intensity_in_flow_ducts}

While it is possible \cite{figurella2012noise} to rely on a single sensor to measure the scalar \emph{sound pressure level} (SPL) of the flow at a certain location of a duct, more sophisticated approaches are needed to estimate the \emph{sound intensity level} (SIL) that is propagating through the duct.

It is important to consider that the scalar magnitude of sound pressure level at a point can be influenced by the precise geometry of that section and by nodes and reflections that may occur in the duct, whereas sound intensity is a vectorial magnitude that remains almost constant along the duct (except for small dissipation losses), making it a more robust indicator of noise emission.

The basis of such intensity measurement methods is to consider that the pressure signal $x(t)$ measured at a given point of a duct is the linear superposition of a pressure wave $x^+$ travelling downstream and another pressure wave $x^-$ travelling upstream (sometimes referred to as forward and backward waves, respectively \cite{pinero2000estimation}) so that:

\begin{equation}
  x(t)=x^+(t)+x^-(t)
\end{equation}

Comparing the pressure information at two or more spatial positions it is thus possible to infer how the waves are propagating along the duct in each direction, a possibility unavailable by means of a single sensor.

When using this wave decomposition approach, the sound intensity can be estimated following the expression proposed by Morfey \cite{morfey1971sound} and successfully derived by Dokumaci \cite{dokumaci2000calculation} from physical principles:

\begin{equation}\label{eq:metod_intensity}
    I=\frac{1}{\rho a}\left(|\mathbf X^+ |^2(1+M)^2-|\mathbf X^- |^2(1-M)^2\right)
\end{equation}

Here $\mathbf X^+$ and $\mathbf X^-$ are, respectively, the complex spectra of the downstream and upstream pressure waves, $\rho$ is the mean density, $a$ the mean sound speed, and $M$ the mean Mach number of the flow.

\subsubsection{Beamforming wave decomposition}\label{sub:method_beamformer}
For a more intuitive comprehension of the beamformer, one can consider its transmitting equivalent: a phased array emitter where the same signal $x(t)$, with its phase shifted by a quantity $w_n^*$, is fed to an $n$-element linear array of fixed transmitters, forming a plane wave emitted in the $\theta$ direction which can be steered at will adjusting each $w_n^*$.
The beamforming method provides a way to mathematically tune the sensitivity of the overall system in order to isolate the downstream and upstream information of the acquired signal. A scheme for a narrowband beamformer can be found in Fig. \ref{fig:schematic}.

\begin{figure}[t!]
\centering
\includegraphics[scale=0.85]{figures/cap_metod/NB_Beamformer}
\caption[Diagram of a narrowband beamformer]{Diagram of a narrowband beamformer with three elements tuned to a Direction of Arrival (DOA) of $\theta$.}
\label{fig:schematic}
\end{figure}

The approach in this caase is the inverse: by tuning the weights $w_n^*$ that multiply each recorded pressure signal $x_n(t)$ the pressure wave coming from the direction of arrival (DOA) $\theta$ that we are interested in may be resolved. Details can be found in \cite{van1988beamforming}.

\begin{figure}[t!]
\centering
\includegraphics[width=0.8\textwidth]{figures/cap_metod/Decomposition_t}
\caption[Sample result of decomposed pressure]{Sample result of the decomposed pressure; at the inlet the amplitude of the pressure wave travelling downstream (towards the compressor) is smaller than the one coming from the compressor; the reverse situation occurs at the outlet.}
\label{fig:decomposition}
\end{figure}

\begin{figure}[t!]
%\centering
\includegraphics[width=0.90\textwidth]{figures/cap_metod/Spectra}
\caption[Sample of raw inlet and outlet pressure PSD]{Sample of raw inlet and outlet pressure power spectral densities (at 160 krpm and 60 g/s), including frequency limitations described in equations \ref{eq:f_nyquist} and \ref{eq:f_modes}. The characteristic Blade Passing Frequency (BPF) can also be identified, specially in the outlet spectra.}
\label{fig:spectra}
\end{figure}

The extension to a wideband beamformer is relatively straightforward \cite{piper2011broadband}. First the Fast Fourier Transform (FFT, denoted by $\mathcal F$) of the recorded pressure signals is computed:

\begin{equation}
  \mathbf X_n(f_k)=\mathcal{F}\{x_n(t_k)\}
\end{equation}

Subscript $k$ indicates that the signal is acquired at discrete time steps. The described narrowband procedure is then followed for each discrete frequency to finally obtain the desired signal through the inverse transform:

\begin{equation}
    x(t)=\mathcal{F}^{-1}\left\{\mathbf w^H(f_k)\; \mathbf X(f_k)\right\}
\end{equation}

Here, $\mathbf X(f_k)$ is the $k \times n$ matrix of transformed signals and $\mathbf w^H(f_k)$ is the matrix of weights for each frequency. In order to compute the optimal weights for the desired DOAs ($\theta=-90^o$ for downstream and $\theta=90^o$ for upstream waves, assuming a typical flow DOA of $\theta=90^o$), several schemes can be used.

For this study a Linearly Constrained Minimum Variance (LCMV) beamformer was  used, a well established procedure, which aims at minimizing the overall output power (variance) of the signal, while maintaining unitary gain in the precise desired direction.

While several beamforming strategies and implementations are available, annex \hyperref[ann:beamformer]{3.A} offers a simple guide on how to compute $\mathbf w^H(f_k)$ and thus $\mathbf X^+$ and $\mathbf X^-$ using this particular approach. Figure \ref{fig:decomposition} shows the result of this decomposition in the time domain.

Further information on its application, including a comparison of various of these schemes against simulated and experimental data is available in the work of Piñero et al. \cite{pinero2000estimation}.

\subsection{Experimental procedure}
\label{sec:experimental_procedure}

Once that a suitable in-duct noise measurement technique was  selected, it is necessary to analyze how to implement the procedure in the experimental setup, perform the measurements, and process the resulting data.

\subsubsection{Turbocharger instrumentation}
\label{sub:turbocharger_instrumentation}

The correct design and installation of the fast pressure sensor arrays is a critical part of the experimental setup.

Ideally, the sensors should be mounted flush (with water-cooled adapters if necessary) in straight pipes, with diameters adjusted to preserve the cross sectional area of the adjacent duct in order to avoid complex structures caused by singularities. A sample of the expected pressure spectra results can be seen in Fig.~\ref{fig:spectra}.

\begin{figure}[b!]
\centering
\includegraphics[scale=0.80]{figures/cap_metod/setup_t}
\caption[Scheme of the proposed turbocharger instrumentation setup]{Scheme of the proposed turbocharger instrumentation setup, highlighting in color the three-sensor beamforming arrays and their clearance requirements.}
\label{fig:setup_schem}
\end{figure}

As a further measure, enough straight clearance should be left before and after the array (as shown on figure \ref{fig:setup_schem}), to ensure essentially plane wave flow at the measurement section \cite{holland2000measurement}. There should be a minimum of 6 diameters upstream of the array and a further 4 diameters downstream, as recommended by Torregrosa et al. \cite{torregrosa2005exp}.

When placing the transducers, it is important that the \nicefrac{1}{5} and \nicefrac{1}{3} nodes of the established standing wave pattern are avoided \cite{dowling1983sound}. Also, a compromise in the spacing $d_s$ between them must be reached, taking into consideration errors in the  low-frequency \cite{aabom1988error} and high-frequency \cite{seybert1988two} bands.

Specially, it should be noted that spatial aliasing effects will occur above the Nyquist frequency imposed by the spacing $d_s$. The time of arrival of a wave with DOA $\theta$ and speed $a$ to two consecutive sensors spaced by a distance $d_s$ is:
\begin{equation}
  T_d = \frac{d_s \sin \theta}{a}
\end{equation}

This imposes a Nyquist-type frequency criterion, as the measured wave frequency $f_n$ should be maximum half the frequency $f_d=1/T_d$ naturally associated with $T_d$ to ensure that no high frequency spatial aliasing effects are present:
\begin{equation}\label{eq:f_nyquist}
  f_n \leq \frac{f_d}{2} = \frac{1}{2T_d}=\frac{a}{2d_s \sin \theta}=\frac{a}{2d_s}
\end{equation}

A further restricting phenomenon is the acoustic mode propagation. As the beamforming method proposed is based on the assumption of one-dimensional wave propagation, it will become unreliable with the onset of 3D effects.

According to Eriksson \cite{eriksson1980higher}, it may be assumed that waves in a circular duct propagate in a planar fashion (i.e., pressure is constant in every cross-sectional area of the duct) below a cut-off frequency of:
\begin{equation}\label{eq:f_modes}
f_a = 1.84 \frac{a}{\pi D}\sqrt{1-M^2}
\end{equation}

Here, $a$ is the sound speed, $D$ the duct diameter and $M$ the mean Mach number. At higher frequencies, the so-called first asymmetric mode will start to propagate, thus rendering the assumption of one-dimensional propagation invalid. The first circular mode will in turn propagate beyond a cut-off frequency of:
\begin{equation}\label{eq:f_modes_circ}
f_c = 3.83 \frac{a}{\pi D}\sqrt{1-M^2}
\end{equation}

Special care should be taken when computing limiting frequencies $f_n$, $f_a$ as they are not constant but dependent on the operating condition, since some of their components vary with temperature and mass flow. Figure \ref{fig:spectra} shows both limits for a certain operating point.

As the figure shows, the Nyquist criterion accurately pinpoints where differences between sensors start to grow, and it is clearly a more restrictive criterion than the acoustic mode onset. This limit should be properly computed for each condition when multi-sensor decomposition techniques are used.

Also, figure \ref{fig:spectra} shows other phenomena, for instance the Blade Passing Frequency (BPF) tonal noise that is identified in many of the works presented in chapter \ref{cap:liter}. As shown in table \ref{tab:lit_gaude_sources}, the main BPF can be estimated by taking into account the shaft speed and the number of main blades:
\begin{equation}
  \text{BPF} \sim 160\,\text{krpm}\,/\,60\,\frac{\text{s}}{\text{min}} \times 6\,\text{main blades} \sim 16\,\text{kHz}
\end{equation}

It is also important to note that the closer the sensors are mounted together for increasing the $f_n$ limit, the poorer the spatial resolution of low frequency waves will be, as the wavelengths $\lambda=a/f$ become too long and the difference between consecutive sensors decreases below the noise threshold. This effect is intrinsically related to each particular sensor model, but it must be considered if very low frequencies are of interest.

Finally, it is essential to ensure that the response of the three sensors of each array is coherent. Before mounting the sensors in the final linear array form, they should be mounted radially in the same section of a duct and individually calibrated, to ensure that the difference between sensors is not due to individual transducer differences, but only to wave propagation.

In this case, the calibration is performed in an impulse test rig (described in detail in Payri et al. \cite{payri2000modified}) so as to obtain a clear reference pressure wave in all sensors. Both the time and frequency responses are observed, and both the amplifier offset and gain are adjusted to find the best match. An example can be seen on figure \ref{fig:6031_calibration}.

\begin{figure}[t!]
\centering
\includegraphics[width=0.85\textwidth]{figures/cap_metod/6031_calibration_t}
\caption[Results of the calibration of the outlet sensors]{Results of the calibration of the outlet sensors in an impulse test rig, showing a pressure measurement match in both time and frequency domains.}
\label{fig:6031_calibration}
\end{figure}

In addition to the fast pressure sensors data, more information is required to compile the compressor operating conditions map. These maps are usually presented as plots of the operation points on a total-to-total pressure ratio $(\Pi_\text{TT})$ versus corrected air mass flow $(\dot m^*)$ axes.

\subsubsection{Steady state measurements}
\label{sub:steady_state_measurements}

The usual approach for measuring the operating points of the compressor map involves reducing the air flow by progressively closing the back-pressure valve (shown in figure \ref{fig:setup_schem}) until the desired air mass flow is reached, while the turbine power is adjusted to preserve a given corrected compressor speed $(N^*)$ and to achieve stable operation conditions.

Eventually the compressor blades will begin to stall, until a deep surge condition is reached, as shown in the top plot of figure \ref{fig:surge_datapoints}. In order to avoid this destructive condition, the valve will be opened again and another constant corrected regime line will be measured, until the desired region of the compressor map is completely covered.

Several reference conditions can be selected when correcting the speed and the air mass flow; in this study the mean sea level standard day (defined as one of 15ºC of temperature and 1 atm of pressure) of the International Standard Atmosphere will be used:
\begin{equation}
    \dot m^* = \dot m\frac{101325}{p_{T}}\sqrt{\frac{T_{T}}{288.15}}
    \qquad N^*=N\sqrt{\frac{288.15}{T_{T}}}
\end{equation}

Here, $p_{T}$ and $T_{T}$ refer to the compressor inlet total pressure and temperature, respectively. These are calculated with the air mass flow and the cross sectional area of the inlet duct, using the continuity and ideal gas relations and assuming adiabatic flow:

\begin{figure}[bthp!]
\centering
\includegraphics[width=0.85\textwidth]{figures/cap_metod/datapoints_surge.pdf}
\caption[Operating conditions for each point and surge criterion]{Operating conditions for each measured data point (right plot), including the precise determination of the deep surge limit: top plot on the left shows the temporal evolution of the raw relative pressure recorded by one piezoelectric sensor; bottom plot on the left depicts the correlation between the frequency content in the 0-20 Hz band and deep surge onset.}
\label{fig:surge_datapoints}
\end{figure}

\begin{equation}
    \rho =\cfrac{p}{RT} \quad
    U =\cfrac{\dot m}{\rho A} \quad
    T_{T}=T+\cfrac{U^2}{2c_p} \quad
    p_{T}=p\left(\cfrac{T_{T}}{T}\right)^\frac{\gamma}{\gamma-1}
\end{equation}

In this case $R$ denotes the air-specific ideal gas constant and $\rho,p,T$ and $U$ the density, pressure, temperature and flow speed in the duct, respectively. Also, $A$ denotes the cross sectional area, and $c_p$ the air-specific heat capacity.

Figure \ref{fig:surge_datapoints} (bottom) shows the recorded data points of the compressor map, after computing the total-to-total pressure ratio and
applying the standard day correction.

\subsubsection{Surge limit characterization}
\label{sub:surge_limit_characterization}

When measuring a turbocharger operation map, it is particularly important to  characterize accurately the surge limit, especially when an acoustic evaluation is sought out \cite{galindo2006surge}.

In order to measure different mass flows, the back-pressure valve is usually closed slowly until a certain surge criterion is met, and then opened to avoid damaging the turbocharger. Several surge criteria for centrifugal turbomachinery have been proposed in the literature, for tests on flow benches \cite{galindo2006surge} and on engine test rigs \cite{galindo2013engine}.

An average of power spectrum of a single pressure sensor in the 0 to 20 Hz range, shown by Galindo et al \cite{galindo2006surge} to be a good surge indicator, was computed once for discrete time steps $k$ of 0.5 seconds, providing a clear indication of deep surge onset, as marked in figure \ref{fig:surge_datapoints} (top) with a solid line.

\begin{equation}
 \mathbf X_k=\left|\mathcal F\{x(t)\}\right|^2,\; t\in[t_k,t_{k+1}]
\end{equation}

To obtain the required near-surge data points to be displayed on the map, represented by $\blacksquare$ in the bottom plot of figure \ref{fig:surge_datapoints}, a 1 second average of all recorded data was computed 1 second before surge as indicated by the dashed lines of the left plot, in order to calculate the map coordinates of the point.

\begin{figure*}[tp!]
\centering
\includegraphics[width=0.88\textwidth]{figures/cap_metod/gr_mapa_ruido_dB_1000-3000.pdf}
\caption[Sample of noise level map in the 1 -- 3 kHz band]{Sample of noise level in the 1 -- 3 kHz band obtained through the processing described in subsection \ref{sub:noise_maps_and_spectrograms}, including the measured data points $(\circ)$ and the paths later expanded as spectrograms shown in figure \ref{fig:spec50-5k}. The left hand side map represents the inlet duct, whereas the right hand side map represents the outlet duct. }
\label{fig:map1k-3k}
\end{figure*}

\subsubsection{Signal postprocessing: maps and spectrograms}
\label{sub:noise_maps_and_spectrograms}

In addition to the intensity spectra determined for each operating point, further results can be obtained by applying an interpolation procedure to the spectra associated with all the points tested.

For instance, it is possible to calculate the overall level of sound intensity for a given frequency range at each operating point $i$ of the compressor operating map. The sound intensity level at a characteristic band (between $f_1$ and $f_2$) can be calculated for each point as:
\begin{equation}\label{eq:point_level}
  L_i=10\log_{10}\left(10^{12}\sum_{f=f_1}^{f_2}I(f)\right)
\end{equation}

Once that several $L_i$ levels have been measured for different conditions of a selected map region, a cubic spline interpolation function $\Phi_{f_1-f_2}$ is used to project the data onto a much more fine and regularly spaced grid, allowing the representation of a map of noise intensity across the measured region of the compressor map, as shown in figure \ref{fig:map1k-3k}.

These ``noise maps'' synthesize the information of the acoustic output of a frequency band of interest in a simple and visual result, allowing an easy estimation of the noise emission through desired operating conditions even if these have not been specifically measured.

Expanding on the necessity of analysing the frequency content evolution as the operating conditions of the compressor change, a different approach can be used, where the measured discrete information can be compiled to estimate a continuous spectrogram.

This is accomplished by sequentially performing the previous map processing for a number $N$ of much smaller frequency bands of width $\Delta f$, resulting in a collection of $N$ interpolation functions $\Phi_{f_j-f_{j+1}}$, where $f_{j+1}-f_j = \Delta f$.

Using these functions, the intensity levels $\mathbf L$ on each small frequency band $k$ are interpolated for a set of $M$ pairs of the compressor map coordinates, such that:

\begin{equation}
 \mathbf L_j=\Phi_{f_j-f_{j+1}}(\bm{\dot m^*},\bm{\Pi_\text{TT}})
\end{equation}

An estimation of the spectrogram $\mathbf S$ that would be obtained in a quasi-stationary measurement by following the set of operation conditions $(\bm{\dot m^*},\bm{\Pi_\text{TT}})$ pairs can then be compiled gathering the levels $\mathbf L_j$. Choosing one coordinate of the set of pairs (in this case $\dot m^*$) as reference for the X axis:

\begin{equation}
  \mathbf S(f_j,\dot m^*)=\begin{bmatrix}
    \mathbf L_1(\dot m^*_1) & \hdots & \mathbf L_1(\dot m^*_M)\\
    \vdots & \ddots & \vdots \\
    \mathbf L_N(\dot m^*_1) & \hdots & \mathbf L_N(\dot m^*_M)
  \end{bmatrix}
\end{equation}

Thus the sound intensity level value $\mathbf S$ for each $\dot m^*$ and each frequency band $j$ can be plotted. An example of this processing for frequencies between 50 and 5000 Hz binned with $\Delta f = 50$ Hz is presented on figure \ref{fig:spec50-5k}, where coordinates $(\bm{\dot{m}^*},\bm{\Pi_\text{TT}})$ correspond to the 160 krpm iso-speed line highlighted in Fig.~\ref{fig:map1k-3k}.

\begin{figure*}[t!]
\centering
\includegraphics[width=0.85\textwidth]{figures/cap_metod/gr_spec_ruido_dB_50-5000.pdf}
\caption[Sample of spectrograms following the 160 krpm line]{Sample of spectrograms calculated using the procedure outlined in subsection \ref{sub:noise_maps_and_spectrograms}, following the 160 krpm paths marked in figure \ref{fig:map1k-3k}. The 1 -- 3 kHz band selected for the maps in figure \ref{fig:map1k-3k} is highlighted, along with some particular flow phenomena discussed in section \ref{sec:sample_meas}.}
\label{fig:spec50-5k}
\end{figure*}

\subsection{Measurement of the sample results}
\label{sec:sample_meas}

In order to illustrate the presented methodology along this section, experimental data was obtained by performing a measurement of the selected turbocharger in the flow bench that is used to provide flow to the anechoic chamber, configured as shown in figure \ref{fig:setup_schem}. Since in this case the only concern was the validation of the in-duct methodology, installation in the anechoic chamber was not required. 

Kistler type 7031 piezoelectric transducers were used for the inlet duct; the outlet duct was fit with type 6031 transducers due to available stock. An array spacing $d_s=32$ mm for the outlet and $d_s=50$ mm for the inlet was selected, allowing for a safe upper limit in terms of the spatial Nyquist criterion $f_n$ of $\sim$6 and $\sim$3.5 kHz respectively (precise values depend on the conditions of each point).

The variables needed in order to characterize the operating condition (mass flow, speed, temperatures and pressures) were averaged during 30 seconds using the in-house DAQ system, while the six piezoelectric sensors acquired $10^5$ pressure samples during 1 second using the Yokogawa DAQ system.

Pressure signals on recorded points (shown in figure \ref{fig:surge_datapoints}) were then processed according to the proposed methodology. Samples of the final results are presented in figures \ref{fig:map1k-3k} and \ref{fig:spec50-5k}.

Figure \ref{fig:map1k-3k} shows the 1 -- 3 kHz sound intensity level distribution across selected operating conditions, approximately between maximum efficiency and the measured surge limit. As shown in chapter \ref{cap:liter}, noise in this frequency is often known as \emph{whoosh}.

Another point of concern that has been identified in the results is that the relevance of this particular phenomenon extends far from the surge region and well into the expected engine operation region.

Looking at either the 160 krpm line on figure \ref{fig:map1k-3k} or at the expanded spectrograms of this line in figure \ref{fig:spec50-5k} where this band has been marked, there is a local maximum in this band between 60 and 70 g/s. For the 180 krpm points on the map of figure \ref{fig:map1k-3k}, the \emph{whoosh} noise may be heard even for up to 80 g/s mass flow rate.

Apart from \emph{whoosh}, both spectrograms show a clear rise in low frequency content when the mass flow is reduced and surge conditions start to appear, even if deep surge conditions have not yet been reached. This is consistent with spectrograms measured experimentally by Galindo et al \cite{galindo2009effect} and indicates that the methodology proposed in this section provides results that are consistent with previous literature. These results will be expanded upon in the following chapter.

\section{External noise measurements}
\label{sec:method_external}

While the accurate characterization of the internal sound field is crucial to understand the different phenomena, the transmission of the noise to the ambient is also relevant as external acoustic radiation is ultimately the main concern of the automotive industry.

To that extent, different external measurements have been carried out as part of the experimental campaign. Noise data has been captured in the orifice of the compressor inlet, around the far field of the turbocharger, and along the radiating surfaces near field.

This section will describe how these measurement were made, while results and discussion will be presented in section \ref{sec:external_results} of chapter \ref{cap:results}, including a correlation of orifice noise and in-duct results and the discussion on the contribution to the overall noise produced by each part of the system.

This section also includes a brief description of an acoustic particle velocity measurement procedure that was tested during the development of this thesis and could be of interest for the characterization of near-field sound emission even in non-anechoic test cells. A sample of the results that this method can achieve will be shown in chapter \ref{cap:engine}, which deals with measurements in these conditions.

\subsection{Orifice noise}
\label{sub:metod_orifice_noise}

\begin{wrapfigure}[19]{r}[1.5cm]{0.4\textwidth}
\centering
\vspace{-4mm}
\includegraphics[width=0.4\textwidth]{figures/cap_metod/mics_photo}
\caption[Image of the orifice free-field microphones]{Image of the two free-field microphones placed at 10 mm (Mic 1) and 20 mm (Mic 2) from the orifice of the compressor inlet duct.}
\label{fig:mics_photo}
\end{wrapfigure}

The noise emitted by the open end of the compressor inlet duct is of special concern, as it is the point where the internal sound field meets the exterior ambient. Furthermore, the compressor may not act as the only source of noise in this case, as the noise produced by the flow interaction with the duct orifice during the aspiration process can be relevant as well.

However, when measuring external noise care should be taken to avoid reflections and perturbations that may interfere with the results. This is why it was decided to install the turbocharger inside the anechoic chamber, so that free-field conditions could be recreated and external noise measurements could be taken and compared against in-duct results.

In this chamber, instrumentation consisted of two \nicefrac{1}{2}'' Brüel \& Kjær type 4190 free-field microphones, fitted with type 2669C pre-amplifiers and placed next to the compressor inlet orifice, separated 10 cm and 20 cm, respectively. These are shown as ``Mic 1'' and ``Mic 2'' in Fig.~\ref{fig:mics_photo}.

These microphones were calibrated before the measurements with a Brüel \& Kjær type 4228 pistonphone. Both the calibration and the data acquisition were performed with a Brüel \& Kjær PULSE system.

As the conditions inside the anechoic chamber can be regarded as being equivalent to those of an acoustic free field, we can estimate the magnitude of the sound intensity at each microphone $i$ as:

\begin{equation}
	|I_i| = \frac{p^2_\text{rms,i}}{\rho c}
\end{equation}

Here, $p^2_\text{rms,i}$ is the root mean squared pressure recorded by microphone $i$, $\rho$ is the ambient density and $c$ the ambient sound speed. These are calculated from pressure and temperature sensors available in the anechoic chamber.

\subsection{Far field}
\label{sub:metod_far_field}

Taking advantage of the free-field conditions of the chamber, additional microphones were installed around the turbocharger group to estimate the far field sound radiation, in a manner similar of the setup used by Teng and Homco for measurements of the whole \cite{teng2009investigation} engine.

To obtain a better estimation of the sound field, four free-field microphones of the same type 4190 were distributed around the the central point of the installation, at a radial distance of 1 m from it, as Fig.~\ref{fig:radiated_jag} shows.

\begin{figure}[b!]
\centering
\includegraphics[width=1\textwidth]{figures/cap_metod/radiated_jag}
\caption[Radiated noise microphones installed around the turbocharger]{Radiated noise microphones installed around the turbocharger, featuring a commercial set of compressor inlet and outlet pipes that will be sequentially insulated for evaluating the contribution of each part.}
\label{fig:radiated_jag}
\end{figure}

Data from all microphones was captured using a Brüel \& Kjær PULSE\textsuperscript{TM} DAQ system, composed by types 3560D, 7536, 2826 and $2\times 3032$A (hardware) and types 7700, 7702, 7705 and 7755 B (software). In order to provide a global metric to compare against orifice noise, results for the four far-field microphones were averaged to compute the overall sound pressure level radiated by the turbocharger system.

\subsection{Near field} % Particle Velocity

While useful to characterize the far field sound information emitted by the turbocharger, successful measurement through free-field microphones depend on the acoustic characteristics of the test ambient being suitable. As standard turbocharger test rigs and engine test cells are usually not acoustically conditioned, reflections and interferences make measurements of external noise difficult. 

These issues increase when the objective is to analyze the acoustic field near the tested turbochargers, as near-field measurements can be adversely impacted by diffuse background noise and reactive sound fields \cite{jacobsen1989active,elliot1981errors}. 

A technique that could be useful in order to address these issues is the direct measurement of acoustic particle velocity level (PVL). A probe featuring a MEMS sensor\footnote{MEMS refers in general to Micro Electro-Mechanical Systems, in this particular case consisting of two thin parallel used as micro hot-wire anemometers.} enabling these kind of measurements in air has been made commercially available by Microflown \cite{debree1992micro}. 

This probe has been used to perform acoustic intensity measurements, demonstrating some advantages against pressure-based probes \cite{jacobsen2005comp}. Applications include acoustic holography for near-field situations \cite{comesana2013scan} and panel-based noise evaluation in automotive cabins \cite{Wolff2009PU}. In chapter \ref{cap:engine} one of these probes will be used to perform a test in a standard engine cell in order to assess its usefulness in identifying the near-field acoustic signature generated by the turbocharger.

\section{Conclusions}
\label{sec:conclusions}

In this chapter a detailed experimental methodology for the noise evaluation of turbocharger compressors in flow benches or engine test cells has been proposed. It includes a description of the kind of turbochargers tested in this thesis and the facilities where the experimental work has been developed.

The characterization method is based on a robust in-duct acoustical beamforming technique that resolves the plane wave range sound intensity, by decomposing the total pressure signal into its upstream and downstream components.

In addition to resolving intensity, another advantage of the wave decomposition technique is the possibility of isolating the pressure data coming from the compressor, in order to enhance the spectral content of interest in situations where actual anechoic setups are not feasible, due to valves, filters, and other elements.

This isolated pressure information is also useful to validate or initialize CFD simulations where the model boundary conditions have to be defined as anechoic because simulating accessories (valves, flexible ducts, etc.) is not worthy or desirable. Such a validation methodology was presented in \cite{broatch2014methodology} and is discussed in chapter \ref{cap:cfd} of this work.

When used to evaluate the acoustic intensity of the noise generated by the compressor, the proposed signal processing allows to easily  understand visual representations in the form of noise maps of selected frequency bands of interest and estimated spectrograms of operating paths. This can provide guidelines for the evaluation of strategies to counter or avoid certain adverse noise phenomena.

The examples presented above illustrate how this technique allows detecting known acoustical phenomena of a centrifugal compressor, such as medium frequency \emph{whoosh} noise and low frequency surge onset, as well as mapping its distribution and relevance across different operating conditions of the turbocharger.

In addition to offering the detailed measurement and signal processing steps, this chapter stresses the importance of taking into account restrictive factors such as calibration, sensor setup clearances and locations, frequency aliasing effects and ranges in which theoretical assumptions hold in order to obtain robust and reproducible results.

Additionally, and taking advantage of the anechoic chamber where the turbocharger was installed, instrumentation added to measure external noise (both orifice noise and the noise radiated to the far field by the system surfaces) is described. Finally, the application of an acoustic particle velocity probe to measure the near-field emission of a turbocharger in engine conditions is proposed.

\newpage
\begin{subappendices}
\begingroup
\titleformat{\section}[hang]{}{}{0mm}{\bfseries\sffamily\Large\textcolor{tema}{\arabic{chapter}.\Alph{section}. }}[\vspace{-1mm}]
\section{Appendix: LCMV beamforming procedure}
\endgroup
\label{ann:beamformer}

This appendix to chapter \ref{cap:metod} shows a short step by step guide to compute the desired $\mathbf X^+$ and $\mathbf X^-$ decomposed spectra required in equation \ref{eq:metod_intensity}, using a LCMV (Linearly Restricted Minimum Covariance) strategy as presented by Verdú \cite{verdu2003contribucion}. Recall that decomposed signals for each frequency $f_k$ are obtained by weighting the transformed measurement matrix $\mathbf X(f_k)=[X_1 X_2 X_3]=\mathcal F\{[x_1 x_2 x_3]\}$ as follows:
\begin{align}
    \mathbf X^+(f_k)&=\mathbf {w^+}^H\mathbf X(f_k)\\\nonumber
    \mathbf X^-(f_k)&=\mathbf {w^-}^H\mathbf X(f_k)
\end{align}

It can be shown that the corresponding weights for filtering the forward and backward signals are obtained for each frequency $f_k$ by:
\begin{align}
    \mathbf w^+ &=\mathbf g^+\left[ \Sigma^{-1}_x\; \mathbf A^H(\Theta)\left[\mathbf A^H(\Theta)\; \Sigma^{-1}_x\; \mathbf A^H(\Theta)\right]^{-1}\right]\\\nonumber
    \mathbf w^- &=\mathbf g^-\left[ \Sigma^{-1}_x \mathbf A^H(\Theta)\left[\mathbf A^H(\Theta)\; \Sigma^{-1}_x\; \mathbf A^H(\Theta)\right]^{-1}\right]
\end{align}

Where $g^+=[1\;0]^T$ and $g^-=[0\;1]^T$ are the desired response vectors (unitary gain in one direction and zero gain in the opposite). $\Sigma_x$ denotes the covariance matrix. For certain discrete frequencies $f_k$ it can be reduced to:
\begin{equation}
  \Sigma_x(f_k)=[\mathbf X(f_k)\;\mathbf X^H(f_k)]
\end{equation}

The constraints matrix $\mathbf A(\Theta)=[\mathbf a^+(\theta)\; \mathbf a^-(\theta)]^T$ contains the beamformer response array (signal lags) in both directions:
\begin{align}
    \mathbf a^+(\theta) &= \mathbf a(-90^\circ)=\left[1,\exp(j\beta^+d_s),\exp(j\beta^+2d_s)\right]^T\\\nonumber
    \mathbf a^-(\theta) &= \mathbf a(90^\circ)=\left[1,\exp(-j\beta^-d_s),\exp(-j\beta^-2d_s)\right]^T
\end{align}

Here $d_s$ is the distance between sensors and $\beta^\pm$ are complex wave numbers corrected for attenuation and mean flow:
\begin{equation}
    \beta^+=\frac{k+\alpha(1-j)}{1+M}\quad\text{and}\quad\beta^-=\frac{k+\alpha(1-j)}{1-M}
\end{equation}

$M$ represents the Mach number, $k=\omega/c$ the acoustic wave number and $\alpha$ the viscothermal attenuation coefficient, which can be computed as:
\begin{equation}
  \alpha=\frac{1}{r\,a}\left(\frac{\nu\,\omega}{2}\right)^{1/2}\left[1-(\gamma-1)\text{\itshape Pr}^{-0.5}\right]
\end{equation}

Duct radius is denoted here by $r$, $a$ is the speed of sound, $\nu$ the cinematic viscosity, $\omega=2\pi f_k$ the angular frequency and \textit{Pr} is the Prandtl number.
\end{subappendices}