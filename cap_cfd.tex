%!TEX root = Tesis.tex
\chapter{Validation of numerical simulations}
\label{cap:cfd}

\noindent As stated in the initial literature review, numerical simulations of physical problems have been demonstrated to be a valuable tool for the analysis and understanding of phenomena that may be difficult to characterize exclusively through experimental means.
\blfootnote{Work in this chapter has been partly published in the following papers:
\begin{itemize}
\item\emph{Methodology for experimental validation of a CFD model for predicting noise generation in centrifugal compressors}~\cite{broatch2014methodology}
\item\emph{Simulations and measurements of automotive turbocharger compressor \emph{whoosh} noise}~\cite{broatch2015simulations}
\item\emph{Numerical and experimental analysis of automotive turbocharger compressor aeroacoustics at different operating conditions}~\cite{broatch2016numerical}
\end{itemize}
}

These kind of simulations are also very valuable in the development stage of engineering applications, where different prototype variations or operating conditions can be simulated simultaneously without the cost and risks of traditional experimental setups. 

However, it is of the utmost importance to properly validate the results offered by these numerical tools in order to ensure that they correspond accurately enough with the physical reality. Small changes in the geometrical model, boundary conditions or numerical algorithms for solving the problem can have a great influence on the outcome of the simulation. Sometimes, it is even possible that the assumptions made in the physical models that are used in the simulation are such that they do not even allow the phenomena of interest to manifest.

In this chapter the use of experimental data and signal processing techniques to validate a numerical model of a turbocharger compressor is presented and discussed, including the comparison of global variables, local flow field, acoustic results and different operating conditions of the system. 

\section{CFD simulation setup}

The development of this numerical model, including the assessment of the influence of spatial and temporal discretization, turbulence models, boundary conditions, tip clearance, etc. was performed in parallel of this thesis, in the framework of the doctoral thesis of R. Navarro \cite{navarro2014numerical}, where full details about the model can be found.

In consequence, in this section only a brief description of the numerical model will be provided, covering only the most relevant details for the validation discussion and the use of simulation results in the analysis of the physical problem.

\subsection{Computational domain}

\begin{wrapfigure}[16]{I}[0cm]{0.4\textwidth}
\raggedleft
\vspace{-4mm}
\includegraphics[width=0.4\textwidth]{figures/cap_cfd/mesh4.pdf}
\caption[Sample of the mesh in the CFD model]{Polyhedral mesh of the rotor walls (grey) and the surrounding fluid (blue) used in the CFD model.}
\label{fig:cfd_mesh}
\end{wrapfigure}

The first step in any numerical model is the decision on the extent at which the real setup will be included in the computational domain. In this case the real setup is the one described in section \ref{sec:method_induct}, this is, the turbocharger and pipes that were installed in the anechoic chamber rig. 

In order to ensure that the fidelity of the simulated geometry to the real one was kept, a real compressor of the same exact model of the one installed in the experimental test rig was disassembled and digitized. A structured-light 3D scanner was used to obtain a point cloud, which was interpolated into a smooth CAD model through software post-processing.

To avoid the influence of manufacturing differences that the real compressor could have had, only one main blade and one splitter blade were digitized, the rest of them being cloned into the final model. Tip clearance and backplate gap were also included in the digitized CAD model. The fluid domain was extracted by considering the spaces between the reconstructed surfaces. A sample of the resulting mesh can be seen in Fig.~\ref{fig:cfd_mesh}.

This non-structured mesh consisted of 9.5 million polyhedral cells arranged as to obtain $y^+$ values\footnote{The so-called $y^+$ is a dimensionless parameter used in boundary layer theory. Some turbulence models work best when it is close to 1 in the first cell adjacent to the wall.} close to 1 near the impeller walls. Inflation cell layers were used near the walls to enhance these values, with the size of the polyhedral cells being enlarged farther from the walls to reduce the total size of the mesh.

Once that the turbocharger compressor fluid volumes were obtained and meshed, inlet and outlet sections of the digitized compressor were used to create short inlet and outlet pipes by extruding these cross sections up to a distance of five of their diameters. Only these short pipes were considered, as simulating the full piping described in chapter~\ref{cap:metod} was considered too computationally expensive. 

\begin{figure}
\centering
\includegraphics[width=1\textwidth]{figures/cap_cfd/piping_plan_2}
\caption[Dimensional drawing of the compressor inlet and outlet pipes]{Dimensional drawing of the real compressor inlet and outlet pipes, including the piezoelectric sensor arrays for the experimental wave decomposition and highlighting the CFD simulation domain.}
\label{fig:cfd_piping_plan}
\end{figure}

In Fig.~\ref{fig:cfd_piping_plan} these short inlet and outlet ducts considered in the numerical simulations are compared with the real experimental setup as installed in the anechoic chamber test rig. It can be seen in the figure that the real pipes were much longer than the simulated ones. 

There was also a change in diameter in the inlet pipe mandated by the available measurement pipes and flow meter diameter. As this figure evidences, extending the CFD domain up to the limit of these ducts would have notably increased the simulation time.

\subsection{Case setup}

Once that the mesh was built, the case was solved using a segregated solver in the commercial CFD software package Star-CCM+ \cite{starccm}. A DES (Detached Eddy Simulation) simulation was performed, using an SST $k-\omega$ turbulence model. This hybrid strategy \cite{shur2008hybrid} uses a combination of sub-grid-scale where possible taking into account grid size, and Reynolds-averaged equations where not \cite{travin2000detached}, and should be helpful to resolve the finer flow structures that may be involved in noise generation. A similar solution was used by Mendonça et al. \cite{baris2011automotive,mendonca2012simulation}.

Outlet pressure and inlet mass flow rate were chosen as boundary conditions for the simulations, with values for them taken from the experimental campaign described in previous chapters of this thesis. Specific operating conditions for the reference simulation can be found in table~\ref{tab:operatingConditions}.

\begin{table}
\centering
\caption[Boundary conditions of the CFD simulation]{Boundary conditions of the CFD simulation as taken from the experimental data measured in the anechoic chamber test rig.}
\footnotesize
\begin{tabular}{rl}
\toprule
Inlet boundary condition & $\dot{m}=77$ g/s, $T_T=293$ K\\
Outlet boundary condition &  $p=223$ kPa \\
Rotational speed & $N=158763$ rpm \\ \bottomrule
\end{tabular}
\label{tab:operatingConditions}
\end{table}

Rigid body motion simulation was chosen for the transient setup, with a time step adjusted to match 1º of rotation, which imposed a maximum frequency according to the Nyquist criterion above hearing range. At least 60 ms after steady state were stored. Regarding heat transfer, the problem is considered adiabatic, as heat transfer is less relevant at high loads \cite{serrano2013importance,serrano2014procedure,serrano2014methodology}. Surface roughness, mechanical deformations and dynamic shaft eccentricity were also disregarded for simplicity.

\subsection{Solution monitors}\label{sub:cfd_sol_monitors}

Besides other monitors dedicated to general fluid field visualization, some additional pressure monitors were added to the simulated inlet and outlet ducts in order to gather data for the specific purpose of acoustic validation of the numerical solution.

Three different kinds of specific solution monitors were considered: a single-point pressure recording flush to the wall of the ducts, a similar single-point monitor in the centre of the duct, and finally an area-averaged pressure recording comprising the whole duct cross section. 

Wall-flush pressure monitors were intended to replicate the position of the piezoelectric transducer used in the experimental campaign. Centerline monitors were used to assess differences between data recorded at this location, which could not be measured experimentally without perturbing the flow, and the data gathered by the wall-flush sensors. 

As cross section area-averaged pressure monitors should produce the same results than the other two at the plane wave frequency range, they were introduced to test if the averaging of the pressure introduced differences.

\begin{figure}
\centering
\includegraphics[width=0.65\textwidth]{figures/cap_cfd/cfd_monitors}
\caption[Image of the CFD domain with different monitors]{Image of the CFD domain, highlighting the compressor wheel and including the three different kind of pressure monitors selected during the setup, at three different sections of the inlet and outlet ducts.}
\label{fig:cfd_monitors}
\end{figure}

The selected monitors can be seen in Fig.~\ref{fig:cfd_monitors}. It can be seen that all three types of monitors were replicated at three different cross sections in both inlet and outlet duct, in order to mimic the array installed in the real setup and also to assess the influence of the longitudinal distance to the compressor on the results. Full results of this assessment can be found in \cite{broatch2014methodology} and \cite{navarro2014numerical}, but in essence the axial position demonstrated small influence. 

\section{Validation methodology}

Once that enough iterations of the simulation solution were performed so that frequency resolution of the simulated pressure signal spectra was suitable for comparison, a validation effort was carried out in order to ascertain if the numerical model reproduced the compressor behaviour with enough accuracy as to use the simulation results to obtain extract information about flow features that are difficult to visualize experimentally.   

\subsection{Global variables}

A first step of this validation methodology consisted in comparing global results of the variables that are normally used to characterize the performance of the compressor since if these simple checks were not good, the particular simulation setup should be  further refined.

For instance, specific work $W_u$ and isentropic efficiency $\eta_s$ are two relevant parameters of the compressor operation, defined from the basic thermodynamic variables as follows:

\begin{align}\label{eq:powerAndEfficiency}
W_u&=\frac{\dot{W}}{\dot{m}}=\frac{\Omega\tau}{\dot{m}}=c_p(T_{out,T}-T_{in,T}) \nonumber \\
\eta_s &= \frac{\dot{W}_{s}}{\dot{W}}=\frac{T_{in,T}\left(\varPi_{TT}^\frac{\gamma-1}{\gamma}-1\right)}{T_{out,T}-T_{in,T}}
\end{align}

Here, $\Omega$ is the rotation speed in rad/s, $\tau$ is the torque as computed by the integration of simulated forces, $c_p$ is the specific heat of air, $\varPi_{TT}$ is the total-to-total pressure ratio, $\gamma$ is the heat capacity ratio and the subscript $T$ refers to total conditions in either inlet or outlet.

Furthermore, this total-to-total pressure ratio $\varPi_{TT}$ is also a result of the simulation that must be validated. Since mass flow rate, outlet pressure and shaft speed were selected as boundary conditions, inlet pressure of the solution will give a pressure ratio that must be compared against the experimental data. 

To avoid small variations in time, both experimental and CFD simulation variables such as pressures, temperatures and velocities were averaged in a 5 seconds window, before applying equations~\ref{eq:powerAndEfficiency}. Table~\ref{tab:cfd_globalVariables} shows the results of the computed global variables in both cases, including the relative error between them, defined for each variable $\psi$ as:

\begin{equation}\label{eq:relative_error_cfd}
	\varepsilon_R\;[\%]=\frac{\left| \psi_{CFD}-\psi_{exp} \right|}{\psi_{exp}}.
\end{equation}

\begin{table}[htb]
\centering
\caption[Validation of global variables]{Comparison between compressor global variables as measured in the experimental test rig and as predicted by the CFD solution.}
\footnotesize
	\begin{tabular}{rccc}
	\toprule
	 & $\varPi$ [-] & $W_u$ [kJ/kg] & $\eta_s$ [$\%$] \\\midrule
	Exp. & 2.240 & 112.27 & 67.82  \\
	CFD & 2.221 & 111.32 & 67.62  \\\midrule
	$\varepsilon_R$ [\%] & 0.87 & 0.84 & 0.30  \\ \bottomrule
	\end{tabular}
\label{tab:cfd_globalVariables}
\end{table}

It can be seen in this table that the accuracy of global parameter prediction is good, with less than 1\% of relative error between the CFD and experimental results, even taking into account the simplifying hypothesis considered for the numerical simulation.

This shows that the numerical model of the compressor is successfully reproducing the operating conditions of the real system so that averaged variables such as mean Mach number, temperatures and pressures that are needed for the acoustic analysis are well reproduced.

\subsection{Acoustic results}

However, the fact that these global time-averaged parameters are in good agreement does not guarantee that the simulation setup is precise enough to reproduce the kind of transient flow phenomena that are related to the acoustic behaviour of the compressor.

In a subsequent step of the validation analysis, the spectral content of the simulated pressure data gathered by the monitors described above was compared with the experimental data gathered by the fast piezoelectric sensors.

\subsubsection{Total pressure spectra}

First, comparison between experimental and numerical results was performed using the spectral content of the total pressure, this is, without applying the wave decomposition algorithm to any of the signals. In Fig.~\ref{fig:cfd_Spec_Inlet_Tot} this comparison is shown taking into account the three types of CFD monitors described in subsection~\ref{sub:cfd_sol_monitors}.

\begin{figure}[htb!]
\centering
\includegraphics[width=0.88\textwidth]{figures/cap_cfd/Spec_Inlet_Tot}\\[5mm]
\includegraphics[width=0.88\textwidth]{figures/cap_cfd/Spec_Outlet_Tot}
\caption[Comparison of spectral content (CFD vs. Exp.)]{Comparison of total spectral content between the experimental data recorded by one piezoelectric sensor and the simulation results for each type of monitor.}
\label{fig:cfd_Spec_Inlet_Tot}
\end{figure}

It can be seen in this figure that as theoretically predicted,  spectral content measured by the three different CFD monitors collapses in the plane wave frequency range for both inlet and outlet as pressure varies evenly in time across the cross-section.

This collapse of the signals ends when the wall monitor starts to show different spectral content after a certain frequency. This cut-off frequency almost exactly matches the theoretically predicted frequency at which the first asymmetric mode starts propagating.

Axis and surface monitor spectra remain collapsed until a higher frequency when another separation happens, again matching almost exactly the cut-off frequency of the first radial mode.

Explanation of how these different kind of monitors are affected by these two higher order modes can be found by considering Fig.~\ref{fig:cfd_modes_monitors}. As the first asymmetric mode starts propagating, the axial monitor is located in its modal line, thus is not influenced by cross-section pressure fluctuations. The area average monitor evens out these fluctuations, thereby the only monitors being influenced being those located at the wall.

\begin{figure}[htb!]
\centering
\includegraphics[width=\textwidth]{figures/cap_cfd/modes_monitors}
\caption[Schematic of acoustic modes modal lines]{Schematic of modal lines of the two relevant acoustic modes in relation to the different types of CFD monitors considered in the model.}
\label{fig:cfd_modes_monitors}
\end{figure}

When the first radial mode starts propagating at a higher frequency, the axial monitor is not in the modal line any more, and thus its pressure spectral content separates from the surface monitor which continues to average the cross-sectional fluctuations.

Thereby it can be concluded that the CFD model is accurately predicting the cut-off frequencies at which higher order acoustic modes start propagating through the duct.

It can also be seen that BPF tonal noise peak is captured by the numerical model, although more clear in the outlet results that in the inlet ones. The pressure monitor located at the wall features the clearest BPF peak, even showing a small peak in the inlet case at the correct frequency of 16 kHz.

However, agreement between the CFD predicted signals and the experimental results is not so good in terms of PSD level and trend in the spectra, specially above the plane wave frequency range. The outlet pipe simulated spectra seems to feature an standing wave pattern, related to the short length of the pipe and lack of non-reflecting boundary conditions.

\subsubsection{Decomposed spectra}

As shown in chapter~\ref{cap:results} the use of pressure wave decomposition can offer clearer spectra as the components coming from the compressor are more isolated from the reflections coming from the duct endings. 

In the case of the experimental data, the decomposition was performed according to the three-sensor LCMV beamformer described in section \ref{sub:method_beamformer}. However, as CFD allows unfettered access to all of the flow variables at any point of the simulated domain, other pressure wave decomposition methods are available, such as the Method of Characteristic (MoC) that was implemented following the formulation proposed in \cite{payri1995acoustic,torregrosa2012dev,galindo2010coupling}: 
\begin{eqnarray}\label{eq:descomposicion}
p_\text{forw}=p_\text{ref}\left[ \frac{1}{2}\left(1+\left(\frac{p}{p_\text{ref}}\right)^{\frac{\gamma-1}{2\gamma}}\left(1+\frac{\gamma-1}{2}\frac{u}{a}\right)\right) \right]^{\frac{2\gamma}{\gamma-1}}\\
p_\text{back}=p_\text{ref}\left[ \frac{1}{2}\left(1+\left(\frac{p}{p_\text{ref}}\right)^{\frac{\gamma-1}{2\gamma}}\left(1-\frac{\gamma-1}{2}\frac{u}{a}\right)\right) \right]^{\frac{2\gamma}{\gamma-1}}\nonumber
\end{eqnarray}

Here, advantage is taken of the CFD ability to obtain instantaneous local values for sound speed $a$ and flow speed $u$. Again, $\gamma$ is the heat capacity ratio and $p_\text{ref}$ is a suitable reference pressure. Fig.~\ref{fig:cfd_Spec_Outlet_Tot_Moc_Wall} shows the result of applying this algorithm to the wall CFD monitor.

\begin{figure}[htb!]
\centering
\includegraphics[width=0.88\textwidth]{figures/cap_cfd/Spec_Outlet_Tot_MoC_Wall}
\caption[Comparison of decomposed CFD spectrum (wall)]{Comparison of experimental and simulated total pressure against simulated decomposed pressure in the outlet case, using the Method of Characteristics and the wall monitor. Black lines indicate mode onset.}
\label{fig:cfd_Spec_Outlet_Tot_Moc_Wall}
\end{figure}

As shown in the figure, the decomposed pressure spectra show a better agreement with the experimental data as the periodic oscillations present in the simulated pressure spectra are alleviated. The trend of both reference and simulated spectra are similar overall, although discrepancies can be observed, specially at lower frequencies below 2 kHz. 

While the numerically simulated spectra is able to capture the characteristic peak of the Blade Passing Frequency (BPF) tonal noise at 16 kHz, the broadband noise that is visible in the experimental data between approximately 13 and 17 kHz is not reproduced by the CFD model, suggesting that the source mechanism of this particular phenomena is not being correctly considered in the simulation.

\begin{figure}[htb!]
\centering
\includegraphics[width=0.88\textwidth]{figures/cap_cfd/Spec_Outlet_Tot_MoC_Area}
\caption[Comparison of decomposed CFD spectrum (section)]{Comparison of experimental and simulated total pressure against simulated decomposed pressure in the outlet case, using the Method of Characteristics and the section monitor. Black lines indicate mode onset.}
\label{fig:cfd_Spec_Outlet_Tot_Moc_Area}
\end{figure}

In Fig.~\ref{fig:cfd_Spec_Outlet_Tot_Moc_Area} a similar comparison is made, but using instead the section monitor (area-averaged pressure). In this case the Method Of Characteristics decomposition procedure allows for a better match between experimental and simulated spectral content at the plane wave range, below the first asymmetric acoustic mode onset at 7.4 kHz.

However, using this monitor type, higher frequency content after the acoustic modes start propagating does not feature a good agreement with the reference experimental data. Blade Passing Frequency (BPF) tonal noise is not captured as clearly as with the single-point wall monitor, and neither is the broadband centred around 15 kHz. 

Specially in the case of the BPF tonal noise, this could be due to the fact that the area-averaging process is neglecting the pressure differences across the cross-section, that are still picked up by the single-point CFD monitor.

Regarding the plane wave frequency range where the agreement between CFD and experimental data is better, Fig.~\ref{fig:cfd_Spec_PlaneW_Dec} shows the comparison of decomposed pressure wave spectra in both inlet and outlet ducts. 

\begin{figure}[htb!]
\centering
\includegraphics[width=0.88\textwidth]{figures/cap_cfd/Spec_Inlet_Dec}\\[5mm]
\includegraphics[width=0.88\textwidth]{figures/cap_cfd/Spec_Outlet_Dec}
\caption[Comparison of decomposed plane wave CFD spectrum (wall)]{Comparison of decomposed pressure spectral content between the experimental data (through beamforming) and the simulation results using the surface monitor (through the method of characteristics).}
\label{fig:cfd_Spec_PlaneW_Dec}
\end{figure}

Data from the decomposed signals is shown both at the highest frequency resolution available in the CFD calculation ($\sim$16 Hz) and smoothed through a moving average filter in order to better show the shape of the spectra. As the experimental data is decomposed though the beamforming procedure described in chapter~\ref{cap:metod}, frequency is limited not only by the onset of the first acoustic mode but also by the spatial Nyquist-like criterion.

In can be seen in this figure that the match between simulated and measured pressure waves coming back through the inlet from the compressor is not so good. While the experimental signal features an approximately plane spectra with only a slight ``hump'' of $\sim$4 dB at the \emph{whoosh} noise frequency between 1.5 and 2.5 kHz, the simulated pressure wave features an oscillatory behaviour.

In the case of the outlet spectra however, a better agreement between the numerical prediction and the experimental data is found. The CFD simulation result appears to accurately follow the trend of the measurements, including the broadband between 1 and 2.5 kHz that is commonly associated with the \emph{whoosh} noise phenomenon.

This ability of the numerical model to predict the \emph{whoosh} noise phenomenon in the outlet duct while not completely matching the experimental inlet results seems to point to the primary source of \emph{whoosh} noise not being located in the inlet flow structures but rather in a downstream point where simulations are more accurate.

Nevertheless, it is shown in Fig.~\ref{fig:cfd_Spec_PlaneW_Dec} that the CFD model appears to contain a source of broadband noise between 1 and 2.5 kHz that matches the experimental results. Thereby, post-processing of the CFD simulation data could be useful to locate the source mechanism of this phenomenon and, furthermore, the effect on \emph{whoosh} noise of mitigating solutions could be simulated using this model before its physical implementation. 

\section{Operating conditions study} %.... Art. op. conditions CFD

After performing the validation and optimization of the CFD model at the reference operating conditions (160 krpm of shaft speed and 77 g/s of corrected air mass flow), a further study was performed where two additional operating settings of the compressor were simulated by modifying the boundary conditions according to the experimentally measured data points, keeping the rest of the model parameters unchanged.

\begin{wrapfigure}[29]{o}[0.8cm]{0.45\textwidth}
\vspace{-3mm}
\centering
\includegraphics[width=0.45\textwidth]{figures/cap_cfd/cfd_ops_globalPressureRatio_iz}\\[4mm]
\includegraphics[width=0.45\textwidth]{figures/cap_cfd/cfd_ops_globalSpecificWork_iz}\\[4mm]
\includegraphics[width=0.44\textwidth]{figures/cap_cfd/cfd_ops_globalEfficiency_iz}\\[2mm]
\caption[Validation of global compressor variables]{Global compressor variables as measured experimentally and as simulated by CFD in the three selected operating conditions.}
\label{fig:cfd_ops_globals}
\end{wrapfigure}

In order to analyze the differences in terms of pressure spectra and fluid field behaviour at different degrees of unstable operating conditions, the two additional simulations were performed at the same shaft speed of 160 krpm, in one case at a higher air mass flow  ($\sim$109 g/s) and in the other case at a lower air mass flow ($\sim$60 g/s).

\subsection{Global variables}

While the original operating condition selected for the CFD study was chosen so that it matched the point at which the experimental measurements showed that the slope of the speed line reached zero, marking the start of the low-flow, marginal surge map zone, in this additional study the new operating conditions were chosen so that the lower flow one was well into this unstable zone and close to the deep surge limit, and the other was in the stable operating zone near the maximum efficiency point.

This would enable a comparison between the flow field of a stable point, the flow field at the point of the instability inception, and that of a point in which the unstable flow structures such as the inlet backflow would be already well developed. 

Simulations carried out in this campaign were also performed using an unsteady RANS turbulence model in addition to the DES model that was considered for the reference study, to assess the influence that the turbulence model could have in the numerical results. Again, the first step of the validation process involved checking the same global parameters of the compressor performance that were analyzed for the reference case. In Fig.~\ref{fig:cfd_ops_globals} the comparison of the numerical predictions for these parameters is shown against the previously measured experimental values.  

It can be seen in this figure that while the prediction of isentropic efficiency (bottom plot) and specific work (middle plot) is in good agreement with the experimental data, in the case of the total-to-total pressure ratio (top plot) the agreement is not so good.

The least accurate prediction is that of the highest flow case, where the difference in terms of pressure ratio is larger. It should be noted as well that in this case, small variations in the predicted temperature and pressure resulted in a slight change in \textit{corrected} air mass flow.

Regarding the differences between the two considered turbulence models, predictions in terms of isentropic efficiency and specific work are quite similar, with the larger differences in the case of the total-to-total pressure ratio. 

This last difference between the two models is large enough so that it changes the slope of the speed line, with the RANS model correctly predicting a decrease in pressure ratio at lower mass flow and the DES model failing to do so, despite having less relative error.

\begin{table}[htb!]
\centering
\caption[Validation of global variables for each condition]{Relative error between the compressor global variables measured in the experimental test rig and those predicted by the numerical model.}
\footnotesize
\begin{tabular}{lcccc}
\toprule
Model & $\dot{m}$ & $\varepsilon_{\varPi_{t,t}}$ [\%] & $\varepsilon_{W_u}$ [\%] & $\varepsilon_{\eta_s}$ [\%] \\ \midrule

\multirow{3}{*}{RANS} & 60 & -0.5 & 0.7 & 0.1  \\ 
					& 77 & -0.8 & -0.6 & -0.5  \\
					& 109 & -2.3 & -2.1 & -1.3  \\ \midrule
					
\multirow{3}{*}{DES} & 60 & 0.4 & 0.6 & 1.6  \\
					& 77 & -0.9 & -0.8 & -0.4  \\
					& 109 & -2.8 & -2.6 & -1.5  \\ \bottomrule
\end{tabular}
\label{tab:ops_globalVariables}
\end{table}

Differences quantified in terms of relative error as defined by expression~\ref{eq:relative_error_cfd} are provided in table~\ref{tab:ops_globalVariables}. It can be seen how, although larger that in the reference case of 77 g/s on which the setup was developed, relative errors are still smaller than 3\% in the two additional operating settings as calculated with both DES and RANS turbulence models.

While this table shows how the use of an unsteady RANS model resulted in less relative error overall, specially in terms of isentropic efficiency at the lower flow operating point, it should be remarked that only the DES model correctly predicted the decrease in pressure ratio as flow was near surge conditions and impeller blades began stalling, suggesting that these unstable flow phenomena are better modelled using DES. 

\subsection{Pressure spectra}

In addition to the validation of global variables, it is necessary to assess if the spectral content of the pressure signal simulated by the model at these new operating conditions still matches the trends observed during the corresponding experimental measurements. In this subsection only the data obtained using a DES model is presented.

\begin{figure}[htb!]
\centering
\includegraphics[width=0.48\textwidth]{figures/cap_cfd/spec_ops_60gs_inlet}
\includegraphics[width=0.48\textwidth]{figures/cap_cfd/spec_ops_60gs_outlet}\\[4mm]
\includegraphics[width=0.48\textwidth]{figures/cap_cfd/spec_ops_77gs_inlet}
\includegraphics[width=0.48\textwidth]{figures/cap_cfd/spec_ops_77gs_outlet}\\[4mm]
\includegraphics[width=0.48\textwidth]{figures/cap_cfd/spec_ops_109gs_inlet}
\includegraphics[width=0.48\textwidth]{figures/cap_cfd/spec_ops_109gs_outlet}
\caption[Comparison of spectra for three conditions (plane wave)]{Comparison of decomposed spectra in the plane wave range for all three operating conditions and both ducts against experimental results.}
\label{fig:cfd_ops_spec_all_PW}
\end{figure}

In Fig.~\ref{fig:cfd_ops_spec_all_PW} a comparison is made in the same fashion as in Fig.~\ref{fig:cfd_Spec_PlaneW_Dec} where decomposed experimental and numerical signals are compared, selecting the surface CFD monitor. The frequency range is thus limited by the experimental spatial criterion, not the first acoustical mode onset. It can be seen that, although not featuring as good match as in the 77 g/s case, the new simulations show a similar trend in spectral content compared against experimental results. 

\begin{figure}[htb!]
\centering
\includegraphics[width=0.48\textwidth]{figures/cap_cfd/spec_ops_60gs_inlet_full}
\includegraphics[width=0.48\textwidth]{figures/cap_cfd/spec_ops_60gs_outlet_full}\\[4mm]
\includegraphics[width=0.48\textwidth]{figures/cap_cfd/spec_ops_77gs_inlet_full}
\includegraphics[width=0.48\textwidth]{figures/cap_cfd/spec_ops_77gs_outlet_full}\\[4mm]
\includegraphics[width=0.48\textwidth]{figures/cap_cfd/spec_ops_109gs_inlet_full}
\includegraphics[width=0.48\textwidth]{figures/cap_cfd/spec_ops_109gs_outlet_full}
\caption[Comparison of spectra for three conditions (full range)]{Comparison of pressure spectra for all three operating conditions and both ducts against experimental results. In this figure, pressure wave decomposition is only performed in the CFD spectra.}
\label{fig:cfd_ops_spec_all_full}
\end{figure}

Regarding the full audible range of the spectrum, in Fig.~\ref{fig:cfd_ops_spec_all_full} a comparison is presented where the decomposed signal of the CFD wall monitor (which was shown previously to result in better prediction of higher frequency spectral content) is compared with the experimental pressure spectra from a single sensor (as decomposition cannot be performed experimentally in the full frequency range). The plane wave range of each plot is shown enlarged in Fig.~\ref{fig:cfd_ops_spec_all_decT}.

\begin{figure}[htb!]
\centering
\includegraphics[width=0.48\textwidth]{figures/cap_cfd/spec_ops_60gs_inlet_decT}
\includegraphics[width=0.48\textwidth]{figures/cap_cfd/spec_ops_60gs_outlet_decT}\\[4mm]
\includegraphics[width=0.48\textwidth]{figures/cap_cfd/spec_ops_77gs_inlet_decT}
\includegraphics[width=0.48\textwidth]{figures/cap_cfd/spec_ops_77gs_outlet_decT}\\[4mm]
\includegraphics[width=0.48\textwidth]{figures/cap_cfd/spec_ops_109gs_inlet_decT}
\includegraphics[width=0.48\textwidth]{figures/cap_cfd/spec_ops_109gs_outlet_decT}
\caption[Comparison of spectra for the three conditions (plane wave, wall)]{Comparison of pressure spectra for all three operating conditions and both ducts against experimental results in the plane wave range. In this figure, pressure wave
decomposition is only performed in the CFD spectra.}
\label{fig:cfd_ops_spec_all_decT}
\end{figure}

Analysis of these last two figures shows that the prediction capability of the model in the higher frequencies is similar in the two new simulations that in the reference one of 77 g/s. The higher frequency broadband seen in both inlet and outlet ducts is still not reproduced by the CFD results, and the level of the high frequency content in the inlet duct is still over-predicted.

Inspecting the plane wave range seen in Fig.~\ref{fig:cfd_ops_spec_all_decT} it is apparent that in the case of the inlet, while still not a perfect match the trends of each flow rate operating point feature good agreement to the experimental data. 

In the case of the outlet results, while the trend at the lowest flow condition is well predicted, for the two higher flow points of 77 and 109 g/s the simulated spectral content captured by the wall monitor does not reproduce correctly the \emph{whoosh} noise hump between 1 and 2.5 kHz, over-predicting the level of the lowest frequencies.

\subsection{Selected CFD results}

While the prediction of spectral content is not totally accurate yet, specially in terms of levels, and thus more research should be devoted to enhance the numerical model of the compressor, it has been shown that the model is precise enough to reproduce the trends in spectral content, allowing some selected results to be presented in order to provide better insight into the flow conditions in the compressor.

\begin{figure}[htb!]
\centering
\includegraphics[width=1.04\textwidth]{figures/cap_cfd/vel_spec_diffuser}
\caption[CFD results for the lowest mass flow rate condition]{Combination of selected CFD results for the lowest mass flow rate operating condition (60 g/s), including pressure and velocity field and diffuser frequency spectrum.}
\label{fig:vel_spec_diffuser}
\end{figure}

The kind of results that can be extracted from the CFD simulation are exemplified in Fig.~\ref{fig:vel_spec_diffuser}, which combines the 3D model of the compressor wheel with plane views of the pressure and velocity fields, and frequency information extracted from the time record of a particular position in the diffuser. Since the CFD setup is using a rotating mesh approach, the whole diffuser fluid domain is rotating, so this monitor is not stationary.

Pressure information is presented in two planes, at the top of the inducer and at the leading edge plane. 3D isosurfaces of $P = 85$ kPa are also shown, forming ``bubbles'' that move from one channel to the next at a frequency lower than that imposed by the wheel rotation. These recirculation ``bubbles'' have been sometimes linked to broadband acoustic phenomena of half rotating order (RO) \cite{sundstrom2015centrifugal,sundstrom2014assessment}.

The radial velocity field is shown in the diffuser and the volute, revealing an asymmetrical field and the presence of a higher speed zone near the volute tongue. Frequency signature of the monitor is shown in the figure, peaking at 2.9 kHz ($\sim$10\% over RO) and presenting two sidelobes. These pulsations could also be related to \emph{whoosh} noise generation.

Taking advantage of the three simulated flow rate conditions. in Fig.~\ref{fig:cfd_isospan_all} the meridional velocity magnitude across the compressor wheel has been plotted into an extended isospan surface. Data to build this surface has been extracted from the inducer, the wheel, the diffuser and the volute, as shown in the accompanying graphic. 

This allows to compare how the velocity field changes across the compressor in all three of the selected operating conditions. Differences are easily noticed between the point of higher flow and higher isentropic efficiency (bottom) and the other two points. 

\begin{figure}[htpb!]
\centering
\includegraphics[height=0.9\textheight]{figures/cap_cfd/isospan_all_small2}
\caption[Comparison of meridional velocity fields]{Comparison of meridional velocity field in a blade-to-blade isospan surface, for the three selected operating conditions of the CFD model.}
\label{fig:cfd_isospan_all}
\end{figure}

In the 109 g/s simulation the velocity field is more axisymmetric, with all blades and channels presenting a similar distribution. Flow in the inducer is very regular, being attached to the blades in all cases. Flow distribution in the diffuser and volute is no longer axisymmetric, featuring instead a wake trail for each blade. Also, vortex shedding can be appreciated even in these conditions close to the maximum efficiency point.

As mass flow rate of the CFD boundary condition was reduced to 77 g/s, a more irregular velocity field is shown in the middle plot of the figure. Flow in the inducer is not so regular as before, with some main blades featuring reversed flow at the suction side (SS) near the leading edge creating flow detachment. In the diffuser, a zone featuring increased velocity can be seen near the tongue, where the volute is smaller, with the wake of the blades not being so apparent as before.

For the lowest flow rate condition at 60 g/s, the flow field in the inducer became highly irregular, featuring some zone with reversed flow. Severe detachment of the flow can be appreciated in most of the blades, accompanied by a similar high velocity zone in the diffuser near the volute tongue. 

The presence of this high velocity zone (which by continuity also affects pressure) can also be seen by attaching a moving pressure monitor to the pressure side of one of the blades (marked with a black dot in Fig.~\ref{fig:cfd_isospan_all}) and plotting the measured spectra, as seen in Fig.~\ref{fig:cfd_spec_blade_p}. Similarly to the previously discussed velocity spectra, a peak is shown at the frequency of the rotating order as the structure is seen by the blade once each revolution. A secondary peak appears also at 2\,RO, but the sidelobes are not reproduced.

\begin{figure}[htb!]
\centering
\includegraphics[width=0.88\textwidth]{figures/cap_cfd/spec_blade_PS_p}\\[5mm]
%\includegraphics[width=0.88\textwidth]{figures/cap_cfd/spec_blade_SS_p}
\caption[Pressure spectra of a trailing edge monitor]{Comparison of pressure spectral content recorded at the pressure side of the trailing edge of an impeller blade between the three simulated operating conditions.}
\label{fig:cfd_spec_blade_p}
\end{figure}

These peaks are much more pronounced for the two cases of lower mass flow rate where this zone was more easily identified, but still appears for the 109 g/s case, indicating that still some influence of the volute restriction affects the blades when they pass through this particular angular position.

Of particular interest among the different results that can be extracted from the CFD simulation is the opportunity to analyze the complex flow field that develops in the inducer and final part of the inlet duct as operating conditions reach the mild surge area of the compressor map and flow is reversed in direction. 

\begin{figure}[htb!]
\centering
\includegraphics[width=0.80\textwidth]{figures/cap_cfd/backflow_isosurface}
\caption[Isosurfaces of null axial velocity]{Isosurfaces of null axial velocity in the compressor inducer/inlet showing the reversed flow extent and distribution, including isosurfaces of low pressure ``bubbles'' in the same region.}
\label{fig:cfd_backflow_isosurfaces}
\end{figure}

In Fig.~\ref{fig:cfd_backflow_isosurfaces} this backflow is characterized for the 60 g/s condition (which is well inside the marginal surge area as demonstrated by the slope of the speed line) by plotting the isosurfaces where $V_z = 0$ m/s. At this operating condition some low pressure ``bubbles'' of stalled flow are also present in the inducer and are also plotted as isosurfaces, as in the previously seen Fig.~\ref{fig:vel_spec_diffuser}. 

The same data is also shown in an inset plot for the 77 g/s operating condition, which is the point that both experimental data and DES CFD simulations identify as the one where the slope of the speed line reaches zero and the marginal surge zone starts.

This 3D view of the pressure and velocity information shows how the low pressure bubbles extend up to the limit of the reversed flow, in both operating conditions. However, these low pressure zones are much more axisymmetric in the 77 g/s point, whereas in the lower flow rate case they become isolated and elongated. Time data shows also how they rotate inside the backflow-affected area in a tornado-like fashion.

Regarding the backflow itself as characterized by the null vertical velocity in the inlet, the figure shows the drastic evolution from an incipient recirculation at 77 g/s very close to the wheel, to the fully developed backflow at 60 g/s, reaching upstream up to 1.2 inducer diameters. It is also shown in the 77 g/s view how this backflow will start appearing close to the inducer walls, leaving  the core flow still unperturbed.

\section{Acoustic source location}

While the analysis of raw pressure and velocity monitors can shed light into the behaviour of the compressor from the fluid dynamics point of view by identifying flow structures and spectral content at selected points of the domain, other data post-processing techniques can be used to obtain further insight into the acoustic behaviour of the system.   

\subsection{Fourier analysis}

A way to explore the spatial distribution of the acoustic sources for different frequencies of interest is to perform the Fourier transform at each cell of different surfaces like walls and planes. Then, the amplitude of the transformed signal at the frequency of interest is used to color each cell the selected surfaces, be it directly or through a smoothing interpolation.

\begin{figure}[htb!]
\hspace{-0.01\textwidth}
\includegraphics[width=1.02\textwidth]{figures/cap_cfd/pressure_fft_walls}
\caption[Selected domain walls colored by 3 kHz amplitude]{Selected domain walls colored by the amplitude of the 3 kHz component of the simulated pressure signal at 160 krpm and 77 g/s.}
\label{fig:pressure_fft_walls}
\end{figure}

These kind of results are shown in Fig.~\ref{fig:pressure_fft_walls} for some selected wall boundaries of the numerical domain. Simulated operating conditions are those of the medium flow rate point at 77 g/s. The chosen frequency component has been 3 kHz, as it is relevant in the phenomenon of \emph{whoosh} noise issue that is of special concern regarding the identification of source flow mechanisms behind its inception.

From the figure it can be surmised that the pressure signal exhibits its stronger 3 kHz component at the diffuser and the downstream half of the blade chords, towards the trailing edges. Color scale needs to be magnified in order to appreciate the distribution in the volute, which is stronger at its narrower part near the tongue. The scale needs to be restricted again to see the pattern at the inlet and outlet ducts, in the form of a standing wave.

\begin{figure}[h!]
\centering
\includegraphics[width=0.95\textwidth]{figures/cap_cfd/pressure_fft_walls_circ}
\caption[Volute, wheel and diffuser colored by 3 kHz amplitude]{Volute, wheel and diffuser walls colored by the amplitude of the 3 kHz component of the simulated pressure signal at 160 krpm and 77 g/s.}
\label{fig:pressure_fft_walls_circ}
\end{figure}

Focusing in the diffuser and the wheel, Fig.~\ref{fig:pressure_fft_walls_circ} shows a perpendicular view of the diffuser plane. As it is apparent from the plot, distribution of the content amplitude is not axisymmetric neither in the diffuser nor in the wheel, with two opposite zones of the rotor appear to feature higher content at the selected frequency. It can also be seen how the trailing edges of the rotor blades seems to feature a higher amplitude that the leading edges, which could indicate an influence of the vortex shedding mechanism evidenced by Fig.~\ref{fig:cfd_isospan_all}.

\subsection{Modal decomposition}

However, while the Fourier analysis method allows the obtaining of interesting results in a simple, straightforward way which moreover can be found already implemented in some commercial CFD codes, it is not without disadvantages.

Among them is the need to manually select both the surfaces and the frequencies of interest, thereby having the risk of missing relevant regions or frequencies, which are not known beforehand. Also, in more complex geometries or flow structures it could be difficult to pinpoint exactly relevant features, requiring the placement of multiple planes in the domain to reconstruct these structures.

\subsubsection{Proper Orthogonal Decomposition}

In order to address these limitations, more sophisticated analysis can be carried out in the form of modal decomposition of the flow field. One of the most used among these is probably the Proper Orthogonal Decomposition (POD) or Karhunen--Loève expansion, which originated in the field of probability theory and was first applied to the analysis of turbulent flow in 1967 \cite{lumley1967structure}. As (rather poetically) put by Aubry in the aptly named paper \emph{``On the Hidden Beauty of the Proper Orthogonal Decomposition''} \cite{aubry1991hidden}, the objective of this technique is that:\\

\emph{``The flow is decomposed into both spatial and temporal orthogonal modes which are coupled: each space component is associated with a time component partner. The latter is the time evolution of the former and the former is the spatial configuration of the latter.''}\\

In this way, this decomposition allows the identification of which spatial structures comprise the most energy of the flow field, which is understood as the superposition of all modes. The ordering of the contribution of each mode however, allows in principle a simplified yet meaningful reconstruction of the flow field. This is realized by diagonalizing the time-averaged spatial correlation matrix assembled with temporal ``snapshots'' of the flow, obtained either numerically or experimentally \cite{schmid2010dynamic,bagheri2013koopman}.

Principal drawbacks of this method include that in some cases energy may not be the most relevant metric in characterizing the relevance of flow structures \cite{schmid2010dynamic}, and that the use of second-order flow statistics can result in a loss of information such as phase \cite{aubry1991hidden}, failing to capture the dynamics of the coherent structures of the flow \cite{schmid2011applications}.

Moreover, while modes resulting from POD exhibit spatial orthogonality, they can contain multi-frequency temporal content \cite{jovanovic2014sparsity,sakowitz2014flow}. This means that a particular flow structure can be excited by different phenomena and at different frequencies, yet remain spatially coherent in time and thus result in a higher energy POD mode. 

Linking the spatial structures with a specific phenomena requires, as a consequence, some guesswork based on prior knowledge of the excitation mechanisms. A good example of this is the application of POD to the CFD simulation of an engine manifold presented by Sakowitz \cite{sakowitz2014flow}, where it can be seen how the first POD mode contains different frequency content originated from both the engine firing and the valve opening.

\subsubsection{Dynamic Mode Decomposition}

An alternative decomposition technique known as Dynamic Mode Decomposition (DMD) that addresses this shortcoming is also presented in the same work. A relatively recent technique for modal flow decomposition \cite{rowley2009spectral,schmid2010dynamic,chen2012variants}, DMD is a \textit{data-based} procedure that aims to extract dynamic information from only the snapshots of a non-linear system while avoiding any need for previous knowledge \cite{schmid2010dynamic}. 

This is done by computing the eigenvalues and eigenvectors of a linear model which approximates the underlying dynamics of a system, which may not be necessarily linear \cite{chen2012variants}. The DMD modes (eigenvectors) calculated using this technique represent again coherent spatial structures of the flow, in this case not necessarily orthogonal but each one featuring a single temporal frequency \cite{jovanovic2014sparsity}, in contrast with POD modes. DMD eigenvalues represent the rates of decay and growth of these modes, along with their associated frequency.

Evaluation of the relevance of each mode to the dynamics of the system is usually performed by computing the norm of the mode, although Dahan et al. reported \cite{dahan2014aero} that in some cases it may be preferable to order the modes attending to their frequency rather that their norm, as some modes may feature a growth rate of less than the unity, and thus they may be damped over time.

Furthermore, another advantage of the DMD and POD techniques is that it can be proved \cite{schmid2010dynamic,schmid2011applications} that processing only subdomain data is possible, be it this subdomain a small region of the domain or even a projection of the 3D field into a 2D plane, as no spatial arrangement is assumed by either method. This is convenient as post-processing can be restricted to an area of the flow where relevant phenomena are expected, saving computational effort.

\subsubsection{DMD implementation}

As several variations for implementing a DMD strategy already exist in the literature only the basic concept originally proposed by Schmid \cite{schmid2010dynamic} will be shown in this section. Generally, information on the flow field evolution coming from either numerical simulations or experimental measurements\footnote{For instance, from experimental Particle Image Velocimetry (PIV) measurements that capture a sequence of velocity field snapshots.} will be naturally presented in a sequence of $N$ vectors $\mathbf v_i$ representing temporal snapshots that can be gathered in a matrix $\mathbf V$:
\begin{equation}\label{eq:dmd_snaps}
	\mathbf V_1^N = \{\mathbf v_1, \mathbf v_2, \dots, \mathbf v_N\}
\end{equation}

Each snapshot contains a quantity $M$ of scalar flow magnitudes which in the case of experimental results usually represent the velocity information. In the case of numerical simulations these snapshots may represent other useful information such as pressure, species, etc. These snapshots should be separated by a constant time step $\Delta t$, and are assumed to be linearly related through a certain matrix $\mathbf A$, which is assumed to remain approximately constant during the time $N\Delta t$ spanned by the sequence:
\begin{equation}\label{eq:dmd_linear}
	\mathbf v_{i+1} = \mathbf A \mathbf v_i
\end{equation}

Even if the flow field is non-linear, matrix $\mathbf A$ will provide a linear tangent approximation of the dynamic characteristics of the flow. The objective is now to extract its eigenvalues and eigenvectors which, will be the DMD eigenvalues and the DMD modes of the flow. Combining both expressions in matrix form we can write that:
\begin{equation}\label{eq:dmd_mat}
	 \mathbf V_{2}^{N}=\mathbf A \mathbf V_{1}^{N-1}
\end{equation}

Where $\mathbf V_{2}^{N}$ contains snapshots 2 to $N$ and $\mathbf V_{1}^{N-1}$ contains snapshots 1 to $N-1$. As $M$ for CFD simulations is usually in the order of $10^6$ and $N$ must comprise a sensible amount of time steps in order to ensure that the dynamic characteristics are well captured, directly performing the eigendecomposition of $\mathbf A$ is usually too computationally expensive.

In order to overcome this problem different strategies exist, such as reformulating \ref{eq:dmd_mat} into a companion matrix that can be solved through a QR decomposition \cite{schmid2011applications,rowley2009spectral,bagheri2013koopman}. A more numerically stable method however is to make use of the Singular Value Decomposition (SVD) \cite{schmid2010dynamic,jovanovic2014sparsity,sakowitz2014flow,chen2012variants} to obtain $\mathbf V_1^{N-1} = \mathbf U \bm\Sigma\mathbf W^T$. Then, \ref{eq:dmd_mat} becomes:
\begin{equation}
	\mathbf V_2^N = \mathbf A \mathbf U \bm\Sigma \mathbf W^T
\end{equation}

From this expression it is now possible to manipulate the terms that are already known in order to define a new matrix $\widetilde{\mathbf{S}}$, which will be thus related to $\mathbf A$ in the following way:
\begin{equation}
	\widetilde{\mathbf{S}} \triangleq \mathbf U^T \mathbf V_2^N \mathbf W \bm\Sigma^{-1} = \mathbf U^T \mathbf A \mathbf U
\end{equation}

It can be seen that this relationship between $\widetilde{\mathbf{S}}$ and $\mathbf A$ matches the definition of \emph{matrix similarity}. Thus, $\widetilde{\mathbf{S}}$ is similar to $\mathbf A$ and as a consequence their eigenvalues $\lambda_i$ are the same, with the advantage that $\widetilde{\mathbf{S}}$ is of reduced size and easier to solve. The DMD modes $\Phi_i$ can finally be calculated from the eigenvectors $y_i$ of $\widetilde{\mathbf{S}}$ as:
\begin{equation}
	\Phi_i = \mathbf U y_i
\end{equation}

The frequency associated with each spatial mode can be recovered by evaluating the imaginary part of each complex eigenvalue, and taking into account the time step $\Delta t$ between snapshots:
\begin{equation}
	f_i = \frac{\omega_i}{2\pi} = \frac{\Im\{\ln(\lambda_i)\}}{2\pi\Delta t}
\end{equation}

As for ranking the contribution of each mode to the total flow field, several approaches can be followed. Since many eigendecomposition routines normalize the eigenvectors \cite{futrzynski2015dymode}, simply computing their $\ell_2$ norm is not always reliable. 

The energy of the modes can be recovered by solving a linear system in which the DMD-reconstructed field multiplied by unknown amplitudes is compared against a selected snapshot \cite{futrzynski2015dymode}. It is also possible to optimize these amplitudes to promote a more sparse reconstruction of the field with fewer modes \cite{jovanovic2014sparsity}. 

Another ranking metric was considered by Schmid \cite{schmid2010dynamic} in his original DMD formulation, based on the computation of the modes' coherence in time. This can be done by first calculating matrix $\mathbf G$ \cite{delorne2014dynamic}:
\begin{equation}
	\mathbf G=\mathbf V_1^{N-1}\bm\Sigma^{-1} \mathbf Y
\end{equation}

Where $\mathbf Y$ is the matrix that gathers the eigenmodes $y_i$ of $\widetilde{\mathbf{S}}$. Then for each column $\mathbf g_{*,i}$ of $\mathbf G$, the coherence $E_i$ of each DMD mode $\Phi_i$ can be obtained as the inverse of its norm:
\begin{equation}
	E_i = ||\mathbf g_{*,i}||^{-1}
\end{equation}

A disadvantage of this solution for ranking the modes is that the true magnitudes of the vectors for flow reconstruction purposes remain unknown. If flow reconstruction is desired, for instance aiming to simplify the application of the Ffowcs Williams-Hawkings acoustical propagation procedure \cite{williams1969sound} as done by Dahan et al. \cite{dahan2014aero}, solving the amplitudes in a reference snapshot as done by the \texttt{dymode} code \cite{futrzynski2015dymode} or the use of the sparsity-promoting procedure by Jovanovi{\'c} et al. \cite{jovanovic2014sparsity} are recommended instead.

\subsubsection{Sample DMD results}

In order to provide an example of application of DMD to the CFD simulation considered in this chapter, preliminary work was done by implementing the original SVD-based DMD procedure with coherence ranking.

\begin{figure}[b!]
\centering
\includegraphics[width=0.88\textwidth]{figures/cap_cfd/dmd_modes_coher}
\caption[Normalized coherence of the outlet surface DMD modes]{Normalized coherence of the outlet surface DMD modes, highlighting the top 10 and including normalized magnitude of the average PSD.}
\label{fig:dmd_modes_coher}
\end{figure}

The procedure was applied to a limited subset of the available data, specifically to the dynamic pressure information at cells near the outlet wall, as in this particular CFD setup information about the pressure at every cell of the domain was not saved. 

Due to the POD/DMD property that allows the selection of a subdomain instead of the whole domain, results in the selected zone should however be representative. Furthermore, data downsampling was performed both spatially and temporally by considering one in ten cells and one in ten snapshots, resulting in a spatial length of $M\sim 7\times 10^3$ and a temporal length of $N\sim 10^3$.

\begin{figure}[hb!]
\centering
\includegraphics[width=0.98\textwidth]{figures/cap_cfd/dmd_modes}
\caption[Normalized amplitude of selected DMD modes (coherence)]{Normalized amplitude (lower values in blue, higher values in red) of selected outlet surface DMD modes featuring a higher temporal coherence.}
\label{fig:dmd_modes}
\end{figure}

In Fig.~\ref{fig:dmd_modes_coher} the normalized coherence $E_i/E_\text{max}$ of each DMD mode $\Phi_i$ has been plotted, highlighting the top 10 modes with the higher coherence. An average PSD of the pressure content is also presented, showing a good agreement with the mode coherence. In both cases the maximum is located at a frequency $f_i\sim 16$ kHz, which in the computed modal space corresponds to mode $\Phi_{323}$ and is related to the blade passing frequency (BPF).

Besides the BPF tonal noise, another relevant DMD mode appears at approximately half of this frequency (7.6 kHz). Coincidentally, this specific frequency is similar to the onset frequency of the first asymmetric acoustic mode (7.4 kHz). At lower frequency (<3kHz) several coherent modes appear, demonstrating the relevance of this particular frequency band, typical of \emph{whoosh} noise. However, it appears that content in this band is not dominated by a particular mode. Finally, modes related to the TCN-like broadband noise seem to be less spatially coherent than those related to \emph{whoosh} and BPF tonal noise.

The spatial distribution of some DMD modes is plotted in Fig.~\ref{fig:dmd_modes}. It can be seen in this figure how the content of the spatial modes is distributed along the outlet duct. The first three modes of lower frequency ($\Phi_\text{0.8 kHz}$, $\Phi_\text{1.6 kHz}$ and $\Phi_\text{2.1 kHz}$) feature a distribution that is related to the plane wave propagation, with the modes at higher frequency showing a more complex distribution as a result of the onset of the different acoustic modes.

The influence of mode frequency and acoustic mode propagation in the distribution of the spatial content of the DMD modes can be seen in Fig.~\ref{fig:dmd_modes_freqs}, where some DMD modes are selected to showcase this evolution. 

In the leftmost plot the mode corresponding to the 0 kHz frequency has been selected, with the result showing the expected uniform distribution along the outlet duct (save for some small regions that are probably the result of numerical error). As frequency increases a pattern typical of a standing wave starts to appear in the simulated duct. Note specially the similarity between the distribution of the $\Phi_\text{3 kHz}$ mode and the magnitude of the FFT in the compressor outlet duct shown in Fig.~\ref{fig:pressure_fft_walls}. 

\begin{figure}[t!]
\centering
\includegraphics[width=0.98\textwidth]{figures/cap_cfd/dmd_modes_freqs}
\caption[Normalized amplitude of selected DMD modes (propagation)]{Normalized amplitude (lower values in blue, higher values in red) of selected outlet surface DMD modes showcasing the effect of acoustic mode propagation.}
\label{fig:dmd_modes_freqs}
\end{figure}

When higher order acoustic modes start propagating in the CFD simulation, the pattern changes. DMD mode $\Phi_\text{11.5 kHz}$ shows the effect of the first asymmetrical mode, denoted as (0,1) in Eriksson's notation \cite{eriksson1980higher}, as the content is no longer homogeneous in each section but dependant on the angle as well. Finally, mode $\Phi_\text{11.5 kHz}$ features a very complex pattern as modes (0,1), (0,2), (0,3), (1,0) and (1,1) are propagating at these higher frequencies.

\section{Conclusions}

In this chapter a numerical simulation of the compressor characterized in chapter \ref{cap:results} has been validated regarding not only the global operating variables but specifically taking into account the acoustic point of view, which is usually not addressed in detail in many works regarding the simulation of automotive compressors.

It has been shown how the definition of the domain and boundary conditions has a large influence on the spectral information that can be recovered from the simulation results, by creating spurious reflections and standing waves that are not present in the experimental setup.

To address this issue while keeping the CFD domain and mesh within the limits of usual computational capacity the use of pressure wave decomposition has been shown both in numerical and experimental results, allowing enhanced isolation and recovery of the spectral information coming from the compressor and filtering spurious interferences.

Selection of the monitor type and location is also crucial from the acoustic point of view, as it is taking into account the propagation of higher order modes that may render results that are dissimilar from experimental reference data. Wall-flush pressure monitors demonstrated better agreement overall with the experimental measurements, although the area-averaged type of monitor showed better agreement in the outlet at plane wave frequencies, specially regarding the prediction of a \emph{whoosh} frequency broadband at the operating condition of higher pressure ratio which signals the onset of marginal surge conditions and the appearance of reversed flow.

Analysis of the pressure and velocity fields simulated by the CFD models has been shown, in order to identify flow phenomena that may be related to the increased acoustic output at these unstable operating conditions. Contours of velocity and pressure at selected planes and surfaces show rotating low pressure ``bubbles'' that are linked to the zone of reversed flow, visualized through null axial velocity isosurfaces.

Inspection of these velocity isosurfaces offers a way of predicting the evolution of the backflow as operating conditions change. The point of maximum pressure ratio shows the inception of these backflows in the periphery of the wheel. At the same time, velocity information from a 50\% isospan surface shows areas of increased velocity in the diffuser near the volute tongue, except for the case of the higher mass flow case. PSD from the velocity signal in the diffuser shows a peak at 2.9 kHz accompanied by two side peaks, which may be related to the \emph{whoosh} phenomenon.

Moreover, vortex shedding from the compressor blades however is detected in the three simulated conditions, which may be linked to the acoustic emission as they induce a pressure fluctuation in the channels that is rotating with the shaft speed.

Further postprocessing techniques to inspect the location of acoustically-relevant flow features includes the use of the FFT of the pressure signal at selected spatial locations and frequencies of interest. An example has been presented regarding the 3 kHz frequency, showing special prominence of this frequency at the compressor wheel and diffuser.

Finally, in order to avoid the necessity of guessing which are the relevant frequencies and spatial locations, the use of modal decompositions of the flow field has been proposed, discussing advantages and disadvantages of different techniques. A limited-scope example of the formulation and implementation of the Dynamic Mode Decomposition has been shown, with the coherence of the model identifying relevant frequencies of the flow and their associated spatial distribution on the outlet duct wall cells, particularly related to \emph{whoosh} and BPF tonal noise.

While implementation of the DMD technique over the full numerical domain is outside of the scope of this work, as it requires the implementation of computational tools to link Star-CCM+ snapshots with outside libraries that allow the manipulation of data in the order of gigabytes, it has been shown that further research on this topic could lead to a better identification of the spatial location of the most relevant frequency components of the flow field.

%------------------------------------------

% \subsection{Comparison with CFD}

% \hl{[To the inlet flow chapter]}

% \begin{figure}[htb!]
% \centering
% \includegraphics[width=0.88\textwidth]{figures/cap_cfd/cfd_circumf_comparison}
% \caption{Isosurfaces of null axial velocity in the compressor inducer/inlet showing the reversed flow extent and distribution, including isosurfaces of low pressure ``bubbles'' in the same region.}
% \label{fig:cfd_circumf_comparison}
% \end{figure}

% \begin{figure}[htb!]
% \centering
% \includegraphics[width=0.88\textwidth]{figures/cap_cfd/cfd_longit_temp_comp}
% \caption{Isosurfaces of null axial velocity in the compressor inducer/inlet showing the reversed flow extent and distribution, including isosurfaces of low pressure ``bubbles'' in the same region.}
% \label{fig:cfd_longit_temp_comp}
% \end{figure}