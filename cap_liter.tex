%!TEX root = Tesis.tex
\chapter{Literature review} \label{cap:liter}

% Chapter text
\noindent In order to provide a better context for the development of this work, a literature review has been carried out. Special emphasis has been put into existing experimental research of automotive turbocharger acoustics, although techniques and results from related fields are also presented when applicable.

For instance, numerical simulations on the topic are becoming more numerous. They offer further insight into the underlying flow structures even if clear, causal correlations have not yet been demonstrated. As such, these studies merit their own specific section. 

Another point related to these flow structures that deserves interest is the analysis of how the particularities of the inlet geometry immediately upstream of the compressor wheel influence the performance and the noise emission of the system. Not only this influence offers clues of which mechanisms are responsible for certain acoustic phenomena, but they also present an interesting possibility of mitigation.

\section{Experimental research}

While the acoustical emission of axial compressors has been a  subject of extensive research since more than fifty years ago \cite{nvemec1967noise,griffiths1964spectrum}, studies performed on centrifugal turbocharger are more comparatively more scarce. 

Of the existing literature dealing with centrifugal turbomachinery, many efforts were directed at large fans or centrifugal pumps, with small turbochargers have begun only recently to be more relevant.

Among the first it is worth to mention the extensive research carried out at Pennsylvania State University, where at least since the nineties \cite{mongeau1993sound} an instrumented centrifugal pump was installed in an anechoic flow bench to characterize its acoustical performance.

This setup allowed several researchers to conduct experiments analysing not only in-duct noise, as in the works of Mongeau et al. \cite{mongeau1993sound,mongeau1995method} and radiated noise, as shown by Choi et al. \cite{choi1994aerodynamic,choi2003experiments}, but also pressure distribution on the blade surface \cite{mongeau1993sound,choi1994aerodynamic}.

A related experimental campaign, starting at least from the turn of the millennium, was carried out at the University of Siegen. Here, Wolfram and Carolus \cite{wolfram2010experimental} used again a centrifugal pump inside an anechoic chamber. 

In this case, measurements were done relying in anemometers directed at the blades that could be rotated at will in order to analyse azimuthal rotating modes \cite{wolfram2009detection}. Radiated noise could be also measured using a single microphone \cite{carolus2000experimental}.

Both Penn State and Siegen campaigns show significant aspects of interest such as the identification of Blade Pass Frequency (BPF) peaks on the acoustic spectrum and hints on the link between stall and noise \cite{mongeau1993sound}. Relevant dimensionless numbers such as the Strouhal and Helmoltz numbers are also suggested \cite{mongeau1995method}. 

However, it is important to notice that there are important differences between these large centrifugal pumps and small turbocharger compressor for automotive purposes.

These differences are found not only in the geometrical design aspect but also in the order of rotational speed. While the considered pumps do not exceed 3600 rpm, small automotive turbochargers can exceed 200 krpm. 

Also, they are affected by the pulsating flow of the reciprocating engine, which will be present in the acoustic signature as well. This later aspect will be shown in chapter \ref{cap:engine}.

In order to address these differences and focus the literature review in the specific issues presented by turbocharger compressors, selected works from the literature will be reviewed in the following subsections, placing special attention in both the measurement techniques and the acoustical results obtained.

\subsection{Methods for turbocharger noise characterization}

Although not numerous, acoustical research works on small, centrifugal turbocharger compressors can be found in the literature. Many of them are published by automotive or compressor manufacturers, who are among the most concerned about compressor noise emissions.

Starting from the work of Trochon \cite{trochon2001new} and Evans and Ward \cite{evans2005minimizing,evans2006reduction} up to more recent investigations such as those of Åbom, Kabral and Rammal \cite{kabral2013acoustical,kabral2014investigation} at the Swedish Royal Institute of Technology (KTH), several different methodologies for characterising the acoustical emission of automotive turbochargers can be found.

A first division can be made between those studies aiming to measure \emph{internal} flow noise, placing sensors inside the compressor housing or inlet and outlet ducts, and those works based on external measurements of the \emph{radiated} noise.

Evans and Ward \cite{evans2005minimizing,evans2006reduction} opted for a simple setup of four microphones to measure sound radiation. These four microphones were situated on the same plane and directed to a turbocharger-equipped engine. They were distributed at 90º with a separation of one meter from the engine. From the pressure readings of these four microphones, Sound Pressure Level (SPL) was computed and then averaged among them, in order to get a single signal from which spectrograms and sound spectra could be derived.

A similar setup was used by Teng and Homco \cite{teng2009investigation} \hl{(see specially Fig. 1)} in order to test different countermeasures for whoosh noise mitigation, such as engine calibration, resonators in the compressor inlet/outlet, compressor trim variation and a swirl generation device. 

This kind of averaging four spatially-distributed microphones method was implemented in the anechoic chamber in order to characterize the far-field radiated sound of the turbocharger group during the experimental campaign; further information will provided on subsection \ref{sub:metod_far_field} of chapter \ref{cap:metod}.

Apart from the far-field radiation, another acoustic problem of interest is that of \emph{orifice noise}, this is, the noise created by the ingestion of the air stream into the opening of the turbocharger inlet duct. 

Lee et al. \cite{lee2011control} used a Brüel \& Kj\ae r free field type microphone pointed at the compressor inlet as seen in \hl{Fig. 8 of the reference}. It can also be noted that in order to avoid ``flow effects'', the microphone was located 0.8 away from the inlet.

The setup selected by Guillou et al. \cite{guillou2010characterization} to measure ingestion noise was very similar. Here, a single microphone was positioned aligned with the axis of the turbocharger at a distance of 0.4 m, again facing directly to the compressor inlet bell mouth.

Another solution was implemented by Figurella et al. in \cite{figurella2012noise}. In this case, a Brüel \& Kj\ae r type 2235 Sound Level Meter was positioned 0.25 m from the inlet duct orifice, at a 45º angle from the inlet axis. In \cite{figurella2014effect} Figurella et al. opted instead to rely on a ¼'' microphone affixed at the same position, 0.25 m away from the orifice and 45º from the axis. Both instruments appear in Fig. 1 of \cite{figurella2014effect}.

During the experimental campaign that was carried out for the development of this thesis, two free field microphones were situated at a distance of 10 and 20 cm from the inlet pipe orifice. This will be expanded in detail in subsection \ref{sub:metod_orifice_noise} of chapter \ref{cap:metod}.

Finally, a third approach for the characterization of compressor noise is based on measuring the sound field inside the inlet and outlet ducts, usually by means of fast pressure sensors. This has the advantage of not needing to ensure anechoic conditions in the test cell. Another remarkable advantage is that, by analysing the flow field alone, one can avoid the influence of the radiative properties of the duct material and geometry.

As these advantages make this last approach specially interesting for a research centred on the sources of aero-acoustical generated noise, emphasis has been put in reviewing existing literature dealing with internal flow noise.

Early work by Trochon \cite{trochon2001new} dealt with the acoustic characterization of turbocharger noise measuring in-duct airborne noise at the compressor exit \hl{(see Fig. 1 of the reference)} during engine operation.

Gaudé et al. \cite{gaude2008experimental} from Honeywell Turbo Technologies presented an extensive review of different vibro-acoustic sources of turbocharger noise. Both accelerometers and pressure sensors affixed to the pipes were used, and their use to identify some vibro-acoustic phenomena was demonstrated. While this work omits a proper, referenced methodology description, it mentions that acoustic noise intensity is calculated using wave decomposition by means of three sensors, indicating the range of plane wave for the installed ducts.

Raitor and Neise \cite{raitor2008sound} used different methods, taking into account the presence of acoustic modes in the duct. The first one consisted on a ½" microphone equipped with a turbulence screen and mounted in a rotatable section of the duct. This allows the measurement of circumferentially averaged sound pressure level (SPL), in accordance with ISO 5136. This rotatable section can also be fitted with four consecutive rings, each with four flush-mounted condenser microphones for analysis of acoustic modes.

This last technique is similar to that used in the previously mentioned work by Guillou et al. \cite{guillou2010characterization}, where four Kulite model XCS-190-5D miniature pressure sensors were mounted on the compressor inlet adaptor, equally separated 90º. A single Kulite model XTEL-160-50G was used on the outlet duct.

A single in-duct fast pressure sensor was also the technique selected by Figurella et al. \cite{figurella2012noise,figurella2014effect}. In this case Kistler piezoresistive transducers were used, both in the compressor inlet duct (model 4045A2) and in the outlet duct (model 4045A2 or 4045A5). 

Pai et al. \cite{pai2013air,pai2015turbocharger} opted as well for a single sensor (in this case labelled as a microphone) to measure sound pressure level inside the compressor inlet duct. An ambient microphone was present as well, but its readings are not included in their work.

A different approach was taken by Tiijoka et al. \cite{tiikoja2011inves}, who describe the use of an array of three pressure transducers in both inlet and outlet pipes of their compressor setup (see Figs. 1-3 of the reference). In order to estimate the strength of the propagating nodes, a cross-spectra averaging technique was used.

\begin{table}[b!]
  \centering
  \caption{Synthesis of different measurement techniques for the acoustical characterization of turbocharger compressors found in the literature, including type of measurement and location.}
  \footnotesize
	\begin{adjustwidth}{-0.65cm}{}
    \begin{tabular}{lp{7cm}cl}
    \toprule
    \textbf{Reference} & \textbf{Measurement} & \textbf{Type} & \textbf{Location} \\
    \midrule
    Trochon \cite{trochon2001new} & 1 in-duct sensor at the outlet &SPL & Out\\
    Evans \cite{evans2005minimizing} & 4-microphone averaging around engine & SPL   & External \\
    Gaudé \cite{gaude2008experimental} & 2 $\times$ 3-sensor arrays (wave decomp.) & SIL   & In+Out \\
    Raitor \cite{raitor2008sound} & 1 mic or 4 $\times$ 4 mic sections + up to 5 outlet sensors & SPL   & In+Out \\
    Teng \cite{teng2009investigation}  & 4-microphone averaging around engine & SPL & External \\
    Guillou \cite{guillou2010characterization} & 4 Kulite (inlet) + 1 Kulite (outlet) + 1 mic at 0.4 m & SPL   & In+Out+Orifice \\
    Lee \cite{lee2011control}  & 1 microphone at 0.8 m from orifice & SPL   & Orifice \\
    Tiikoja \cite{tiikoja2011inves} & 2 $\times$ 3-sensor arrays (cross spectra avg.) & SPL   & In+Out \\
    Figurella \cite{figurella2012noise} & 2 $\times$ piezores. transducers + SLM at 0.25 m and 45º & SPL   & In+Out+Orifice \\
    Figurella \cite{figurella2014effect} & 2 $\times$ piezores. transducers + 1 mic at 0.25 m and 45º & SPL   & In+Out+Orifice \\
    Pai \cite{pai2015turbocharger}   & 1 microphone (inlet) + 1 microphone (ambient) & SPL   & In+External \\
    \bottomrule
    \end{tabular}
    \end{adjustwidth}
  \label{tab:lit_meas_tech}
\end{table}

In table \ref{tab:lit_meas_tech} a synthesis of the information compiled in this subsection is presented. It can be seen how only one reference \cite{gaude2008experimental} used sound intensity, which is inherently a more robust magnitude. Sound pressure level, on the other hand, is easily influenced by the particular geometric setup and measurement location. It is also apparent that, while measurements have been presented regarding inlet, outlet, orifice and external noise, no study as far (as the author knows) has been presented regarding all four options simultaneously. 


\subsection{Relevant results}

After reviewing the different methods available in the literature regarding turbocharger noise, a selection of relevant acoustical results from these works is analyzed in this subsection. It is worth noting that while some of the referred research was directly aimed at characterizing noise output, other works had different objectives and only used noise measurement as a metric, without discussing its characteristics or generation process.

An interesting introduction is provided by Gaudé et al.  whose work collects a variety of different noise-generating phenomena in turbochargers \cite{gaude2008experimental}. Table \ref{tab:lit_gaude_sources} reproduces this classification. Several tone sources are proposed, mainly linked to the imbalance of the rotor and irregularities and lack of symmetry of the compressor wheel. Another source of tones is the passing of the blades of both turbine and compressor, in special the main ones in the case of the later.

\begin{table}[b!]
  \centering
  \caption{Classification of turbocharger noise phenomena, including frequencies and source mechanisms, as proposed by Gaudé et al. in Fig. 1 of \cite{gaude2008experimental}.}
  \footnotesize
  \begin{adjustwidth}{-0.6cm}{}
    \begin{tabular}{p{2cm}p{3cm}p{2.5cm}p{4cm}}
    \toprule
    \textbf{Name} & \textbf{Spectrum} & \textbf{Frequency range} & \textbf{Source mechanism} \\
    \midrule
    Oil whirl & Tone  & 0 -- 1000 Hz
(function of oil temp.) & Rotor-dynamics vibration of rotating group caused by oil film. \\
    Synchronous vibration & Tone synchronous to turbocharger speed & 0 -- 4000 Hz & Rotor-dynamics
- Residual unbalance of  the rotor. \\
    Synchronous pulsation & Tone synchronous to turbocharger speed & 1000 -- 4000 Hz
(Depends on wheel outer diameter) & Aerodynamics: Pressure fluctuations induced by lack of symmetry of compressor wheel. \\
    Harmonics & Tone = number of irregularities $\times$ speed & 0 -- 20 kHz & Rotor-dynamics or aerodynamics due to irregularities. \\
    Turbine BPF & Tone = number of blades $\times$ speed & $\rightarrow$ 20 kHz & Aerodynamics. \\
    Compressor BPF & Tone = number of full blades $\times$ speed & $\rightarrow$ 20 kHz & Aerodynamics. \\
    Hiss  & broadband noise & 0 -- 20 kHz & Aerodynamics due to turbulence in the compressor stage. \\
    \bottomrule
    \end{tabular}
    \end{adjustwidth}
  \label{tab:lit_gaude_sources}
\end{table}

Besides the different tones and their associated harmonics, a broadband noise is identified as ``hiss'', whose source mechanism is listed as related to turbulence in the compressor stage. The frequency range of this turbulence-related phenomenon is proposed as 0 -- 20 kHz, but no further insight is given.

A similar division was anticipated by Trochon \cite{trochon2001new}, who classified turbocharger compressor noise in ``pulsation noise'' (the tonal noise caused by the blade rotation) and ``blow noise'', a broadband noise (named this way for its resemblance to noise produced by a leakage). It is attributed by the author to aerodynamic turbulence ``inside the compressor'' at high mass flow and low shaft speed. By examining Fig. 1 of the reference we can assess the presence of such a broadband noise between 1.5 and 2.5 kHz.

Evans and Ward \cite{evans2005minimizing} address explicitly the issue of a so-called ``whoosh'' noise. This noise is identified as being ``in general'' of the broadband type. The authors state that this noise is typically present from 750 Hz to in excess of 10 kHz. they note however that most of the energy is apparently distributed between 1.5 and 3 kHz. This appears to be similar to the ``hiss'' noise mentioned in the previous reference, but in this case the authors point out that the occurrence of whoosh appears to be specifically associated with operation close to surge. In this situation, blade stall, reverse flow and recirculation of intake air generate increased turbulence.

Raitor and Neise \cite{raitor2008sound} mention instead the so-called ``tip clearance noise'' (TCN), referring to a noise with narrow band spectral components below the blade passing frequency (BPF). A similarity is drawn with the work of Kameier and Neise \cite{kameier1997rotating} regarding axial turbo compressors, where a rotating instability linked to stall cell propagation is discussed. This TCN is marked in Fig. 8 of the reference as located between 3 kHz and 5.5 kHz in the inlet duct. BPF is also marked in the figure, including its harmonics. An intense ``buzz-saw'' multi-tonal noise is apparent at higher speeds in the inlet, but not so in the outlet (see Fig. 10 of the reference). A TCN band appears in the outlet duct as well. Mach speed at the impeller tip is also highlighted as signalling the transition between TCN and BPF being the dominant source mechanism.  Further on, the appearance of buzz-saw noise in the inlet appears to be marked by the transition from transonic to supersonic regime.

Following the work of Evans and Ward, Teng and Homco \cite{teng2009investigation} obtained further insight on the precise location of the whoosh noise for their setup. Especially, in Fig. 5 of the reference it can be seen that whoosh noise appears clearly in the 4.5 to 7 kHz band. BPF tone is not seen on the figure, which is limited to a maximum frequency of 10 kHz. This suggests that BPF is either of higher frequency, a condition which for the kind of compressor described would be normal, thus indicating that this narrow band noise is again below this frequency. In order to confirm the relation between surge inception and whoosh noise, a type K thermocouple was affixed very close to the compressor inlet. An apparent correlation between temperature rise and 5 to 7 kHz sound pressure content is found in Fig. 7 of the reference, suggesting that the presence of hot back-flow air due blade stall is related to the identified whoosh noise.

Guillou et al. \cite{guillou2010characterization} do not directly address the noise spectra in their study. Nonetheless, the authors plot the pressure FFT in the outlet duct for various flow regimes. It can be seen in Fig. 7, specially in the lower right plot (stable regime), how a clear narrow band appears between 2 and 7 kHz. This narrow band continues to be present in both stall and surge regimes, but it is vastly made irrelevant by the strong low frequency signals that are typical of surge.

Lee et al. \cite{lee2011control} focus their work on ``pulsation noise'', asserting that it is caused by unbalance of shaft and blades, and BPF noise, caused by the cyclic motion of the blades. Its harmonics are also observed and mentioned (see Figs. 9 and 10 of the reference). The authors also observe a ``narrow band noise'' in the same figures, but they argue that, since it was not reported by the turbocharger manufacturer, it can be dismissed as caused by the experimental setup. Note that this narrow band noise appeared in this case between 3 and 4.25 kHz. Interestingly, it was also present even when the compressor was running without inlet and outlet ducts (see Fig. 11 of the reference).

Tiikoja et al. \cite{tiikoja2011inves} revisit the topic making again a distinction between two contributing phenomena. Stationary distortions are identified as leading to periodic tones. Given an absence of inflow disturbances these tones are referred to as rotor-alone tonal noise. Fluctuating pressure fields caused by the interaction of inlet and outlet guide vanes and the rotor is regarded as a secondary source which will introduce tonal noise at harmonics of the BPF. On the other hand, a wide band noise is ascribed to non-stationary disturbances (turbulence), but its importance for turbochargers is dismissed. Denomination of buzz-saw noise for the tonal phenomena is used and attributed to rotating shock waves attached to the blades whose tips reach supersonic conditions. TCN is also named and marked as such in the results, appearing in this case between 5 and 10 kHz in the outlet and between 4 and 7.5 kHz in the inlet, again below the BPF tone clearly visible at 12 kHz (see Fig. 13 of the reference).

\begin{wrapfigure}{O}[1.5cm]{0.4\textwidth}
\centering
\vspace{-5mm}
\includegraphics[width=0.35\textwidth]{./figures/cap_liter/freq_whoosh_lit}
\caption{Frequency amplitude of broadband noise found in selected turbocharger compressor literature, by principal author.}
\label{fig:freq_whoosh_lit}
\end{wrapfigure}

Similar results were obtained by Figurella et al. \cite{figurella2012noise}. Discrete sound peaks (tones) were cited as the dominant phenomena, located at synchronous (rotor) speed and blade pass frequencies (rotor speed times number of main blades). A ``broadband flow noise'' is named as ``whoosh or blow noise''. Results were presented locating this noise as concentrated between 4 and 12 kHz. However, inlet sound pressure level spectra presented on Figs. 4 and 5 of the reference seem to indicate a more narrow band elevation, only between 5 and 7 kHz, well below the BPF at $\sim$11 kHz. Later work by Figurella et al. \cite{figurella2014effect} continued to make use of the 4--12 kHz band and no further individual spectra were presented. Sound maps of the compressor located a clear increase of noise in this band at conditions close to ``marginal surge'', where the gradient of isospeed lines is close to zero, suggesting again the relation with the onset of recirculating backflows.

In the work of Pai et al. \cite{pai2015turbocharger}, whoosh noise is mentioned only as a ``broadband noise'' but no specific frequency band is proposed. Again it is linked at operation near surge, for instance in transient conditions such as throttle tip-in or tip-in tip-out, but also to certain steady driving condition in petrol engines. A mechanism is proposed where acoustic radial mode resonance may be acting as amplifier for pressure perturbations in the 3 -- 4 kHz band. Careful inspection of the spectrogram in Fig. 6 of the reference seems to suggest, however, that if whoosh noise was indeed present at the measured operating condition, its frequency location was between 4 and 5 kHz.

In synthesis, two different phenomena are consistently identified in the literature about turbocharger compressors. One consists of the tonal noises linked to the rotation of the wheel. In order to work, compressor blades must maintain a pressure difference between their pressure and suction sides. This creates rotating pressure differences, being the rotation of the blades the mechanism imposing the frequency. Harmonics of this frequency may appear, especially in transonic or supersonic conditions where shock waves attach to the blades, thus creating more rotating pressure gradients.

On the other hand, there is sufficient evidence of a recurrent broadband noise at frequencies below the BPF, having been reported between 1.5 and 7.5 kHz, as seen in Fig. \ref{fig:freq_whoosh_lit}. In contrast with BPF tonal noise, there is no clear mechanism proposed as being the cause of this phenomenon. As seen above, some authors vaguely ascribe it to ``turbulence'', others to stalled backflow interaction (including modal resonance), tip clearance effects, etc. It is apparent that further research is needed to clarify this issue.

\section{Numerical simulations}

Given the clues about the relationship between this broadband noise and three-dimensional flow effects in and near the compressor impeller, and at the same time the difficulties of their accurate experimental characterization, it is worth considering the numerical simulation of the problem through Computational Fluid Dynamics (CFD).

The previously mentioned work by Lee et al. \cite{lee2011control} included a CFD simulation of the compressor, matching experimentally measured operating conditions. However no agreement was found between predicted and measured noise spectra (see Figs. 17 and 18 of the reference), possibly due to reuse of only a single-revolution simulation. Karim et al. \cite{karim2013computational} successfully used experimental results to confirm a CFD prediction of whoosh noise diminishing, achieved by reducing length of leading edge step. However, only overall results are shown, not noise spectra.

Plausible noise spectra of turbocharger compressor were shown by Mendonça et al. \cite{mendonca2012simulation}. By examining Fig. 7 of their work, it is possible to recognize the two aforementioned leading noise phenomena: BPF tonal noise and its harmonics, and a broadband noise at a lower frequency. In this case, the broadband is expressed as percentage of rotation order (RO\footnote{Rotation order (RO) is frequency normalized by shaft speed (Hz/rotations per second) so that RO = 1 is the frequency of shaft rotation and RO of BPF is equal to the number of main blades.}), being located approximately at 0.5 to 0.9 RO (50\% to 90\% of shaft rotating frequency). BPF peaks are noticeable only in the outlet ducts. 

This last work appears to demonstrate that CFD simulations are indeed able to reproduce noise phenomena whose existence was experimentally assessed. However, caution should be exercised since (as shown for instance by the results of Lee \cite{lee2011control}) it is possible to produce CFD results that accurately predict overall or averaged variables (efficiency, compression ratio, mass flow) or even overall noise levels, but whose frequency content does not match with experimental observations at all.

As a consequence, while plenty of works regarding turbocharger CFD simulation exist in the literature, only those that present plausible validation against \emph{acoustic} experimental measurements (and not only against overall or averaged thermodynamic measurements) can be relied upon in order to extract insights about noise generation.

For instance, Hellstrom \cite{hellstrom2010stall} presented CFD simulation results related to the experimental visualization work done by Guillou \cite{guillou2011flow}, where qualitative agreement is found between the predicted and measured flow structures. Of special interest when considering CFD results is data such as the iso-surface of negative axial velocity shown in Fig. 10 of the reference, showing a spiral-like backflow structure that would be more difficult to identify experimentally.

However, it is also possible to consider research works that, while not specifically acoustic in nature, offer experimentally validated information about the flow field near the compressor inlet, specially if simulated operating conditions were chosen with the intention of analysing stall or surge phenomena. The work of Margot et al. \cite{margot2008combination} demonstrates how deep surge can be numerically predicted and how low pressure zones upstream of the compressor wheel lead to reversed flow in the periphery of the impeller (see Fig. 11 of the reference).

Numerical analysis of this kind of flow reversion was carried out by Lang \cite{lang2011contribucion}. Unsteady simulations of the compressor, including the outlet duct with a throttle valve (replicating that of the experimental facility that was used for validation) were performed, decreasing valve area to reduce flow from stable to surge operating conditions. Good agreement was found between predicted and measured instantaneous pressure. 

Furthermore, three different inlet geometries were simulated: a straight duct, an elbow and a reservoir. In the straight inlet case, a slight tapering of the duct is present. Even in this case, reversed flow seems to extend up to $\sim 2.5$ inducer diameters from the wheel leading edge. Different patterns for reversed flow are attested for each geometry. Influence in surge margin is clearly demonstrated both numerically and experimentally.

\section{Local inlet flow field}
\label{sec:liter_local_flow}

% Regarding unstable fluid phenomena in compressors, the most known issue since the early days of large steam-powered turbomachinery \cite{stodola1927steam} is deep surge. Deep surge occurs when the flow completely detaches from the impeller blades and, free of opposition, the compressed flow reverses direction and surges upstream the impeller and into the inlet duct until the momentum of the incoming air forces air back into the compressor, starting a new surging cycle.

% This hazardous phenomenon does not, however, appear spontaneously. It was promptly observed \cite{bullock1946experimental} that surge only happens after the slope of characteristic curve (this is, the curve drawn by the pressure ratio as air mass changes at a given constant shaft speed) reaches a maximum and changes direction, the pressure ratio decreasing instead of increasing as flow is further reduced.

% Sometimes called mild surge conditions \cite{fink1992surge}, this change in slope is produced by blades partially stalling and hot compressed flow reversing along the periphery of the impeller and then reintegrating into the core flow as the central part of the impeller still keeps operating.

% Unstable operation conditions where these complex backflows are present in the inlet zone have also been linked to an increase in the unpleasant noise emitted by the compressor \cite{evans2005minimizing,teng2009investigation}, a concerning issue for automotive manufacturers as public demand for quieter vehicles  rises.

Accurate characterization and modelling of the aforementioned unstable reversed flows is thus necessary to validate CFD results and design and evaluate the effect of corrective measures. Some indirect measurements have been made taking advantage of the temperature rise that the inlet experiences because of the high temperature of the compressed flow \cite{andersen2009surge,figurella2012noise}. Low frequency content and standard deviation of the pressure signals have also been used to track the onset and growth of these unstable phenomena \cite{liu2013methods}.

Other tests have been performed to directly measure the flow velocity fields through different velocimetry techniques such as Laser Doppler Anemometry (LDA) \cite{pedersen2003flow}, hot-wire anemometry (HWA) \cite{oro2009forced,kalpakli2015combined} and Particle Imaging Velocimetry (PIV) \cite{wernet2000development,liu2006application,pedersen2003flow}.

The latter technique is desirable for CFD validation as it allows the measurement of whole sections of the field instead of a single point such as in the HWA and LDA techniques. Regarding the use of PIV in turbomachinery, many of the studies available in the literature have been performed on large centrifugal pumps \cite{pedersen2003flow,dazin2011high,wu2011piv}, axial fans \cite{fike2014visualisation,oro2009forced} and axial compressors \cite{liu2006application,wernet2000development,wernet2000application,wernet2001investigation}, usually with the aim to study the evolution of the flow through the passages and the interaction with the diffuser vanes. 

In the case of the works presented by Liu et al. \cite{liu2006application} and Wernet \cite{wernet2000development,wernet2000application} an effort was also made to analyze the distribution of flow in planes perpendicular to the channel direction so that the transversal velocity field could be observed.

Even when proper centrifugal compressors were studied, attention was still put in the diffuser passages, such as in the works of Voges et al. \cite{voges2007application}, Ohuchida et al. \cite{ohuchida2013internal} and Cukurel et al. \cite{cukurel2010particle}. In the three cases a small transparent window is placed in the diffuser casing to record the passage flow and interaction with the impeller wake.

The most relevant studies that instead of the diffuser considered the compressor inlet and the aforementioned backflow issue are those published by Guillou, Gancedo et al. \cite{guillou2010characterization,guillou2010surge,guillou2012piv,gancedo2016piv} who used a test rig where a turbocharger compressor was fitted with a bell mouth open inlet so that images of the flow could be taken directly without any barrier as the seeded ambient air was ingested. Hellström et al. \cite{hellstrom2010stall} also used this setup to perform a comparison with numerical CFD simulations.

During these experimental campaigns measurements of the transversal velocity fields were conducted by illuminating the flow outside of the inlet with a laser sheet perpendicular to the compressor axis and close to the bell mouth. An interesting development of this approach was made in \cite{gancedo2016piv} where a short polycarbonate ported shroud was fitted between the bell mouth and the compressor. The laser sheet was then made to pass at different planes through this transparent shroud.

On the other hand, axial velocity fields were measured in \cite{guillou2010characterization,guillou2012piv} by aligning the laser sheet with the compressor axis, and taking the snapshots not perpendicular to it but at an angle, as the bell mouth was not transparent. By doing this with two cameras an stereoscopic algorithm could be used to reconstruct the true axial velocity field as if the camera were indeed perpendicular to the illuminated plane.

However, it must be noted that in real world automotive applications the compressor inlet is not open in this fashion but attached to a long duct that houses the filter and flow meter. These ducts confine the reversed flow and force its reintegration into the main flow. And since has also been proved that the inlet geometry close to the compressor holds special relevance regarding onset and frequency of the deep surge instabilities \cite{galindo2013engine} effort should be put into a characterization where the reversed flow is confined in he duct.

\section{Influence of inlet geometry}
\label{sec:liter_influence}

Besides the aforementioned investigation of Lang \cite{lang2011contribucion}, this influence of the inlet geometry modification immediately upstream of the compressor wheel (and thus, air presentation) on the performance and surge margin has been established on the literature, as seen for example in Galindo et al. \cite{galindo2007potential}, Serrano et al. \cite{serrano2013optimization} and Wang et al. \cite{wang2013change}.

Nonetheless, research has been focused on the improvement of surge margin and efficiency, while information about the influence of air presentation on turbocharger compressor noise is scarce, if existent at all. Consequently, one of the objectives of the present work is to experimentally assess this acoustical influence and analyze the features of the flow field that may be related to this acoustic emission.

This assessment must consider inlet modifications that are relevant to recent industry developments and requirements, while at the same time being simple enough to allow the establishment of general guidelines instead of conclusions limited to very specific devices.

Reviewing published works, one of the most simple modifications consists of the installation of an elbow upstream of the compressor inlet. These elbows are used sometimes to overcome packaging limitations. However, they cause pressure loss \cite{engeda2003inlet,li2012numerical} and flow distortion \cite{serrano2013optimization} that have adverse impact on compressor efficiency.

Concerning the acoustical influence, Li et al. \cite{li2012numerical} do consider the influence of the elbow over pressure spectral content at various locations. Of special interest is the analysis of BPF amplitude variation, which is attested experimentally (see Figs. 16 and 18 of the reference). The authors concluded that some interaction between elbow-induced flow non-uniformity and the volute tongue was indeed affecting acoustic output, at least regarding tonal BPF noise, but also that more research was needed.

Other geometries of interest that have been tested in the literature are tapered ducts \cite{galindo2011measurement,lang2011contribucion} and reservoirs \cite{lang2011contribucion}. As mentioned earlier, Lang appreciated up to 15\% improvement in surge margin optimizing reservoir volume, a figure similar to that reported by Galindo et al. \cite{galindo2011measurement} regarding improvement achieved by tapered ducts. Similarly, a convergent-divergent nozzle was proposed by Desantes et al. \cite{desantes2011potential}, improving surge margin up to 25\% at the cost of decreased maximum air flow. However, impact of these solutions on noise output was not considered.

More sophisticated options to enhance surge margin have been also proposed. For instance, a large amount of authors have researched the use of swirl generator devices placed upstream of the compressor inlet \cite{kyrtatos1980application,coppinger2000performance}. 

The aforementioned work by Karim et al. \cite{karim2013computational} showed how a certain combination of swirl vanes could diminish overall SPL over the 6--12 kHz band. Furthermore, Galindo et al. \cite{galindo2007potential} considered both positive and negative rotation devices, assessing their impact on surge margin and efficiency. Still, acoustic output was not taken into account.

Addition of a ported shroud has also been reported as an effective method of improving surge margin \cite{guillou2010characterization,semlitsch2014numerical}. In this solution, flow bleed slots are introduced on the impeller wall, connecting with the compressor inlet. The objective is that reversed flow should recirculate through these slots that present lower pressures due to Venturi effect, instead of causing the usual flow disturbance reversing across the tip clearance. At high mass flow, these slots act as bypasses, improving choke limits. 

Regarding the acoustical influence of these solutions, Chen and Yin \cite{chen2006turbocharger} evaluated the influence of different ported shroud geometrical aspects, including the effect of splitter blades on the compressor wheel. The authors reported that full bladed compressors (as opposed of those with full and splitter blades) equipped with a ported shroud exhibited a decrease in both BPF (tonal) and overall noise.

\section{Conclusions}

After reviewing several works concerning diverse aspects of turbocharger compressor noise generation, some general conclusions can be drawn about the state of the art:

\begin{itemize}
	\item While axial compressors have been subject to aero-acoustical studies since more than fifty years ago, noise generation in small centrifugal compressors has been comparatively more scarce.

	\item Several different methods for experimental acoustical characterization of turbochargers can be found in the literature. Among others, variations include instrumentation, measured variables, sensor location, etc.

	\item Studies consistently identify two main acoustical phenomena: tonal noises related to a blade passing frequency (BPF) and its harmonics, and a broadband noise at frequencies lower than those BPFs. The later is usually linked to stalling blades and correspondent back flows, but a clear mechanism is not yet firmly established.

	\item Computational Fluid Dynamics (CFD) simulations offer relevant insights about these stalling and recirculation issues, offering at the same time the capacity of testing different geometries to mitigate them. However, consistent and accurate prediction of measured noise spectra is still difficult to achieve.

  \item Characterization of the unstable flow phenomena in the compressor inducer and inlet has been performed using temperature and pressure indicators. More complete velocity field measurements using PIV are usually performed compressors fitted with an open bell mouth, withouth confining the reversed flow in a straight pipe. 

	\item Modification of compressor inlet geometry immediately upstream of the compressor wheel has a demonstrated effect on surge margin and compressor efficiency, but its influence on noise generation has been scarcely analyzed.

	\end{itemize}