%!TEX root = Tesis.tex
\chapter{Literature review} \label{cap:liter}

% Chapter text
\noindent In order to provide a better context for the development of this work, a literature review has been carried out. Special emphasis has been put into existing experimental research of automotive turbocharger acoustics, although techniques and results from related fields are also presented when applicable.

For instance, numerical simulations on the topic are becoming more numerous. They offer further insight into the underlying flow structures even if clear, causal correlations have not yet been demonstrated. As such, these studies merit their own specific section. 

Another point related to these flow structures that deserves interest is the analysis of how the particularities of the inlet geometry immediately upstream of the compressor wheel influence the performance and the noise emission of the system. Not only this influence offers clues of which mechanisms are responsible for certain acoustic phenomena, but they also present an interesting possibility of mitigation.

\section{Experimental research}

While the acoustical emission of axial compressors has been a  subject of extensive research since more than fifty years ago \cite{nvemec1967noise,griffiths1964spectrum}, studies performed on centrifugal turbocharger are more comparatively more scarce. 

Of the existing literature dealing with centrifugal turbomachinery, many efforts were directed at large fans or centrifugal pumps, with small turbochargers have begun only recently to be more relevant.

Among the first it is worth to mention the extensive research carried out at Pennsylvania State University, where at least since the nineties \cite{mongeau1993sound} an instrumented centrifugal pump was installed in an anechoic flow bench to characterize its acoustical performance.

This setup allowed several researchers to conduct experiments analysing not only in-duct noise, as in the works of Mongeau et al \cite{mongeau1993sound,mongeau1995method} and radiated noise, as shown by Choi et al \cite{choi1994aerodynamic,choi2003experiments}, but also pressure distribution on the blade surface \cite{mongeau1993sound,choi1994aerodynamic}.

A related experimental campaign, starting at least from the turn of the millennium, was carried out at the University of Siegen. Here, Wolfram and Carolus \cite{wolfram2010experimental} used again a centrifugal pump inside an anechoic chamber. 

In this case, measurements were done relying in anemometers directed at the blades that could be rotated at will in order to analyse azimuthal rotating modes \cite{wolfram2009detection}. Radiated noise could be also measured using a single microphone \cite{carolus2000experimental}.

Both Penn State and Siegen campaigns show significant aspects of interest such as the identification of Blade Pass Frequency (BPF) peaks on the acoustic spectrum and hints on the link between stall and noise \cite{mongeau1993sound}. Relevant dimensionless numbers such as the Strouhal and Helmoltz numbers are also suggested \cite{mongeau1995method}. 

However, it is important to notice that there are important differences between these large centrifugal pumps and small turbocharger compressor for automotive purposes.

These differences are found not only in the geometrical design aspect but also in the order of rotational speed. While the considered pumps do not exceed 3600 rpm, small automotive turbochargers can exceed 200 krpm. 

Also, they are affected by the pulsating flow of the alternating engine, which will be present in the acoustic signature as well. This later aspect will be shown in chapter \ref{cap:engine}.

In order to address these differences and focus the literature review in the specific issues presented by turbocharger compressors, selected works from the literature will be reviewed in the following subsections, placing special interest in both the measurement techniques and the acoustical results obtained.

\subsection{Measurement methods}

Although not numerous, acoustical research works on small, centrifugal turbocharger compressors can be found in the literature. Many of them are published by automotive or compressor manufacturers, who are among the most concerned about compressor noise emissions.

Starting from the work of Evans and Ward \cite{evans2005minimizing,evans2006reduction} up to more recent investigations such as those of Åbom, Kabral and Rammal \cite{kabral2013acoustical,kabral2014investigation} at the Swedish Royal Institute of Technology (KTH), several different methodologies for characterising the acoustical emission of automotive turbochargers can be found. 

A first division can be made between those studies aiming to measure \emph{internal} flow noise, placing sensors inside the compressor housing or inlet and outlet ducts, and those works based on external measurements of the \emph{radiated} noise. 

Evans and Ward \cite{evans2005minimizing,evans2006reduction} opted for a simple setup of four microphones, situated on the same plane and directed to a turbocharger-equipped engine. They were distributed at 90º with a separation of one meter from the engine. From the pressure readings of these four microphones, Sound Pressure Level (SPL) was computed and then averaged among them, in order to get a single signal from which spectrograms and spectra could be derived.

\subsection{Relevant results}
\section{Numerical simulations} 
\section{Influence of inlet geometry} 