%!TEX root = Tesis.tex
\chapter{Noise measurement results} \label{cap:results}

% % Quote
% \null\vspace{-10mm}
% \begin{flushright}
% \begin{minipage}[t]{0.5\textwidth}
% {\itshape\small``Take care of the sense, and the sounds will take care of themselves.''}\\[-3mm]

% {\cabincondensed\small\hfill \textbf{Lewis Carroll}, \emph{Alice in Wonderland.}}\\[3mm]
% \end{minipage}
% \end{flushright}

% Chapter text
\noindent In this chapter, the results of applying the described acoustic measurement methodology to the main turbocharger selected in this investigation will be presented and analyzed. A first section of the chapter will deal with in-duct results, regarding both the full spectrum of the turbocharger noise output and the plane wave frequency range, the later being  analyzed in detail with the help of the wave decomposition technique described in the previous chapter.\blfootnote{Work on this chapter has been partly published in the following paper:
\begin{itemize}
\item\emph{Simulations and measurements of automotive turbocharger compressor \emph{whoosh} noise}~\cite{broatch2015simulations}
\end{itemize}
}

Externally radiated noise produced by the turbocharger group will also be discussed, results comprising both inlet orifice noise and the acoustic radiation around the system generated by the turbocharger itself and its inlet and outlet ducts. Additionally, a correlation will be obtained between the internal and orifice noise levels, that could be useful to estimate these external noise levels without the need to test the system in a large anechoic environment.

\section{Internal noise field}\label{sec:res_internal}

The first part of the experimental campaign consisted of gathering sufficient acoustic data across several operating conditions of the compressor, in order to compute not only the acoustic signature in the most typical operating points but also how this signature varied at different conditions.

\begin{figure}[tbh!]
\centering
\vspace{3mm}
\includegraphics[width=0.8\textwidth]{figures/cap_result/map_comp}
\caption[Manufacturer compressor map and measured points]{Comparison of the compressor map provided by the manufacturer and the critical operating points measured in the experimental campaign, including the deep surge limit measured in the gas stand, the approximate working line with the target engine and the 160 krpm speed line.}
\label{fig:result_map_comp}
\end{figure}

Data for this in-duct noise analysis was recorded using the previously described piezoelectric sensor arrays in both inlet and outlet ducts. This data was then processed in the form of pressure and sound intensity and pressure spectra, noise maps, and spectrograms to show different aspects of the turbocharger acoustic behaviour and its evolution across different operating conditions.

These operating conditions at which measurements were performed were selected taking into account the full compressor map provided by the manufacturer and the approximate working line\footnote{Note that at shaft speeds grater than 190 krpm a second turbocharger is activated in this engine, creating a shift in the working line noticeable in the range between 2.5 and 3 pressure ratio.} of the engine operating points in real-life conditions and the capabilities of the compressor test bench. 

Regarding air mass flow, conditions were measured from approximately the maximum efficiency point to a limit slightly beyond the conservative deep surge line indicated by the manufacturer. Pressure ratios were measured up to the maximum capability of the available screw compressor to provide continuous flow.

The two limits conform a measurement subset of the compressor map depicted in Fig. \ref{fig:result_map_comp} in color. This subset encompasses the working line up to a shaft speed of 170 krpm. In addition, the deep surge limit was measured in a separate gas stand to locate more accurately the true limit.

These surge measurements were performed with a tapered duct in the compressor inlet line to match the gas stand pipe diameter, which has an influence on the surge margin. Although, these tests confirmed that the manufacturer margin was close enough to the deep surge measured according to the methodology of low frequency content increase described in chapter \ref{cap:metod}. 

As it can be seen in Fig. \ref{fig:result_map_comp} for most of the measured shaft speed lines the last data point (the one with less air mass flow) was situated approximately in between the conservative limit established by the manufacturer and the limit experimentally measurable in the gas stand. 

This allowed the comparison between optimum compressor operating points situated at near the maximum thermodynamic efficiency conditions and the most critical conditions near deep surge, passing through the unstable mild surge (or marginal surge) region between the two extremes.

\subsection{Spectral content}

Inlet and outlet pressure spectra were measured at each one of the selected points, allowing a comparison in terms of frequency content of the pressure signals and distribution of acoustic intensity across the measured compressor map. In this subsection results will be divided between plane wave frequency range, where the wave decomposition algorithm based on beamforming can be applied to the array signals to compute the forward and backward waves, and the higher frequencies where acoustic modes are present and thus the decomposition cannot be performed.

To provide a better insight on how the spectral content of the pressure signals changes across the different operating conditions two subsets of the measured map have been highlighted.

The first of these subsets is a constant shaft speed line at 160 krpm where the air mass flow is progressively reduced from approximately highest efficiency. This will allow the analysis of how the spectral content changes with mass flow from relatively stable conditions to unstable ones.

A second subset forms a distribution of data points following the working line of the real engine which is depicted in Fig. \ref{fig:result_map_comp}. Since this line is situated at an  approximately constant distance from the surge limit it allows to track the evolution of spectral content given by the change in shaft speed while keeping similarly mild surge operating conditions. 

\subsubsection{Plane wave range}

Though dependent on pipe diameter, flow temperature and velocity, typical automotive turbocharger compressor setups can maintain plane wave conditions up to 5--7 kHz \cite{gaude2008experimental}. It is specially relevant because it comprises the frequencies of relevant acoustic phenomena such as \emph{whoosh} noise, tonal noise induced by BPF harmonics, deep surge, etc.

From the research point of view, this frequency range is interesting because considering the pressure change constant along each pipe section allows for easier analytical and numerical modelling. 

Working in this range allows the use of wave decomposition procedures to isolate the contributions coming forward or backward through the compressor inlet and outlet pipes. These decomposed pressure waves can also be used together to compute the acoustic intensity through the duct.

From chapter \ref{cap:metod} it can be recalled that in a circular duct with mean flow, the cut-off frequency at which the plane wave assumption cannot be considered to be valid any longer is given by equation \ref{eq:f_modes}. At higher frequencies, the first asymmetric acoustic mode starts propagating.

In the case of the pipes considered in this study, evaluation of this expression led to cut-off frequencies of approximately 4.7 kHz for the inlet duct and 7.4 kHz in the case of the outlet duct, due to differences in flow conditions and pipe diameter. Also, it must be taken into account that this limit varied slightly because of changes in velocity and temperature at different operating conditions.

However, the spatial Nyquist-like criterion related to the separation of the sensors of the decomposition array described by equation \ref{eq:f_nyquist} was more restrictive than the asymmetric mode limit, leading to cut-off frequencies for plane wave decomposition to 3.4 kHz for the inlet and 6.4 kHz for the outlet.

\begin{figure}[tbh!]
\centering
\includegraphics[width=0.83\textwidth]{figures/cap_result/spectra_LugLine_PW}
\caption[Spectra along the working line in the plane wave range]{Comparison of pressure spectra along the working line plotted in Fig. \ref{fig:result_map_comp} in the plane wave frequency range, for both inlet (top) and outlet (bottom) ducts. Legend shows the different shaft speeds in krpm.}
\label{fig:result_spectra_LugLine_PW}
\end{figure}

In Fig. \ref{fig:result_spectra_LugLine_PW} the Power Spectral Density (PSD) of the raw pressure signals captured by one piezoelectric sensor of each array is shown. For both inlet and outlet ducts, spectra are plotted at different shaft speeds along the working line depicted in Fig. \ref{fig:result_map_comp}. It can be seen in this figure how, generally, spectra on the plane wave range feature higher levels at lower frequencies. Higher speed points in the outlet feature a distinctive broadband between 1 and 2.5 kHz, described as \emph{whoosh} noise in the literature.

\begin{figure}[tbh!]
\centering
\includegraphics[width=0.83\textwidth]{figures/cap_result/spectra_160_PW}
\caption[Spectra along the 160 krpm line in the plane wave range]{Comparison of pressure spectra along the 160 krpm speed line plotted in Fig. \ref{fig:result_map_comp} in the plane wave frequency range, for both inlet (top) and outlet (bottom) ducts. Legend shows the different mass flow rates in g/s.}
\label{fig:result_spectra_60_PW}
\end{figure}

In Fig. \ref{fig:result_spectra_60_PW} the same procedure has been carried out for the 160 krpm shaft speed line marked with a dashed line in Fig. \ref{fig:result_map_comp}, in order to evaluate the evolution of spectral content as air mass flow is reduced through the mild surge conditions. In the outlet where the \emph{whoosh} phenomenon can be seen between 1 kHz or less and up to 3 kHz with greater clarity. Again, spectral content below approximately 3 kHz appear more ``jagged'' that at higher frequencies.

It is also apparent, specially examining the spectra at 60 g/s and 55 g/s that the broadband is not so apparent at the lowest flow conditions not because it decreases in level but because of the increase in the surrounding frequency content that matches the broadband level, thereby masking it. 

Moreover, it should be remarked that this broadband phenomena at typical \emph{whoosh} noise frequencies can be identified even at the operating point with higher air mass flow, which is located near the maximum efficiency region of the compressor map; a zone which is presumed to be stable.

\subsubsection{Decomposed pressure waves}

In order to better isolate the spectral content coming from the compressor from the reflections and interferences of the upstream and downstream ducting and components, the wave decomposition algorithm based on LCMV beamforming presented in chapter \ref{cap:metod} was applied to the aforementioned pressure data recorded by the three-sensor arrays while following the working line and 160 krpm line operating conditions.

The differences in spectral content between the decomposed pressure waves and the reconstructed total pressure signal can be seen in Fig. \ref{fig:result_spectra_DOPA_tot_comp} where the three are plotted for two opposite operating conditions: higher flow at lower speed and lower flow at higher speed.

\begin{figure}[b!]
\centering
\includegraphics[width=0.83\textwidth]{figures/cap_result/spectra_DOPA_tot_comp2}
\caption[Comparison of total and decomposed spectra]{Comparison of total (solid, thin line) and decomposed pressure spectra (coming from the compressor in solid thick line, going towards the compressor in dotted line) for two extreme points: 80 krpm and highest mass flow and 160 krpm and lower mass flow.}
\label{fig:result_spectra_DOPA_tot_comp}
\end{figure}

It can be seen how, as previously mentioned, the decomposed signal coming from the compressor (backward wave in the inlet, forward wave in the outlet) are almost coincident with the total pressure spectra along the plane wave range, smoothing out the buzz-saw noise at the lower frequencies. 

In contrast, the opposite-travelling wave presents lower amplitudes, specially between 1 and 3 kHz in the high speed, lower flow plot where there is a difference of approximately 10 dB between 2 and 3 kHz.

\begin{figure}[tbh!]
\centering
\includegraphics[width=0.83\textwidth]{figures/cap_result/spectra_LugLine_DOPA}
\caption[Decomposed plane wave spectra along the working line]{Comparison of decomposed pressure spectra along the working line plotted in Fig. \ref{fig:result_map_comp} for both the inlet duct (top plot, backwards travelling waves) and outlet duct (bottom plot, forwards travelling waves).}
\label{fig:result_spectra_LugLine_DOPA}
\end{figure}

Fig. \ref{fig:result_spectra_LugLine_DOPA} shows the PSD of this decomposed pressure signal. In the case of the top plot that contains the information from the inlet duct, PSD from the backward travelling wave is presented, since it is the one coming from the compressor. Conversely, the forward travelling wave PSD is plotted in the case of the bottom plot depicting outlet duct data. Note the spurious aliasing peaks as the spatial Nyquist limit is reached.

It can thus be seen in Fig. \ref{fig:result_spectra_LugLine_DOPA} that the decomposition algorithm allows a clearer representation of the signal, since it is not affected by the buzz-saw effect at lower frequencies (lower than 3 kHz), specially visible in the inlet spectra (see also Fig.~\ref{fig:result_spectra_DOPA_tot_comp} for a comparison).

\begin{figure}[tbh!]
\centering
\includegraphics[width=0.83\textwidth]{figures/cap_result/spectra_160_DOPA}
\caption[Decomposed plane wave spectra along the 160 krpm line]{Comparison of decomposed pressure spectra along the 160 krpm speed line plotted in Fig. \ref{fig:result_map_comp} for both the inlet duct (top plot, backwards travelling waves) and outlet duct (bottom plot, forwards travelling waves).}
\label{fig:result_spectra_160_DOPA}
\end{figure}

A similar result can be seen in Fig. \ref{fig:result_spectra_160_DOPA}, where the decomposed pressure waves are plotted for the data points in the 160 krpm, from higher to lower air mass flows. It shows a clear identification of the \emph{whoosh} broadband phenomenon between 1 kHz and 3 kHz and how it becomes masked by the rise in lower and higher content as air mass flow diminishes. A rotation order tone at 160\,krpm/60\,s = 2.66 kHz is also noticeable in both ducts.

\subsubsection{Higher frequencies}
\label{sub:result_high_freq}

Beyond the previously mentioned cut-off frequencies, the propagation of acoustic modes invalidates the plane wave assumption that it is made in the wave decomposition process, thereby rendering this treatment unusable. It is however interesting to consider at least the raw pressure PSD as it contains relevant acoustic content.

\begin{figure}[b!]
\centering
\includegraphics[width=0.83\textwidth]{figures/cap_result/spectra_LugLine}
\caption[Spectra along the working line in the full range]{Comparison of pressure spectra along the working line  plotted in Fig. \ref{fig:result_map_comp} up to 35 kHz, for both inlet (top) and outlet (bottom) ducts. Cut-off frequencies at 160 krpm for the spatial Nyquist criterion $f_n$, 1${}^\text{st}$ asymmetric ($f_a$) and 1${}^\text{st}$ circular ($f_c$) acoustic modes are also plotted as reference.}
\label{fig:result_spectra_LugLine}
\end{figure}

In Fig. \ref{fig:result_spectra_LugLine} the PSD of the pressure signals recorded by one sensor of the array is presented, for both inlet and outlet ducts. It can be seen how, in the case of the inlet pressure spectra, amplitude increases with shaft speed along the whole frequency range. Inlet spectra also feature a broadband  elevation that starts at 12 kHz and collapses with plane wave propagation at 5 kHz, which could be related to the TCN mentioned in the literature as described in chapter \ref{cap:liter}.

Characteristic Blade Passing Frequency (BPF) peaks associated with each shaft speed are also visible in the inlet spectra, although only for some speeds such as 160 krpm or 170 krpm.

\begin{figure}[tbh!]
\centering
\includegraphics[width=0.83\textwidth]{figures/cap_result/spectra_160}
\caption[Spectra along the 160 krpm line in the full range]{Comparison of pressure spectra along the 160 krpm speed line plotted in Fig. \ref{fig:result_map_comp} up to 35 kHz, for both inlet (top) and outlet (bottom) ducts. Cut-off frequencies for the spatial Nyquist criterion $f_n$, 1${}^\text{st}$ asymmetric ($f_a$) and 1${}^\text{st}$ circular ($f_c$) acoustic modes are also plotted as reference.}
\label{fig:result_spectra_160}
\end{figure}

Evolution of the spectral content in the outlet pressure signal shows also the previously discussed increase at plane wave frequencies, followed by a broadband at higher frequencies which again could be related to the so-called TCN. BPF peaks including some harmonics are visible in the outlet spectra, even for conditions where they were not apparent in the inlet spectra.

Similar evolution can be observed in Fig. \ref{fig:result_spectra_160} which tracks the spectral content of the raw pressure signal along the 160 krpm shaft speed line. Again, BPF peaks are much more visible in the outlet spectra. The plane wave broadband between is also noticeable in the outlet, along with other higher frequency broadbands after the first mode cut-off frequency.

\subsection{Non-dimensional analysis}

Further insight into the spectral content of the signal can be obtained through the use of non-dimensional analysis in order to identify the possible sources or mechanisms behind certain features of the spectra, as done by numerous authors such as Choi \cite{choi1994aerodynamic,choi2003experiments}, Mongeau \cite{mongeau1993sound,mongeau1995method}, Wolfram \cite{wolfram2009detection,wolfram2010experimental} and Carolus \cite{carolus2000experimental} regarding specially the development of similarity laws for centrifugal pumps and fans \cite{neise1982acoustic}.

In this way, non-dimensional power spectral density of the pressure signal $S(f)$ can be written \cite{mongeau1993sound,choi1994aerodynamic} as being the product of two functions: 

\begin{equation}\label{eq:non-dim_spectra}
	\frac{\sqrt{S(f)}}{\rho V^2_\text{tip}\sqrt{D_o/V_\text{tip}}}=
	G(He,\varphi)\cdot F(St,\varphi)
\end{equation}

On the left hand side, $\rho$ is the mean density of the flow, $D_o$ is the outer diameter of the rotor and $V_\text{tip}$ is the velocity of the blade tips. On the right hand side, $G(He,\varphi)$ is an acoustic frequency response function that depends on the Helmholtz number and comprises the influence of propagation effects such as duct resonances and directivity \cite{mongeau1993sound,wolfram2009detection}. 

The source spectral distribution $F(St,\varphi)$ depends on the Strouhal number and comprises the phenomena related to flow-induced acoustic generation such as flow fluctuations and near field acoustic interactions \cite{mongeau1995method}.

Both functions depend of the operating conditions of the system, as remarked by their dependence on the flow coefficient $\varphi$. They are however assumed to be independent of $Re$ (Reynolds number). Definitions of the non-dimensional parameters can be found in table \ref{tab:nondim_numbers}. 

Two Strouhal numbers have been considered: one is related to the tip speed $V_\text{tip}$ and is adjusted so that the blade passing frequency (BPF) has \textit{St}~$=1$. The other takes into account the mean flow velocity $U$. Finally, the Helmholtz number takes into account the speed of sound $a$ that is affected by temperature at each operating condition. All three are normalized with the outer diameter $D_o$ of the rotor.

\begin{figure}[tb!]
\centering
\includegraphics[width=0.46\textwidth]{figures/cap_result/PlotsNormMiniDOPA_st2_Map_Anec_ST51_Straight_160_outlet_sensor1}\hspace{3mm}
\includegraphics[width=0.46\textwidth]{figures/cap_result/PlotsNormMini_st2_Map_Anec_ST51_Straight_160_outlet_sensor1}\\[4mm]
\includegraphics[width=0.46\textwidth]{figures/cap_result/PlotsNormMiniDOPA_st_Map_Anec_ST51_Straight_160_outlet_sensor1}\hspace{3mm}
\includegraphics[width=0.46\textwidth]{figures/cap_result/PlotsNormMini_st_Map_Anec_ST51_Straight_160_outlet_sensor1}\\[4mm]
\includegraphics[width=0.46\textwidth]{figures/cap_result/PlotsNormMiniDOPA_he_Map_Anec_ST51_Straight_160_outlet_sensor1}\hspace{3mm}
\includegraphics[width=0.46\textwidth]{figures/cap_result/PlotsNormMini_he_Map_Anec_ST51_Straight_160_outlet_sensor1}
\caption[Non-dimensional analysis for the 160 krpm line]{Relationship of decomposed non-dimensional spectral density (NSD) in the plane wave range (left) and higher frequency spectra (right) with selected non-dimensional numbers for the different operating conditions of the 160 krpm speed line.}
\label{fig:result_norm_all}
\end{figure}

\begin{figure}[tb!]
\centering
\includegraphics[width=0.46\textwidth]{figures/cap_result/PlotsNormMiniDOPA_st2_Map_Anec_ST51_Straight_LugLine_outlet_sensor1}\hspace{3mm}
\includegraphics[width=0.46\textwidth]{figures/cap_result/PlotsNormMini_st2_Map_Anec_ST51_Straight_LugLine_outlet_sensor1}\\[4mm]
\includegraphics[width=0.46\textwidth]{figures/cap_result/PlotsNormMiniDOPA_st_Map_Anec_ST51_Straight_LugLine_outlet_sensor1}\hspace{3mm}
\includegraphics[width=0.46\textwidth]{figures/cap_result/PlotsNormMini_st_Map_Anec_ST51_Straight_LugLine_outlet_sensor1}\\[4mm]
\includegraphics[width=0.46\textwidth]{figures/cap_result/PlotsNormMiniDOPA_he_Map_Anec_ST51_Straight_LugLine_outlet_sensor1}\hspace{3mm}
\includegraphics[width=0.46\textwidth]{figures/cap_result/PlotsNormMini_he_Map_Anec_ST51_Straight_LugLine_outlet_sensor1}
\caption[Non-dimensional analysis for the working line]{Relationship of decomposed non-dimensional spectral density (NSD) in the plane wave range (left) and higher frequency spectra (right) with selected non-dimensional numbers for the different operating conditions of the engine working line.}
\label{fig:result_norm_all_LL}
\end{figure}

\begin{table}
\centering
\caption[Symbols and definitions of the non-dimensional numbers]{Symbols and definitions of the three non-dimensional numbers considered in the analysis of acoustic spectra.}
\footnotesize
\begin{tabular}{lcc}
\toprule
\textbf{Name} & \textbf{Symbol} & \textbf{Definition}\\\midrule
Strouhal (tip speed)& \textit{St}$(V_\text{tip})$ & $\cfrac{f D_o}{V_\text{tip}}\cfrac{\pi}{N}=\cfrac{f}{BPF}$ \\[4mm]
Strouhal (mean flow) & \textit{St}$(U)$ & $\cfrac{f D_o}{U}$ \\[4mm]
Helmholtz & \textit{He} & $\cfrac{f D_o}{a}$ \\[2mm] \bottomrule
\end{tabular}
\label{tab:nondim_numbers}
\end{table}

In Fig.~\ref{fig:result_norm_all} the non-dimensional spectra as computed in the left hand side of equation \ref{eq:non-dim_spectra} as a function of the non-dimensional number has been plotted for the outlet duct, whose acoustic features appear more clear. Different mass flow rates are considered while shaft speed was kept constant at 160 krpm. 

It can be seen in Fig.~\ref{fig:result_norm_all} how when the spectra Strouhal number that takes into account the mean flow velocity \textit{St}$(U)$ is used to normalize the spectra, the \emph{whoosh} hump appears to align between 1 and 3. Broadband features at higher frequencies (right plots) however do not align.

However, when the Strouhal number takes into account the blade passing frequency these high frequency features appear more aligned, specially the tonal noise at the BPF and its harmonic at \textit{St}$(V_\text{tip})=1$ and 2. 

Helmholtz normalization appears to be quite similar, with the difference that in this case the BPF tones are not aligned. by inspecting the decomposed spectra on the left however, it can be seen that the spurious peak that limits the range of valid beamforming decomposition is aligned as the spatial Nyquist criterion depends on the speed of sound.

In Fig.~\ref{fig:result_norm_all_LL} a similar analysis is carried out considering the spectral information along the working line instead of along a fixed shaft speed. In this case the the results are similar, although in this case the Strouhal normalization that takes into account the tip velocity is more meaningful, as different shaft speeds are considered.

Again, the broadband hump in the plane wave range that is typically identified with \emph{whoosh} appears to collapse when considering the mean flow Strouhal, albeit in this figure tracking of this effect is more complicated due to the simultaneous change in both shaft speed and mass flow rate. 

In the case of the non-decomposed spectral information, it can be seen how the higher frequency TCN-like broadband is now only collapsing when the Helmholtz normalization is performed, whereas for both Strouhal normalizations clear differences can be seen between lower and higher shaft speeds. This could point to this broadband frequency being highly influenced by the speed of sound rather than by the flow mechanisms.

\subsection{Interpolation}

As described in section \ref{sub:noise_maps_and_spectrograms}, the information contained in the discrete data points measured in this experimental campaign can be used to interpolate the results across the selected relevant section of the compressor map. This allows a clearer understanding into how the noise content varies across the different operating conditions of air mass flow and shaft speed.

\subsubsection{Noise maps}

In order to paint a general picture of the global sound emission of the compressor across the selected operating range, in-duct sound intensity has been computed through equation \ref{eq:metod_intensity} for the described data points, and then overall level between frequencies of interest has been interpolated. 

\begin{figure}[tbh!]
\centering
\includegraphics[width=0.75\textwidth]{figures/cap_result/maps_induct_10-3400}
\vspace{-2.25cm}
\caption[In-duct SIL maps between 10 and 3400 Hz]{In-duct Sound Intensity Level maps for inlet (top) and outlet (bottom) ducts between 10 and 3400 Hz, obtained through the interpolation procedure.}
\label{fig:maps_induct_10-3400}
\end{figure}

\begin{figure}[tbh!]
\centering
\includegraphics[width=0.75\textwidth]{figures/cap_result/maps_induct_10-100}
\vspace{-2.25cm}
\caption[In-duct SIL maps between 10 and 100 Hz]{In-duct Sound Intensity Level maps for inlet (top) and outlet (bottom) ducts between 10 and 100 Hz, obtained through the interpolation procedure.}
\label{fig:maps_induct_10-100}
\end{figure}

\begin{figure}[tbh!]
\centering
\includegraphics[width=0.75\textwidth]{figures/cap_result/maps_induct_1000-3000}
\vspace{-2.25cm}
\caption[In-duct SIL maps between 1000 and 3000 Hz]{In-duct Sound Intensity Level maps for inlet (top) and outlet (bottom) ducts between 1000 and 3000 Hz, obtained through the interpolation procedure.}
\label{fig:maps_induct_1000-3000}
\end{figure}

\begin{figure}[tbh!]
\centering
\includegraphics[width=0.89\textwidth]{figures/cap_result/maps_induct_pres_5000-21000_sens1}\\[4mm]
\includegraphics[width=0.89\textwidth]{figures/cap_result/maps_induct_pres_5000-21000_sens2}\\[4mm]
\includegraphics[width=0.89\textwidth]{figures/cap_result/maps_induct_pres_5000-21000_sens3}
\caption[In-duct SPL maps between 5 and 21 kHz]{In-duct Sound Pressure Level maps for inlet (left) and outlet (right) ducts between 5 and 21 kHz Hz, obtained through the interpolation procedure and considering the three sensors.}
\label{fig:maps_induct_SPL_3sens}
\end{figure}

For instance, Fig. \ref{fig:maps_induct_10-3400} shows global in-duct Sound Intensity Level (SIL) contours for both inlet and outlet in the frequency range where the plane wave assumption holds true for both ducts ($f<3.4$ kHz). 

It can be seen that, as previously observed when analysing the pressure spectra, sound intensity levels are generally higher in the outlet duct than in the inlet duct as pressure fluctuations are amplified. 

Distribution of SIL across the different conditions of the compressor map holds some similarities between both ducts, with SIL increasing towards the surge limit and with higher shaft speeds. However, the outlet seems to feature an abrupt increase in levels at the lower allowable air mass flow conditions, whereas the increase in the case of the inlet appears smoother. 

In order to provide a better insight into how frequency content is distributed across the compressor map, Fig. \ref{fig:maps_induct_10-100} shows sound intensity levels at very low frequency, between 10 and 100 Hz.

Comparing Fig. \ref{fig:maps_induct_10-100} with \ref{fig:maps_induct_10-3400} it can be observed how, in the case of the inlet in-duct intensity, very low frequency content increase is parallel to the surge limit line, thus making the slope of the contours different in the two figures.

On the other hand, outlet intensity distribution \hl{is very similar between the two figures, highlighting the importance of the contribution of the low frequency content to the whole plane wave range}.

In order to analyze the distribution of \emph{whoosh} noise which, albeit still in the plane wave range, does not comprise so low frequency content, maps for intensity between 1 and 3 kHz have been plotted in Fig. \ref{fig:maps_induct_1000-3000}.

Differences in the outlet in-duct SIL between between this range and those shown previously in Figs \ref{fig:maps_induct_10-100} and \ref{fig:maps_induct_10-3400} are noticeable. The slope of the SIL contours has changed, showing that the increase in this frequency band is more gradual with shaft speed, whereas in the two previous figures it rose abruptly for the 160 krpm line marked in the plots with dashes.

Finally, data from higher frequencies (above plane wave range) has been plotted in Fig. \ref{fig:maps_induct_SPL_3sens}, showing the differences between the pressure levels recorded by each one of the three sensors of the arrays. This data shows a different pattern in the inlet duct: while still increasing with shaft speed, highest levels are located at medium mass flow settings, not at the lower ones.

As for the outlet, there are again differences when comparing against previously shown distributions in the plane wave range. In this higher frequency case, while content increases again with shaft speed, the point at which the maximum is located varies with mass flow rate.

It can also be seen that in the case of the outlet, there are more differences between the spectral content recorded by each of the sensors of the array that in the case of the inlet where differences are much more difficult to appreciate. This is probably due the flow presentation being much more uniform in the inlet as it is coming from the ambient, while in the case of the outlet it is leaving the compressor with all its associated flow phenomena such as increased swirl.

Furthermore, apart from the different distribution, it can be shown in the data from sensor 1 of the outlet that there was an operating conditions were a spurious low content is recorded, probably due a node affecting the sensors. This highlights the risk of using SPL to characterize compressor in-duct noise levels and the advantages of relying into sound intensity instead.

\subsubsection{Spectrograms}

Beside noise maps, an alternative way of analysing the spectral content across different conditions consists of processing the interpolated data to reconstruct spectrograms along lines of interest; in this case the working line and the 160 krpm lines have been selected as in the previously presented spectral content discussion.

\begin{figure}[tbh!]
\centering
\includegraphics[width=0.9\textwidth]{figures/cap_result/Spec_Inlet_LugLine_SG}\\[5mm]
\includegraphics[width=0.9\textwidth]{figures/cap_result/Spec_Outlet_LugLine_SG}
\caption[Spectrograms along the working line]{Spectrograms for inlet (top) and outlet (bottom) in-duct SIL reconstructed from the interpolated spectral content along the working line.}
\label{fig:result_LugLine_SG}
\end{figure}

\begin{figure}[tbh!]
\centering
\includegraphics[width=0.9\textwidth]{figures/cap_result/Spec_Inlet_iso160krpm_SG}\vspace{5mm}
\includegraphics[width=0.9\textwidth]{figures/cap_result/Spec_Outlet_iso160krpm_SG}
\caption[Spectrograms along the 160 krpm line]{Spectrograms for inlet (top) and outlet (bottom) in-duct SIL reconstructed from the interpolated spectral content along the 160 krpm line.}
\label{fig:result_iso160krpm_SG}
\end{figure}

Figure \ref{fig:result_LugLine_SG} shows how the spectral content of the sound intensity changes along the path of the working line first shown in Fig. \ref{fig:result_map_comp}. This figure shows how, specially in the outlet, a certain, typical \emph{whoosh} content (1 -- 2.5 kHz) is present along the line. Intensity levels present a sudden step after 160 krpm of shaft speed.

In a similar manner, Fig. \ref{fig:result_iso160krpm_SG} shows the evolution of the intensity spectral content as air mass flow is reduced, while keeping a constant corrected shaft speed of 160 krpm. In this case, and again specially in the outlet duct, broadband content between 1 and 3 kHz consistent with \emph{whoosh} noise is evident even at the highest considered mass flows which should be quite stable. This broadband then becomes masked by the rise in levels as mass flow rate is reduced.

It can also be seen in this figure how very low frequency content appears as operating conditions get close to the deep surge limit. In the inlet case this content under 1 kHz is dominant at these unstable conditions, whereas in the case of the outlet the content is more broadband. As for higher frequency content in the inlet duct, in both Figs. \ref{fig:result_LugLine_SG} and \ref{fig:result_iso160krpm_SG} spectral content is not as consistent as in the outlet duct.

\section{External noise}
\label{sec:external_results}

Besides the measurement of in-duct sound pressure and intensity, external measurements of noise were performed during the experimental campaign as well, taking advantage of the anechoic conditions of the chamber where the turbocharger test rig was installed. 

The equipment and methodology used in these experiments was described in section \ref{sec:method_external}. As seen in the literature review presented in chapter \ref{cap:liter}, external measurements of noise are commonplace in the available scientific works, being also a common metric used by manufacturers. It is thus important to relate the obtained in-duct results with these kind of external measurements.

\subsection{Orifice noise}

External measurements of the orifice noise were carried out simultaneously with the in-duct measurements at the same data points, to ensure that the comparison between the two could be meaningful. 

\begin{figure}[b!]
\centering
\includegraphics[width=0.83\textwidth]{figures/cap_result/sil_corr_anec}
\caption[Correlation between in-duct and orifice sound intensity]{Correlation between inlet in-duct sound intensity and orifice sound intensity considering both LCMV beamforming and the Two Microphone Method.}
\label{fig:sil_corr_aneic}
\end{figure}

As previously detailed, data was captured with two free-field omnidirectional microphones placed perpendicular to the turbocharger axis line at distances of 10 and 20 mm from the inlet duct orifice.

In Fig. \ref{fig:sil_corr_aneic} a correlation is shown between the  in-duct sound intensity levels measured in the inlet and those measured by the first microphone situated at 10 mm from the orifice by applying free-field conditions to compute intensity, as detailed in chapter \ref{cap:metod}. In the case of the in-duct SIL, the results from the LCMV beamforming method and the traditional Seybert's \cite{seybert1988two} Two Microphone Method (TMM) are shown, with the LCMV procedure showing a better statistical agreement in terms of $R^2$ with the free-field orifice data. Results from the second microphone offered a similar result.

\subsection{Surface radiation}

Apart from taking advantage of the anechoic environment to measure orifice noise, noise radiation data was also recorded by four additional free-field microphones distributed around the turbocharger assembly, as shown previously in chapter \ref{cap:metod}.

In this case however, unlike the previously described orifice measurements where acquisition was simultaneous with the compressor map characterization and thus they covered the whole selected region, measurements were restricted to the two characteristic lines used in this chapter: the engine working line and the 160 krpm speed line.

\begin{figure}[tbh!]
\centering
\includegraphics[width=0.75\textwidth]{figures/cap_result/LL_SPL_barchart_jag}
\caption[Averaged radiated SPL along the working line]{Averaged RMS Sound Pressure Level as recorded during the sequential damping measurements for different corrected shaft speeds of the engine working line.}
\label{fig:result_spl_radiated_comp_bar_LL}
\end{figure}

Furthermore, a sequential damping of the different parts of the assembly (inlet duct, outlet duct, the turbocharger group itself and nothing at all) was performed using acoustic insulation blankets wrapped around them in order to isolate the distinct contribution of each part. 

Results from this sequential testing procedure can be seen in Fig.~\ref{fig:result_spl_radiated_comp_bar_LL} considering the operating conditions of the working line. In this case full spectrum RMS SPL averaged between the four encircling microphones are shown. 

As expected radiated sound increases with shaft speed as predicted by the in-duct measurements. In almost all conditions sound levels radiated by the outlet are higher that those radiated by the inlet, and those are, in turn, higher than those radiated by the turbocharger group itself.

\begin{figure}[tbh!]
\centering
\includegraphics[width=0.75\textwidth]{figures/cap_result/160_SPL_barchart_jag}
\caption[Averaged radiated SPL along the 160 krpm line]{Averaged RMS Sound Pressure Level as recorded during the sequential damping measurements for different air mass flow rates of the 160 krpm speed line.}
\label{fig:result_spl_radiated_comp_bar_160}
\end{figure}

In Fig.~\ref{fig:result_spl_radiated_comp_bar_160} a similar analysis is carried out for four points of the 160 krpm speed line, from a higher flow rate stable point to a lower flow rate at more unstable conditions. In this case it can be seen again how, as expected, noise content rises towards deep surge.

However, at higher mass flow it is now the inlet duct the one exhibiting higher noise levels. As flow rate is reduced, the outlet duct takes against the lead on noise radiation. Between the last two points that are closer to surge, inlet contribution does not increase, only those of the of the outlet and turbocharger group.

\section{Conclusions}

In this chapter the main results of the experimental measurement campaign are presented. These concern specially the data gathered through in-duct measurements in both inlet and outlet pipes of the compressor, although some data from external measurements is also presented.

Analysis was first performed by considering the spectral signature of the pressure data in two sets of measurements: the working line of the engine which is representative of the different air flow rate and shaft speeds that the compressor may experience in normal operation, and a speed line or characteristic line where corrected shaft speed was maintained at 160 krpm, and mass flow rate was reduced from stable conditions to a point close to the deep surge limit.

Pressure wave decomposition was shown to be a helpful technique to enhance the spectral information coming from the compressor and filter out the reflections coming from the opposite ends of the pipes, thereby aiding to eliminate standing waves and offering  cleaner spectral information.

In these improved results it became easier to identify relevant features such as the Blade Passing Frequency and its harmonics, along with a broadband hump between 1 and 3 kHz that is consistent with a broadband phenomenon known in the literature as \emph{whoosh} noise, being of special concern due to its negative perception in automotive applications.

This particular noise phenomenon was more easily detected in the acoustic signature of the outlet duct, being more prominent at higher shaft speeds. At these higher speeds the hump was noticeable even at the more stable higher mass flow rate conditions. As flow rate was diminished, rising levels of lower and higher frequency phenomena masked the presence of this hump. 

Non-dimensional analysis showed that the hump identified with \emph{whoosh} was sensible to the mean velocity of the flow, rather than to the shaft speed or the sound speed, thus pointing to a flow-related generation rather than to a transmission-related one. The higher frequency TCN-like broadband on the other hand, seems to be heavily dependent of transmission factors such as the sound speed. 

Furthermore, this later phenomenon shows an abrupt collapse at frequencies of plane wave transmission at all the considered points, suggesting that its source mechanism cannot propagate in this condition. This would confirm that this TCN-like phenomenon is indeed different from \emph{whoosh}, which is a broadband noise that, on the other hand, does appear at the plane wave range.

Noise maps were also constructed from the interpolation of in-duct acoustic intensity across the different operating conditions considered in the study. In this way it was shown how onset of low frequency (<100 Hz) phenomena was parallel to the surge limit, whereas content on the \emph{whoosh} band followed a different distribution, with its onset appearing more closely related to the null slope point of the characteristic line rather than to proximity of deep surge.

Analysis of the higher frequency maps became more difficult as pressure wave decomposition could not be performed due to the propagation of higher order acoustic modes that nullify the plane wave assumption required by the procedure. It was observed that while inlet PSD maps were similar for the three sensors of the array, distribution of the levels in the outlet presented important differences even considering the relatively small distance between each sensor, including in some cases spurious readings. 

This reinforces the notion that PSD measurements with a single sensor, specially in the outlet duct where the flow is less uniform and features a high swirl, should be avoided if possible, preferring if feasible acoustic intensity measurements.

In order to visualize how frequency content changed through the two different sets of conditions, spectrograms were constructed from the interpolated data of the maps for both inlet and outlet. Specially in the later it can be seen how content in the \emph{whoosh} noise band is present even at lower shaft speeds and higher mass flow rate conditions. 

Taking advantage of the anechoic environment in which the turbocharger was installed, orifice noise measurements performed with free-field microphones near the inlet mouth were used to compute a correlation with the in-duct intensity. In-duct intensity was calculated with both the beamforming procedure and the two microphone method, with the first resulting in better agreement and less dispersion, thereby confirming that the technique is suitable to estimate orifice noise without the need of an anechoic environment.

Finally, surface radiation was measured in a sequential test in which the different parts were damped: inlet and outlet ducts and turbocharger group. Results from these tests showed that the outlet duct featured generally a higher contribution than the inlet to the radiated noise, except at the higher mass flow rate conditions. It is thereby important to take into account the target operating conditions when deciding which duct design and damping should be prioritized.