%!TEX root = Tesis.tex

\chapter{Noise measurement results \& analysis} \label{cap:results}

% Chapter text
\noindent In this chapter, the results of applying the described acoustic measurement methodology to the \hl{test turbochargers} will be presented and analyzed. The measurements described in this chapter have been carried out in the turbocharger test rig installed in the anechoic chamber of the Institute.

A first section of the chapter will deal with the in-duct results, both regarding the full spectrum of the turbocharger noise output and the plane wave frequency range, which will be analyzed in detail with the help of the wave decomposition technique to calculate the sound intensity.

Externally radiated noise produced by the turbocharger group will be also discussed, the results comprising both inlet orifice noise and the radiation around the system generated by the turbocharger itself and its inlet and outlet ducts.

Finally, some correlations will be offered between the internal and external noise levels, that could be useful to estimate the external noise levels without the need to test the system in a large anechoic environment.

\section{Internal noise field}\label{sec:res_internal}

The first part of the experimental campaign consisted of gathering sufficient acoustic data across several operating conditions of the compressor, in order to compute not only the acoustic signature in the most typical points but also how this signature varied at different conditions.

Data for this in-duct noise analysis was recorded using the previously described piezoelectric sensor arrays in both inlet and outlet ducts. This data was then processed in the form of pressure and sound intensity and pressure spectra, noise maps, and spectrograms to show different aspects of the turbocharger acoustic behaviour and its evolution across different operating conditions.

These operating conditions at which measurements were performed were selected taking into account the full compressor map provided by the manufacturer and the approximate working line\footnote{Note that at shaft speeds grater than 190 krpm a second turbocharger is activated in this engine, creating a shift the working line noticeable between 2.5 and 3 pressure ratios.} of the engine operating points in real-life conditions and the capabilities of the compressor test bench. 

Regarding air mass flow, conditions were measured from approximately the maximum efficiency point to a limit slightly beyond the conservative deep surge line indicated by the manufacturer. Pressure ratios were measured up to the maximum capability of the available screw compressor to provide continuous flow.

\begin{figure}[tbh!]
\centering
\vspace{3mm}
\includegraphics[width=0.8\textwidth]{figures/cap_result/map_comp}
\caption{Comparison of the compressor map provided by the manufacturer and the critical operating points measured in the experimental campaign, including the deep surge limit measured in the gas stand and the approximate lug line of the target engine.}
\label{fig:result_map_comp}
\end{figure}

The two limits conform a measurement subset of the compressor map depicted in Fig. \ref{fig:result_map_comp} in color. This subset encompasses the working line up to a shaft speed of 170 krpm. In addition, the deep surge limit was measured in a separate gas stand to locate more accurately the true limit.

These surge measurements were performed with a tapered duct in the compressor inlet line to match the gas stand pipe diameter, which has an influence on the surge margin. Although, these tests confirmed that the manufacturer margin was close enough to the deep surge measured according to the methodology of low frequency content increase described in chapter \ref{cap:metod}. 

As it can be seen in Fig. \ref{fig:result_map_comp} for most of the measured shaft speed lines the last data point (the one with less air mass flow) was situated approximately in between the conservative limit established by the manufacturer and the limit experimentally measured in the gas stand. 

This allowed the comparison between optimum compressor operating points situated at near the maximum thermodynamic efficiency conditions and the most critical conditions near deep surge, passing through the unstable mild surge (or marginal surge) region between the two extremes.

\subsection{Spectral content}

Inlet and outlet pressure spectra were measured at each one of the selected points, allowing a comparison in terms of frequency content of the pressure signals and distribution of acoustic intensity across the measured compressor map.

In this subsection results will be divided between plane wave frequency range, where the wave decomposition algorithm based on beamforming can be applied to the array signals to compute the forward and backward waves, and the higher frequencies where acoustic modes are present and thus the decomposition cannot be performed.

To provide a better insight on how the spectral content of the pressure signals changes across the different operating conditions two subsets of the measured map have been highlighted.

The first of these subsets is a constant shaft speed line at 160 krpm where the air mass flow is progressively reduced from approximately highest efficiency. This will allow the analysis of how the spectral content changes with mass flow from relatively stable conditions to unstable ones.

A second subset forms a distribution of data points following the working line of the real engine which is depicted in Fig. \ref{fig:result_map_comp}. Since this line is situated at an  approximately constant distance from the surge limit it allows to track the evolution of spectral content given by the change in shaft speed while keeping similarly mild surge operating conditions. 

\subsubsection{Plane wave range}

Though dependent on pipe diameter, flow temperature and velocity, typical automotive turbocharger compressor setups can maintain plane wave conditions up to 5--7 kHz \cite{gaude2008experimental}. It is specially relevant because it comprises the frequencies of acoustic phenomena such as ``whoosh'' noise, tonal noise induced by BPF harmonics, deep surge, etc.

From the research point of view, this frequency range is interesting because considering the pressure change constant along each pipe section allows for easier analytical and numerical modelling. 

Experimentally, working in this range allows the use of wave decomposition procedures to isolate the contributions coming forward or backward through the compressor inlet and outlet pipes. These decomposition procedures can also be used together to compute acoustic intensity through the duct.

\subsubsection{Higher frequencies}

\subsubsection{Nondimensional analysis}

\subsection{Sound intensity evolution}


\section{External noise}
\label{sec:external_results}
\subsection{Orifice noise}
\subsection{Surface radiation}

\section{Correlations}
\label{sec:correlations}