%!TEX root = Tesis.tex

\chapter{Results and analysis} \label{cap:results} %............ Futuro artículo paramétricos

% Chapter text
\noindent 

\section{Internal noise field}
\subsection{Acoustic spectra} 
\subsection{Maps and spectrograms}
\subsection{Adimensional numbers}

\section{External noise radiation}
\label{sec:external_results}
\subsection{Mouth noise} 
\subsection{Surface radiation}

\section{Local variables}
\label{sec:local_variables}

For this investigation, a production automotive turbocharger was chosen. However, prior to its installation in the gas stand, modifications were made to accommodate the required local temperature and pressure sensors and to allow for the installation of transparent ducting for velocity field measurement at a later stage.

The compressor housing was modified by removing the original inducer up to the plane of the blade leading edges and substituting it by a removable straight adaptor piece which incorporated guides for a circumferential array of thermocouples and a miniaturized pressure probe. This adaptor is depicted in red in Figs. \ref{fig:metod_local_meas} and \ref{fig:metod_local_meas_dimen}.

A detachable straight pipe section with a linear thermocouple guide was affixed to this adaptor. This configuration offers the possibility of easily substituting the straight section for others with different geometries such as elbows or tapered ducts.

\hl{For the characterization of the velocity field, both the adaptor and the straight pipe segment were replaced by a transparent straight duct that allowed the use of a Particle Image Velocimetry (PIV) measurement system.}

\begin{figure}[htb!]
\centering
\includegraphics[width=0.60\textwidth]{figures/cap_result/locales_sin_brida}
\caption{CAD view of the proposed local temperature and pressure measurement instrumentation, showing the distribution of thermocouple arrays and pressure probes.}
\label{fig:metod_local_meas}
\end{figure}

\begin{figure}[b!]
\centering
\includegraphics[width=0.40\textwidth]{figures/cap_result/medidas_locales_corte.PDF}
\raisebox{6mm}{\includegraphics[width=0.55\textwidth]{figures/cap_result/front_ann.PDF}}
\caption{Annotated section view of the pressure probes (purple) and thermocouples (green) location.}
\label{fig:metod_local_meas_dimen}
\end{figure}

\subsection{Temperature}
\label{sub:temperature}

\subsubsection{Radial profile}
Before the measurement of the whole map, a sweep of temperature measurements along the inlet radius was performed for both the thermocouple located at 14mm and the thermocouple located at 84mm. For these measurements, the turbocharger compressor was brought to conditions close to deep surge at 80 krpm.

The results can be seen in Fig. \ref{fig:barrido_temp}. As expected, overall temperature profile is higher at 14mm from the leading edge plane than at 84mm. Temperature seems to be approximately constant up to a distance of 5mm from the internal wall, then decaying up to a distance of approximately 10mm from the centreline (which is at 21mm from the wall). From there onwards, temperature is again approximately constant.

\begin{figure}[tb!]
\centering
\includegraphics[width=0.70\textwidth]{figures/cap_result/barridoTemp_isol}
\caption{Evolution of temperature measured by the first and last thermocouples of the linear array for different distances from the wall.}
\label{fig:barrido_temp}
\end{figure}

Ambient temperature and surface temperature of the adaptor piece (seen if Figs. \ref{fig:metod_local_meas} and \ref{fig:metod_local_meas_dimen} in red) can also be seen in Fig. \ref{fig:barrido_temp}. As expected, centreline temperature at the most upstream point is almost coincident with the ambient, while fluid temperatures near the wall at the most downstream point (at just 14mm from the wheel) are almost equal to those of the piece (the small initial difference is due to the cold start and thermal resistance).

\begin{figure}[htb!]%[34]{e}[1.25cm]{0.55\textwidth}
\centering
\includegraphics[width=0.48\textwidth]{figures/cap_result/temperatures_rise_line-80.pdf}
\includegraphics[width=0.48\textwidth]{figures/cap_result/temperatures_rise_line-100.pdf}
\includegraphics[width=0.48\textwidth]{figures/cap_result/temperatures_rise_line-120.pdf}
\includegraphics[width=0.48\textwidth]{figures/cap_result/temperatures_rise_line-140.pdf}
\includegraphics[width=0.48\textwidth]{figures/cap_result/temperatures_rise_line-160.pdf}
\caption{Evolution of temperature rise measured by the linear array thermocouples for different mass flow settings and different shaft speeds, taking the most upstream temperature as reference.}
\label{fig:temp_line_linear}
\end{figure}

From this results it is apparent that measuring the fluid temperature near the wall is not practical due to the thermal transfer from the wall to the thermocouple. Standardized procedures for characterization of flow temperature call for measurements at R/2 and R/3. In this case, this would correspond to distances of 7mm and 10.5mm from the wall. It can be seen in Fig. \ref{fig:barrido_temp} that those distances show indeed a depart from wall-influenced temperatures. Ultimately, the distance of R/2 (10.5mm) was chosen for this study.

\subsubsection{Backflow growth}

By analysing the evolution of temperature results over different operating conditions (speed and air mass flow) it is possible to estimate the growth of the recirculating backflow.

Figure \ref{fig:temp_line_linear} shows different temperature profiles measured by the linear thermocouple array at a same shaft speed and different mass flows. It can be seen how at higher mass flows the temperature along the inlet pipe is approximately equal to that of the ambient.

When mass flows is reduced a temperature increase coming from the compressor wheel starts to be noticeable, denoting the presence of air heated by the compression process which is reversing in direction due to flow instabilities on the wheel channels. The operating condition of backflow onset appears to correspond to that of maximum pressure ratio at each shaft speed.

Temperature increase grows both in difference from the ambient and distance from the wheel. However, this growth appears to be highly dependent of the particular shaft speed. At lower shaft speeds, growth reaches greater distances from the wheel at lower air flows, but maximum temperature difference is lower. Conversely, at higher shaft speed temperature difference is greater (due to higher pressure ratio) but backflow extent along the inlet pipe is more constrained.

The reason for this constriction of backflow length at higher shaft speed could be probably attributed to \hl{higher inlet flow momentum} at those operating conditions. Reduced momentum at lower shaft speeds would instead allow the compressed hot backflow to reach farthest into the inlet line when backflow occurs. 

\hl{In order to ensure that the temperature increase along the inlet pipe measured through the thermocouples is indeed caused by compressor backflow and not by thermal diffusion from the compressor housing through the metallic pipe, thermographic images were captured.

These were synchronized with the operating conditions at which flow temperatures were measured inside the duct. Inlet pipe radiated wall temperature profiles were extracted to compare their growth against those of the internal flow. In this case, profiles appear continuous and coherent with heat diffusion. No onset point of temperature difference with air mass flow could be identified, contrary to the case of internal flow measurement.}

\subsubsection{Circumferential profile}

\begin{figure*}[htb!]
\centering
\includegraphics[width=0.25\textwidth]{figures/cap_result/temperatures_circ_line-80.pdf}
\includegraphics[width=0.25\textwidth]{figures/cap_result/temperatures_circ_line-100.pdf}
\includegraphics[width=0.25\textwidth]{figures/cap_result/temperatures_circ_line-120.pdf}
\includegraphics[width=0.25\textwidth]{figures/cap_result/temperatures_circ_line-140.pdf}
\includegraphics[width=0.25\textwidth]{figures/cap_result/temperatures_circ_line-160.pdf}
\caption{Evolution of temperature measured by the circumferential array thermocouples as air mass flow was reduced keeping shaft speed constant. A polar temperature plot is shown for each speed, colors indicating mass flow from higher (blues) to lower (reds). Legends for colors at each speed are available in Fig. \ref{fig:temp_line_linear}. See Fig. \ref{fig:metod_local_meas_dimen} for reference position of the compressor volute.}
\label{fig:temp_circ_all}
\end{figure*}

Temperature measurements were taken using the circumferential thermocouple array situated at 14mm from the compressor leading edge. The objective in this case was to characterize the angular distribution of temperature at different operating conditions.

An example of these circumferential profiles can be seen in Fig. \ref{fig:temp_circ_all}, where temperature readings at different mass flow conditions are plotted, for the different shaft speeds considered in the study. 

\begin{figure*}[htb!]
\centering
\includegraphics[width=0.98\textwidth]{figures/cap_result/temp_maps.pdf}
\caption{\hl{Evolution of temperature rise measured by the linear array thermocouples for different mass flow settings at 140 krpm, taking the most upstream temperature as reference.}}
\label{fig:temp_maps}
\end{figure*}

It can be seen that the skewness of the circumferential profile grows with the reduction of mass flow. At higher flow, skewness is practically zero, the profile being completely regular. However, again at the point of higher pressure ratio, some skewness starts to appear along with an increase in temperature due to hot backflow.

As air mass flow is further decreased towards compressor surge limit, both overall temperature and profile skewness increased. \hl{The pattern of the skewness appeared to be similar at all conditions: temperature was lower at angles corresponding to the start of the volute and higher at angles corresponding to the end of the volute, where its cross-section is larger.}

\subsection{Velocity}
\label{sub:velocity}

\subsubsection{Longitudinal plane}

\begin{figure*}[htb!]
\centering
\includegraphics[width=1.05\textwidth]{figures/cap_result/longitudinal}
\caption{Results of the axial velocity field measurement (longitudinal plane) for the two operating points considered: low}
\label{fig:longitudinal}
\end{figure*}

As described in the previous section, longitudinal plane measurements were the most difficult since the seeding oil particles blocked the line of sight of the camera. However with the right combination of seeding flow, oil dilution and quick operation of the system, valid snapshot pairs were obtained.

In order to capture the difference between stable and partially stalled conditions (also called mild surge or marginal surge conditions) two back-pressure valve positions were tested, while maintaining the same shaft speed of 140 krpm.

PIV postprocessing was carried out for 10 snapshot pairs for each conditions. While the lower third of the frames were obscured by a reflection, at least 2/3 of the longitudinal plane could be resolved in sufficient quality. The two resulting averaged velocity fields can be seen in Fig. \ref{fig:longitudinal}.

In these test there was no direct measurement of inlet air mass flow, so an average of the measured velocity field is taken to identify the precise operating point of the compressor.

At the higher mass flow, it can be seen that the velocity field is approximately homogeneous along the duct and only near the wall the velocity diminishes. However, when flow is reduced, the spatial distribution of velocity changes. The expected compressor backflow is apparent as the low velocity region near the wall increases in thickness, reducing the useful diameter from $\sim 19\,$mm to $\sim 13\,$mm.

This reduction in diameter extends to approximately $\sim 55\,$mm, this is, approximately half of the transparent glass. It can also be seen in Fig. \ref{fig:longitudinal} how this reduction in useful diameter appears to be paired with an increase in flow velocity. 

To better quantify this increase in speed, velocity data along the turbocharger axis line (marked with dotted line in Fig. \ref{fig:longitudinal}) from each valid snapshot has been plotted in Fig. \ref{fig:line_comp_long} for both operating conditions. 

While at the high flow operating point the flow exhibits an approximately constant velocity along the line, only decreasing adjacent to the axis nut, at the lower flow condition it can be seen how after a steady initial segment the velocity increases to a top average speed of 91.73 ms${}^{-1}$. The average speed at the initial segment between 80 and 90 mm was found to be 43.68 ms${}^{-1}$.

Using this data it is possible to draw a simple correlation between the area constriction and the increase of speed. Using the mass conservation equation between the initial segment (1) and the top speed point (2):

\begin{equation}
  \dot{m}_1 = \dot{m_2} = \rho_1 V_1 \pi r_1^2 = \rho_2 V_2 \pi r_2^2
\end{equation}

If a small enough increase of temperature between point (1) and point (2) is assumed so that $\rho_1\sim\rho_2$ the expression can be rearranged to obtain the relationship between radius and velocity changes:

\begin{equation}
  \frac{V_1}{V_2} = \frac{r_2^2}{r_1^2}
\end{equation}

If now the aforementioned approximate values for radii and velocities are substituted into the equation, we can obtain an estimation of the relative error $\varepsilon$ between the theoretical value and the experimental observation:

\begin{equation}
  \varepsilon\, (\%) = \left|\frac{V_1/V_2-r_2^2/r_1^2}{r_2^2/r_1^2}\right| \cdot 100 = 1.71\%
\end{equation}
% \left|\frac{43.68/91.73-13^2/19^2}{13^2/19^2}\right| \cdot 100 = 

The small relative difference between the prediction and the observation reinforces the assumption that the measurement technique is in fact able to characterize the reduction in useful area associated with the backflow ans its associated increase in velocity, even if the axial speed of backflow itself cannot be quantitatively measured.

\begin{figure}[htb!]
\centering
\includegraphics[width=0.65\textwidth]{figures/cap_result/line_comparison_longCent_ann}
\caption{Velocity magnitude over the centreline of the longitudinal plane at high and low flow. Solid line indicates the mean and points correspond to individual snapshot pairs.}
\label{fig:line_comp_long}
\end{figure}

\subsubsection{Transversal plane}

\begin{figure*}[htb!]
\centering
\includegraphics[width=1.05\textwidth]{figures/cap_result/slices_2}
\caption{CAD view of the straight transparent duct for PIV measurements with its mounting assembly.}
\label{fig:res_trans_20mm}
\end{figure*}

\begin{figure*}[b!]
\centering
\includegraphics[width=1.05\textwidth]{figures/cap_result/slices_3}
\caption{Velocity magnitude averaged over the all the captured frame pairs for three different transversal planes.}
\label{fig:res_trans_20mm}
\end{figure*}

\begin{figure}[b!]
\centering
\includegraphics[width=0.65\textwidth]{figures/cap_result/line_comparison_3slices}
\caption{Velocity magnitude over the centreline of three transversal planes. Solid line indicates the mean and points correspond to individual snapshot pairs.}
\label{fig:line_comp_3slices}
\end{figure}

\subsection{Pressure}
\label{sub:pressure}

\subsubsection{Local pressure}

As shown in Figs. \ref{fig:metod_local_meas} and \ref{fig:metod_local_meas_dimen}, miniature pressure probes were fitted in both the inducer and the diffuser of the compressor.

A simple metric of sound level, used by most authors, is Sound Pressure Level (SPL). It is a scalar magnitude, and can be calculated from the pressure signal as:
\begin{equation}
  \text{SPL}_\text{dB} = 20\log_{10}\left(\frac{\bar P}{p_0}\right)
\end{equation}

Where $\bar P$ is the RMS mean of the pressure wave amplitude, restricted at the desired frequency range, and $p_0$ is a reference pressure value of $2\cdot 10^{-5}$ Pa.

\section{Correlations}
\label{sec:correlations}

\subsection{Local variables}
\subsubsection{Backflow and sound intensity}

\begin{figure}[htb!]
\centering
\includegraphics[width=0.6\textwidth]{figures/cap_result/corr_delta_int.pdf}
\caption{Correlation between standard deviation $\sigma$ at the circumferential thermocouple array and inlet sound intensity level in the 1--3 kHz band. Size indicates shaft speed from low to high and color indicates mass flow from higher (blues) to lower (reds).}
\label{fig:corr1}
\end{figure}

In order to attest for a linkage between inlet flow instabilities at marginal surge conditions and the broadband noise at subsynchronous speeds identified by several authors, it is possible to correlate the aforementioned thermal phenomena with sound level in that range.

Figure \ref{fig:corr1} shows a plot of \hl{inlet sound intensity level against standard deviation of temperature in the circumferential thermocouple array}. It can be seen how for lower \hl{temperature deviation (near 0.5$^{\circ}$C, which is the expected deviation of the thermocouples)} sound intensity appears to be uncorrelated, varying mainly due to higher shaft speeds. After \hl{temperature deviation reaches approximately 1$^{\circ}$C} however, a linear trend emerges. 

A least squares linear fitting was performed, showing that there is a good correlation between increase in sound intensity at the selected frequency band and increase in the skewness of the inlet circumferential temperature distribution, thereby suggsting a link between the two phenomena.

